\documentclass{report}
    \usepackage[a4paper,margin=2cm]{geometry}
    \usepackage[utf8]{inputenc}
    \usepackage[T2A]{fontenc}
    \usepackage[russian]{babel}
    \usepackage{indentfirst}
    \setlength{\parindent}{0pt}
    \usepackage{textcomp}
    \usepackage{soulutf8}
    \setuloverlap{0pt}
    \usepackage[ampersand]{easylist}
    \newcommand{\beginEasyList}{
        \begin{easylist}[enumerate]
            \ListProperties(Numbers=a,Hide2=1,Hide3=1,Style*=,Mark=.,FinalMark={)},FinalMark1=.)
    }
    \newcommand{\eEasyList}{\end{easylist}}
    \setcounter{secnumdepth}{-2}
\begin{document}
\setcounter{page}{2}

\section{{\bf Раздел I. Общие положения}}
\section{{\bf Подраздел 2. Лица}}
\subsection{{\bf Глава 3. Граждане (физические лица)}}

\subsubsection{{\bf Статья 17.} Правоспособность гражданина}
\beginEasyList
& Способность иметь гражданские права и нести обязанности (гражданская правоспособность) признается в равной мере за всеми гражданами.
& Правоспособность гражданина возникает в момент его рождения и прекращается смертью.
\eEasyList
\subsubsection{{\bf Статья 18.} Содержание правоспособности граждан}
\par Граждане могут иметь имущество на праве собственности; наследовать и завещать имущество; заниматься предпринимательской и любой иной не запрещенной законом деятельностью; создавать юридические лица самостоятельно или совместно с другими гражданами и юридическими лицами; совершать любые не противоречащие закону сделки и участвовать в обязательствах; избирать место жительства; иметь права авторов произведений науки, литературы и искусства, изобретений и иных охраняемых законом результатов интеллектуальной деятельности; иметь иные имущественные и личные неимущественные права.

\subsubsection{{\bf Статья 23.} Предпринимательская деятельность гражданина}
\beginEasyList
& Гражданин вправе заниматься предпринимательской деятельностью без образования юридического лица с момента государственной регистрации в качестве индивидуального предпринимателя, за исключением случаев, предусмотренных абзацем вторым настоящего пункта.
\par В отношении отдельных видов предпринимательской деятельности законом могут быть предусмотрены условия осуществления гражданами такой деятельности без государственной регистрации в качестве индивидуального предпринимателя.
& Утратил силу с 1 марта 2013 г.
& К предпринимательской деятельности граждан, осуществляемой без образования юридического лица, соответственно применяются правила настоящего Кодекса, которые регулируют деятельность юридических лиц, являющихся коммерческими организациями, если иное не вытекает из закона, иных правовых актов или существа правоотношения.
& Гражданин, осуществляющий предпринимательскую деятельность без образования юридического лица с нарушением требований \ul{пункта 1} настоящей статьи, не вправе ссылаться в отношении заключенных им при этом сделок на то, что он не является предпринимателем. Суд может применить к таким сделкам правила настоящего Кодекса об обязательствах, связанных с осуществлением предпринимательской деятельности.
& Граждане вправе заниматься производственной или иной хозяйственной деятельностью в области сельского хозяйства без образования юридического лица на основе соглашения о создании крестьянского (фермерского) хозяйства, заключенного в соответствии с законом о крестьянском (фермерском) хозяйстве.
\par Главой крестьянского (фермерского) хозяйства может быть гражданин, зарегистрированный в качестве индивидуального предпринимателя.
\eEasyList
\subsubsection{{\bf Статья 24.} Имущественная ответственность гражданина}
\par Гражданин отвечает по своим обязательствам всем принадлежащим ему имуществом, за исключением имущества, на которое в соответствии с законом не может быть обращено взыскание.
\par Перечень имущества граждан, на которое не может быть обращено взыскание, устанавливается гражданским процессуальным законодательством.
\subsubsection{{\bf Статья 25.} Несостоятельность (банкротство) гражданина}
\beginEasyList
& Гражданин, который не способен удовлетворить требования кредиторов по денежным обязательствам и (или) исполнить обязанность по уплате обязательных платежей, может быть признан несостоятельным (банкротом) по решению арбитражного суда.
& Основания, порядок и последствия признания арбитражным судом гражданина несостоятельным (банкротом), очередность удовлетворения требований кредиторов, порядок применения процедур в деле о несостоятельности (банкротстве) гражданина устанавливаются законом, регулирующим вопросы несостоятельности (банкротства).
\eEasyList
\subsubsection{{\bf Статья 26.} Дееспособность несовершеннолетних в возрасте от четырнадцати до восемнадцати лет}
\beginEasyList
& Несовершеннолетние в возрасте от четырнадцати до восемнадцати лет совершают сделки, за исключением названных в \ul{пункте 2} настоящей статьи, с письменного согласия своих законных представителей --- родителей, усыновителей или попечителя.
\par Сделка, совершенная таким несовершеннолетним, действительна также при ее последующем письменном одобрении его родителями, усыновителями или попечителем.
& Несовершеннолетние в возрасте от четырнадцати до восемнадцати лет вправе самостоятельно, без согласия родителей, усыновителей и попечителя:
&& распоряжаться своими заработком, стипендией и иными доходами;
&& осуществлять права автора произведения науки, литературы или искусства, изобретения или иного охраняемого законом результата своей интеллектуальной деятельности;
&& в соответствии с \ul{законом} вносить вклады в кредитные организации и распоряжаться ими;
&& совершать мелкие бытовые сделки и иные сделки, предусмотренные \ul{пунктом 2 статьи 28} настоящего Кодекса.
\par По достижении шестнадцати лет несовершеннолетние также вправе быть членами кооперативов в соответствии с законами о кооперативах.
& Несовершеннолетние в возрасте от четырнадцати до восемнадцати лет самостоятельно несут имущественную ответственность по сделкам, совершенным ими в соответствии с \ul{пунктами 1} и \ul{2} настоящей статьи. За причиненный ими вред такие несовершеннолетние несут ответственность в соответствии с настоящим \ul{Кодексом.}
& При наличии достаточных оснований суд по ходатайству родителей, усыновителей или попечителя либо органа опеки и попечительства может ограничить или лишить несовершеннолетнего в возрасте от четырнадцати до восемнадцати лет права самостоятельно распоряжаться своими заработком, стипендией или иными доходами, за исключением случаев, когда такой несовершеннолетний приобрел дееспособность в полном объеме в соответствии с \ul{пунктом 2 статьи 21} или со \ul{статьей 27} настоящего Кодекса.
\eEasyList
\subsubsection{{\bf Статья 27.} Эмансипация}
\beginEasyList
& Несовершеннолетний, достигший шестнадцати лет, может быть объявлен полностью дееспособным, если он работает по трудовому договору, в том числе по контракту, или с согласия родителей, усыновителей или попечителя занимается предпринимательской деятельностью.
\par Объявление несовершеннолетнего полностью дееспособным (эмансипация) производится по решению органа опеки и попечительства --- с согласия обоих родителей, усыновителей или попечителя либо при отсутствии такого согласия --- по решению суда.
& Родители, усыновители и попечитель не несут ответственности по обязательствам эмансипированного несовершеннолетнего, в частности по обязательствам, возникшим вследствие причинения им вреда.
\eEasyList
\subsubsection{{\bf Статья 28.} Дееспособность малолетних}
\beginEasyList
& За несовершеннолетних, не достигших четырнадцати лет (малолетних), сделки, за исключением указанных в \ul{пункте 2} настоящей статьи, могут совершать от их имени только их родители, усыновители или опекуны.
\par К сделкам законных представителей несовершеннолетнего с его имуществом применяются правила, предусмотренные \ul{пунктами 2} и \ul{3 статьи 37} настоящего Кодекса.
& Малолетние в возрасте от шести до четырнадцати лет вправе самостоятельно совершать:
&& мелкие бытовые сделки;
&& сделки, направленные на безвозмездное получение выгоды, не требующие нотариального удостоверения либо государственной регистрации;
&& сделки по распоряжению средствами, предоставленными законным представителем или с согласия последнего третьим лицом для определенной цели или для свободного распоряжения.
& Имущественную ответственность по сделкам малолетнего, в том числе по сделкам, совершенным им самостоятельно, несут его родители, усыновители или опекуны, если не докажут, что обязательство было нарушено не по их вине. Эти лица в соответствии с \ul{законом} также отвечают за вред, причиненный малолетними.
\eEasyList

\subsection{{\bf Глава 4. Юридические лица}}
\subsection{{\bf § 1. Основные положения}}
\subsubsection{{\bf Статья 48.} Понятие юридического лица}
\beginEasyList
& Юридическим лицом признается организация, которая имеет обособленное имущество и отвечает им по своим обязательствам, может от своего имени приобретать и осуществлять гражданские права и нести гражданские обязанности, быть истцом и ответчиком в суде.
& Юридическое лицо должно быть зарегистрировано в едином государственном реестре юридических лиц в одной из организационно-правовых форм, предусмотренных настоящим \ul{Кодексом}.
& К юридическим лицам, на имущество которых их учредители имеют вещные права, относятся государственные и муниципальные унитарные предприятия, а также учреждения.
К юридическим лицам, в отношении которых их участники имеют корпоративные права, относятся корпоративные организации (\ul{статья 65.1}).
& Правовое положение Центрального банка Российской Федерации (Банка России) определяется \ul{Конституцией} Российской Федерации и \ul{законом} о Центральном банке Российской Федерации.
\eEasyList
\subsubsection{{\bf Статья 49.} Правоспособность юридического лица}
\beginEasyList
    & Юридическое лицо может иметь гражданские права, соответствующие целям деятельности, предусмотренным в его учредительном документе (\ul{статья 52}), и нести связанные с этой деятельностью обязанности.
    \par Коммерческие организации, за исключением унитарных предприятий и иных видов организаций, предусмотренных законом, могут иметь гражданские права и нести гражданские обязанности, необходимые для осуществления любых видов деятельности, не запрещенных законом.
    \par В случаях, предусмотренных законом, юридическое лицо может заниматься отдельными видами деятельности только на основании специального разрешения (лицензии), членства в саморегулируемой организации или выданного саморегулируемой организацией свидетельства о допуске к определенному виду работ.
    & Юридическое лицо может быть ограничено в правах лишь в случаях и в порядке, предусмотренных законом. Решение об ограничении прав может быть оспорено юридическим лицом в суде.
    & Правоспособность юридического лица возникает с момента внесения в единый государственный реестр юридических лиц сведений о его создании и прекращается в момент внесения в указанный реестр сведений о его прекращении.
    \par Право юридического лица осуществлять деятельность, для занятия которой необходимо получение специального разрешения (лицензии), членство в саморегулируемой организации или получение свидетельства саморегулируемой организации о допуске к определенному виду работ, возникает с момента получения такого разрешения (лицензии) или в указанный в нем срок либо с момента вступления юридического лица в саморегулируемую организацию или выдачи саморегулируемой организацией свидетельства о допуске к определенному виду работ и прекращается при прекращении действия разрешения (лицензии), членства в саморегулируемой организации или выданного саморегулируемой организацией свидетельства о допуске к определенному виду работ.
    & Гражданско-правовое положение юридических лиц и порядок их участия в гражданском обороте (статья 2) регулируются настоящим Кодексом. Особенности гражданско-правового положения юридических лиц отдельных организационно-правовых форм, видов и типов, а также юридических лиц, созданных для осуществления деятельности в определенных сферах, определяются настоящим Кодексом, другими законами и иными правовыми актами.
    & К юридическим лицам, создаваемым Российской Федерацией на основании специальных федеральных законов, положения настоящего Кодекса о юридических лицах применяются постольку, поскольку иное не предусмотрено специальным федеральным законом о соответствующем юридическом лице.
\eEasyList
\subsubsection{{\bf Статья 50.} Коммерческие и некоммерческие организации}
\beginEasyList
& Юридическими лицами могут быть организации, преследующие извлечение прибыли в качестве основной цели своей деятельности (коммерческие организации) либо не имеющие извлечение прибыли в качестве такой цели и не распределяющие полученную прибыль между участниками (некоммерческие организации).
& Юридические лица, являющиеся коммерческими организациями, могут создаваться в организационно-правовых формах хозяйственных товариществ и обществ, крестьянских (фермерских) хозяйств, хозяйственных партнерств, производственных кооперативов, государственных и муниципальных унитарных предприятий.
& Юридические лица, являющиеся некоммерческими организациями, могут создаваться в организационно-правовых формах:
&& потребительских кооперативов, к которым относятся в том числе жилищные, жилищно-строительные и гаражные кооперативы, общества взаимного страхования, кредитные кооперативы, фонды проката, сельскохозяйственные потребительские кооперативы;
&& общественных организаций, к которым относятся в том числе политические партии и созданные в качестве юридических лиц профессиональные союзы (профсоюзные организации), органы общественной самодеятельности, территориальные общественные самоуправления;
&&& общественных движений;
&& ассоциаций (союзов), к которым относятся в том числе некоммерческие партнерства, саморегулируемые организации, объединения работодателей, объединения профессиональных союзов, кооперативов и общественных организаций, торгово-промышленные палаты;
&& товариществ собственников недвижимости, к которым относятся в том числе товарищества собственников жилья, садоводческие или огороднические некоммерческие товарищества;
&& казачьих обществ, внесенных в государственный реестр казачьих обществ в Российской Федерации;
&& общин коренных малочисленных народов Российской Федерации;
&& фондов, к которым относятся в том числе общественные и благотворительные фонды;
&& учреждений, к которым относятся государственные учреждения (в том числе государственные академии наук), муниципальные учреждения и частные (в том числе общественные) учреждения;
&& автономных некоммерческих организаций;
&& религиозных организаций;
&& публично-правовых компаний;
&& адвокатских палат;
&& адвокатских образований (являющихся юридическими лицами);
&& государственных корпораций;
&& нотариальных палат.
& Некоммерческие организации могут осуществлять приносящую доход деятельность, если это предусмотрено их уставами, лишь постольку, поскольку это служит достижению целей, ради которых они созданы, и если это соответствует таким целям.
& Некоммерческая организация, уставом которой предусмотрено осуществление приносящей доход деятельности, за исключением казенного и частного учреждений, должна иметь достаточное для осуществления указанной деятельности имущество рыночной стоимостью не менее минимального размера уставного капитала, предусмотренного для обществ с ограниченной ответственностью (\ul{пункт 1 статьи 66.2}).
& К отношениям по осуществлению некоммерческими организациями своей основной деятельности, а также к другим отношениям с их участием, не относящимся к предмету гражданского законодательства (статья 2), правила настоящего Кодекса не применяются, если законом или уставом некоммерческой организации не предусмотрено иное.
\eEasyList
\subsubsection{{\bf Статья 50.1.} Решение об учреждении юридического лица}
\beginEasyList 
    & Юридическое лицо может быть создано на основании решения учредителя (учредителей) об учреждении юридического лица.
    & В случае учреждения юридического лица одним лицом решение о его учреждении принимается учредителем единолично.
    \par В случае учреждения юридического лица двумя и более учредителями указанное решение принимается всеми учредителями единогласно.
    & В решении об учреждении юридического лица указываются сведения об учреждении юридического лица, утверждении его устава, а в случае, предусмотренном \ul{пунктом 2 статьи 52} настоящего Кодекса, о том, что юридическое лицо действует на основании типового устава, утвержденного уполномоченным государственным органом, о порядке, размере, способах и сроках образования имущества юридического лица, об избрании (назначении) органов юридического лица.
    \par В решении об учреждении корпоративного юридического лица (\ul{статья 65.1}) указываются также сведения о результатах голосования учредителей по вопросам учреждения юридического лица, о порядке совместной деятельности учредителей по созданию юридического лица.
    \par В решении об учреждении юридического лица указываются также иные сведения, предусмотренные законом.
    & В случае создания наследственного фонда (\ul{статья 123.20-1}) решение об учреждении наследственного фонда принимается гражданином при составлении им завещания и должно содержать сведения об учреждении наследственного фонда после смерти этого гражданина, об утверждении этим гражданином устава наследственного фонда и условий управления наследственным фондом, о порядке, размере, способах и сроках образования имущества наследственного фонда, лицах, назначаемых в состав органов данного фонда, или о порядке определения таких лиц.
    \par После смерти гражданина нотариус, ведущий наследственное дело, направляет в уполномоченный государственный орган заявление о государственной регистрации наследственного фонда с указанием имени или наименования лица (лиц), осуществляющего полномочия единоличного исполнительного органа фонда.
\eEasyList
\subsubsection{{\bf Статья 51.} Государственная регистрация юридических лиц}
\beginEasyList
& Юридическое лицо подлежит государственной регистрации в уполномоченном государственном органе в порядке, предусмотренном законом о государственной регистрации юридических лиц.
& Данные государственной регистрации включаются в единый государственный реестр юридических лиц, открытый для всеобщего ознакомления.
\par Лицо, добросовестно полагающееся на данные единого государственного реестра юридических лиц, вправе исходить из того, что они соответствуют действительным обстоятельствам. Юридическое лицо не вправе в отношениях с лицом, полагавшимся на данные единого государственного реестра юридических лиц, ссылаться на данные, не включенные в указанный реестр, а также на недостоверность данных, содержащихся в нем, за исключением случаев, если соответствующие данные включены в указанный реестр в результате неправомерных действий третьих лиц или иным путем помимо воли юридического лица.
\par Юридическое лицо обязано возместить убытки, причиненные другим участникам гражданского оборота вследствие непредставления, несвоевременного представления или представления недостоверных данных о нем в единый государственный реестр юридических лиц.
& До государственной регистрации юридического лица, изменений его устава или до включения иных данных, не связанных с изменениями устава, в единый государственный реестр юридических лиц уполномоченный государственный орган обязан провести в порядке и в срок, которые предусмотрены законом, проверку достоверности данных, включаемых в указанный реестр.
& В случаях и в порядке, которые предусмотрены законом о государственной регистрации юридических лиц, уполномоченный государственный орган обязан заблаговременно сообщить заинтересованным лицам о предстоящей государственной регистрации изменений устава юридического лица и о предстоящем включении данных в единый государственный реестр юридических лиц.
\par Заинтересованные лица вправе направить в уполномоченный государственный орган возражения относительно предстоящей государственной регистрации изменений устава юридического лица или предстоящего включения данных в единый государственный реестр юридических лиц в порядке, предусмотренном законом о государственной регистрации юридических лиц. Уполномоченный государственный орган обязан рассмотреть эти возражения и принять соответствующее решение в порядке и в срок, которые предусмотрены законом о государственной регистрации юридических лиц.
& Отказ в государственной регистрации юридического лица, а также во включении данных о нем в единый государственный реестр юридических лиц допускается только в случаях, предусмотренных законом о государственной регистрации юридических лиц.
\par Отказ в государственной регистрации юридического лица и уклонение от такой регистрации могут быть оспорены в суде.
& Государственная регистрация юридического лица может быть признана судом недействительной в связи с допущенными при его создании грубыми нарушениями закона, если эти нарушения носят неустранимый характер.
\par Включение в единый государственный реестр юридических лиц данных о юридическом лице может быть оспорено в суде, если такие данные недостоверны или включены в указанный реестр с нарушением закона.
& Убытки, причиненные незаконным отказом в государственной регистрации юридического лица, уклонением от государственной регистрации, включением в единый государственный реестр юридических лиц недостоверных данных о юридическом лице либо нарушением порядка государственной регистрации, предусмотренного законом о государственной регистрации юридических лиц, по вине уполномоченного государственного органа, подлежат возмещению за счет казны Российской Федерации.
& Юридическое лицо считается созданным, а данные о юридическом лице считаются включенными в единый государственный реестр юридических лиц со дня внесения соответствующей записи в этот реестр.
\eEasyList
\subsubsection{{\bf Статья 52.} Учредительные документы юридических лиц}
\beginEasyList
    & Юридические лица, за исключением хозяйственных товариществ и государственных корпораций, действуют на основании уставов, которые утверждаются их учредителями (участниками), за исключением случая, предусмотренного \ul{пунктом 2} настоящей статьи.
    \par Хозяйственное товарищество действует на основании учредительного договора, который заключается его учредителями (участниками) и к которому применяются правила настоящего Кодекса об уставе юридического лица.
    \par Государственная корпорация действует на основании федерального закона о такой государственной корпорации.
    & Юридические лица могут действовать на основании типового устава, утвержденного уполномоченным государственным органом. Сведения о том, что юридическое лицо действует на основании типового устава, утвержденного уполномоченным государственным органом, указываются в едином государственном реестре юридических лиц.
    \par Типовой устав, утвержденный уполномоченным государственным органом, не содержит сведений о наименовании, фирменном наименовании, месте нахождения и размере уставного капитала юридического лица. Такие сведения указываются в едином государственном реестре юридических лиц.
    & В случаях, предусмотренных законом, учреждение может действовать на основании единого типового устава, утвержденного его учредителем или уполномоченным им органом для учреждений, созданных для осуществления деятельности в определенных сферах.
    & Устав юридического лица, утвержденный учредителями (участниками) юридического лица, должен содержать сведения о наименовании юридического лица, его организационно-правовой форме, месте его нахождения, порядке управления деятельностью юридического лица, а также другие сведения, предусмотренные законом для юридических лиц соответствующих организационно-правовой формы и вида. В уставах некоммерческих организаций, уставах унитарных предприятий и в предусмотренных законом случаях в уставах других коммерческих организаций должны быть определены предмет и цели деятельности юридических лиц. Предмет и определенные цели деятельности коммерческой организации могут быть предусмотрены уставом также в случаях, если по закону это не является обязательным.
    & Учредители (участники) юридического лица вправе утвердить регулирующие корпоративные отношения (\ul{пункт 1 статьи 2}) и не являющиеся учредительными документами внутренний регламент и иные внутренние документы юридического лица.
    \par Во внутреннем регламенте и в иных внутренних документах юридического лица могут содержаться положения, не противоречащие учредительному документу юридического лица.
    & Изменения, внесенные в учредительные документы юридических лиц, приобретают силу для третьих лиц с момента государственной регистрации учредительных документов, а в случаях, установленных законом, с момента уведомления органа, осуществляющего государственную регистрацию, о таких изменениях. Однако юридические лица и их учредители (участники) не вправе ссылаться на отсутствие регистрации таких изменений в отношениях с третьими лицами, действовавшими с учетом таких изменений.
\eEasyList
\subsubsection{{\bf Статья 53.} Органы юридического лица}
\beginEasyList
    & Юридическое лицо приобретает гражданские права и принимает на себя гражданские обязанности через свои органы, действующие в соответствии с законом, иными правовыми актами и учредительным документом.
    \par Порядок образования и компетенция органов юридического лица определяются законом и учредительным документом.
    \par Учредительным документом может быть предусмотрено, что полномочия выступать от имени юридического лица предоставлены нескольким лицам, действующим совместно или независимо друг от друга. Сведения об этом подлежат включению в единый государственный реестр юридических лиц.
    & В предусмотренных настоящим Кодексом случаях юридическое лицо может приобретать гражданские права и принимать на себя гражданские обязанности через своих участников.
    & Лицо, которое в силу закона, иного правового акта или учредительного документа юридического лица уполномочено выступать от его имени, должно действовать в интересах представляемого им юридического лица добросовестно и разумно. Такую же обязанность несут члены коллегиальных органов юридического лица (наблюдательного или иного совета, правления и т.п.).
    & Отношения между юридическим лицом и лицами, входящими в состав его органов, регулируются настоящим Кодексом и принятыми в соответствии с ним законами о юридических лицах.
\eEasyList
\subsubsection{{\bf Статья 53.1.} Ответственность лица, уполномоченного выступать от имени юридического лица, членов коллегиальных органов юридического лица и лиц, определяющих действия юридического лица}
\beginEasyList 
    & Лицо, которое в силу закона, иного правового акта или учредительного документа юридического лица уполномочено выступать от его имени (\ul{пункт 3 статьи 53}), обязано возместить по требованию юридического лица, его учредителей (участников), выступающих в интересах юридического лица, убытки, причиненные по его вине юридическому лицу.
    \par Лицо, которое в силу закона, иного правового акта или учредительного документа юридического лица уполномочено выступать от его имени, несет ответственность, если будет доказано, что при осуществлении своих прав и исполнении своих обязанностей оно действовало недобросовестно или неразумно, в том числе если его действия (бездействие) не соответствовали обычным условиям гражданского оборота или обычному предпринимательскому риску.
    & Ответственность, предусмотренную \ul{пунктом 1} настоящей статьи, несут также члены коллегиальных органов юридического лица, за исключением тех из них, кто голосовал против решения, которое повлекло причинение юридическому лицу убытков, или, действуя добросовестно, не принимал участия в голосовании.
    & Лицо, имеющее фактическую возможность определять действия юридического лица, в том числе возможность давать указания лицам, названным в \ul{пунктах 1 и 2} настоящей статьи, обязано действовать в интересах юридического лица разумно и добросовестно и несет ответственность за убытки, причиненные по его вине юридическому лицу.
    & В случае совместного причинения убытков юридическому лицу лица, указанные в \ul{пунктах 1 - 3} настоящей статьи, обязаны возместить убытки солидарно.
    & Соглашение об устранении или ограничении ответственности лиц, указанных в \ul{пунктах 1 и 2} настоящей статьи, за совершение недобросовестных действий, а в публичном обществе за совершение недобросовестных и неразумных действий (\ul{пункт 3 статьи 53}) ничтожно.
    \par Соглашение об устранении или ограничении ответственности лица, указанного в \ul{пункте 3} настоящей статьи, ничтожно.
\eEasyList
\subsubsection{{\bf Статья 53.2.} Аффилированность}
\beginEasyList 
    \par В случаях, если настоящий Кодекс или другой закон ставит наступление правовых последствий в зависимость от наличия между лицами отношений связанности (аффилированности), наличие или отсутствие таких отношений определяется в соответствии с законом.
\eEasyList
\subsubsection{{\bf Статья 54.} Наименование, место нахождения и адрес юридического лица}
\beginEasyList
& Юридическое лицо имеет свое наименование, содержащее указание на организационно-правовую форму, а в случаях, когда законом предусмотрена возможность создания вида юридического лица, указание только на такой вид. Наименование некоммерческой организации и в предусмотренных законом случаях наименование коммерческой организации должны содержать указание на характер деятельности юридического лица.
\par Включение в наименование юридического лица официального наименования Российская Федерация или Россия, а также слов, производных от этого наименования, допускается в случаях, предусмотренных законом, указами Президента Российской Федерации или актами Правительства Российской Федерации, либо по разрешению, выданному в порядке, установленном Правительством Российской Федерации.
\par Полные или сокращенные наименования федеральных органов государственной власти не могут использоваться в наименованиях юридических лиц, за исключением случаев, предусмотренных законом, указами Президента Российской Федерации или актами Правительства Российской Федерации.
\par Нормативными правовыми актами субъектов Российской Федерации может быть установлен порядок использования в наименованиях юридических лиц официального наименования субъектов Российской Федерации.
& Место нахождения юридического лица определяется местом его государственной регистрации на территории Российской Федерации путем указания наименования населенного пункта (муниципального образования). Государственная регистрация юридического лица осуществляется по месту нахождения его постоянно действующего исполнительного органа, а в случае отсутствия постоянно действующего исполнительного органа - иного органа или лица, уполномоченных выступать от имени юридического лица в силу закона, иного правового акта или учредительного документа, если иное не установлено законом о государственной регистрации юридических лиц.
& В едином государственном реестре юридических лиц должен быть указан адрес юридического лица в пределах места нахождения юридического лица.
\par Юридическое лицо несет риск последствий неполучения юридически значимых сообщений (статья 165.1), доставленных по адресу, указанному в едином государственном реестре юридических лиц, а также риск отсутствия по указанному адресу своего органа или представителя. Сообщения, доставленные по адресу, указанному в едином государственном реестре юридических лиц, считаются полученными юридическим лицом, даже если оно не находится по указанному адресу.
\par При наличии у иностранного юридического лица представителя на территории Российской Федерации сообщения, доставленные по адресу такого представителя, считаются полученными иностранным юридическим лицом.
& Юридическое лицо, являющееся коммерческой организацией, должно иметь фирменное наименование.
\par Требования к фирменному наименованию устанавливаются настоящим Кодексом и другими законами. Права на фирменное наименование определяются в соответствии с правилами \ul{раздела VII} настоящего Кодекса.
& Наименование, фирменное наименование и место нахождения юридического лица указываются в его учредительном документе и в едином государственном реестре юридических лиц, а в случае, если юридическое лицо действует на основании типового устава, утвержденного уполномоченным государственным органом, - только в едином государственном реестре юридических лиц.
\eEasyList
\subsubsection{{\bf Статья 55.} Представительства и филиалы юридического лица}
\beginEasyList
& Представительством является обособленное подразделение юридического лица, расположенное вне места его нахождения, которое представляет интересы юридического лица и осуществляет их защиту.
& Филиалом является обособленное подразделение юридического лица, расположенное вне места его нахождения и осуществляющее все его функции или их часть, в том числе функции представительства.
& Представительства и филиалы не являются юридическими лицами. Они наделяются имуществом создавшим их юридическим лицом и действуют на основании утвержденных им положений.
\par Руководители представительств и филиалов назначаются юридическим лицом и действуют на основании его доверенности.
\par Представительства и филиалы должны быть указаны в едином государственном реестре юридических лиц.
\eEasyList
\subsubsection{{\bf Статья 56.} Ответственность юридического лица}
\beginEasyList
    & Юридическое лицо отвечает по своим обязательствам всем принадлежащим ему имуществом.
    \par Особенности ответственности казенного предприятия и учреждения по своим обязательствам определяются правилами \ul{абзаца третьего пункта 6 статьи 113}, \ul{пункта 3 статьи 123.21}, \ul{пунктов 3 - 6 статьи 123.22} и \ul{пункта 2 статьи 123.23} настоящего Кодекса. Особенности ответственности религиозной организации определяются правилами \ul{пункта 2 статьи 123.28} настоящего Кодекса.
    & Учредитель (участник) юридического лица или собственник его имущества не отвечает по обязательствам юридического лица, а юридическое лицо не отвечает по обязательствам учредителя (участника) или собственника, за исключением случаев, предусмотренных настоящим Кодексом или другим законом.
\eEasyList
\subsubsection{{\bf Статья 57.} Реорганизация юридического лица}
\beginEasyList
    & Реорганизация юридического лица (слияние, присоединение, разделение, выделение, преобразование) может быть осуществлена по решению его учредителей (участников) или органа юридического лица, уполномоченного на то учредительным документом.
    \par Допускается реорганизация юридического лица с одновременным сочетанием различных ее форм, предусмотренных абзацем первым настоящего пункта.
    \par Допускается реорганизация с участием двух и более юридических лиц, в том числе созданных в разных организационно-правовых формах, если настоящим Кодексом или другим законом предусмотрена возможность преобразования юридического лица одной из таких организационно-правовых форм в юридическое лицо другой из таких организационно-правовых форм.
    \par Ограничения реорганизации юридических лиц могут быть установлены законом.
    \par Особенности реорганизации кредитных, страховых, клиринговых организаций, специализированных финансовых обществ, специализированных обществ проектного финансирования, профессиональных участников рынка ценных бумаг, акционерных инвестиционных фондов, управляющих компаний инвестиционных фондов, паевых инвестиционных фондов и негосударственных пенсионных фондов, негосударственных пенсионных фондов и иных некредитных финансовых организаций, акционерных обществ работников (народных предприятий) определяются законами, регулирующими деятельность таких организаций.
    & В случаях, установленных законом, реорганизация юридического лица в форме его разделения или выделения из его состава одного или нескольких юридических лиц осуществляется по решению уполномоченных государственных органов или по решению суда.
    \par Если учредители (участники) юридического лица, уполномоченный ими орган или орган юридического лица, уполномоченный на реорганизацию его учредительным документом, не осуществят реорганизацию юридического лица в срок, определенный в решении уполномоченного государственного органа, суд по иску указанного государственного органа назначает в установленном законом порядке арбитражного управляющего юридическим лицом и поручает ему осуществить реорганизацию юридического лица. С момента назначения арбитражного управляющего к нему переходят полномочия по управлению делами юридического лица. Арбитражный управляющий выступает от имени юридического лица в суде, составляет передаточный акт и передает его на рассмотрение суда вместе с учредительными документами юридических лиц, создаваемых в результате реорганизации. Решение суда об утверждении указанных документов является основанием для государственной регистрации вновь создаваемых юридических лиц.
    & В случаях, установленных законом, реорганизация юридических лиц в форме слияния, присоединения или преобразования может быть осуществлена лишь с согласия уполномоченных государственных органов.
    & Юридическое лицо считается реорганизованным, за исключением случаев реорганизации в форме присоединения, с момента государственной регистрации юридических лиц, создаваемых в результате реорганизации.
    \par При реорганизации юридического лица в форме присоединения к нему другого юридического лица первое из них считается реорганизованным с момента внесения в единый государственный реестр юридических лиц записи о прекращении деятельности присоединенного юридического лица.
    \par Государственная регистрация юридического лица, создаваемого в результате реорганизации (в случае регистрации нескольких юридических лиц - первого по времени государственной регистрации), допускается не ранее истечения соответствующего срока для обжалования решения о реорганизации (\ul{пункт 1 статьи 60.1}).
\eEasyList
\subsubsection{{\bf Статья 58.} Правопреемство при реорганизации юридических лиц}
\beginEasyList
& При слиянии юридических лиц права и обязанности каждого из них переходят к вновь возникшему юридическому лицу.
& При присоединении юридического лица к другому юридическому лицу к последнему переходят права и обязанности присоединенного юридического лица.
& При разделении юридического лица его права и обязанности переходят к вновь возникшим юридическим лицам в соответствии с передаточным актом.
& При выделении из состава юридического лица одного или нескольких юридических лиц к каждому из них переходят права и обязанности реорганизованного юридического лица в соответствии с передаточным актом.
& При преобразовании юридического лица одной организационно-правовой формы в юридическое лицо другой организационно-правовой формы права и обязанности реорганизованного юридического лица в отношении других лиц не изменяются, за исключением прав и обязанностей в отношении учредителей (участников), изменение которых вызвано реорганизацией.
\par К отношениям, возникающим при реорганизации юридического лица в форме преобразования, правила \ul{статьи 60} настоящего Кодекса не применяются.
\eEasyList
\subsubsection{{\bf Статья 59.} Передаточный акт и разделительный баланс}
\beginEasyList
    & Передаточный акт должен содержать положения о правопреемстве по всем обязательствам реорганизованного юридического лица в отношении всех его кредиторов и должников, включая обязательства, оспариваемые сторонами, а также порядок определения правопреемства в связи с изменением вида, состава, стоимости имущества, возникновением, изменением, прекращением прав и обязанностей реорганизуемого юридического лица, которые могут произойти после даты, на которую составлен передаточный акт.
    & Передаточный акт утверждается учредителями (участниками) юридического лица или органом, принявшим решение о реорганизации юридического лица, и представляется вместе с учредительными документами для государственной регистрации юридических лиц, создаваемых в результате реорганизации, или внесения изменений в учредительные документы существующих юридических лиц.
    & Непредставление вместе с учредительными документами передаточного акта, отсутствие в нем положений о правопреемстве по всем обязательствам реорганизованного юридического лица влекут отказ в государственной регистрации юридических лиц, создаваемых в результате реорганизации.
\eEasyList
\subsubsection{{\bf Статья 60.} Гарантии прав кредиторов реорганизуемого юридического лица}
\beginEasyList
    & В течение трех рабочих дней после даты принятия решения о реорганизации юридического лица оно обязано уведомить в письменной форме уполномоченный государственный орган, осуществляющий государственную регистрацию юридических лиц, о начале процедуры реорганизации с указанием формы реорганизации. В случае участия в реорганизации двух и более юридических лиц такое уведомление направляется юридическим лицом, последним принявшим решение о реорганизации или определенным решением о реорганизации. На основании такого уведомления уполномоченный государственный орган, осуществляющий государственную регистрацию юридических лиц, вносит в единый государственный реестр юридических лиц запись о том, что юридические лица находятся в процессе реорганизации.
    \par Реорганизуемое юридическое лицо после внесения в единый государственный реестр юридических лиц записи о начале процедуры реорганизации дважды с периодичностью один раз в месяц опубликовывает в средствах массовой информации, в которых опубликовываются данные о государственной регистрации юридических лиц, уведомление о своей реорганизации. В случае участия в реорганизации двух и более юридических лиц уведомление о реорганизации опубликовывается от имени всех участвующих в реорганизации юридических лиц юридическим лицом, последним принявшим решение о реорганизации или определенным решением о реорганизации. В уведомлении о реорганизации указываются сведения о каждом участвующем в реорганизации, создаваемом или продолжающем деятельность в результате реорганизации юридическом лице, форма реорганизации, описание порядка и условий заявления кредиторами своих требований, иные сведения, предусмотренные законом.
    \par Законом может быть предусмотрена обязанность реорганизуемого юридического лица уведомить в письменной форме кредиторов о своей реорганизации.
    & Кредитор юридического лица, если его права требования возникли до опубликования первого уведомления о реорганизации юридического лица, вправе потребовать в судебном порядке досрочного исполнения соответствующего обязательства должником, а при невозможности досрочного исполнения - прекращения обязательства и возмещения связанных с этим убытков, за исключением случаев, установленных законом или соглашением кредитора с реорганизуемым юридическим лицом.
    \par Требования о досрочном исполнении обязательства или прекращении обязательства и возмещении убытков могут быть предъявлены кредиторами не позднее чем в течение тридцати дней после даты опубликования последнего уведомления о реорганизации юридического лица.
    \par Право, предусмотренное \ul{абзацем первым} настоящего пункта, не предоставляется кредитору, уже имеющему достаточное обеспечение.
    \par Предъявленные в указанный срок требования должны быть исполнены до завершения процедуры реорганизации, в том числе внесением долга в депозит в случаях, предусмотренных статьей 327 настоящего Кодекса.
    \par Кредитор не вправе требовать досрочного исполнения обязательства или прекращения обязательства и возмещения убытков, если в течение тридцати дней с даты предъявления кредитором этих требований ему будет предоставлено обеспечение, признаваемое достаточным в соответствии с \ul{пунктом 4} настоящей статьи.
    \par Предъявление кредиторами требований на основании настоящего пункта не является основанием для приостановления процедуры реорганизации юридического лица.
    & Если кредитору, потребовавшему в соответствии с правилами настоящей статьи досрочного исполнения обязательства или прекращения обязательства и возмещения убытков, такое исполнение не предоставлено, убытки не возмещены и не предложено достаточное обеспечение исполнения обязательства, солидарную ответственность перед кредитором наряду с юридическими лицами, созданными в результате реорганизации, несут лица, имеющие фактическую возможность определять действия реорганизованных юридических лиц (\ul{пункт 3 статьи 53.1}), члены их коллегиальных органов и лицо, уполномоченное выступать от имени реорганизованного юридического лица (\ul{пункт 3 статьи 53}), если они своими действиями (бездействием) способствовали наступлению указанных последствий для кредитора, а при реорганизации в форме выделения солидарную ответственность перед кредитором наряду с указанными лицами несет также реорганизованное юридическое лицо.
    & Предложенное кредитору обеспечение исполнения обязательств реорганизуемого юридического лица или возмещения связанных с его прекращением убытков считается достаточным, если:
    && кредитор согласился принять такое обеспечение;
    && кредитору выдана независимая безотзывная гарантия кредитной организацией, кредитоспособность которой не вызывает обоснованных сомнений, со сроком действия, не менее чем на три месяца превышающим срок исполнения обеспечиваемого обязательства, и с условием платежа по предъявлении кредитором требований к гаранту с приложением доказательств неисполнения обязательства реорганизуемого или реорганизованного юридического лица.
    && Если передаточный акт не позволяет определить правопреемника по обязательству юридического лица, а также если из передаточного акта или иных обстоятельств следует, что при реорганизации недобросовестно распределены активы и обязательства реорганизуемых юридических лиц, что привело к существенному нарушению интересов кредиторов, реорганизованное юридическое лицо и созданные в результате реорганизации юридические лица несут солидарную ответственность по такому обязательству.
\eEasyList
\subsubsection{{\bf Статья 60.1.} Последствия признания недействительным решения о реорганизации юридического лица}
\beginEasyList 
    & Решение о реорганизации юридического лица может быть признано недействительным по требованию участников реорганизуемого юридического лица, а также иных лиц, не являющихся участниками юридического лица, если такое право им предоставлено законом.
    \par Указанное требование может быть предъявлено в суд не позднее чем в течение трех месяцев после внесения в единый государственный реестр юридических лиц записи о начале процедуры реорганизации, если иной срок не установлен законом.
    & Признание судом недействительным решения о реорганизации юридического лица не влечет ликвидации образовавшегося в результате реорганизации юридического лица, а также не является основанием для признания недействительными сделок, совершенных таким юридическим лицом.
    & В случае признания решения о реорганизации юридического лица недействительным до окончания реорганизации, если осуществлена государственная регистрация части юридических лиц, подлежащих созданию в результате реорганизации, правопреемство наступает только в отношении таких зарегистрированных юридических лиц, в остальной части права и обязанности сохраняются за прежними юридическими лицами.
    & Лица, недобросовестно способствовавшие принятию признанного судом недействительным решения о реорганизации, обязаны солидарно возместить убытки участнику реорганизованного юридического лица, голосовавшему против принятия решения о реорганизации или не принимавшему участия в голосовании, а также кредиторам реорганизованного юридического лица. Солидарно с данными лицами, недобросовестно способствовавшими принятию решения о реорганизации, отвечают юридические лица, созданные в результате реорганизации на основании указанного решения.
    \par Если решение о реорганизации юридического лица принималось коллегиальным органом, солидарная ответственность возлагается на членов этого органа, голосовавших за принятие соответствующего решения.
\eEasyList
\subsubsection{{\bf Статья 60.2.} Признание реорганизации корпорации несостоявшейся}
\beginEasyList 
    & Суд по требованию участника корпорации, голосовавшего против принятия решения о реорганизации этой корпорации или не принимавшего участия в голосовании по данному вопросу, может признать реорганизацию несостоявшейся в случае, если решение о реорганизации не принималось участниками реорганизованной корпорации, а также в случае представления для государственной регистрации юридических лиц, создаваемых путем реорганизации, документов, содержащих заведомо недостоверные данные о реорганизации.
    & Решение суда о признании реорганизации несостоявшейся влечет следующие правовые последствия:
    && восстанавливаются юридические лица, существовавшие до реорганизации, с одновременным прекращением юридических лиц, созданных в результате реорганизации, о чем делаются соответствующие записи в едином государственном реестре юридических лиц;
    && сделки юридических лиц, созданных в результате реорганизации, с лицами, добросовестно полагавшимися на правопреемство, сохраняют силу для восстановленных юридических лиц, которые являются солидарными должниками и солидарными кредиторами по таким сделкам;
    && переход прав и обязанностей признается несостоявшимся, при этом предоставление (платежи, услуги и т.п.), осуществленное в пользу юридического лица, созданного в результате реорганизации, должниками, добросовестно полагавшимися на правопреемство на стороне кредитора, признается совершенным в пользу управомоченного лица. Если за счет имущества (активов) одного из юридических лиц, участвовавших в реорганизации, исполнены обязанности другого из них, перешедшие к юридическому лицу, созданному в результате реорганизации, к отношениям указанных лиц применяются правила об обязательствах вследствие неосновательного обогащения (\ul{глава 60}). Произведенные выплаты могут быть оспорены по заявлению лица, за счет средств которого они были произведены, если получатель исполнения знал или должен был знать о незаконности реорганизации;
    && участники ранее существовавшего юридического лица признаются обладателями долей участия в нем в том размере, в котором доли принадлежали им до реорганизации, а при смене участников юридического лица в ходе такой реорганизации или по ее окончании доли участия участников ранее существовавшего юридического лица возвращаются им по правилам, предусмотренным пунктом 3 статьи 65.2 настоящего Кодекса.
\eEasyList
\subsubsection{{\bf Статья 61.} Ликвидация юридического лица}
\beginEasyList
    & Ликвидация юридического лица влечет его прекращение без перехода в порядке универсального правопреемства его прав и обязанностей к другим лицам.
    & Юридическое лицо ликвидируется по решению его учредителей (участников) или органа юридического лица, уполномоченного на то учредительным документом, в том числе в связи с истечением срока, на который создано юридическое лицо, с достижением цели, ради которой оно создано.
    & Юридическое лицо ликвидируется по решению суда:
    && по иску государственного органа или органа местного самоуправления, которым право на предъявление требования о ликвидации юридического лица предоставлено законом, в случае признания государственной регистрации юридического лица недействительной, в том числе в связи с допущенными при его создании грубыми нарушениями закона, если эти нарушения носят неустранимый характер;
    && по иску государственного органа или органа местного самоуправления, которым право на предъявление требования о ликвидации юридического лица предоставлено законом, в случае осуществления юридическим лицом деятельности без надлежащего разрешения (лицензии) либо при отсутствии обязательного членства в саморегулируемой организации или необходимого в силу закона свидетельства о допуске к определенному виду работ, выданного саморегулируемой организацией;
    && по иску государственного органа или органа местного самоуправления, которым право на предъявление требования о ликвидации юридического лица предоставлено законом, в случае систематического осуществления общественной организацией, общественным движением, благотворительным и иным фондом, религиозной организацией деятельности, противоречащей уставным целям таких организаций;
    && по иску учредителя (участника) юридического лица в случае невозможности достижения целей, ради которых оно создано, в том числе в случае, если осуществление деятельности юридического лица становится невозможным или существенно затрудняется;
    && в иных случаях, предусмотренных законом.
    & С момента принятия решения о ликвидации юридического лица срок исполнения его обязательств перед кредиторами считается наступившим.
    & Решением суда о ликвидации юридического лица на его учредителей (участников) или на орган, уполномоченный на ликвидацию юридического лица его учредительным документом, могут быть возложены обязанности по осуществлению ликвидации юридического лица. Неисполнение решения суда является основанием для осуществления ликвидации юридического лица арбитражным управляющим (\ul{пункт 5 статьи 62}) за счет имущества юридического лица. При недостаточности у юридического лица средств на расходы, необходимые для его ликвидации, эти расходы возлагаются на учредителей (участников) юридического лица солидарно (\ul{пункт 2 статьи 62}).
    & Юридические лица, за исключением предусмотренных \ul{статьей 65} настоящего Кодекса юридических лиц, по решению суда могут быть признаны несостоятельными (банкротами) и ликвидированы в случаях и в порядке, которые предусмотрены законодательством о несостоятельности (банкротстве).
    \par Общие правила о ликвидации юридических лиц, содержащиеся в настоящем Кодексе, применяются к ликвидации юридического лица в порядке конкурсного производства в случаях, если настоящим Кодексом или законодательством о несостоятельности (банкротстве) не установлены иные правила.
\eEasyList
\subsubsection{{\bf Статья 62.} Обязанности лиц, принявших решение о ликвидации юридического лица}
\beginEasyList
    & Учредители (участники) юридического лица или орган, принявшие решение о ликвидации юридического лица, в течение трех рабочих дней после даты принятия данного решения обязаны сообщить в письменной форме об этом в уполномоченный государственный орган, осуществляющий государственную регистрацию юридических лиц, для внесения в единый государственный реестр юридических лиц записи о том, что юридическое лицо находится в процессе ликвидации, а также опубликовать сведения о принятии данного решения в порядке, установленном законом.
    & Учредители (участники) юридического лица независимо от оснований, по которым принято решение о его ликвидации, в том числе в случае фактического прекращения деятельности юридического лица, обязаны совершить за счет имущества юридического лица действия по ликвидации юридического лица. При недостаточности имущества юридического лица учредители (участники) юридического лица обязаны совершить указанные действия солидарно за свой счет.
    & Учредители (участники) юридического лица или орган, принявшие решение о ликвидации юридического лица, назначают ликвидационную комиссию (ликвидатора) и устанавливают порядок и сроки ликвидации в соответствии с законом.
    & С момента назначения ликвидационной комиссии к ней переходят полномочия по управлению делами юридического лица. Ликвидационная комиссия от имени ликвидируемого юридического лица выступает в суде. Ликвидационная комиссия обязана действовать добросовестно и разумно в интересах ликвидируемого юридического лица, а также его кредиторов.
    \par Если ликвидационной комиссией установлена недостаточность имущества юридического лица для удовлетворения всех требований кредиторов, дальнейшая ликвидация юридического лица может осуществляться только в порядке, установленном законодательством о несостоятельности (банкротстве).
    & В случае неисполнения или ненадлежащего исполнения учредителями (участниками) юридического лица обязанностей по его ликвидации заинтересованное лицо или уполномоченный государственный орган вправе потребовать в судебном порядке ликвидации юридического лица и назначения для этого арбитражного управляющего.
    & При невозможности ликвидации юридического лица ввиду отсутствия средств на расходы, необходимые для его ликвидации, и невозможности возложить эти расходы на его учредителей (участников) юридическое лицо подлежит исключению из единого государственного реестра юридических лиц в порядке, установленном законом о государственной регистрации юридических лиц.
\eEasyList
\subsubsection{{\bf Статья 63.} Порядок ликвидации юридического лица}
\beginEasyList
& Ликвидационная комиссия опубликовывает в средствах массовой информации, в которых опубликовываются данные о государственной регистрации юридического лица, сообщение о его ликвидации и о порядке и сроке заявления требований его кредиторами. Этот срок не может быть менее двух месяцев с момента опубликования сообщения о ликвидации.
\par Ликвидационная комиссия принимает меры по выявлению кредиторов и получению дебиторской задолженности, а также уведомляет в письменной форме кредиторов о ликвидации юридического лица.
& После окончания срока предъявления требований кредиторами ликвидационная комиссия составляет промежуточный ликвидационный баланс, который содержит сведения о составе имущества ликвидируемого юридического лица, перечне требований, предъявленных кредиторами, результатах их рассмотрения, а также о перечне требований, удовлетворенных вступившим в законную силу решением суда, независимо от того, были ли такие требования приняты ликвидационной комиссией.
\par Промежуточный ликвидационный баланс утверждается учредителями (участниками) юридического лица или органом, принявшими решение о ликвидации юридического лица. В случаях, установленных законом, промежуточный ликвидационный баланс утверждается по согласованию с уполномоченным государственным органом.
& В случае возбуждения дела о несостоятельности (банкротстве) юридического лица его ликвидация, осуществляемая по правилам настоящего Кодекса, прекращается и ликвидационная комиссия уведомляет об этом всех известных ей кредиторов. Требования кредиторов в случае прекращения ликвидации юридического лица при возбуждении дела о его несостоятельности (банкротстве) рассматриваются в порядке, установленном законодательством о несостоятельности (банкротстве).
& Если имеющиеся у ликвидируемого юридического лица (кроме учреждений) денежные средства недостаточны для удовлетворения требований кредиторов, ликвидационная комиссия осуществляет продажу имущества юридического лица, на которое в соответствии с законом допускается обращение взыскания, с торгов, за исключением объектов стоимостью не более ста тысяч рублей (согласно утвержденному промежуточному ликвидационному балансу), для продажи которых проведение торгов не требуется.
& В случае недостаточности имущества ликвидируемого юридического лица для удовлетворения требований кредиторов или при наличии признаков банкротства юридического лица ликвидационная комиссия обязана обратиться в арбитражный суд с заявлением о банкротстве юридического лица, если такое юридическое лицо может быть признано несостоятельным (банкротом).
& Выплата денежных сумм кредиторам ликвидируемого юридического лица производится ликвидационной комиссией в порядке очередности, установленной статьей 64 настоящего Кодекса, в соответствии с промежуточным ликвидационным балансом со дня его утверждения.
& После завершения расчетов с кредиторами ликвидационная комиссия составляет ликвидационный баланс, который утверждается учредителями (участниками) юридического лица или органом, принявшими решение о ликвидации юридического лица. В случаях, установленных законом, ликвидационный баланс утверждается по согласованию с уполномоченным государственным органом.
& В случаях, если настоящим Кодексом предусмотрена субсидиарная ответственность собственника имущества учреждения или казенного предприятия по обязательствам этого учреждения или этого предприятия, при недостаточности у ликвидируемых учреждения или казенного предприятия имущества, на которое в соответствии с законом может быть обращено взыскание, кредиторы вправе обратиться в суд с иском об удовлетворении оставшейся части требований за счет собственника имущества этого учреждения или этого предприятия.
& Оставшееся после удовлетворения требований кредиторов имущество юридического лица передается его учредителям (участникам), имеющим вещные права на это имущество или корпоративные права в отношении юридического лица, если иное не предусмотрено законом, иными правовыми актами или учредительным документом юридического лица. При наличии спора между учредителями (участниками) относительно того, кому следует передать вещь, она продается ликвидационной комиссией с торгов. Если иное не установлено настоящим Кодексом или другим законом, при ликвидации некоммерческой организации оставшееся после удовлетворения требований кредиторов имущество направляется в соответствии с уставом некоммерческой организации на цели, для достижения которых она была создана, и (или) на благотворительные цели.
& Ликвидация юридического лица считается завершенной, а юридическое лицо - прекратившим существование после внесения сведений о его прекращении в единый государственный реестр юридических лиц в порядке, установленном законом о государственной регистрации юридических лиц.
\eEasyList
\subsubsection{{\bf Статья 64.} Удовлетворение требований кредиторов ликвидируемого юридического лица}
\beginEasyList
& При ликвидации юридического лица после погашения текущих расходов, необходимых для осуществления ликвидации, требования его кредиторов удовлетворяются в следующей очередности:
\par в первую очередь удовлетворяются требования граждан, перед которыми ликвидируемое юридическое лицо несет ответственность за причинение вреда жизни или здоровью, путем капитализации соответствующих повременных платежей, о компенсации сверх возмещения вреда, причиненного вследствие разрушения, повреждения объекта капитального строительства, нарушения требований безопасности при строительстве объекта капитального строительства, требований к обеспечению безопасной эксплуатации здания, сооружения;
\par во вторую очередь производятся расчеты по выплате выходных пособий и оплате труда лиц, работающих или работавших по трудовому договору, и по выплате вознаграждений авторам результатов интеллектуальной деятельности;
\par в третью очередь производятся расчеты по обязательным платежам в бюджет и во внебюджетные фонды;
\par в четвертую очередь производятся расчеты с другими кредиторами;
\par абзац шестой утратил силу.
\par При ликвидации банков, привлекающих средства граждан, в первую очередь удовлетворяются также требования граждан, являющихся кредиторами банков по заключенным с ними или в их пользу договорам банковского вклада или банковского счета, за исключением договоров, связанных с осуществлением гражданином предпринимательской или иной профессиональной деятельности, в части основной суммы задолженности и причитающихся процентов, требования организации, осуществляющей обязательное страхование вкладов, в связи с выплатой возмещения по вкладам в соответствии с законом о страховании вкладов граждан в банках и требования Банка России в связи с осуществлением выплат по вкладам граждан в банках в соответствии с законом.
\par Требования кредиторов о возмещении убытков в виде упущенной выгоды, о взыскании неустойки (штрафа, пени), в том числе за неисполнение или ненадлежащее исполнение обязанности по уплате обязательных платежей, удовлетворяются после удовлетворения требований кредиторов первой, второй, третьей и четвертой очереди.
& Требования кредиторов каждой очереди удовлетворяются после полного удовлетворения требований кредиторов предыдущей очереди, за исключением требований кредиторов по обязательствам, обеспеченным залогом имущества ликвидируемого юридического лица.
\par Требования кредиторов по обязательствам, обеспеченным залогом имущества ликвидируемого юридического лица, удовлетворяются за счет средств, полученных от продажи предмета залога, преимущественно перед иными кредиторами, за исключением обязательств перед кредиторами первой и второй очереди, права требования по которым возникли до заключения соответствующего договора залога.
\par Не удовлетворенные за счет средств, полученных от продажи предмета залога, требования кредиторов по обязательствам, обеспеченным залогом имущества ликвидируемого юридического лица, удовлетворяются в составе требований кредиторов четвертой очереди.
& При недостаточности имущества ликвидируемого юридического лица, когда такое юридическое лицо в случаях, предусмотренных настоящим Кодексом, не может быть признано несостоятельным (банкротом), имущество такого юридического лица распределяется между кредиторами соответствующей очереди пропорционально размеру требований, подлежащих удовлетворению, если иное не установлено законом.
& Утратил силу с 3 июня 2018 г. - Федеральный закон от 23 мая 2018 г. N 116-ФЗ
& Утратил силу с 1 сентября 2014 г.
&& Считаются погашенными при ликвидации юридического лица:
&&& требования кредиторов, не удовлетворенные по причине недостаточности имущества ликвидируемого юридического лица и не удовлетворенные за счет имущества лиц, несущих субсидиарную ответственность по таким требованиям, если ликвидируемое юридическое лицо в случаях, предусмотренных \ul{статьей 65} настоящего Кодекса, не может быть признано несостоятельным (банкротом);
&&& требования, не признанные ликвидационной комиссией, если кредиторы по таким требованиям не обращались с исками в суд;
&&& требования, в удовлетворении которых решением суда кредиторам отказано.
&& В случае обнаружения имущества ликвидированного юридического лица, исключенного из единого государственного реестра юридических лиц, в том числе в результате признания такого юридического лица несостоятельным (банкротом), заинтересованное лицо или уполномоченный государственный орган вправе обратиться в суд с заявлением о назначении процедуры распределения обнаруженного имущества среди лиц, имеющих на это право. К указанному имуществу относятся также требования ликвидированного юридического лица к третьим лицам, в том числе возникшие из-за нарушения очередности удовлетворения требований кредиторов, вследствие которого заинтересованное лицо не получило исполнение в полном объеме. В этом случае суд назначает арбитражного управляющего, на которого возлагается обязанность распределения обнаруженного имущества ликвидированного юридического лица.
\par Заявление о назначении процедуры распределения обнаруженного имущества ликвидированного юридического лица может быть подано в течение пяти лет с момента внесения в единый государственный реестр юридических лиц сведений о прекращении юридического лица. Процедура распределения обнаруженного имущества ликвидированного юридического лица может быть назначена при наличии средств, достаточных для осуществления данной процедуры, и возможности распределения обнаруженного имущества среди заинтересованных лиц.
\par Процедура распределения обнаруженного имущества ликвидированного юридического лица осуществляется по правилам настоящего Кодекса о ликвидации юридических лиц.
& Утратил силу с 1 сентября 2014 г.
\eEasyList
\subsubsection{{\bf Статья 64.1.} Защита прав кредиторов ликвидируемого юридического лица}
\beginEasyList 
    & В случае отказа ликвидационной комиссии удовлетворить требование кредитора или уклонения от его рассмотрения кредитор до утверждения ликвидационного баланса юридического лица вправе обратиться в суд с иском об удовлетворении его требования к ликвидируемому юридическому лицу. В случае удовлетворения судом иска кредитора выплата присужденной ему денежной суммы производится в порядке очередности, установленной \ul{статьей 64} настоящего Кодекса.
    & Члены ликвидационной комиссии (ликвидатор) по требованию учредителей (участников) ликвидированного юридического лица или по требованию его кредиторов обязаны возместить убытки, причиненные ими учредителям (участникам) ликвидированного юридического лица или его кредиторам, в порядке и по основаниям, которые предусмотрены \ul{статьей 53.1} настоящего Кодекса.
\eEasyList 
\subsubsection{{\bf Статья 64.2.} Прекращение недействующего юридического лица}
\beginEasyList 
    & Считается фактически прекратившим свою деятельность и подлежит исключению из единого государственного реестра юридических лиц в порядке, установленном законом о государственной регистрации юридических лиц, юридическое лицо, которое в течение двенадцати месяцев, предшествующих его исключению из указанного реестра, не представляло документы отчетности, предусмотренные законодательством Российской Федерации о налогах и сборах, и не осуществляло операций хотя бы по одному банковскому счету (недействующее юридическое лицо).
    & Исключение недействующего юридического лица из единого государственного реестра юридических лиц влечет правовые последствия, предусмотренные настоящим Кодексом и другими законами применительно к ликвидированным юридическим лицам.
    & Исключение недействующего юридического лица из единого государственного реестра юридических лиц не препятствует привлечению к ответственности лиц, указанных в \ul{статье 53.1} настоящего Кодекса.
\eEasyList
\subsubsection{{\bf Статья 65.} Несостоятельность (банкротство) юридического лица}
\beginEasyList
& Юридическое лицо, за исключением казенного предприятия, учреждения, политической партии и религиозной организации, по решению суда может быть признано несостоятельным (банкротом). Государственная корпорация или государственная компания может быть признана несостоятельной (банкротом), если это допускается федеральным законом, предусматривающим ее создание. Фонд не может быть признан несостоятельным (банкротом), если это установлено законом, предусматривающим создание и деятельность такого фонда. Публично-правовая компания не может быть признана несостоятельной (банкротом).
\par Признание юридического лица банкротом судом влечет его ликвидацию.
& Утратил силу.
& Основания признания судом юридического лица несостоятельным (банкротом), порядок ликвидации такого юридического лица, а также очередность удовлетворения требований кредиторов устанавливается законом о несостоятельности (банкротстве).
\eEasyList
\subsubsection{{\bf Статья 65.1.} Корпоративные и унитарные юридические лица}
\beginEasyList 
    & Юридические лица, учредители (участники) которых обладают правом участия (членства) в них и формируют их высший орган в соответствии с \ul{пунктом 1 статьи 65.3} настоящего Кодекса, являются корпоративными юридическими лицами (корпорациями). К ним относятся хозяйственные товарищества и общества, крестьянские (фермерские) хозяйства, хозяйственные партнерства, производственные и потребительские кооперативы, общественные организации, общественные движения, ассоциации (союзы), нотариальные палаты, товарищества собственников недвижимости, казачьи общества, внесенные в государственный реестр казачьих обществ в Российской Федерации, а также общины коренных малочисленных народов Российской Федерации.
    \par Юридические лица, учредители которых не становятся их участниками и не приобретают в них прав членства, являются унитарными юридическими лицами. К ним относятся государственные и муниципальные унитарные предприятия, фонды, учреждения, автономные некоммерческие организации, религиозные организации, государственные корпорации, публично-правовые компании.
    & В связи с участием в корпоративной организации ее участники приобретают корпоративные (членские) права и обязанности в отношении созданного ими юридического лица, за исключением случаев, предусмотренных настоящим Кодексом.
\eEasyList
\subsubsection{{\bf Статья 65.2.} Права и обязанности участников корпорации}
\beginEasyList 
    & Участники корпорации (участники, члены, акционеры и т.п.) вправе:
    \par участвовать в управлении делами корпорации, за исключением случая, предусмотренного \ul{пунктом 2 статьи 84} настоящего Кодекса;
    \par в случаях и в порядке, которые предусмотрены законом и учредительным документом корпорации, получать информацию о деятельности корпорации и знакомиться с ее бухгалтерской и иной документацией;
    \par обжаловать решения органов корпорации, влекущие гражданско-правовые последствия, в случаях и в порядке, которые предусмотрены законом;
    \par требовать, действуя от имени корпорации (\ul{пункт 1 статьи 182}), возмещения причиненных корпорации убытков (\ul{статья 53.1});
    \par оспаривать, действуя от имени корпорации (\ul{пункт 1 статьи 182}), совершенные ею сделки по основаниям, предусмотренным \ul{статьей 174} настоящего Кодекса или законами о корпорациях отдельных организационно-правовых форм, и требовать применения последствий их недействительности, а также применения последствий недействительности ничтожных сделок корпорации.
    \par Участники корпорации могут иметь и другие права, предусмотренные законом или учредительным документом корпорации.
    & Участник корпорации или корпорация, требующие возмещения причиненных корпорации убытков (\ul{статья 53.1}) либо признания сделки корпорации недействительной или применения последствий недействительности сделки, должны принять разумные меры по заблаговременному уведомлению других участников корпорации и в соответствующих случаях корпорации о намерении обратиться с такими требованиями в суд, а также предоставить им иную информацию, имеющую отношение к делу. Порядок уведомления о намерении обратиться в суд с иском может быть предусмотрен законами о корпорациях и учредительным документом корпорации.
    \par Участники корпорации, не присоединившиеся в порядке, установленном процессуальным законодательством, к иску о возмещении причиненных корпорации убытков (\ul{статья 53.1}) либо к иску о признании недействительной совершенной корпорацией сделки или о применении последствий недействительности сделки, в последующем не вправе обращаться в суд с тождественными требованиями, если только суд не признает причины этого обращения уважительными.
    & Если иное не установлено настоящим Кодексом, участник коммерческой корпорации, утративший помимо своей воли в результате неправомерных действий других участников или третьих лиц права участия в ней, вправе требовать возвращения ему доли участия, перешедшей к иным лицам, с выплатой им справедливой компенсации, определяемой судом, а также возмещения убытков за счет лиц, виновных в утрате доли. Суд может отказать в возвращении доли участия, если это приведет к несправедливому лишению иных лиц их прав участия или повлечет крайне негативные социальные и другие публично значимые последствия. В этом случае лицу, утратившему помимо своей воли права участия в корпорации, лицами, виновными в утрате доли участия, выплачивается справедливая компенсация, определяемая судом.
    & Участник корпорации обязан:
    \par участвовать в образовании имущества корпорации в необходимом размере в порядке, способом и в сроки, которые предусмотрены настоящим Кодексом, другим законом или учредительным документом корпорации;
    \par не разглашать конфиденциальную информацию о деятельности корпорации;
    \par участвовать в принятии корпоративных решений, без которых корпорация не может продолжать свою деятельность в соответствии с законом, если его участие необходимо для принятия таких решений;
    \par не совершать действия, заведомо направленные на причинение вреда корпорации;
    \par не совершать действия (бездействие), которые существенно затрудняют или делают невозможным достижение целей, ради которых создана корпорация.
    \par Участники корпорации могут нести и другие обязанности, предусмотренные законом или учредительным документом корпорации.
\eEasyList
\subsubsection{{\bf Статья 65.3.} Управление в корпорации}
\beginEasyList 
    & Высшим органом корпорации является общее собрание ее участников.
    \par В некоммерческих корпорациях и производственных кооперативах с числом участников более ста высшим органом может являться съезд, конференция или иной представительный (коллегиальный) орган, определяемый их уставами в соответствии с законом. Компетенция этого органа и порядок принятия им решений определяются настоящим Кодексом, другими законами и уставом корпорации.
    & Если иное не предусмотрено настоящим Кодексом или другим законом, к исключительной компетенции высшего органа корпорации относятся:
    \par определение приоритетных направлений деятельности корпорации, принципов образования и использования ее имущества;
    \par утверждение и изменение устава корпорации;
    \par определение порядка приема в состав участников корпорации и исключения из числа ее участников, кроме случаев, если такой порядок определен законом;
    \par образование других органов корпорации и досрочное прекращение их полномочий, если уставом корпорации в соответствии с законом это правомочие не отнесено к компетенции иных коллегиальных органов корпорации;
    \par утверждение годовых отчетов и бухгалтерской (финансовой) отчетности корпорации, если уставом корпорации в соответствии с законом это правомочие не отнесено к компетенции иных коллегиальных органов корпорации;
    \par принятие решений о создании корпорацией других юридических лиц, об участии корпорации в других юридических лицах, о создании филиалов и об открытии представительств корпорации, за исключением случаев, если уставом хозяйственного общества в соответствии с законами о хозяйственных обществах принятие таких решений по указанным вопросам отнесено к компетенции иных коллегиальных органов корпорации;
    \par принятие решений о реорганизации и ликвидации корпорации, о назначении ликвидационной комиссии (ликвидатора) и об утверждении ликвидационного баланса;
    \par избрание ревизионной комиссии (ревизора) и назначение аудиторской организации или индивидуального аудитора корпорации.
    \par Законом и учредительным документом корпорации к исключительной компетенции ее высшего органа может быть отнесено решение иных вопросов.
    \par Вопросы, отнесенные настоящим Кодексом и другими законами к исключительной компетенции высшего органа корпорации, не могут быть переданы им для решения другим органам корпорации, если иное не предусмотрено настоящим Кодексом или другим законом.
    & В корпорации образуется единоличный исполнительный орган (директор, генеральный директор, председатель и т.п.). Уставом корпорации может быть предусмотрено предоставление полномочий единоличного исполнительного органа нескольким лицам, действующим совместно, или образование нескольких единоличных исполнительных органов, действующих независимо друг от друга (\ul{абзац третий пункта 1 статьи 53}). В качестве единоличного исполнительного органа корпорации может выступать как физическое лицо, так и юридическое лицо.
    \par В случаях, предусмотренных настоящим Кодексом, другим законом или уставом корпорации, в корпорации образуется коллегиальный исполнительный орган (правление, дирекция и т.п.).
    \par К компетенции указанных в настоящем пункте органов корпорации относится решение вопросов, не входящих в компетенцию ее высшего органа и созданного в соответствии с \ul{пунктом 4} настоящей статьи коллегиального органа управления.
    & Наряду с исполнительными органами, указанными в \ul{пункте 3} настоящей статьи, в корпорации может быть образован в случаях, предусмотренных настоящим Кодексом, другим законом или уставом корпорации, коллегиальный орган управления (наблюдательный или иной совет), контролирующий деятельность исполнительных органов корпорации и выполняющий иные функции, возложенные на него законом или уставом корпорации. Лица, осуществляющие полномочия единоличных исполнительных органов корпораций, и члены их коллегиальных исполнительных органов не могут составлять более одной четверти состава коллегиальных органов управления корпораций и не могут являться их председателями.
    \par Члены коллегиального органа управления корпорации имеют право получать информацию о деятельности корпорации и знакомиться с ее бухгалтерской и иной документацией, требовать возмещения причиненных корпорации убытков (\ul{статья 53.1}), оспаривать совершенные корпорацией сделки по основаниям, предусмотренным \ul{статьей 174} настоящего Кодекса или законами о корпорациях отдельных организационно-правовых форм, и требовать применения последствий их недействительности, а также требовать применения последствий недействительности ничтожных сделок корпорации в порядке, установленном \ul{пунктом 2 статьи 65.2} настоящего Кодекса.
\eEasyList
\subsection{{\bf § 2. Коммерческие корпоративные организации}}
\subsection{{\bf 1. Общие положения о хозяйственных товариществах и обществах}}
\subsubsection{{\bf Статья 66.} Основные положения о хозяйственных товариществах и обществах}
\beginEasyList
    & Хозяйственными товариществами и обществами признаются корпоративные коммерческие организации с разделенным на доли (вклады) учредителей (участников) уставным (складочным) капиталом. Имущество, созданное за счет вкладов учредителей (участников), а также произведенное и приобретенное хозяйственным товариществом или обществом в процессе деятельности, принадлежит на праве собственности хозяйственному товариществу или обществу.
    \par Объем правомочий участников хозяйственного общества определяется пропорционально их долям в уставном капитале общества. Иной объем правомочий участников непубличного хозяйственного общества может быть предусмотрен уставом общества, а также корпоративным договором при условии внесения сведений о наличии такого договора и о предусмотренном им объеме правомочий участников общества в единый государственный реестр юридических лиц.
    & В случаях, предусмотренных настоящим Кодексом, хозяйственное общество может быть создано одним лицом, которое становится его единственным участником.
    \par Хозяйственное общество не может иметь в качестве единственного участника другое хозяйственное общество, состоящее из одного лица, если иное не установлено настоящим Кодексом или другим законом.
    & Хозяйственные товарищества могут создаваться в организационно-правовой форме полного товарищества или товарищества на вере (коммандитного товарищества).
    & Хозяйственные общества могут создаваться в организационно-правовой форме акционерного общества или общества с ограниченной ответственностью.
    & Участниками полных товариществ и полными товарищами в товариществах на вере могут быть индивидуальные предприниматели и коммерческие организации.
    \par Участниками хозяйственных обществ и вкладчиками в товариществах на вере могут быть граждане и юридические лица, а также публично-правовые образования (\ul{статья 125}).
    & Государственные органы и органы местного самоуправления не вправе участвовать от своего имени в хозяйственных товариществах и обществах.
    \par Учреждения могут быть участниками хозяйственных обществ и вкладчиками в товариществах на вере с разрешения собственника имущества учреждения, если иное не установлено законом.
    \par Законом может быть запрещено или ограничено участие отдельных категорий лиц в хозяйственных товариществах и обществах.
    \par Хозяйственные товарищества и общества могут быть учредителями (участниками) других хозяйственных товариществ и обществ, за исключением случаев, предусмотренных законом.
    & Особенности правового положения кредитных организаций, страховых организаций, клиринговых организаций, специализированных финансовых обществ, специализированных обществ проектного финансирования, профессиональных участников рынка ценных бумаг, акционерных инвестиционных фондов, управляющих компаний инвестиционных фондов, паевых инвестиционных фондов и негосударственных пенсионных фондов, негосударственных пенсионных фондов и иных некредитных финансовых организаций, акционерных обществ работников (народных предприятий), а также права и обязанности их участников определяются законами, регулирующими деятельность таких организаций.
\eEasyList
\subsubsection{{\bf Статья 66.1.} Вклады в имущество хозяйственного товарищества или общества}
\beginEasyList 
    & Вкладом участника хозяйственного товарищества или общества в его имущество могут быть денежные средства, вещи, доли (акции) в уставных (складочных) капиталах других хозяйственных товариществ и обществ, государственные и муниципальные облигации. Таким вкладом также могут быть подлежащие денежной оценке исключительные, иные интеллектуальные права и права по лицензионным договорам, если иное не установлено законом.
    & Законом или учредительными документами хозяйственного товарищества или общества могут быть установлены виды указанного \ul{в пункте 1} настоящей статьи имущества, которое не может быть внесено для оплаты долей в уставном (складочном) капитале хозяйственного товарищества или общества.
\eEasyList 
\subsubsection{{\bf Статья 66.2.} Основные положения об уставном капитале хозяйственного общества}
\beginEasyList 
    & Минимальный размер уставных капиталов хозяйственных обществ определяется законами о хозяйственных обществах.
    \par Минимальные размеры уставных капиталов хозяйственных обществ, осуществляющих банковскую, страховую или иную подлежащую лицензированию деятельность, а также акционерных обществ, использующих открытую (публичную) подписку на свои акции, устанавливаются законами, определяющими особенности правового положения указанных хозяйственных обществ.
    & При оплате уставного капитала хозяйственного общества должны быть внесены денежные средства в сумме не ниже минимального размера уставного капитала (\ul{пункт 1} настоящей статьи).
    \par Денежная оценка неденежного вклада в уставный капитал хозяйственного общества должна быть проведена независимым оценщиком. Участники хозяйственного общества не вправе определять денежную оценку неденежного вклада в размере, превышающем сумму оценки, определенную независимым оценщиком.
    & При оплате долей в уставном капитале общества с ограниченной ответственностью не денежными средствами, а иным имуществом участники общества и независимый оценщик в случае недостаточности имущества общества солидарно несут субсидиарную ответственность по его обязательствам в пределах суммы, на которую завышена оценка имущества, внесенного в уставный капитал, в течение пяти лет с момента государственной регистрации общества или внесения в устав общества соответствующих изменений. При внесении в уставный капитал акционерного общества не денежных средств, а иного имущества акционер, осуществивший такую оплату, и независимый оценщик в случае недостаточности имущества общества солидарно несут субсидиарную ответственность по его обязательствам в пределах суммы, на которую завышена оценка имущества, внесенного в уставный капитал, в течение пяти лет с момента государственной регистрации общества или внесения в устав общества соответствующих изменений.
    \par Правила настоящего пункта об ответственности участника общества и независимого оценщика не применяются к хозяйственным обществам, созданным в соответствии с законами о приватизации путем приватизации государственных или муниципальных унитарных предприятий.
    & Если иное не предусмотрено законами о хозяйственных обществах, учредители хозяйственного общества обязаны оплатить не менее трех четвертей его уставного капитала до государственной регистрации общества, а остальную часть уставного капитала хозяйственного общества - в течение первого года деятельности общества.
    \par В случаях, если в соответствии с законом допускается государственная регистрация хозяйственного общества без предварительной оплаты трех четвертей уставного капитала, участники общества несут субсидиарную ответственность по его обязательствам, возникшим до момента полной оплаты уставного капитала.
\eEasyList
\subsubsection{{\bf Статья 66.3.} Публичные и непубличные общества}
\beginEasyList 
    & Публичным является акционерное общество, акции которого и ценные бумаги которого, конвертируемые в его акции, публично размещаются (путем открытой подписки) или публично обращаются на условиях, установленных законами о ценных бумагах. Правила о публичных обществах применяются также к акционерным обществам, устав и фирменное наименование которых содержат указание на то, что общество является публичным.
    & Общество с ограниченной ответственностью и акционерное общество, которое не отвечает признакам, указанным в \ul{пункте 1} настоящей статьи, признаются непубличными.
    & По решению участников (учредителей) непубличного общества, принятому единогласно, в устав общества могут быть включены следующие положения:
    && о передаче на рассмотрение коллегиального органа управления общества (\ul{пункт 4 статьи 65.3}) или коллегиального исполнительного органа общества вопросов, отнесенных законом к компетенции общего собрания участников хозяйственного общества, за исключением вопросов:
    \par внесения изменений в устав хозяйственного общества, утверждения устава в новой редакции;
    \par реорганизации или ликвидации хозяйственного общества;
    \par определения количественного состава коллегиального органа управления общества (\ul{пункт 4 статьи 65.3}) и коллегиального исполнительного органа (если его формирование отнесено к компетенции общего собрания участников хозяйственного общества), избрания их членов и досрочного прекращения их полномочий;
    \par определения количества, номинальной стоимости, категории (типа) объявленных акций и прав, предоставляемых этими акциями;
    \par увеличения уставного капитала общества с ограниченной ответственностью непропорционально долям его участников или за счет принятия третьего лица в состав участников такого общества;
    \par утверждения не являющихся учредительными документами внутреннего регламента или иных внутренних документов (\ul{пункт 5 статьи 52}) хозяйственного общества;
    && о закреплении функций коллегиального исполнительного органа общества за коллегиальным органом управления общества (пункт 4 статьи 65.3) полностью или в части либо об отказе от создания коллегиального исполнительного органа, если его функции осуществляются указанным коллегиальным органом управления;
    && о передаче единоличному исполнительному органу общества функций коллегиального исполнительного органа общества;
    && об отсутствии в обществе ревизионной комиссии или о ее создании исключительно в случаях, предусмотренных уставом общества;
    && о порядке, отличном от установленного законами и иными правовыми актами порядка созыва, подготовки и проведения общих собраний участников хозяйственного общества, принятия ими решений, при условии, что такие изменения не лишают его участников права на участие в общем собрании непубличного общества и на получение информации о нем;
    && о требованиях, отличных от установленных законами и иными правовыми актами требований к количественному составу, порядку формирования и проведения заседаний коллегиального органа управления общества (\ul{пункт 4 статьи 65.3}) или коллегиального исполнительного органа общества;
    && о порядке осуществления преимущественного права покупки доли или части доли в уставном капитале общества с ограниченной ответственностью или преимущественного права приобретения размещаемых акционерным обществом акций либо ценных бумаг, конвертируемых в его акции, а также о максимальной доле участия одного участника общества с ограниченной ответственностью в уставном капитале общества;
    && об отнесении к компетенции общего собрания акционеров вопросов, не относящихся к ней в соответствии с настоящим Кодексом или законом об акционерных обществах;
    && иные положения в случаях, предусмотренных законами о хозяйственных обществах.
    & В случаях, если положения, предусмотренные \ul{пунктом 3} настоящей статьи, не относятся к числу положений, подлежащих в соответствии с настоящим Кодексом или другими законами обязательному включению в устав непубличного хозяйственного общества, они могут быть предусмотрены корпоративным договором, сторонами которого являются все участники этого общества.
\eEasyList
\subsubsection{{\bf Статья 67.} Права и обязанности участника хозяйственного товарищества и общества}
\beginEasyList
    & Участник хозяйственного товарищества или общества наряду с правами, предусмотренными для участников корпораций пунктом 1 \ul{статьи 65.2} настоящего Кодекса, также вправе:
    \par принимать участие в распределении прибыли товарищества или общества, участником которого он является;
    \par получать в случае ликвидации товарищества или общества часть имущества, оставшегося после расчетов с кредиторами, или его стоимость;
    \par требовать исключения другого участника из товарищества или общества (кроме публичных акционерных обществ) в судебном порядке с выплатой ему действительной стоимости его доли участия, если такой участник своими действиями (бездействием) причинил существенный вред товариществу или обществу либо иным образом существенно затрудняет его деятельность и достижение целей, ради которых оно создавалось, в том числе грубо нарушая свои обязанности, предусмотренные законом или учредительными документами товарищества или общества. Отказ от этого права или его ограничение ничтожны.
    \par Участники хозяйственных товариществ или обществ могут иметь и другие права, предусмотренные настоящим Кодексом, законами о хозяйственных обществах, учредительными документами товарищества или общества.
    & Участник хозяйственного товарищества или общества наряду с обязанностями, предусмотренными для участников корпораций \ul{пунктом 4 статьи 65.2} настоящего Кодекса, также обязан вносить вклады в уставный (складочный) капитал товарищества или общества, участником которого он является, в порядке, в размерах, способами, которые предусмотрены учредительным документом хозяйственного товарищества или общества, и вклады в иное имущество хозяйственного товарищества или общества.
    \par Участники хозяйственных товариществ и обществ могут нести и другие обязанности, предусмотренные законом и их учредительными документами.
\eEasyList
\subsubsection{{\bf Статья 67.1.} Особенности управления и контроля в хозяйственных товариществах и обществах}
\beginEasyList
    & Управление в полном товариществе и товариществе на вере осуществляется в порядке, установленном \ul{статьями 71 и 84} настоящего Кодекса.
    & К исключительной компетенции общего собрания участников хозяйственного общества наряду с вопросами, указанными в \ul{пункте 2 статьи 65.3} настоящего Кодекса, относятся:
    && изменение размера уставного капитала общества, если иное не предусмотрено законами о хозяйственных обществах;
    && принятие решения о передаче полномочий единоличного исполнительного органа общества другому хозяйственному обществу (управляющей организации) или индивидуальному предпринимателю (управляющему), а также утверждение такой управляющей организации или такого управляющего и условий договора с такой управляющей организацией или с таким управляющим, если уставом общества решение указанных вопросов не отнесено к компетенции коллегиального органа управления общества (\ul{пункт 4 статьи 65.3});
    && распределение прибылей и убытков общества.
    & Принятие общим собранием участников хозяйственного общества решения посредством очного голосования и состав участников общества, присутствовавших при его принятии, подтверждаются в отношении:
    && публичного акционерного общества лицом, осуществляющим ведение реестра акционеров такого общества и выполняющим функции счетной комиссии (\ul{пункт 4 статьи 97});
    && непубличного акционерного общества путем нотариального удостоверения или удостоверения лицом, осуществляющим ведение реестра акционеров такого общества и выполняющим функции счетной комиссии;
    && общества с ограниченной ответственностью путем нотариального удостоверения, если иной способ (подписание протокола всеми участниками или частью участников; с использованием технических средств, позволяющих достоверно установить факт принятия решения; иным способом, не противоречащим закону) не предусмотрен уставом такого общества либо решением общего собрания участников общества, принятым участниками общества единогласно.
    & Общество с ограниченной ответственностью для проверки и подтверждения правильности годовой бухгалтерской (финансовой) отчетности вправе, а в случаях, предусмотренных законом, обязано ежегодно привлекать аудитора, не связанного имущественными интересами с обществом или его участниками (внешний аудит). Такой аудит также может быть проведен по требованию любого из участников общества.
    & Акционерное общество для проверки и подтверждения правильности годовой бухгалтерской (финансовой) отчетности должно ежегодно привлекать аудитора, не связанного имущественными интересами с обществом или его участниками.
    \par В случаях и в порядке, которые предусмотрены законом, уставом общества, аудит бухгалтерской (финансовой) отчетности акционерного общества должен быть проведен по требованию акционеров, совокупная доля которых в уставном капитале акционерного общества составляет десять и более процентов.
\eEasyList
\subsubsection{{\bf Статья 67.2.} Корпоративный договор}
\beginEasyList
    & Участники хозяйственного общества или некоторые из них вправе заключить между собой корпоративный договор об осуществлении своих корпоративных прав (договор об осуществлении прав участников общества с ограниченной ответственностью, акционерное соглашение), в соответствии с которым они обязуются осуществлять эти права определенным образом или воздерживаться (отказаться) от их осуществления, в том числе голосовать определенным образом на общем собрании участников общества, согласованно осуществлять иные действия по управлению обществом, приобретать или отчуждать доли в его уставном капитале (акции) по определенной цене или при наступлении определенных обстоятельств либо воздерживаться от отчуждения долей (акций) до наступления определенных обстоятельств.
    & Корпоративный договор не может обязывать его участников голосовать в соответствии с указаниями органов общества, определять структуру органов общества и их компетенцию.
    \par Условия корпоративного договора, противоречащие правилам \ul{абзаца первого} настоящего пункта, ничтожны.
    \par Корпоративным договором может быть установлена обязанность его сторон проголосовать на общем собрании участников общества за включение в устав общества положений, определяющих структуру органов общества и их компетенцию, если в соответствии с настоящим Кодексом и законами о хозяйственных обществах допускается изменение структуры органов общества и их компетенции уставом общества.
    & Корпоративный договор заключается в письменной форме путем составления одного документа, подписанного сторонами.
    & Участники хозяйственного общества, заключившие корпоративный договор, обязаны уведомить общество о факте заключения корпоративного договора, при этом его содержание раскрывать не требуется. В случае неисполнения данной обязанности участники общества, не являющиеся сторонами корпоративного договора, вправе требовать возмещения причиненных им убытков.
    \par Информация о корпоративном договоре, заключенном акционерами публичного акционерного общества, должна быть раскрыта в пределах, в порядке и на условиях, которые предусмотрены законом об акционерных обществах.
    \par Если иное не установлено законом, информация о содержании корпоративного договора, заключенного участниками непубличного общества, не подлежит раскрытию и является конфиденциальной.
    & Корпоративный договор не создает обязанностей для лиц, не участвующих в нем в качестве сторон (\ul{статья 308}).
    & Нарушение корпоративного договора может являться основанием для признания недействительным решения органа хозяйственного общества по иску стороны этого договора при условии, что на момент принятия органом хозяйственного общества соответствующего решения сторонами корпоративного договора являлись все участники хозяйственного общества.
    \par Признание решения органа хозяйственного общества недействительным в соответствии с настоящим пунктом само по себе не влечет недействительности сделок хозяйственного общества с третьими лицами, совершенных на основании такого решения.
    \par Сделка, заключенная стороной корпоративного договора в нарушение этого договора, может быть признана судом недействительной по иску участника корпоративного договора только в случае, если другая сторона сделки знала или должна была знать об ограничениях, предусмотренных корпоративным договором.
    & Стороны корпоративного договора не вправе ссылаться на его недействительность в связи с его противоречием положениям устава хозяйственного общества.
    & Прекращение права одной из сторон корпоративного договора на долю в уставном капитале (акции) хозяйственного общества не влечет прекращения действия корпоративного договора в отношении остальных его сторон, если иное не предусмотрено этим договором.
    & Кредиторы общества и иные третьи лица могут заключить договор с участниками хозяйственного общества, по которому последние в целях обеспечения охраняемого законом интереса таких третьих лиц обязуются осуществлять свои корпоративные права определенным образом или воздерживаться (отказаться) от их осуществления, в том числе голосовать определенным образом на общем собрании участников общества, согласованно осуществлять иные действия по управлению обществом, приобретать или отчуждать доли в его уставном капитале (акции) по определенной цене или при наступлении определенных обстоятельств либо воздерживаться от отчуждения долей (акций) до наступления определенных обстоятельств. К этому договору соответственно применяются правила о корпоративном договоре.
    & Правила о корпоративном договоре соответственно применяются к соглашению о создании хозяйственного общества, если иное не установлено законом или не вытекает из существа отношений сторон такого соглашения.
\eEasyList
\subsubsection{{\bf Статья 67.3.} Дочернее хозяйственное общество}
\beginEasyList
    & Хозяйственное общество признается дочерним, если другое (основное) хозяйственное товарищество или общество в силу преобладающего участия в его уставном капитале, либо в соответствии с заключенным между ними договором, либо иным образом имеет возможность определять решения, принимаемые таким обществом.
    & Дочернее общество не отвечает по долгам основного хозяйственного товарищества или общества.
    \par Основное хозяйственное товарищество или общество отвечает солидарно с дочерним обществом по сделкам, заключенным последним во исполнение указаний или с согласия основного хозяйственного товарищества или общества (\ul{пункт 3 статьи 401}), за исключением случаев голосования основного хозяйственного товарищества или общества по вопросу об одобрении сделки на общем собрании участников дочернего общества, а также одобрения сделки органом управления основного хозяйственного общества, если необходимость такого одобрения предусмотрена уставом дочернего и (или) основного общества.
    \par В случае несостоятельности (банкротства) дочернего общества по вине основного хозяйственного товарищества или общества последнее несет субсидиарную ответственность по его долгам.
    \par Участники (акционеры) дочернего общества вправе требовать возмещения основным хозяйственным товариществом или обществом убытков, причиненных его действиями или бездействием дочернему обществу (\ul{статья 1064}).
\eEasyList
\subsubsection{{\bf Статья 68.} Преобразование хозяйственных товариществ и обществ}
\beginEasyList
& Хозяйственные товарищества и общества одного вида могут преобразовываться в хозяйственные товарищества и общества другого вида или в производственные кооперативы по решению общего собрания участников в порядке, установленном настоящим Кодексом и законами о хозяйственных обществах.
& При преобразовании товарищества в общество каждый полный товарищ, ставший участником (акционером) общества, в течение двух лет несет субсидиарную ответственность всем своим имуществом по обязательствам, перешедшим к обществу от товарищества. Отчуждение бывшим товарищем принадлежавших ему долей (акций) не освобождает его от такой ответственности. Правила, изложенные в настоящем пункте, соответственно применяются при преобразовании товарищества в производственный кооператив.
& Хозяйственные товарищества и общества не могут быть реорганизованы в некоммерческие организации, а также в унитарные коммерческие организации.
\eEasyList
\subsection{{\bf 2. Полное товарищество}}
\subsubsection{{\bf Статья 69.} Основные положения о полном товариществе}
\beginEasyList
& Полным признается товарищество, участники которого (полные товарищи) в соответствии с заключенным между ними договором занимаются предпринимательской деятельностью от имени товарищества и несут ответственность по его обязательствам принадлежащим им имуществом.
& Лицо может быть участником только одного полного товарищества.
& Фирменное наименование полного товарищества должно содержать либо имена (наименования) всех его участников и слова «полное товарищество», либо имя (наименование) одного или нескольких участников с добавлением слов «и компания» и слова «полное товарищество».
\eEasyList
\subsubsection{{\bf Статья 70.} Учредительный договор полного товарищества}
\beginEasyList
& Полное товарищество создается и действует на основании учредительного договора. Учредительный договор подписывается всеми его участниками.
& Учредительный договор полного товарищества должен содержать сведения о фирменном наименовании и месте нахождения товарищества, условия о размере и составе его складочного капитала; о размере и порядке изменения долей каждого из участников в складочном капитале; о размере, составе, сроках и порядке внесения ими вкладов; об ответственности участников за нарушение обязанностей по внесению вкладов.
\eEasyList
\subsubsection{{\bf Статья 71.} Управление в полном товариществе}
\beginEasyList
& Управление деятельностью полного товарищества осуществляется по общему согласию всех участников. Учредительным договором товарищества могут быть предусмотрены случаи, когда решение принимается большинством голосов участников.
& Каждый участник полного товарищества имеет один голос, если учредительным договором не предусмотрен иной порядок определения количества голосов его участников.
& Каждый участник товарищества независимо от того, уполномочен ли он вести дела товарищества, вправе получать всю информацию о деятельности товарищества и знакомиться со всей документацией по ведению дел. Отказ от этого права или его ограничение, в том числе по соглашению участников товарищества, ничтожны.
\eEasyList
\subsubsection{{\bf Статья 72.} Ведение дел полного товарищества}
\beginEasyList
& Каждый участник полного товарищества вправе действовать от имени товарищества, если учредительным договором не установлено, что все его участники ведут дела совместно, либо ведение дел поручено отдельным участникам.
\par При совместном ведении дел товарищества его участниками для совершения каждой сделки требуется согласие всех участников товарищества.
\par Если ведение дел товарищества поручается его участниками одному или некоторым из них, остальные участники для совершения сделок от имени товарищества должны иметь доверенность от участника (участников), на которого возложено ведение дел товарищества.
\par В отношениях с третьими лицами товарищество не вправе ссылаться на положения учредительного договора, ограничивающие полномочия участников товарищества, за исключением случаев, когда товарищество докажет, что третье лицо в момент совершения сделки знало или заведомо должно было знать об отсутствии у участника товарищества права действовать от имени товарищества.
& Полномочия на ведение дел товарищества, предоставленные одному или нескольким участникам, могут быть прекращены судом по требованию одного или нескольких других участников товарищества при наличии к тому серьезных оснований, в частности вследствие грубого нарушения уполномоченным лицом (лицами) своих обязанностей или обнаружившейся неспособности его к разумному ведению дел. На основании судебного решения в учредительный договор товарищества вносятся необходимые изменения.
\eEasyList
\subsubsection{{\bf Статья 73.} Обязанности участника полного товарищества}
\beginEasyList
& Участник полного товарищества обязан участвовать в его деятельности в соответствии с условиями учредительного договора.
& Участник полного товарищества обязан внести не менее половины своего вклада в складочный капитал товарищества до его государственной регистрации. Остальная часть должна быть внесена участником в сроки, установленные учредительным договором. При невыполнении указанной обязанности участник обязан уплатить товариществу десять процентов годовых с невнесенной части вклада и возместить причиненные убытки, если иные последствия не установлены учредительным договором.
& Участник полного товарищества не вправе без согласия остальных участников совершать от своего имени в своих интересах или в интересах третьих лиц сделки, однородные с теми, которые составляют предмет деятельности товарищества.
\par При нарушении этого правила товарищество вправе по своему выбору потребовать от такого участника возмещения причиненных товариществу убытков либо передачи товариществу всей приобретенной по таким сделкам выгоды.
\eEasyList
\subsubsection{{\bf Статья 74.} Распределение прибыли и убытков полного товарищества}
\beginEasyList
& Прибыль и убытки полного товарищества распределяются между его участниками пропорционально их долям в складочном капитале, если иное не предусмотрено учредительным договором или иным соглашением участников. Не допускается соглашение об устранении кого-либо из участников товарищества от участия в прибыли или в убытках.
& Если вследствие понесенных товариществом убытков стоимость его чистых активов станет меньше размера его складочного капитала, полученная товариществом прибыль не распределяется между участниками до тех пор, пока стоимость чистых активов не превысит размер складочного капитала.
\eEasyList
\subsubsection{{\bf Статья 75.} Ответственность участников полного товарищества по его обязательствам}
\beginEasyList
& Участники полного товарищества солидарно несут субсидиарную ответственность своим имуществом по обязательствам товарищества.
& Участник полного товарищества, не являющийся его учредителем, отвечает наравне с другими участниками по обязательствам, возникшим до его вступления в товарищество.
\par Участник, выбывший из товарищества, отвечает по обязательствам товарищества, возникшим до момента его выбытия, наравне с оставшимися участниками в течение двух лет со дня утверждения отчета о деятельности товарищества за год, в котором он выбыл из товарищества.
& Соглашение участников товарищества об ограничении или устранении ответственности, предусмотренной в настоящей статье, ничтожно.
\eEasyList
\subsubsection{{\bf Статья 76.} Изменение состава участников полного товарищества}
\beginEasyList
& В случаях выхода или смерти кого-либо из участников полного товарищества, признания одного из них безвестно отсутствующим, недееспособным, или ограниченно дееспособным, либо несостоятельным (банкротом), открытия в отношении одного из участников реорганизационных процедур по решению суда, ликвидации участвующего в товариществе юридического лица либо обращения кредитором одного из участников взыскания на часть имущества, соответствующую его доле в складочном капитале, товарищество может продолжить свою деятельность, если это предусмотрено учредительным договором товарищества или соглашением остающихся участников.
& Участники полного товарищества вправе требовать в судебном порядке исключения кого-либо из участников из товарищества по единогласному решению остающихся участников и при наличии к тому серьезных оснований, в частности вследствие грубого нарушения этим участником своих обязанностей или обнаружившейся неспособности его к разумному ведению дел.
\eEasyList
\subsubsection{{\bf Статья 77.} Выход участника из полного товарищества}
\beginEasyList
& Участник полного товарищества вправе выйти из него, заявив об отказе от участия в товариществе.
\par Отказ от участия в полном товариществе, учрежденном без указания срока, должен быть заявлен участником не менее чем за шесть месяцев до фактического выхода из товарищества. Досрочный отказ от участия в полном товариществе, учрежденном на определенный срок, допускается лишь по уважительной причине.
& Соглашение между участниками товарищества об отказе от права выйти из товарищества ничтожно.
\eEasyList
\subsubsection{{\bf Статья 78.} Последствия выбытия участника из полного товарищества}
\beginEasyList
& Участнику, выбывшему из полного товарищества, выплачивается стоимость части имущества товарищества, соответствующей доле этого участника в складочном капитале, если иное не предусмотрено учредительным договором. По соглашению выбывающего участника с остающимися участниками выплата стоимости имущества может быть заменена выдачей имущества в натуре.
\par Причитающаяся выбывающему участнику часть имущества товарищества или ее стоимость определяется по балансу, составляемому, за исключением случая, предусмотренного в \ul{статье 80} настоящего Кодекса, на момент его выбытия.
& В случае смерти участника полного товарищества его наследник может вступить в полное товарищество лишь с согласия других участников.
\par Юридическое лицо, являющееся правопреемником участвовавшего в полном товариществе реорганизованного юридического лица, вправе вступить в товарищество с согласия других его участников, если иное не предусмотрено учредительным договором товарищества.
\par Расчеты с наследником (правопреемником), не вступившим в товарищество, производятся в соответствии с \ul{пунктом 1} настоящей статьи. Наследник (правопреемник) участника полного товарищества несет ответственность по обязательствам товарищества перед третьими лицами, по которым в соответствии с \ul{пунктом 2 статьи 75} настоящего Кодекса отвечал бы выбывший участник, в пределах перешедшего к нему имущества выбывшего участника товарищества.
& Если один из участников выбыл из товарищества, доли оставшихся участников в складочном капитале товарищества соответственно увеличиваются, если иное не предусмотрено учредительным договором или иным соглашением участников.
\eEasyList
\subsubsection{{\bf Статья 79.} Передача доли участника в складочном капитале полного товарищества}
\par Участник полного товарищества вправе с согласия остальных его участников передать свою долю в складочном капитале или ее часть другому участнику товарищества либо третьему лицу.
\par При передаче доли (части доли) иному лицу к нему переходят полностью или в соответствующей части права, принадлежавшие участнику, передавшему долю (часть доли). Лицо, которому передана доля (часть доли), несет ответственность по обязательствам товарищества в порядке, установленном абзацем первым \ul{пункта 2 статьи 75} настоящего Кодекса.
\par Передача всей доли иному лицу участником товарищества прекращает его участие в товариществе и влечет последствия, предусмотренные \ul{пунктом 2 статьи 75} настоящего Кодекса.
\subsubsection{{\bf Статья 80.} Обращение взыскания на долю участника в складочном капитале полного товарищества}
\par Обращение взыскания на долю участника в складочном капитале полного товарищества по собственным долгам участника допускается лишь при недостатке иного его имущества для покрытия долгов. Кредиторы такого участника вправе потребовать от полного товарищества выдела части имущества товарищества, соответствующей доле должника в складочном капитале, с целью обращения взыскания на это имущество. Подлежащая выделу часть имущества товарищества или его стоимость определяется по балансу, составленному на момент предъявления кредиторами требования о выделе.
\par Обращение взыскания на имущество, соответствующее доле участника в складочном капитале полного товарищества, прекращает его участие в товариществе и влечет последствия, предусмотренные абзацем вторым \ul{пункта 2 статьи 75} настоящего Кодекса.
\subsubsection{{\bf Статья 81.} Ликвидация полного товарищества}
\par Полное товарищество ликвидируется по основаниям, указанным в \ul{статье 61} настоящего Кодекса, а также в случае, когда в товариществе остается единственный участник. Такой участник вправе в течение шести месяцев с момента, когда он стал единственным участником товарищества, преобразовать такое товарищество в хозяйственное общество в порядке, установленном настоящим Кодексом.
\par Полное товарищество ликвидируется также в случаях, указанных в \ul{пункте 1 статьи 76} настоящего Кодекса, если учредительным договором товарищества или соглашением остающихся участников не предусмотрено, что товарищество продолжит свою деятельность.
\subsection{{\bf 3. Товарищество на вере}}
\subsubsection{{\bf Статья 82.} Основные положения о товариществе на вере}
\beginEasyList
& Товариществом на вере (коммандитным товариществом) признается товарищество, в котором наряду с участниками, осуществляющими от имени товарищества предпринимательскую деятельность и отвечающими по обязательствам товарищества своим имуществом (полными товарищами), имеется один или несколько участников --- вкладчиков (коммандитистов), которые несут риск убытков, связанных с деятельностью товарищества, в пределах сумм внесенных ими вкладов и не принимают участия в осуществлении товариществом предпринимательской деятельности.
& Положение полных товарищей, участвующих в товариществе на вере, и их ответственность по обязательствам товарищества определяются правилами настоящего \ul{Кодекса} об участниках полного товарищества.
& Лицо может быть полным товарищем только в одном товариществе на вере.
\par Участник полного товарищества не может быть полным товарищем в товариществе на вере.
\par Полный товарищ в товариществе на вере не может быть участником полного товарищества.
\par Число коммандитистов в товариществе на вере не должно превышать двадцать. В противном случае оно подлежит преобразованию в хозяйственное общество в течение года, а по истечении этого срока - ликвидации в судебном порядке, если число его коммандитистов не уменьшится до указанного предела.
& Фирменное наименование товарищества на вере должно содержать либо имена (наименования) всех полных товарищей и слова «товарищество на вере» или «коммандитное товарищество», либо имя (наименование) не менее чем одного полного товарища с добавлением слов «и компания» и слова «товарищество на вере» или «коммандитное товарищество».
\par Если в фирменное наименование товарищества на вере включено имя вкладчика, такой вкладчик становится полным товарищем.
& К товариществу на вере применяются правила настоящего \ul{Кодекса} о полном товариществе постольку, поскольку это не противоречит правилам настоящего Кодекса о товариществе на вере.
\eEasyList
\subsubsection{{\bf Статья 83.} Учредительный договор товарищества на вере}
\beginEasyList
& Товарищество на вере создается и действует на основании учредительного договора. Учредительный договор подписывается всеми полными товарищами.
& Учредительный договор товарищества на вере должен содержать сведения о фирменном наименовании и месте нахождения товарищества, условия о размере и составе складочного капитала товарищества; о размере и порядке изменения долей каждого из полных товарищей в складочном капитале; о размере, составе, сроках и порядке внесения ими вкладов, их ответственности за нарушение обязанностей по внесению вкладов; о совокупном размере вкладов, вносимых вкладчиками.
\eEasyList
\subsubsection{{\bf Статья 84.} Управление в товариществе на вере и ведение его дел}
\beginEasyList
& Управление деятельностью товарищества на вере осуществляется полными товарищами. Порядок управления и ведения дел такого товарищества его полными товарищами устанавливается ими по правилам настоящего \ul{Кодекса} о полном товариществе.
& Вкладчики не вправе участвовать в управлении и ведении дел товарищества на вере, выступать от его имени иначе, как по доверенности. Они не вправе оспаривать действия полных товарищей по управлению и ведению дел товарищества.
\eEasyList
\subsubsection{{\bf Статья 85.} Права и обязанности вкладчика товарищества на вере}
\beginEasyList
& Вкладчик товарищества на вере обязан внести вклад в складочный капитал. Внесение вклада удостоверяется свидетельством об участии, выдаваемым вкладчику товариществом.
& Вкладчик товарищества на вере имеет право:
&& получать часть прибыли товарищества, причитающуюся на его долю в складочном капитале, в порядке, предусмотренном учредительным договором;
&& знакомиться с годовыми отчетами и балансами товарищества;
&& по окончании финансового года выйти из товарищества и получить свой вклад в порядке, предусмотренном учредительным договором;
&& передать свою долю в складочном капитале или ее часть другому вкладчику или третьему лицу. Вкладчики пользуются преимущественным перед третьими лицами правом покупки доли (ее части) применительно к условиям и порядку, предусмотренным \ul{пунктом 2 статьи 93} настоящего Кодекса. Передача всей доли иному лицу вкладчиком прекращает его участие в товариществе.
\par Учредительным договором товарищества на вере могут предусматриваться и иные права вкладчика.
\eEasyList
\subsubsection{{\bf Статья 86.} Ликвидация товарищества на вере}
\beginEasyList
& Товарищество на вере ликвидируется при выбытии всех участвовавших в нем вкладчиков. Однако полные товарищи вправе вместо ликвидации преобразовать товарищество на вере в полное товарищество.
\par Товарищество на вере ликвидируется также по основаниям ликвидации полного товарищества (\ul{статья 81}). Однако товарищество на вере сохраняется, если в нем остаются по крайней мере один полный товарищ и один вкладчик.
& При ликвидации товарищества на вере, в том числе в случае банкротства, вкладчики имеют преимущественное перед полными товарищами право на получение вкладов из имущества товарищества, оставшегося после удовлетворения требований его кредиторов.
\par Оставшееся после этого имущество товарищества распределяется между полными товарищами и вкладчиками пропорционально их долям в складочном капитале товарищества, если иной порядок не установлен учредительным договором или соглашением полных товарищей и вкладчиков.
\eEasyList
\subsection{{\bf 3.1. Крестьянское (фермерское) хозяйство}}
\subsubsection{{\bf Статья 86.1.} Крестьянское (фермерское) хозяйство}
\beginEasyList
& Граждане, ведущие совместную деятельность в области сельского хозяйства без образования юридического лица на основе соглашения о создании крестьянского (фермерского) хозяйства (\ul{статья 23}), вправе создать юридическое лицо --- крестьянское (фермерское) хозяйство.
\par Крестьянским (фермерским) хозяйством, создаваемым в соответствии с настоящей статьей в качестве юридического лица, признается добровольное объединение граждан на основе членства для совместной производственной или иной хозяйственной деятельности в области сельского хозяйства, основанной на их личном участии и объединении членами крестьянского (фермерского) хозяйства имущественных вкладов.
& Имущество крестьянского (фермерского) хозяйства принадлежит ему на праве собственности.
& Гражданин может быть членом только одного крестьянского (фермерского) хозяйства, созданного в качестве юридического лица.
& При обращении взыскания кредиторов крестьянского (фермерского) хозяйства на земельный участок, находящийся в собственности хозяйства, земельный участок подлежит продаже с публичных торгов в пользу лица, которое в соответствии с законом вправе продолжать использование земельного участка по целевому назначению.
\par Члены крестьянского (фермерского) хозяйства, созданного в качестве юридического лица, несут по обязательствам крестьянского (фермерского) хозяйства субсидиарную ответственность.
& Особенности правового положения крестьянского (фермерского) хозяйства, созданного в качестве юридического лица, определяются законом.
\eEasyList
\subsection{{\bf 4. Общество с ограниченной ответственностью}}
\subsubsection{{\bf Статья 87.} Основные положения об обществе с ограниченной ответственностью}
\beginEasyList
& Обществом с ограниченной ответственностью признается хозяйственное общество, уставный капитал которого разделен на доли; участники общества с ограниченной ответственностью не отвечают по его обязательствам и несут риск убытков, связанных с деятельностью общества, в пределах стоимости принадлежащих им долей.
\par Участники общества, не полностью оплатившие доли, несут солидарную ответственность по обязательствам общества в пределах стоимости неоплаченной части доли каждого из участников.
& Фирменное наименование общества с ограниченной ответственностью должно содержать наименование общества и слова «с ограниченной ответственностью».
& Правовое положение общества с ограниченной ответственностью и права и обязанности его участников определяются настоящим Кодексом и законом об обществах с ограниченной ответственностью.
\par Абзац второй утратил силу с 1 сентября 2014 г.
\eEasyList
\subsubsection{{\bf Статья 88.} Участники общества с ограниченной ответственностью}
\beginEasyList
& Число участников общества с ограниченной ответственностью не должно превышать пятьдесят. В противном случае оно подлежит преобразованию в акционерное общество в течение года, а по истечении этого срока --- ликвидации в судебном порядке, если число его участников не уменьшится до указанного предела.
& Общество с ограниченной ответственностью может быть учреждено одним лицом или может состоять из одного лица, в том числе при создании в результате реорганизации.
\par Абзац второй утратил силу с 1 сентября 2014 г.
\eEasyList
\subsubsection{{\bf Статья 89.} Создание общества с ограниченной ответственностью и его устав}
\beginEasyList
& Учредители общества с ограниченной ответственностью заключают между собой договор об учреждении общества с ограниченной ответственностью, определяющий порядок осуществления ими совместной деятельности по учреждению общества, размер уставного капитала общества, размер их долей в уставном капитале общества и иные установленные законом об обществах с ограниченной ответственностью условия.
\par Договор об учреждении общества с ограниченной ответственностью заключается в письменной форме.
& Учредители общества с ограниченной ответственностью несут солидарную ответственность по обязательствам, связанным с его учреждением и возникшим до его государственной регистрации.
\par Общество с ограниченной ответственностью несет ответственность по обязательствам учредителей общества, связанным с его учреждением, только в случае последующего одобрения действий учредителей общества общим собранием участников общества. Размер ответственности общества по этим обязательствам учредителей общества может быть ограничен законом об обществах с ограниченной ответственностью.
& Учредительным документом общества с ограниченной ответственностью является его устав.
\par Устав общества с ограниченной ответственностью должен содержать сведения о фирменном наименовании общества и месте его нахождения, размере его уставного капитала (за исключением случая, предусмотренного \ul{пунктом 2 статьи 52} настоящего Кодекса), составе и компетенции его органов, порядке принятия ими решений (в том числе решений по вопросам, принимаемым единогласно или квалифицированным большинством голосов) и иные сведения, предусмотренные законом об обществах с ограниченной ответственностью.
& Порядок совершения иных действий по учреждению общества с ограниченной ответственностью определяется законом об обществах с ограниченной ответственностью.
\eEasyList
\subsubsection{{\bf Статья 90.} Уставный капитал общества с ограниченной ответственностью}
\beginEasyList
& Уставный капитал общества с ограниченной ответственностью (\ul{статья 66.2}) составляется из номинальной стоимости долей участников.
& Не допускается освобождение участника общества с ограниченной ответственностью от обязанности оплаты доли в уставном капитале общества.
\par Оплата уставного капитала общества с ограниченной ответственностью при увеличении уставного капитала путем зачета требований к обществу допускается в случаях, предусмотренных законом об обществах с ограниченной ответственностью.
& Уставный капитал общества с ограниченной ответственностью оплачивается его участниками в сроки и в порядке, которые предусмотрены законом об обществах с ограниченной ответственностью.
\par Последствия нарушения участниками общества сроков и порядка оплаты уставного капитала общества определяются законом об обществах с ограниченной ответственностью.
& Если по окончании второго или каждого последующего финансового года стоимость чистых активов общества с ограниченной ответственностью окажется меньше его уставного капитала, общество в порядке и в срок, которые предусмотрены законом об обществах с ограниченной ответственностью, обязано увеличить стоимость чистых активов до размера уставного капитала или зарегистрировать в установленном порядке уменьшение уставного капитала. Если стоимость указанных активов общества становится меньше определенного законом минимального размера уставного капитала, общество подлежит ликвидации.
& Уменьшение уставного капитала общества с ограниченной ответственностью допускается после уведомления всех его кредиторов. В этом случае последние вправе потребовать досрочного прекращения или исполнения соответствующих обязательств общества и возмещения им убытков.
\par Права и обязанности кредиторов кредитных организаций и некредитных финансовых организаций, созданных в организационно-правовой форме общества с ограниченной ответственностью, определяются также законами, регулирующими деятельность таких организаций.Права и обязанности кредиторов кредитных организаций и некредитных финансовых организаций, созданных в организационно-правовой форме общества с ограниченной ответственностью, определяются также законами, регулирующими деятельность таких организаций.
& Увеличение уставного капитала общества допускается после полной оплаты всех его долей.
\eEasyList
\subsubsection{{\bf Статья 91.} Утратила силу с 1 сентября 2014 г.}
\subsubsection{{\bf Статья 92.} Реорганизация и ликвидация общества с ограниченной ответственностью}
\beginEasyList
& Общество с ограниченной ответственностью может быть реорганизовано или ликвидировано добровольно по единогласному решению его участников.
\par Иные основания реорганизации и ликвидации общества, а также порядок его реорганизации и ликвидации определяются настоящим Кодексом и другими законами.
& Общество с ограниченной ответственностью вправе преобразоваться в акционерное общество, хозяйственное товарищество или производственный кооператив.
\eEasyList
\subsubsection{{\bf Статья 93.} Переход доли в уставном капитале общества с ограниченной ответственностью к другому лицу}
\beginEasyList
& Переход доли или части доли участника общества в уставном капитале общества с ограниченной ответственностью к другому лицу допускается на основании сделки или в порядке правопреемства либо на ином законном основании с учетом особенностей, предусмотренных настоящим Кодексом и законом об обществах с ограниченной ответственностью.
& Продажа либо отчуждение иным образом доли или части доли в уставном капитале общества с ограниченной ответственностью третьим лицам допускается с соблюдением требований, предусмотренных законом об обществах с ограниченной ответственностью, если это не запрещено уставом общества.
\par Участники общества пользуются преимущественным правом покупки доли или части доли участника общества. Порядок осуществления преимущественного права и срок, в течение которого участники общества могут воспользоваться указанным правом, определяются законом об обществах с ограниченной ответственностью и уставом общества. Уставом общества также может быть предусмотрено преимущественное право покупки обществом доли или части доли участника общества, если другие участники общества не использовали свое преимущественное право покупки доли или части доли в уставном капитале общества.
& В случае, если уставом общества отчуждение доли или части доли, принадлежащих участнику общества, третьим лицам запрещено и другие участники общества отказались от их приобретения либо не получено согласие на отчуждение доли или части доли участнику общества или третьему лицу при условии, что необходимость получить такое согласие предусмотрена уставом общества, общество обязано приобрести по требованию участника общества принадлежащую ему долю или часть доли.
& Доля участника общества с ограниченной ответственностью может быть отчуждена до полной ее оплаты только в части, в которой она уже оплачена.
& В случае приобретения доли или части доли участника самим обществом с ограниченной ответственностью оно обязано реализовать их другим участникам или третьим лицам в сроки и в порядке, которые предусмотрены законом об обществах с ограниченной ответственностью и уставом, либо уменьшить свой уставный капитал в соответствии с \ul{пунктами 4} и \ul{5 статьи 90} настоящего Кодекса.
& Доли в уставном капитале общества переходят к наследникам граждан и к правопреемникам юридических лиц, являвшихся участниками общества, если иное не предусмотрено уставом общества с ограниченной ответственностью. Уставом общества может быть предусмотрено, что переход доли в уставном капитале общества к наследникам граждан и правопреемникам юридических лиц, являвшихся участниками общества, передача доли, принадлежавшей ликвидированному юридическому лицу, его учредителям (участникам), имеющим вещные права на его имущество или обязательственные права в отношении этого юридического лица, допускаются только с согласия остальных участников общества. Отказ в согласии на переход доли влечет за собой обязанность общества выплатить указанным лицам ее действительную стоимость или выдать им в натуре имущество, соответствующее такой стоимости, в порядке и на условиях, которые предусмотрены законом об обществах с ограниченной ответственностью и уставом общества.
& Переход доли участника общества с ограниченной ответственностью к другому лицу влечет за собой прекращение его участия в обществе.
\eEasyList
\subsubsection{{\bf Статья 94.} Выход участника общества с ограниченной ответственностью из общества}
\beginEasyList
& Участник общества с ограниченной ответственностью вправе выйти из общества независимо от согласия других его участников или общества путем:
&& подачи заявления о выходе из общества, если такая возможность предусмотрена уставом общества;
&& предъявления к обществу требования о приобретении обществом доли в случаях, предусмотренных \ul{пунктом 3 статьи 93} настоящего Кодекса и законом об обществах с ограниченной ответственностью.
& При подаче участником общества с ограниченной ответственностью заявления о выходе из общества или предъявлении им требования о приобретении обществом принадлежащей ему доли в случаях, предусмотренных \ul{пунктом 1} настоящей статьи, доля переходит к обществу с момента получения обществом соответствующего заявления (требования). Этому участнику должна быть выплачена действительная стоимость его доли в уставном капитале или с его согласия должно быть выдано в натуре имущество такой же стоимости в порядке, способом и в сроки, которые предусмотрены законом об обществах с ограниченной ответственностью и уставом общества.
\eEasyList
\subsection{{\bf 5. Общество с дополнительной ответственностью}}
\par Утратил силу с 1 сентября 2014 г.
\subsection{{\bf 6. Акционерное общество}}
\subsubsection{{\bf Статья 96.} Основные положения об акционерном обществе}
\beginEasyList
& Акционерным обществом признается хозяйственное общество, уставный капитал которого разделен на определенное число акций; участники акционерного общества (акционеры) не отвечают по его обязательствам и несут риск убытков, связанных с деятельностью общества, в пределах стоимости принадлежащих им акций.
\par Акционеры, не полностью оплатившие акции, несут солидарную ответственность по обязательствам акционерного общества в пределах неоплаченной части стоимости принадлежащих им акций.
& Фирменное наименование акционерного общества должно содержать его наименование и указание на то, что общество является акционерным.
& Правовое положение акционерного общества и права и обязанности акционеров определяются в соответствии с настоящим Кодексом и законом об акционерных обществах.
\par Особенности правового положения акционерных обществ, созданных путем приватизации государственных и муниципальных предприятий, определяются также законами и иными правовыми актами о приватизации этих предприятий.
\par Особенности правового положения кредитных организаций, созданных в организационно-правовой форме акционерного общества, права и обязанности их акционеров определяются также законами, регулирующими деятельность кредитных организаций.
\eEasyList
\subsubsection{{\bf Статья 97.} Публичное акционерное общество}
\beginEasyList
    & Публичное акционерное общество (\ul{пункт 1 статьи 66.3}) обязано представить для внесения в единый государственный реестр юридических лиц сведения о фирменном наименовании общества, содержащем указание на то, что такое общество является публичным.
    \par Акционерное общество вправе представить для внесения в единый государственный реестр юридических лиц сведения о фирменном наименовании общества, содержащем указание на то, что такое общество является публичным.
    \par Акционерное общество приобретает право публично размещать (путем открытой подписки) акции и ценные бумаги, конвертируемые в его акции, которые могут публично обращаться на условиях, установленных законами о ценных бумагах, со дня внесения в единый государственный реестр юридических лиц сведений о фирменном наименовании общества, содержащем указание на то, что такое общество является публичным.
    & Приобретение непубличным акционерным обществом статуса публичного общества (\ul{пункт 1} настоящей статьи) влечет недействительность положений устава и внутренних документов общества, противоречащих правилам о публичном акционерном обществе, установленным настоящим Кодексом, законом об акционерных обществах и законами о ценных бумагах.
    & В публичном акционерном обществе образуется коллегиальный орган управления общества (\ul{пункт 4 статьи 65.3}), число членов которого не может быть менее пяти. Порядок образования и компетенция указанного коллегиального органа управления определяются законом об акционерных обществах и уставом публичного акционерного общества.
    & Обязанности по ведению реестра акционеров публичного акционерного общества и исполнение функций счетной комиссии осуществляются организацией, имеющей предусмотренную законом лицензию.
    & В публичном акционерном обществе не могут быть ограничены количество акций, принадлежащих одному акционеру, их суммарная номинальная стоимость, а также максимальное число голосов, предоставляемых одному акционеру. Уставом публичного акционерного общества не может быть предусмотрена необходимость получения чьего-либо согласия на отчуждение акций этого общества. Никому не может быть предоставлено право преимущественного приобретения акций публичного акционерного общества, кроме случаев, предусмотренных \ul{пунктом 3 статьи 100} настоящего Кодекса.
    \par Уставом публичного акционерного общества не может быть отнесено к исключительной компетенции общего собрания акционеров решение вопросов, не относящихся к ней в соответствии с настоящим Кодексом и законом об акционерных обществах.
    & Публичное акционерное общество обязано раскрывать публично информацию, предусмотренную законом.
    & Дополнительные требования к созданию и деятельности, а также к прекращению публичных акционерных обществ устанавливаются законом об акционерных обществах и законами о ценных бумагах.
\eEasyList
\subsubsection{{\bf Статья 98.} Создание акционерного общества}
\beginEasyList
& Учредители акционерного общества заключают между собой договор, определяющий порядок осуществления ими совместной деятельности по созданию общества, размер уставного капитала общества, категории выпускаемых акций и порядок их размещения, а также иные условия, предусмотренные законом об акционерных обществах.
\par Договор о создании акционерного общества заключается в письменной форме путем составления одного документа, подписанного сторонами.
& Учредители акционерного общества несут солидарную ответственность по обязательствам, возникшим до регистрации общества.
\par Общество несет ответственность по обязательствам учредителей, связанным с его созданием, только в случае последующего одобрения их действий общим собранием акционеров.
& Учредительным документом акционерного общества является его устав, утвержденный учредителями.
\par Устав акционерного общества должен содержать сведения о фирменном наименовании общества и месте его нахождения, условия о категориях выпускаемых обществом акций, об их номинальной стоимости и количестве, о размере уставного капитала общества, правах акционеров, составе и компетенции органов общества и порядке принятия ими решений, в том числе по вопросам, решения по которым принимаются единогласно или квалифицированным большинством голосов. В уставе акционерного общества также должны содержаться иные сведения, предусмотренные законом.
& Порядок совершения иных действий по созданию акционерного общества, в том числе компетенция учредительного собрания, определяется законом об акционерных обществах.
& Особенности создания акционерных обществ при приватизации государственных и муниципальных предприятий определяются законами и иными правовыми актами о приватизации этих предприятий.
& Акционерное общество может быть создано одним лицом или состоять из одного лица в случае приобретения одним акционером всех акций общества. Сведения об этом подлежат внесению в единый государственный реестр юридических лиц.
\par Акционерное общество не может иметь в качестве единственного участника другое хозяйственное общество, состоящее из одного лица, если иное не установлено законом.
\eEasyList
\subsubsection{{\bf Статья 99.} Уставный капитал акционерного общества}
\beginEasyList
& Уставный капитал акционерного общества составляется из номинальной стоимости акций общества, приобретенных акционерами.
\par Абзац второй утратил силу с 1 сентября 2014 г.
& Не допускается освобождение акционера от обязанности оплаты акций общества.
\par Оплата размещаемых обществом дополнительных акций путем зачета требований к обществу допускается в случаях, предусмотренных законом об акционерных обществах.
& Открытая подписка на акции акционерного общества не допускается до полной оплаты уставного капитала. При учреждении акционерного общества все его акции должны быть распределены среди учредителей.
& Если по окончании второго или каждого последующего финансового года стоимость чистых активов акционерного общества окажется меньше его уставного капитала, общество в порядке и в срок, которые предусмотрены законом об акционерных обществах, обязано увеличить стоимость чистых активов до размера уставного капитала либо зарегистрировать в установленном порядке уменьшение уставного капитала. Если стоимость указанных активов общества становится меньше определенного законом минимального размера уставного капитала, общество подлежит ликвидации.
& Законом или уставом общества, не являющегося публичным, могут быть установлены ограничения числа, суммарной номинальной стоимости акций или максимального числа голосов, принадлежащих одному акционеру.
\eEasyList
\subsubsection{{\bf Статья 100.} Увеличение уставного капитала акционерного общества}
\beginEasyList
& Акционерное общество вправе в соответствии с законом об акционерных обществах увеличить уставный капитал путем увеличения номинальной стоимости акций или выпуска дополнительных акций.
& Увеличение уставного капитала акционерного общества допускается после его полной оплаты.
& В случаях и в порядке, которые предусмотрены законом об акционерных обществах, акционерам и лицам, которым принадлежат ценные бумаги общества, конвертируемые в его акции, может быть предоставлено преимущественное право покупки дополнительно выпускаемых обществом акций или конвертируемых в акции ценных бумаг.
\eEasyList
\subsubsection{{\bf Статья 101.} Уменьшение уставного капитала акционерного общества}
\beginEasyList
& Акционерное общество вправе в соответствии с законом об акционерных обществах уменьшить уставный капитал путем уменьшения номинальной стоимости акций либо путем покупки части акций в целях сокращения их общего количества.
\par Уменьшение уставного капитала общества допускается после уведомления всех его кредиторов в порядке, определяемом законом об акционерных обществах. Права кредиторов в случае уменьшения уставного капитала общества или снижения стоимости его чистых активов определяются законом об акционерных обществах.
\par Права и обязанности кредиторов кредитных организаций и некредитных финансовых организаций, созданных в организационно-правовой форме акционерного общества, определяются также законами, регулирующими деятельность таких организаций.
& Уменьшение уставного капитала акционерного общества путем покупки и погашения части акций допускается, если такая возможность предусмотрена в уставе общества.
\eEasyList
\subsubsection{{\bf Статья 102.} Ограничения на выпуск ценных бумаг и выплату дивидендов акционерного общества}
\beginEasyList
& Доля привилегированных акций в общем объеме уставного капитала акционерного общества не должна превышать двадцати пяти процентов. При этом публичное акционерное общество не вправе размещать привилегированные акции, номинальная стоимость которых ниже номинальной стоимости обыкновенных акций.
& Утратил силу с 1 сентября 2014 г.
& Акционерное общество не вправе объявлять и выплачивать дивиденды:
\par до полной оплаты всего уставного капитала;
\par если стоимость чистых активов акционерного общества меньше его уставного капитала и резервного фонда либо станет меньше их размера в результате выплаты дивидендов.
\par в иных случаях, предусмотренных законом об акционерных обществах.
\eEasyList
\subsubsection{{\bf Статья 103.} Утратила силу с 1 сентября 2014 г.}
\subsubsection{{\bf Статья 104.} Реорганизация и ликвидация акционерного общества}
\beginEasyList
& Акционерное общество может быть реорганизовано или ликвидировано добровольно по решению общего собрания акционеров.
\par Иные основания и порядок реорганизации и ликвидации акционерного общества определяются законом.
& Акционерное общество вправе преобразоваться в общество с ограниченной ответственностью или в производственный кооператив, а также в некоммерческую организацию в соответствии с законом.
\eEasyList
\subsection{{\bf 7. Дочерние и зависимые общества}}
\par Утратил силу с 1 сентября 2014 г.
\subsection{{\bf 8. Производственные кооперативы}}
\subsubsection{{\bf Статья 106.1.} Понятие производственного кооператива}
\beginEasyList
& Производственным кооперативом (артелью) признается добровольное объединение граждан на основе членства для совместной производственной или иной хозяйственной деятельности (производство, переработка, сбыт промышленной, сельскохозяйственной и иной продукции, выполнение работ, торговля, бытовое обслуживание, оказание других услуг), основанной на их личном трудовом и ином участии и объединении его членами (участниками) имущественных паевых взносов. Законом и учредительными документами производственного кооператива может быть предусмотрено участие в его деятельности юридических лиц. Производственный кооператив является коммерческой организацией.
& Члены производственного кооператива несут по обязательствам кооператива субсидиарную ответственность в размерах и в порядке, предусмотренных законом о производственных кооперативах и уставом кооператива.
\eEasyList
\subsubsection{{\bf Статья 106.2.} Создание производственного кооператива и его устав}
\beginEasyList
& Учредительным документом производственного кооператива является его устав, утверждаемый общим собранием его членов.
& Устав производственного кооператива должен содержать сведения о фирменном наименовании кооператива и месте его нахождения, условия о размере паевых взносов членов кооператива, составе и порядке внесения паевых взносов членами кооператива и об их ответственности за нарушение обязательства по внесению паевых взносов, о характере и порядке трудового участия его членов в деятельности кооператива и об их ответственности за нарушение обязанности принимать личное трудовое участие в деятельности кооператива, о порядке распределения прибыли и убытков кооператива, размере и об условиях субсидиарной ответственности его членов по обязательствам кооператива, о составе и компетенции органов кооператива и порядке принятия ими решений, в том числе по вопросам, решения по которым принимаются единогласно или квалифицированным большинством голосов.
& Фирменное наименование кооператива должно содержать его наименование и слова «производственный кооператив» или «артель».
& Число членов кооператива не должно быть менее пяти.
\eEasyList
\subsubsection{{\bf Статья 106.3.} Имущество производственного кооператива}
\beginEasyList
& Имущество, находящееся в собственности производственного кооператива, делится на паи его членов в соответствии с уставом кооператива.
\par Уставом кооператива может быть установлено, что определенная часть принадлежащего кооперативу имущества составляет неделимые фонды, используемые на цели, определяемые уставом.
\par Решение об образовании неделимых фондов принимается членами кооператива единогласно, если иное не предусмотрено уставом кооператива.
& Член кооператива обязан внести к моменту регистрации кооператива не менее десяти процентов паевого взноса, а остальную часть --- в течение года с момента регистрации.
& Кооператив не вправе выпускать акции.
& Прибыль кооператива распределяется между его членами в соответствии с их трудовым участием, если иной порядок не предусмотрен законом и уставом кооператива.
\par В таком же порядке распределяется имущество, оставшееся после ликвидации кооператива и удовлетворения требований его кредиторов.
\eEasyList
\subsubsection{{\bf Статья 106.4.} Особенности управления в производственном кооперативе}
\beginEasyList
    & Исполнительными органами производственного кооператива являются председатель и правление кооператива, если его образование предусмотрено законом или уставом кооператива
    & Членами правления производственного кооператива и председателем кооператива могут быть только члены кооператива.
    & Член производственного кооператива имеет один голос при принятии решений общим собранием.
\eEasyList
\subsubsection{{\bf Статья 106.5.} Прекращение членства в производственном кооперативе и переход пая}
\beginEasyList
& Член кооператива вправе по своему усмотрению выйти из кооператива. В этом случае ему должна быть выплачена стоимость пая или выдано имущество, соответствующее его паю, а также осуществлены другие выплаты, предусмотренные уставом кооператива.
\par Выплата стоимости пая или выдача другого имущества выходящему члену кооператива производится по окончании финансового года и утверждении бухгалтерского баланса кооператива, если иное не предусмотрено уставом кооператива.
& Член производственного кооператива может быть исключен из кооператива по решению общего собрания в случае неисполнения или ненадлежащего исполнения обязанностей, возложенных на него уставом кооператива, а также в других случаях, предусмотренных законом и уставом кооператива.
\par Член наблюдательного совета или исполнительного органа может быть исключен из кооператива по решению общего собрания в связи с его членством в аналогичном кооперативе.
\par Член кооператива, исключенный из него, имеет право на получение пая и других выплат, предусмотренных уставом кооператива, в соответствии с \ul{пунктом 1} настоящей статьи.
& Член кооператива вправе передать свой пай или его часть другому члену кооператива, если иное не предусмотрено законом и уставом кооператива.
\par Передача пая или его части гражданину, не являющемуся членом кооператива, допускается лишь с согласия кооператива. В этом случае другие члены кооператива пользуются преимущественным правом покупки такого пая или его части.
& В случае смерти члена производственного кооператива его наследники могут быть приняты в члены кооператива, если иное не предусмотрено уставом кооператива. В противном случае кооператив выплачивает наследникам стоимость пая умершего члена кооператива.
& Обращение взыскания на пай члена производственного кооператива по собственным долгам члена кооператива допускается лишь при недостатке иного его имущества для покрытия таких долгов в порядке, предусмотренном законом и уставом кооператива. Взыскание по долгам члена кооператива не может быть обращено на неделимые фонды кооператива.
\eEasyList
\subsubsection{{\bf Статья 106.6.} Преобразование производственного кооператива}
\par Производственный кооператив по решению его членов, принятому единогласно, может преобразоваться в хозяйственное товарищество или общество.
\subsection{{\bf § 3.} Производственные кооперативы}
\par Утратил силу с 1 сентября 2014 г.
\subsection{{\bf § 4. Государственные и муниципальные унитарные предприятия}}
\subsubsection{{\bf Статья 113.} Основные положения об унитарном предприятии}
\beginEasyList
& Унитарным предприятием признается коммерческая организация, не наделенная правом собственности на закрепленное за ней собственником имущество. Имущество унитарного предприятия является неделимым и не может быть распределено по вкладам (долям, паям), в том числе между работниками предприятия.
\par В организационно-правовой форме унитарных предприятий действуют государственные и муниципальные предприятия.
\par В случаях и в порядке, которые предусмотрены законом о государственных и муниципальных унитарных предприятиях, на базе государственного или муниципального имущества может быть создано унитарное казенное предприятие (казенное предприятие).
& Имущество государственного или муниципального унитарного предприятия находится в государственной или муниципальной собственности и принадлежит такому предприятию на праве хозяйственного ведения или оперативного управления.
\par Права унитарного предприятия на закрепленное за ним имущество определяются в соответствии с настоящим Кодексом и законом о государственных и муниципальных унитарных предприятиях.
& Учредительным документом унитарного предприятия является его устав, утверждаемый уполномоченным государственным органом или органом местного самоуправления, если иное не предусмотрено законом.
\par Устав унитарного предприятия должен содержать сведения о его фирменном наименовании и месте его нахождения, предмете и целях его деятельности. Устав унитарного предприятия, не являющегося казенным, должен содержать также сведения о размере уставного фонда унитарного предприятия.
& Фирменное наименование унитарного предприятия должно содержать указание на собственника его имущества. Фирменное наименование казенного предприятия, кроме того, должно содержать указание на то, что такое предприятие является казенным.
& Органом унитарного предприятия является руководитель предприятия, который назначается уполномоченным собственником органом, если иное не предусмотрено законом, и ему подотчетен.
& Унитарное предприятие отвечает по своим обязательствам всем принадлежащим ему имуществом.
\par Унитарное предприятие не несет ответственность по обязательствам собственника его имущества.
\par Собственник имущества унитарного предприятия, за исключением собственника имущества казенного предприятия, не отвечает по обязательствам своего унитарного предприятия. Собственник имущества казенного предприятия несет субсидиарную ответственность по обязательствам такого предприятия при недостаточности его имущества.
& Правовое положение унитарных предприятий определяется настоящим Кодексом и законом о государственных и муниципальных унитарных предприятиях.
& Унитарное предприятие может быть реорганизовано в соответствии с законом о государственных и муниципальных унитарных предприятиях и законами о приватизации.
\eEasyList
\subsubsection{{\bf Статья 114.} Создание унитарного предприятия и его уставный фонд}
\beginEasyList
    & Унитарное предприятие создается от имени публично-правового образования (\ul{статья 125}) решением уполномоченного на то государственного органа или органа местного самоуправления.
    & Минимальный размер уставного фонда унитарного предприятия определяется законом о государственных и муниципальных унитарных предприятиях.
    & Порядок формирования уставного фонда унитарного предприятия устанавливается законом о государственных и муниципальных унитарных предприятиях.
    & Если по окончании финансового года стоимость чистых активов унитарного предприятия окажется меньше размера уставного фонда, орган, уполномоченный создавать такие предприятия, обязан произвести в установленном порядке уменьшение уставного фонда. Если стоимость чистых активов становится меньше размера, определенного законом, унитарное предприятие может быть ликвидировано по решению суда.
    & В случае принятия решения об уменьшении уставного фонда унитарное предприятие обязано уведомить об этом в письменной форме своих кредиторов.
    \par Кредитор унитарного предприятия вправе потребовать прекращения или досрочного исполнения обязательства, должником по которому является это предприятие, и возмещения убытков.
\eEasyList
\subsubsection{{\bf Статья 115.} Утратила силу с 1 сентября 2014 г.}
\subsection{{\bf § 5. Некоммерческие организации}}
\par Утратил силу с 1 сентября 2014 г.
\subsection{{\bf § 6. Некоммерческие корпоративные организации}}
\subsection{{\bf 1. Общие положения о некоммерческих корпоративных организациях}}
\subsubsection{{\bf Статья 123.1.} Основные положения о некоммерческих корпоративных организациях}
\beginEasyList
    & Некоммерческими корпоративными организациями признаются юридические лица, которые не преследуют извлечение прибыли в качестве основной цели своей деятельности и не распределяют полученную прибыль между участниками (\ul{пункт 1 статьи 50} и \ul{статья 65.1}), учредители (участники) которых приобретают право участия (членства) в них и формируют их высший орган в соответствии с \ul{пунктом 1 статьи 65.3} настоящего Кодекса.
    & Некоммерческие корпоративные организации создаются в организационно-правовых формах потребительских кооперативов, общественных организаций, ассоциаций (союзов), нотариальных палат, товариществ собственников недвижимости, казачьих обществ, внесенных в государственный реестр казачьих обществ в Российской Федерации, а также общин коренных малочисленных народов Российской Федерации (\ul{пункт 3 статьи 50}).
    & Некоммерческие корпоративные организации создаются по решению учредителей, принятому на их общем (учредительном) собрании, конференции, съезде и т.п. Указанные органы утверждают устав соответствующей некоммерческой корпоративной организации и образуют ее органы.
    & Некоммерческая корпоративная организация является собственником своего имущества.
    & Уставом некоммерческой корпоративной организации может быть предусмотрено, что решения о создании корпорацией других юридических лиц, а также решения об участии корпорации в других юридических лицах, о создании филиалов и об открытии представительств корпорации принимаются коллегиальным органом корпорации.
\eEasyList
\subsection{{\bf 2. Потребительский кооператив}}
\subsubsection{{\bf Статья 123.2.} Основные положения о потребительском кооперативе}
\beginEasyList
    & Потребительским кооперативом признается основанное на членстве добровольное объединение граждан или граждан и юридических лиц в целях удовлетворения их материальных и иных потребностей, осуществляемое путем объединения его членами имущественных паевых взносов. Общество взаимного страхования может быть основано на членстве юридических лиц.
    & Устав потребительского кооператива должен содержать сведения о наименовании и месте нахождения кооператива, предмете и целях его деятельности, условия о размере паевых взносов членов кооператива, составе и порядке внесения паевых взносов членами кооператива и об их ответственности за нарушение обязательства по внесению паевых взносов, о составе и компетенции органов кооператива и порядке принятия ими решений, в том числе по вопросам, решения по которым принимаются единогласно или квалифицированным большинством голосов, порядке покрытия членами кооператива понесенных им убытков.
    \par Наименование потребительского кооператива должно содержать указание на основную цель его деятельности, а также слово "кооператив". Наименование общества взаимного страхования должно содержать слова "потребительское общество".
    & Потребительский кооператив по решению своих членов может быть преобразован в общественную организацию, ассоциацию (союз), автономную некоммерческую организацию или фонд. Жилищный или жилищно-строительный кооператив по решению своих членов может быть преобразован только в товарищество собственников недвижимости. Общество взаимного страхования по решению своих членов может быть преобразовано только в хозяйственное общество - страховую организацию.
\eEasyList
\subsubsection{{\bf Статья 123.3.} Обязанность членов потребительского кооператива по внесению дополнительных взносов}
\beginEasyList
    & В течение трех месяцев после утверждения ежегодного баланса члены потребительского кооператива обязаны покрыть образовавшиеся убытки путем внесения дополнительных взносов. В случае невыполнения этой обязанности кооператив может быть ликвидирован в судебном порядке по требованию кредиторов.
    & Члены потребительского кооператива солидарно несут субсидиарную ответственность по его обязательствам в пределах невнесенной части дополнительного взноса каждого из членов кооператива.
\eEasyList
\subsection{{\bf 3. Общественные организации}}
\subsubsection{{\bf Статья 123.4.} Основные положения об общественных организациях}
\beginEasyList
    & Общественными организациями признаются добровольные объединения граждан, объединившихся в установленном законом порядке на основе общности их интересов для удовлетворения духовных или иных нематериальных потребностей, для представления и защиты общих интересов и достижения иных не противоречащих закону целей.
    & Общественная организация является собственником своего имущества. Ее участники (члены) не сохраняют имущественные права на переданное ими в собственность организации имущество, в том числе на членские взносы.
    \par Участники (члены) общественной организации не отвечают по обязательствам организации, в которой участвуют в качестве членов, а организация не отвечает по обязательствам своих членов.
    & Общественные организации могут объединяться в ассоциации (союзы) в порядке, установленном настоящим Кодексом.
    & Общественная организация по решению ее участников (членов) может быть преобразована в ассоциацию (союз), автономную некоммерческую организацию или фонд.
\eEasyList
\subsubsection{{\bf Статья 123.5.} Учредители и устав общественной организации}
\beginEasyList
    & Количество учредителей общественной организации не может быть менее трех.
    & Устав общественной организации должен содержать сведения о ее наименовании и месте нахождения, предмете и целях ее деятельности, а также условия о порядке вступления (принятия) в общественную организацию и выхода из нее, составе и компетенции ее органов и порядке принятия ими решений, в том числе по вопросам, решения по которым принимаются единогласно или квалифицированным большинством голосов, об имущественных правах и обязанностях участника (члена) организации и о порядке распределения имущества, оставшегося после ликвидации организации.
\eEasyList
\subsubsection{{\bf Статья 123.6.} Права и обязанности участника (члена) общественной организации}
\beginEasyList
    & Участник (член) общественной организации осуществляет корпоративные права, предусмотренные \ul{пунктом 1 статьи 65.2} настоящего Кодекса, в порядке, установленном уставом организации. Он также вправе на равных началах с другими участниками (членами) организации безвозмездно пользоваться оказываемыми ею услугами.
    & Участник (член) общественной организации наряду с обязанностями, предусмотренными для участников корпорации \ul{пунктом 4 статьи 65.2} настоящего Кодекса, также несет обязанность уплачивать предусмотренные ее уставом членские и иные имущественные взносы.
    \par Участник (член) общественной организации по своему усмотрению в любое время вправе выйти из организации, в которой он участвует.
    & Членство в общественной организации неотчуждаемо. Осуществление прав участника (члена) общественной организации не может быть передано другому лицу.
\eEasyList
\subsubsection{{\bf Статья 123.7} Особенности управления в общественной организации}
\beginEasyList
    & К исключительной компетенции высшего органа общественной организации наряду с вопросами, указанными в \ul{пункте 2 статьи 65.3} настоящего Кодекса, относится также принятие решений о размере и порядке уплаты ее участниками (членами) членских и иных имущественных взносов.
    & В общественной организации образуется единоличный исполнительный орган (председатель, президент и т.п.) и могут образовываться постоянно действующие коллегиальные исполнительные органы (совет, правление, президиум и т.п.).
    \par По решению общего собрания членов общественной организации полномочия ее органа могут быть досрочно прекращены в случаях грубого нарушения этим органом своих обязанностей, обнаружившейся неспособности к надлежащему ведению дел или при наличии иных серьезных оснований.
\eEasyList
\subsection{{\bf 3.1. Общественные движения}}
\subsubsection{{\bf Статья 123.7-1.} Общественные движения}
\beginEasyList
    & Общественным движением является состоящее из участников общественное объединение, преследующее социальные, политические и иные общественно полезные цели, поддерживаемые участниками общественного движения.
    & Положения настоящего Кодекса о некоммерческих организациях применяются к общественным движениям, если иное не предусмотрено Федеральным законом от 19 мая 1995 года N 82-ФЗ "Об общественных объединениях".
\eEasyList
\subsection{{\bf 4. Ассоциации и союзы}}
\subsubsection{{\bf Статья 123.8.} Основные положения об ассоциации (союзе)}
\beginEasyList
    & Ассоциацией (союзом) признается объединение юридических лиц и (или) граждан, основанное на добровольном или в установленных законом случаях на обязательном членстве и созданное для представления и защиты общих, в том числе профессиональных, интересов, для достижения общественно полезных целей, а также иных не противоречащих закону и имеющих некоммерческий характер целей.
    \par В организационно-правовой форме ассоциации (союза) создаются, в частности, объединения лиц, имеющие целями координацию их предпринимательской деятельности, представление и защиту общих имущественных интересов, профессиональные объединения граждан, не имеющие целью защиту трудовых прав и интересов своих членов, профессиональные объединения граждан, не связанные с их участием в трудовых отношениях (объединения оценщиков, лиц творческих профессий и другие), саморегулируемые организации и их объединения.
    & Ассоциации (союзы) могут иметь гражданские права и нести гражданские обязанности, соответствующие целям их создания и деятельности, предусмотренным уставами таких ассоциаций (союзов).
    & Ассоциация (союз) является собственником своего имущества. Ассоциация (союз) отвечает по своим обязательствам всем своим имуществом, если иное не предусмотрено законом в отношении ассоциаций (союзов) отдельных видов.
    \par Ассоциация (союз) не отвечает по обязательствам своих членов, если иное не предусмотрено законом.
    \par Члены ассоциации (союза) не отвечают по ее обязательствам, за исключением случаев, если законом или уставом ассоциации (союза) предусмотрена субсидиарная ответственность ее членов.
    & Ассоциация (союз) по решению своих членов может быть преобразована в общественную организацию, автономную некоммерческую организацию или фонд.
    & Особенности правового положения ассоциаций (союзов) отдельных видов могут быть установлены законами.
\eEasyList
\subsubsection{{\bf Статья 123.9.} Учредители ассоциации (союза) и устав ассоциации (союза)}
\beginEasyList
    & Число учредителей ассоциации (союза) не может быть менее двух. Законами, устанавливающими особенности правового положения ассоциаций (союзов) отдельных видов, могут быть установлены иные требования к минимальному числу учредителей таких ассоциаций (союзов).
    & Устав ассоциации (союза) должен содержать сведения о ее наименовании и месте нахождения, предмете и целях ее деятельности, условия о порядке вступления (принятия) членов в ассоциацию (союз) и выхода из нее, сведения о составе и компетенции органов ассоциации (союза) и порядке принятия ими решений, в том числе по вопросам, решения по которым принимаются единогласно или квалифицированным большинством голосов, об имущественных правах и обязанностях членов ассоциации (союза), о порядке распределения имущества, оставшегося после ликвидации ассоциации (союза).
\eEasyList
\subsubsection{{\bf Статья 123.10.} Особенности управления в ассоциации (союзе)}
\beginEasyList
    & К исключительной компетенции высшего органа ассоциации (союза) наряду с вопросами, указанными в \ul{пункте 2 статьи 65.3} настоящего Кодекса, относится также принятие решений о порядке определения размера и способа уплаты членских взносов, о дополнительных имущественных взносах членов ассоциации (союза) в ее имущество и о размере их субсидиарной ответственности по обязательствам ассоциации (союза), если такая ответственность предусмотрена законом или уставом.
    & В ассоциации (союзе) образуется единоличный исполнительный орган (председатель, президент и т.п.) и могут образовываться постоянно действующие коллегиальные исполнительные органы (совет, правление, президиум и т.п.).
    \par По решению высшего органа ассоциации (союза) полномочия органа ассоциации (союза) могут быть досрочно прекращены в случаях грубого нарушения этим органом своих обязанностей, обнаружившейся неспособности к надлежащему ведению дел или при наличии иных серьезных оснований.
\eEasyList
\subsubsection{{\bf Статья 123.11.} Права и обязанности члена ассоциации (союза)}
\beginEasyList
    & Член ассоциации (союза) осуществляет корпоративные права, предусмотренные \ul{пунктом 1 статьи 65.2} настоящего Кодекса, в порядке, установленном в соответствии с законом уставом ассоциации (союза). Он также вправе на равных началах с другими членами ассоциации (союза) безвозмездно, если иное не предусмотрено законом, пользоваться оказываемыми ею услугами.
    \par Член ассоциации (союза) вправе выйти из нее по своему усмотрению в любое время.
    & Члены ассоциации (союза) наряду с обязанностями, предусмотренными для участников корпорации \ul{пунктом 4 статьи 65.2} настоящего Кодекса, также обязаны уплачивать предусмотренные уставом членские взносы и по решению высшего органа ассоциации (союза) вносить дополнительные имущественные взносы в имущество ассоциации (союза).
    \par Член ассоциации (союза) может быть исключен из нее в случаях и в порядке, которые установлены в соответствии с законом уставом ассоциации (союза).
    & Членство в ассоциации (союзе) неотчуждаемо. Последствия прекращения членства в ассоциации (союзе) устанавливаются законом и (или) ее уставом.
\eEasyList
\subsection{{\bf 5. Товарищества собственников недвижимости}}
\subsubsection{{\bf Статья 123.12.} Основные положения о товариществе собственников недвижимости}
\beginEasyList
    & Товариществом собственников недвижимости признается добровольное объединение собственников недвижимого имущества (помещений в здании, в том числе в многоквартирном доме, или в нескольких зданиях, жилых домов, садовых домов, садовых или огородных земельных участков и т.п.), созданное ими для совместного владения, пользования и в установленных законом пределах распоряжения имуществом (вещами), в силу закона находящимся в их общей собственности или в общем пользовании, а также для достижения иных целей, предусмотренных законами.
    & Устав товарищества собственников недвижимости должен содержать сведения о его наименовании, включающем слова "товарищество собственников недвижимости", месте нахождения, предмете и целях его деятельности, составе и компетенции органов товарищества и порядке принятия ими решений, в том числе по вопросам, решения по которым принимаются единогласно или квалифицированным большинством голосов, а также иные сведения, предусмотренные законом.
    & Товарищество собственников недвижимости не отвечает по обязательствам своих членов. Члены товарищества собственников недвижимости не отвечают по его обязательствам.
    & Товарищество собственников недвижимости по решению своих членов может быть преобразовано в потребительский кооператив.
\eEasyList
\subsubsection{{\bf Статья 123.13.} Имущество товарищества собственников недвижимости}
\beginEasyList
    & Товарищество собственников недвижимости является собственником своего имущества.
    & Общее имущество в многоквартирном доме принадлежит членам товарищества собственников недвижимости на праве общей долевой собственности, если иное не предусмотрено законом. Состав такого имущества и порядок определения долей в праве общей собственности на него устанавливаются законом.
    && Имущество общего пользования в садоводческом или огородническом некоммерческом товариществе принадлежит на праве общей долевой собственности лицам, являющимся собственниками земельных участков, расположенных в границах территории ведения гражданами садоводства или огородничества для собственных нужд, если иное не предусмотрено законом.
    & Доля в праве общей собственности на общее имущество в многоквартирном доме собственника помещения в этом доме, доля в праве общей собственности на имущество общего пользования, расположенное в границах территории ведения гражданами садоводства или огородничества для собственных нужд, собственника садового или огородного земельного участка следуют судьбе права собственности на указанные помещение или земельный участок.
\eEasyList
\subsubsection{{\bf Статья 123.14.} Особенности управления в товариществе собственников недвижимости}
\beginEasyList
    & К исключительной компетенции высшего органа товарищества собственников недвижимости наряду с вопросами, указанными в \ul{пункте 2 статьи 65.3} настоящего Кодекса, относится также принятие решений об установлении размера обязательных платежей и взносов членов товарищества.
    & В товариществе собственников недвижимости создаются единоличный исполнительный орган (председатель) и постоянно действующий коллегиальный исполнительный орган (правление).
    \par По решению высшего органа товарищества собственников недвижимости (\ul{пункт 1 статьи 65.3}) полномочия постоянно действующих органов товарищества могут быть досрочно прекращены в случаях грубого нарушения ими своих обязанностей, обнаружившейся неспособности к надлежащему ведению дел или при наличии иных серьезных оснований.
\eEasyList
\subsection{{\bf 6. Казачьи общества, внесенные в государственный реестр казачьих обществ в Российской Федерации}}
\subsubsection{{\bf Статья 123.15.} Казачье общество, внесенное в государственный реестр казачьих обществ в Российской Федерации}
\beginEasyList
    & Казачьими обществами признаются внесенные в государственный реестр казачьих обществ в Российской Федерации объединения граждан, созданные в целях сохранения традиционных образа жизни, хозяйствования и культуры российского казачества, а также в иных целях, предусмотренных Федеральным законом от 5 декабря 2005 года N 154-ФЗ "О государственной службе российского казачества", добровольно принявших на себя в порядке, установленном законом, обязательства по несению государственной или иной службы.
    & Казачье общество по решению его членов может быть преобразовано в ассоциацию (союз) или автономную некоммерческую организацию.
    & Положения настоящего Кодекса о некоммерческих организациях применяются к казачьим обществам, внесенным в государственный реестр казачьих обществ в Российской Федерации, если иное не установлено Федеральным законом от 5 декабря 2005 года N 154-ФЗ "О государственной службе российского казачества".
\eEasyList
\subsection{{\bf 7. Общины коренных малочисленных народов Российской Федерации}}
\subsubsection{{\bf Статья 123.16.} Община коренных малочисленных народов Российской Федерации}
\beginEasyList
    & Общинами коренных малочисленных народов Российской Федерации признаются добровольные объединения граждан, относящихся к коренным малочисленным народам Российской Федерации и объединившихся по кровнородственному и (или) территориально-соседскому признаку в целях защиты исконной среды обитания, сохранения и развития традиционных образа жизни, хозяйствования, промыслов и культуры.
    & Члены общины коренных малочисленных народов Российской Федерации имеют право на получение части ее имущества или компенсации стоимости такой части при выходе из общины или ее ликвидации в порядке, установленном законом.
    & Община коренных малочисленных народов Российской Федерации по решению ее членов может быть преобразована в ассоциацию (союз) или автономную некоммерческую организацию.
    & Положения настоящего Кодекса о некоммерческих организациях применяются к общинам коренных малочисленных народов Российской Федерации, если иное не установлено законом.
\eEasyList
\subsection{{\bf 8. Адвокатские палаты}}
\subsubsection{{\bf Статья 123.16-1.} Адвокатские палаты}
\beginEasyList
    & Адвокатскими палатами признаются некоммерческие организации, основанные на обязательном членстве и созданные в виде адвокатской палаты субъекта Российской Федерации или Федеральной палаты адвокатов Российской Федерации для реализации целей, предусмотренных законодательством об адвокатской деятельности и адвокатуре.
    & Адвокатская палата субъекта Российской Федерации является некоммерческой организацией, основанной на обязательном членстве всех адвокатов одного субъекта Российской Федерации.
    & Федеральная палата адвокатов Российской Федерации является некоммерческой организацией, объединяющей адвокатские палаты субъектов Российской Федерации на основе обязательного членства.
    & Особенности создания, правового положения и деятельности адвокатских палат субъектов Российской Федерации и Федеральной палаты адвокатов Российской Федерации определяются законодательством об адвокатской деятельности и адвокатуре.
\eEasyList
\subsection{{\bf 9. Адвокатские образования, являющиеся юридическими лицами}}
\subsubsection{{\bf Статья 123.16-2.} Адвокатские образования, являющиеся юридическими лицами}
\beginEasyList
    & Адвокатскими образованиями, являющимися юридическими лицами, признаются некоммерческие организации, созданные в соответствии с законодательством об адвокатской деятельности и адвокатуре в целях осуществления адвокатами адвокатской деятельности.
    & Адвокатские образования, являющиеся юридическими лицами, создаются в виде коллегии адвокатов, адвокатского бюро или юридической консультации.
    & Особенности создания, правового положения и деятельности адвокатских образований, являющихся юридическими лицами, определяются законодательством об адвокатской деятельности и адвокатуре.
\eEasyList
\subsection{{\bf 10. Нотариальные палаты}}
\subsubsection{{\bf Статья 123.16-3.} Нотариальные палаты}
\beginEasyList
    & Нотариальными палатами признаются некоммерческие организации, которые представляют собой профессиональные объединения, основанные на обязательном членстве, и созданы в виде нотариальной палаты субъекта Российской Федерации или Федеральной нотариальной палаты для реализации целей, предусмотренных законодательством о нотариате.
    & Нотариальная палата субъекта Российской Федерации является некоммерческой организацией, представляющей собой профессиональное объединение, основанное на обязательном членстве нотариусов, занимающихся частной практикой.
    & Федеральная нотариальная палата является некоммерческой организацией, представляющей собой профессиональное объединение нотариальных палат субъектов Российской Федерации, основанное на их обязательном членстве.
    & Особенности создания, правового положения и деятельности нотариальных палат субъектов Российской Федерации и Федеральной нотариальной палаты определяются законодательством о нотариате.
\eEasyList
\subsection{{\bf § 7. Некоммерческие унитарные организации}}
\subsection{{\bf 1. Фонды}}
\subsubsection{{\bf Статья 123.17.} Основные положения о фонде}
\beginEasyList
    & Фондом в целях настоящего Кодекса признается унитарная некоммерческая организация, не имеющая членства, учрежденная гражданами и (или) юридическими лицами на основе добровольных имущественных взносов и преследующая благотворительные, культурные, образовательные или иные социальные, общественно полезные цели.
    & Устав фонда должен содержать сведения о наименовании фонда, включающем слово "фонд", месте его нахождения, предмете и целях его деятельности, об органах фонда, в том числе о высшем коллегиальном органе и о попечительском совете, осуществляющем надзор за деятельностью фонда, порядке назначения должностных лиц фонда и их освобождения от исполнения обязанностей, судьбе имущества фонда в случае его ликвидации.
    & Реорганизация фонда не допускается, за исключением случаев, предусмотренных \ul{пунктом 4} настоящей статьи, а также законами, устанавливающими основания и порядок реорганизации фонда.
    & Правовое положение негосударственных пенсионных фондов, включая случаи и порядок их возможной реорганизации, определяется настоящей статьей и \ul{статьями 123.18 - 123.20} настоящего Кодекса с учетом особенностей, предусмотренных законом о негосударственных пенсионных фондах.
    & Правовое положение наследственных фондов определяется настоящей статьей и \ul{статьями 123.18 - 123.20} настоящего Кодекса с учетом особенностей, предусмотренных \ul{статьями 123.20-1 - 123.20-3} настоящего Кодекса.
\eEasyList
\subsubsection{{\bf Статья 123.18.} Имущество фонда}
\beginEasyList
    & Имущество, переданное фонду его учредителями (учредителем), является собственностью фонда. Учредители фонда не имеют имущественных прав в отношении созданного ими фонда и не отвечают по его обязательствам, а фонд не отвечает по обязательствам своих учредителей.
    & Фонд использует имущество для целей, определенных в его уставе.
    \par Ежегодно фонд обязан опубликовывать отчеты об использовании своего имущества.
\eEasyList
\subsubsection{{\bf Статья 123.19.} Управление фондом}
\beginEasyList
    & Если иное не предусмотрено законом или иным правовым актом, к исключительной компетенции высшего коллегиального органа фонда относятся:
    \par определение приоритетных направлений деятельности фонда, принципов образования и использования его имущества;
    \par образование других органов фонда и досрочное прекращение их полномочий;
    \par утверждение годовых отчетов и годовой бухгалтерской (финансовой) отчетности фонда;
    \par принятие решений о создании фондом хозяйственных обществ и (или) об участии в них фонда, за исключением случаев, когда уставом фонда принятие решений по указанным вопросам отнесено к компетенции иных коллегиальных органов фонда;
    \par принятие решений о создании филиалов и (или) об открытии представительств фонда;
    \par изменение устава фонда, если эта возможность предусмотрена уставом;
    \par одобрение совершаемых фондом сделок в случаях, предусмотренных законом.
    \par Законом или уставом фонда к исключительной компетенции высшего коллегиального органа фонда может быть отнесено принятие решений по иным вопросам.
    & Высший коллегиальный орган фонда избирает единоличный исполнительный орган фонда (председателя, генерального директора и т.д.) и может назначить коллегиальный исполнительный орган фонда (правление) или иной коллегиальный орган фонда, если законом или другим правовым актом указанные полномочия не отнесены к компетенции учредителя фонда.
    \par К компетенции единоличного исполнительного и (или) коллегиальных органов фонда относится решение вопросов, не входящих в исключительную компетенцию высшего коллегиального органа фонда.
    & Лица, уполномоченные выступать от имени фонда, обязаны по требованию членов его высшего коллегиального органа, действующих в интересах фонда, в соответствии со \ul{статьей 53.1} настоящего Кодекса возместить убытки, причиненные ими фонду.
    & Попечительский совет фонда является органом фонда и осуществляет надзор за деятельностью фонда, принятием другими органами фонда решений и обеспечением их исполнения, использованием средств фонда, соблюдением фондом законодательства. Попечительский совет фонда осуществляет свою деятельность на общественных началах.
\eEasyList
\subsubsection{{\bf Статья 123.20.} Изменение устава и ликвидация фонда}
\beginEasyList
    & Устав фонда может быть изменен высшим коллегиальным органом фонда, если уставом не предусмотрена возможность его изменения по решению учредителя.
    \par Устав фонда может быть изменен решением суда, принятым по заявлению органов фонда или государственного органа, уполномоченного осуществлять надзор за деятельностью фонда, в случае, если сохранение устава фонда в неизменном виде влечет последствия, которые было невозможно предвидеть при учреждении фонда, а высший коллегиальный орган фонда или учредитель фонда не изменяет его устав.
    & Фонд может быть ликвидирован только на основании решения суда, принятого по заявлению заинтересованных лиц, в случае, если:
    && имущества фонда недостаточно для осуществления его целей и вероятность получения необходимого имущества нереальна;
    && цели фонда не могут быть достигнуты, а необходимые изменения целей фонда не могут быть произведены;
    && фонд в своей деятельности уклоняется от целей, предусмотренных уставом;
    && в других случаях, предусмотренных законом.
    & В случае ликвидации фонда его имущество, оставшееся после удовлетворения требований кредиторов, направляется на цели, указанные в уставе фонда, за исключением случаев, если законом предусмотрен возврат такого имущества учредителям фонда.
\eEasyList
\subsubsection{{\bf Статья 123.20-1.} Создание наследственного фонда, условия управления им и его ликвидация}
\beginEasyList
    & Наследственным фондом признается создаваемый в порядке, предусмотренном настоящим Кодексом, во исполнение завещания гражданина и на основе его имущества фонд, осуществляющий деятельность по управлению полученным в порядке наследования имуществом этого гражданина бессрочно или в течение определенного срока в соответствии с условиями управления наследственным фондом.
    & Наследственный фонд подлежит созданию после смерти гражданина, который предусмотрел в своем завещании создание наследственного фонда, по заявлению, направляемому в уполномоченный государственный орган нотариусом, ведущим наследственное дело, с приложением к заявлению составленного при жизни указанного гражданина его решения об учреждении наследственного фонда и утвержденного этим гражданином устава фонда и после его создания призывается к наследованию по завещанию в порядке, предусмотренном разделом V настоящего Кодекса.
    \par Завещание, условия которого предусматривают создание наследственного фонда, должно включать в себя решение завещателя об учреждении наследственного фонда, устав фонда, а также условия управления наследственным фондом. Такое завещание подлежит нотариальному удостоверению.
    \par Наследственный фонд может быть создан на основании решения суда по требованию душеприказчика или выгодоприобретателя наследственного фонда в случае неисполнения нотариусом обязанности по созданию наследственного фонда.
    \par Нотариус, ведущий наследственное дело, обязан направить в уполномоченный государственный орган заявление о государственной регистрации наследственного фонда не позднее трех рабочих дней со дня открытия наследственного дела после смерти гражданина, который предусмотрел в своем завещании создание наследственного фонда. Наследственный фонд не подлежит регистрации по истечении одного года со дня открытия наследства.
    \par Действия нотариуса по созданию наследственного фонда могут быть оспорены выгодоприобретателями наследственного фонда, душеприказчиком или наследниками, если нотариусом нарушены содержащиеся в завещании или решении об учреждении наследственного фонда распоряжения наследодателя относительно создания наследственного фонда и условий управления им.
    & Имущество наследственного фонда формируется при создании фонда, в ходе осуществления им своей деятельности, а также за счет доходов от управления имуществом наследственного фонда. Безвозмездная передача иными лицами имущества в наследственный фонд не допускается.
    \par При создании наследственного фонда и принятии им наследства нотариус обязан выдать фонду свидетельство о праве на наследство в срок, указанный в решении об учреждении наследственного фонда, но не позднее срока, предусмотренного \ul{статьей 1154} настоящего Кодекса. В случае неисполнения нотариусом указанных обязанностей наследственный фонд вправе обжаловать бездействие нотариуса.
    & Условия управления наследственным фондом должны включать в себя положения о передаче определенным третьим лицам (далее также - выгодоприобретатели фонда) или отдельным категориям лиц из неопределенного круга лиц (далее - отдельные категории лиц) всего имущества наследственного фонда или его части, в том числе при наступлении обстоятельств, относительно которых неизвестно, наступят они или нет.
    \par Условиями управления наследственным фондом может быть предусмотрено, что выгодоприобретатели фонда или отдельные категории лиц, которым подлежит передаче имущество фонда, определяются органами фонда в соответствии с условиями управления фондом.
    \par Порядок передачи выгодоприобретателям наследственного фонда или отдельным категориям лиц всего имущества фонда или его части, в том числе доходов от деятельности фонда, должен быть определен условиями управления фондом путем указания на вид и размер передаваемого имущества или порядок определения вида и размера имущества, в том числе имущественного права (например, права пользования имуществом, права на оплату работ, услуг, оказываемых третьими лицами выгодоприобретателям или отдельным категориям лиц), срок или периодичность передачи имущества, а также на обстоятельства, при наступлении которых осуществляется такая передача.
    & Устав наследственного фонда и условия управления наследственным фондом не могут быть изменены после создания наследственного фонда, за исключением изменения на основании решения суда по требованию любого органа фонда в случаях, если управление наследственным фондом на прежних условиях стало невозможно по обстоятельствам, возникновение которых при создании фонда нельзя было предполагать, а также в случае, если будет установлено, что выгодоприобретатель является недостойным наследником (\ul{статья 1117}), если только это обстоятельство не было известно в момент создания наследственного фонда.
    & Условия управления наследственным фондом до направления нотариусом указанного в \ul{абзаце четвертом пункта 2} настоящей статьи заявления доводятся им до сведения лиц, входящих в состав органов фонда, и могут быть раскрыты только выгодоприобретателям, а также в предусмотренных законом случаях органам государственной власти и органам местного самоуправления.
    & Ликвидация наследственного фонда осуществляется по решению суда по основаниям, предусмотренным \ul{подпунктами 1 - 4 пункта 3 статьи 61} настоящего Кодекса, а также в связи с наступлением срока, до истечения которого создавался фонд, наступлением указанных в условиях управления наследственным фондом обстоятельств или невозможностью формирования органов фонда (\ul{пункт 4 статьи 123.20-2}).
    \par Оставшееся после ликвидации наследственного фонда имущество подлежит передаче выгодоприобретателям соразмерно объему их прав на получение имущества или дохода от деятельности фонда, если условиями управления наследственным фондом не предусмотрены иные правила распределения оставшегося имущества, в том числе его передача лицам, не являющимся выгодоприобретателями. При отсутствии возможности определить лиц, которым подлежит передаче оставшееся после ликвидации наследственного фонда имущество, такое имущество в соответствии с решением суда подлежит передаче в собственность Российской Федерации.
    & Наименование наследственного фонда должно включать слова "наследственный фонд".
\eEasyList
\subsubsection{{\bf Статья 123.20-2.} Управление наследственным фондом}
\beginEasyList
    & В качестве единоличного исполнительного органа наследственного фонда или члена коллегиального органа наследственного фонда может выступать физическое или юридическое лицо. Выгодоприобретатель наследственного фонда не может выступать в качестве единоличного исполнительного органа фонда или члена коллегиального исполнительного органа наследственного фонда.
    & В случаях, предусмотренных уставом наследственного фонда, в нем создаются высший коллегиальный орган и попечительский совет. В состав высшего коллегиального органа наследственного фонда могут входить выгодоприобретатели фонда.
    & До направления нотариусом указанного в \ul{абзаце четвертом пункта 2 статьи 123.20-1} заявления о государственной регистрации наследственного фонда в уполномоченный государственный орган нотариус предлагает лицам, указанным в решении об учреждении фонда, или лицам, которые могут быть определены в порядке, установленном решением об учреждении фонда, войти в состав органов фонда. При согласии указанных лиц войти в состав органов фонда нотариус направляет сведения о них в уполномоченный государственный орган.
    \par В случае отказа лица, указанного в решении об учреждении фонда, войти в состав органов фонда и невозможности сформировать органы фонда в соответствии с решением об учреждении фонда нотариус не вправе направлять в уполномоченный государственный орган заявление о создании наследственного фонда.
    & Замена членов коллегиальных органов наследственного фонда и лица, осуществляющего полномочия единоличного исполнительного органа наследственного фонда, осуществляется в порядке, предусмотренном уставом фонда. Уставом фонда может быть предусмотрен порядок определения членов коллегиальных органов фонда и лица, осуществляющего полномочия единоличного исполнительного органа наследственного фонда, в случае их выбытия, в том числе предусмотрено подназначение указанных лиц из определенного списка.
    \par Если в течение года со дня возникновения необходимости формирования органов наследственного фонда (отсутствие кворума в коллегиальных органах фонда, отсутствие единоличного исполнительного органа) такие органы не будут сформированы, фонд подлежит ликвидации (\ul{пункт 7 статьи 123.20-1}) по требованию выгодоприобретателя или уполномоченного государственного органа. До истечения указанного срока единоличный исполнительный орган наследственного фонда (при наличии такого органа) продолжает осуществлять деятельность наследственного фонда в соответствии с условиями управления наследственным фондом.
    & Условиями управления наследственным фондом могут быть предусмотрены порядок выплаты и размер вознаграждения лицу, осуществляющему полномочия единоличного исполнительного органа фонда, членам попечительского совета фонда или членам иных органов фонда за исполнение своих обязанностей.
    & Уставом фонда может быть предусмотрена необходимость получения согласия высшего коллегиального органа фонда или иного органа фонда на совершение наследственным фондом указанных в уставе сделок.
    & Аудит деятельности наследственного фонда проводится по основаниям, предусмотренным условиями управления наследственным фондом, а также по требованию выгодоприобретателя в порядке, предусмотренном \ul{пунктом 5 статьи 123.20-3} настоящего Кодекса.
    & Единоличный исполнительный орган наследственного фонда обязан хранить устав фонда и внесенные в него изменения и дополнения, которые зарегистрированы в установленном порядке, решение об учреждении фонда, документы, подтверждающие права фонда на его имущество, документ, содержащий условия управления наследственным фондом, годовые отчеты, документы бухгалтерского учета, документы бухгалтерской (финансовой) отчетности, протоколы собраний коллегиальных органов фонда, отчеты оценщиков, заключения ревизионной комиссии (ревизора) фонда, аудитора фонда, государственных и муниципальных органов финансового контроля, судебные акты по спорам, связанным с управлением фондом, иные документы, предусмотренные настоящим Кодексом, уставом фонда и условиями управления наследственным фондом.
    \par Уставом фонда может быть предусмотрено хранение документов, указанных в \ul{абзаце первом} настоящего пункта, у нотариуса по правилам, предусмотренным законодательством о нотариате.
    & Отчет об использовании имущества наследственного фонда не подлежит опубликованию, за исключением случаев, предусмотренных условиями управления наследственным фондом.
\eEasyList
\subsubsection{{\bf Статья 123.20-3.} Права выгодоприобретателей наследственного фонда}
\beginEasyList
    & Выгодоприобретатель наследственного фонда имеет право на получение в соответствии с условиями управления наследственным фондом всего или части имущества фонда, а также иные права, предусмотренные настоящей статьей. Права выгодоприобретателя наследственного фонда неотчуждаемы, на них не может быть обращено взыскание по обязательствам выгодоприобретателя. Сделки, совершенные с нарушением этих правил, являются ничтожными.
    & Выгодоприобретателями наследственного фонда могут быть любые участники регулируемых гражданским законодательством отношений, за исключением коммерческих организаций.
    & Права гражданина-выгодоприобретателя наследственного фонда не переходят по наследству. Права выгодоприобретателя - юридического лица прекращаются в случае его реорганизации, за исключением случая преобразования, если условиями управления наследственным фондом не предусмотрено прекращение прав такого выгодоприобретателя при его преобразовании.
    \par После смерти гражданина-выгодоприобретателя или ликвидации выгодоприобретателя - юридического лица, а также в случае заявленного наследственному фонду в нотариальной форме отказа выгодоприобретателя от права на получение имущества новые выгодоприобретатели определяются в соответствии с условиями управления наследственным фондом, в частности, они могут быть определены путем подназначения.
    & В случаях, предусмотренных уставом наследственного фонда, выгодоприобретатель вправе запрашивать и получать у наследственного фонда информацию о деятельности фонда.
    & Выгодоприобретатель наследственного фонда вправе потребовать проведения аудита деятельности фонда выбранным им аудитором. В случае проведения такого аудита оплата услуг аудитора осуществляется за счет выгодоприобретателя наследственного фонда, по требованию которого он проводится. Расходы выгодоприобретателя фонда на оплату услуг аудитора могут быть ему возмещены по решению попечительского совета за счет средств фонда.
    & В случае нарушения условий управления наследственным фондом, повлекшего возникновение у выгодоприобретателя убытков, последний вправе потребовать их возмещения, если это право предусмотрено уставом фонда.
    & Выгодоприобретатель не отвечает по обязательствам наследственного фонда, а фонд не отвечает по обязательствам выгодоприобретателя.
\eEasyList
\subsection{{\bf 2. Учреждения}}
\subsubsection{{\bf Статья 123.21.} Основные положения об учреждениях}
\beginEasyList 
    & Учреждением признается унитарная некоммерческая организация, созданная собственником для осуществления управленческих, социально-культурных или иных функций некоммерческого характера.
    \par Учредитель является собственником имущества созданного им учреждения. На имущество, закрепленное собственником за учреждением и приобретенное учреждением по иным основаниям, оно приобретает право оперативного управления в соответствии с настоящим Кодексом.
    & Учреждение может быть создано гражданином или юридическим лицом (частное учреждение) либо соответственно Российской Федерацией, субъектом Российской Федерации, муниципальным образованием (государственное учреждение, муниципальное учреждение).
    \par При создании учреждения не допускается соучредительство нескольких лиц.
    & Учреждение отвечает по своим обязательствам находящимися в его распоряжении денежными средствами, а в случаях, установленных законом, также иным имуществом. При недостаточности указанных денежных средств или имущества субсидиарную ответственность по обязательствам учреждения в случаях, предусмотренных пунктами \ul{4 - 6 статьи 123.22} и \ul{пунктом 2 статьи 123.23} настоящего Кодекса, несет собственник соответствующего имущества.
    & Учредитель учреждения назначает его руководителя, являющегося органом учреждения. В случаях и в порядке, которые предусмотрены законом, руководитель государственного или муниципального учреждения может избираться его коллегиальным органом и утверждаться его учредителем.
    \par По решению учредителя в учреждении могут быть созданы коллегиальные органы, подотчетные учредителю. Компетенция коллегиальных органов учреждения, порядок их создания и принятия ими решений определяются законом и уставом учреждения.
\eEasyList 
\subsubsection{{\bf Статья 123.22.} Государственное учреждение и муниципальное учреждение}
\beginEasyList
    & Государственное или муниципальное учреждение может быть казенным, бюджетным или автономным учреждением.
    & Порядок финансового обеспечения деятельности государственных и муниципальных учреждений определяется законом.
    & Государственные и муниципальные учреждения не отвечают по обязательствам собственников своего имущества.
    & Казенное учреждение отвечает по своим обязательствам находящимися в его распоряжении денежными средствами. При недостаточности денежных средств субсидиарную ответственность по обязательствам казенного учреждения несет собственник его имущества.
    & Бюджетное учреждение отвечает по своим обязательствам всем находящимся у него на праве оперативного управления имуществом, в том числе приобретенным за счет доходов, полученных от приносящей доход деятельности, за исключением особо ценного движимого имущества, закрепленного за бюджетным учреждением собственником этого имущества или приобретенного бюджетным учреждением за счет средств, выделенных собственником его имущества, а также недвижимого имущества независимо от того, по каким основаниям оно поступило в оперативное управление бюджетного учреждения и за счет каких средств оно приобретено.
    \par По обязательствам бюджетного учреждения, связанным с причинением вреда гражданам, при недостаточности имущества учреждения, на которое в соответствии с \ul{абзацем первым} настоящего пункта может быть обращено взыскание, субсидиарную ответственность несет собственник имущества бюджетного учреждения.
    & Автономное учреждение отвечает по своим обязательствам всем находящимся у него на праве оперативного управления имуществом, за исключением недвижимого имущества и особо ценного движимого имущества, закрепленных за автономным учреждением собственником этого имущества или приобретенных автономным учреждением за счет средств, выделенных собственником его имущества.
    \par По обязательствам автономного учреждения, связанным с причинением вреда гражданам, при недостаточности имущества учреждения, на которое в соответствии с \ul{абзацем первым} настоящего пункта может быть обращено взыскание, субсидиарную ответственность несет собственник имущества автономного учреждения.
    \par Ежегодно автономное учреждение обязано опубликовывать отчеты о своей деятельности и об использовании закрепленного за ним имущества.
    & Государственное или муниципальное учреждение может быть преобразовано в некоммерческую организацию иных организационно-правовых форм в случаях, предусмотренных законом.
    & Особенности правового положения государственных и муниципальных учреждений отдельных типов определяются законом.
\eEasyList
\subsubsection{{\bf Статья 123.23.} Частное учреждение}
\beginEasyList
    & Частное учреждение полностью или частично финансируется собственником его имущества.
    & Частное учреждение отвечает по своим обязательствам находящимися в его распоряжении денежными средствами. При недостаточности указанных денежных средств субсидиарную ответственность по обязательствам частного учреждения несет собственник его имущества.
    & Частное учреждение может быть преобразовано учредителем в автономную некоммерческую организацию или фонд.
\eEasyList
\subsection{{\bf 3. Автономные некоммерческие организации}}
\subsubsection{{\bf Статья 123.24.} Основные положения об автономной некоммерческой организации}
\beginEasyList
    & Автономной некоммерческой организацией признается унитарная некоммерческая организация, не имеющая членства и созданная на основе имущественных взносов граждан и (или) юридических лиц в целях предоставления услуг в сферах образования, здравоохранения, культуры, науки и иных сферах некоммерческой деятельности.
    \par Автономная некоммерческая организация может быть создана одним лицом (может иметь одного учредителя).
    & Устав автономной некоммерческой организации должен содержать сведения о ее наименовании, включающем слова "автономная некоммерческая организация", месте нахождения, предмете и целях ее деятельности, составе, порядке образования и компетенции органов автономной некоммерческой организации, а также иные предусмотренные законом сведения.
    & Имущество, переданное автономной некоммерческой организации ее учредителями, является собственностью автономной некоммерческой организации. Учредители автономной некоммерческой организации не сохраняют права на имущество, переданное ими в собственность этой организации.
    \par Учредители не отвечают по обязательствам созданной ими автономной некоммерческой организации, а она не отвечает по обязательствам своих учредителей.
    & Учредители автономной некоммерческой организации могут пользоваться ее услугами только на равных условиях с другими лицами.
    & Автономная некоммерческая организация вправе заниматься предпринимательской деятельностью, необходимой для достижения целей, ради которых она создана, и соответствующей этим целям, создавая для осуществления предпринимательской деятельности хозяйственные общества или участвуя в них.
    & Лицо может по своему усмотрению выйти из состава учредителей автономной некоммерческой организации.
    \par По решению учредителей автономной некоммерческой организации, принятому единогласно, в состав ее учредителей могут быть приняты новые лица.
    & Автономная некоммерческая организация по решению своих учредителей может быть преобразована в фонд.
    & В части, не урегулированной настоящим Кодексом, правовое положение автономных некоммерческих организаций, а также права и обязанности их учредителей устанавливаются законом.
\eEasyList
\subsubsection{{\bf Статья 123.25.} Управление автономной некоммерческой организацией}
\beginEasyList 
    & Управление деятельностью автономной некоммерческой организации осуществляют ее учредители в порядке, установленном ее уставом, утвержденным ее учредителями.
    & По решению учредителей (учредителя) автономной некоммерческой организации в ней может быть создан постоянно действующий коллегиальный орган (органы), компетенция которого устанавливается уставом автономной некоммерческой организации.
    & Учредители (учредитель) автономной некоммерческой организации назначают единоличный исполнительный орган автономной некоммерческой организации (председателя, генерального директора и т.п.). Единоличным исполнительным органом автономной некоммерческой организации может быть назначен один из ее учредителей-граждан.
\eEasyList
\subsection{{\bf 4. Религиозные организации}}
\subsubsection{{\bf Статья 123.26.} Основные положения о религиозных организациях}
\beginEasyList
    & Религиозной организацией признается добровольное объединение постоянно и на законных основаниях проживающих на территории Российской Федерации граждан Российской Федерации или иных лиц, образованное ими в целях совместного исповедания и распространения веры и зарегистрированное в установленном законом порядке в качестве юридического лица (местная религиозная организация), объединение этих организаций (централизованная религиозная организация), а также созданная указанным объединением в соответствии с законом о свободе совести и о религиозных объединениях в целях совместного исповедания и распространения веры организация и (или) созданный указанным объединением руководящий или координирующий орган.
    & Гражданско-правовое положение религиозных организаций определяется настоящим Кодексом и законом о свободе совести и о религиозных объединениях. Положения настоящего Кодекса применяются к религиозным организациям, если иное не установлено законом о свободе совести и о религиозных объединениях и другими законами.
    \par Религиозные организации действуют в соответствии со своими уставами и внутренними установлениями, не противоречащими закону.
    \par Порядок образования органов религиозной организации и их компетенция, порядок принятия решений этими органами, а также отношения между религиозной организацией и лицами, входящими в состав ее органов, определяются в соответствии с \ul{законом о свободе совести и о религиозных объединениях} уставом и внутренними установлениями религиозной организации.
    & Религиозная организация не может быть преобразована в юридическое лицо другой организационно-правовой формы.
\eEasyList
\subsubsection{{\bf Статья 123.27.} Учредители и устав религиозной организации}
\beginEasyList
    & Местная религиозная организация создается в соответствии с законом о свободе совести и о религиозных объединениях не менее чем десятью гражданами-учредителями, централизованная религиозная организация - не менее чем тремя местными религиозными организациями или другой централизованной религиозной организацией.
    & Учредительным документом религиозной организации является устав, утвержденный ее учредителями или централизованной религиозной организацией.
    \par Устав религиозной организации должен содержать сведения о ее виде, наименовании и месте нахождения, предмете и целях ее деятельности, составе, компетенции ее органов и порядке принятия ими решений, об источниках образования ее имущества, о направлениях его использования и порядке распределения имущества, остающегося после ее ликвидации, а также иные сведения, предусмотренные \ul{законом о свободе совести и о религиозных объединениях}.
    & Учредитель (учредители) религиозной организации может выполнять функции органа управления или членов коллегиального органа управления данной религиозной организации в порядке, установленном в соответствии с \ul{законом о свободе совести и о религиозных объединениях} уставом религиозной организации и внутренними установлениями.
\eEasyList
\subsubsection{{\bf Статья 123.28.} Имущество религиозной организации}
\beginEasyList
    & Религиозные организации являются собственниками принадлежащего им имущества, в том числе имущества, приобретенного или созданного ими за счет собственных средств, а также пожертвованного религиозным организациям или приобретенного ими по иным предусмотренным законом основаниям.
    & На принадлежащее религиозным организациям имущество богослужебного назначения не может быть обращено взыскание по требованиям их кредиторов. Перечень такого имущества определяется в порядке, установленном законом о свободе совести и о религиозных объединениях.
    & Учредители религиозной организации не сохраняют имущественные права на имущество, переданное ими этой организации в собственность.
    & Учредители религиозных организаций не отвечают по обязательствам этих организаций, а эти организации не отвечают по обязательствам своих учредителей.
\eEasyList

\end{document}
