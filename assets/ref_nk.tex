\documentclass{report}
    \usepackage[a4paper,margin=2cm]{geometry}
    \usepackage[utf8]{inputenc}
    \usepackage[T2A]{fontenc}
    \usepackage[russian]{babel}
    \usepackage{indentfirst}
    \setlength{\parindent}{0pt}
    \usepackage{textcomp}
    \usepackage{soulutf8}
    \setuloverlap{0pt}
    \usepackage[ampersand]{easylist}
    \newcommand{\beginEasyList}{
        \begin{easylist}[enumerate]
            \ListProperties(Numbers=a,Hide2=1,Hide3=1,Style*=,Mark=.,FinalMark={)},FinalMark1=.)
    }
    \newcommand{\eEasyList}{\end{easylist}}
    \setcounter{secnumdepth}{-2}
\begin{document}
\begin{titlepage}
    \center
    \Large

    Федеральное государственное автономное образовательное\\
    учреждение высшего профессионального образования\\
    «Уральский федеральный университет\\
    имени первого Президента России Б. Н. Ельцина»\\
    Специализированный учебно-научный центр\\

    \vfill
    \textbf{Реферативный обзор}\\
    статей (19--45, 52, 80--81) Налогового кодекса РФ\\
    $ofStudentOf$ 11 $group$ класса $name$\\

    \vfill
    {Екатеринбург}\\
    {\the\year}\\
\end{titlepage}

\section{{\bf Раздел II. Налогоплательщики и плательщики сборов.}\\{\bf Налоговые агенты. Представительство в налоговых правоотношениях}}
\subsection{{\bf Глава 3. Налогоплательщики и плательщики сборов. Налоговые агенты}}
\subsubsection{{\bf Статья 19.} Налогоплательщики и плательщики сборов}
\par Налогоплательщиками и плательщиками сборов признаются \ul{организации} и \ul{физические лица}, на которых в соответствии с настоящим Кодексом возложена обязанность уплачивать соответственно налоги и (или) сборы.
\par В порядке, предусмотренном настоящим Кодексом, филиалы и иные обособленные подразделения российских организаций исполняют обязанности этих организаций по уплате налогов и сборов по месту нахождения этих филиалов и иных обособленных подразделений.
\subsubsection{{\bf Статья 20.} Взаимозависимые лица}
\beginEasyList
& Взаимозависимыми лицами для целей налогообложения признаются \ul{физические лица} и (или) \ul{организации}, отношения между которыми могут оказывать влияние на условия или экономические результаты их деятельности или деятельности представляемых ими лиц, а именно:
&& одна организация непосредственно и (или) косвенно участвует в другой организации, и суммарная доля такого участия составляет более 20 процентов. Доля косвенного участия одной организации в другой через последовательность иных организаций определяется в виде произведения долей непосредственного участия организаций этой последовательности одна в другой;
&& одно физическое лицо подчиняется другому физическому лицу по должностному положению;
&& лица состоят в соответствии с семейным законодательством Российской Федерации в брачных отношениях, отношениях родства или свойства, усыновителя и усыновленного, а также попечителя и опекаемого.
& Суд может признать лица взаимозависимыми по иным основаниям, не предусмотренным пунктом 1 настоящей статьи, если отношения между этими лицами могут повлиять на результаты сделок по реализации товаров (работ, услуг).
\eEasyList
\subsubsection{{\bf Статья 21.} Права налогоплательщиков (плательщиков сборов)}
\beginEasyList
& Налогоплательщики имеют право:
&& получать по месту своего учета от налоговых органов бесплатную информацию (в том числе в письменной форме) о действующих налогах и сборах, законодательстве о налогах и сборах и принятых в соответствии с ним нормативных правовых актах, порядке исчисления и уплаты налогов и сборов, правах и обязанностях налогоплательщиков, полномочиях налоговых органов и их должностных лиц, а также получать формы налоговых деклараций (расчетов) и разъяснения о порядке их заполнения;
&& получать от Министерства финансов Российской Федерации письменные разъяснения по вопросам применения законодательства Российской Федерации о налогах и сборах, от финансовых органов субъектов Российской Федерации и муниципальных образований --- по вопросам применения соответственно законодательства субъектов Российской Федерации о налогах и сборах и нормативных правовых актов муниципальных образований о местных налогах и сборах;
&& использовать налоговые льготы при наличии оснований и в порядке, установленном законодательством о налогах и сборах;
&& получать отсрочку, рассрочку или инвестиционный налоговый кредит в порядке и на условиях, установленных настоящим Кодексом;
&& на своевременный зачет или возврат сумм излишне уплаченных либо излишне взысканных налогов, пени, штрафов;
&&& на осуществление совместной с налоговыми органами сверки расчетов по налогам, сборам, пеням и штрафам, а также на получение акта совместной сверки расчетов по налогам, сборам, пеням и штрафам;
&& представлять свои интересы в отношениях, регулируемых законодательством о налогах и сборах, лично либо через своего представителя;
&& представлять налоговым органам и их должностным лицам пояснения по исчислению и уплате налогов, а также по актам проведенных налоговых проверок;
&& присутствовать при проведении выездной налоговой проверки;
&& получать копии акта налоговой проверки и решений налоговых органов, а также налоговые уведомления и требования об уплате налогов;
&& требовать от должностных лиц налоговых органов и иных уполномоченных органов соблюдения законодательства о налогах и сборах при совершении ими действий в отношении налогоплательщиков;
&& не выполнять неправомерные акты и требования налоговых органов, иных уполномоченных органов и их должностных лиц, не соответствующие настоящему Кодексу или иным федеральным законам;
&& обжаловать в установленном порядке акты налоговых органов, иных уполномоченных органов и действия (бездействие) их должностных лиц;
&& на соблюдение и сохранение \ul{налоговой тайны};
&& на возмещение в полном объеме убытков, причиненных незаконными актами налоговых органов или незаконными действиями (бездействием) их должностных лиц;
&& на участие в процессе рассмотрения материалов налоговой проверки или иных актов налоговых органов в случаях, предусмотренных настоящим Кодексом.
& Налогоплательщики имеют также иные права, установленные настоящим Кодексом и другими актами законодательства о налогах и сборах.
& Плательщики сборов имеют те же права, что и налогоплательщики.
& Любой из участников договора инвестиционного товарищества имеет право обжаловать в установленном порядке акты налоговых органов и действия (бездействие) их должностных лиц.
\eEasyList
\subsubsection{{\bf Статья 22.} Обеспечение и защита прав налогоплательщиков (плательщиков сборов)}
\beginEasyList
& Налогоплательщикам (плательщикам сборов) гарантируется административная и судебная защита их прав и законных интересов.
\par Порядок защиты прав и законных интересов налогоплательщиков (плательщиков сборов) определяется настоящим Кодексом и иными федеральными законами.
& Права налогоплательщиков (плательщиков сборов) обеспечиваются соответствующими обязанностями должностных лиц налоговых органов и иных уполномоченных органов.
\par Неисполнение или ненадлежащее исполнение обязанностей по обеспечению прав налогоплательщиков (плательщиков сборов) влечет ответственность, предусмотренную федеральными законами.
\eEasyList
\subsubsection{{\bf Статья 23.} Обязанности налогоплательщиков (плательщиков сборов)}
\beginEasyList
& Налогоплательщики обязаны:
&& уплачивать законно установленные налоги;
&& встать на учет в налоговых органах, если такая обязанность предусмотрена настоящим Кодексом;
&& вести в установленном порядке учет своих доходов (расходов) и объектов налогообложения, если такая обязанность предусмотрена законодательством о налогах и сборах;
&& представлять в установленном порядке в налоговый орган по месту учета налоговые декларации (расчеты), если такая обязанность предусмотрена законодательством о налогах и сборах;
&& представлять в налоговый орган по месту жительства индивидуального предпринимателя, нотариуса, занимающегося частной практикой, адвоката, учредившего адвокатский кабинет, по запросу налогового органа книгу учета доходов и расходов и хозяйственных операций; представлять в налоговый орган по месту нахождения организации годовую бухгалтерскую (финансовую) отчетность не позднее трех месяцев после окончания отчетного года, за исключением случаев, когда организация в соответствии с Федеральным законом от 6 декабря 2011 года N 402-ФЗ «О бухгалтерском учете» не обязана вести бухгалтерский учет;
&& представлять в налоговые органы и их должностным лицам в случаях и в порядке, которые предусмотрены настоящим Кодексом, документы, необходимые для исчисления и уплаты налогов;
&& выполнять законные требования налогового органа об устранении выявленных нарушений законодательства о налогах и сборах, а также не препятствовать законной деятельности должностных лиц налоговых органов при исполнении ими своих служебных обязанностей;
&& в течение четырех лет обеспечивать сохранность данных бухгалтерского и налогового учета и других документов, необходимых для исчисления и уплаты налогов, в том числе документов, подтверждающих получение доходов, осуществление расходов (для организаций и индивидуальных предпринимателей), а также уплату (удержание) налогов, если иное не предусмотрено настоящим \ul{Кодексом};
&& нести иные обязанности, предусмотренные законодательством о налогах и сборах.
& Налогоплательщики --- организации и индивидуальные предприниматели помимо обязанностей, предусмотренных \ul{пунктом 1} настоящей статьи, обязаны сообщать в налоговый орган соответственно по месту нахождения организации, месту жительства индивидуального предпринимателя:
&& об открытии или о закрытии счетов (лицевых счетов) --- в течение семи дней со дня открытия (закрытия) таких счетов. Индивидуальные предприниматели сообщают в налоговый орган о счетах, используемых ими в предпринимательской деятельности;
&&& о возникновении или прекращении права использовать корпоративные электронные средства платежа для переводов электронных денежных средств --- в течение семи дней со дня возникновения (прекращения) такого права;
&& обо всех случаях участия в российских организациях (за исключением случаев участия в хозяйственных товариществах и обществах с ограниченной ответственностью) и иностранных организациях --- в срок не позднее одного месяца со дня начала такого участия;
&& обо всех обособленных подразделениях российской организации, созданных на территории Российской Федерации (за исключением филиалов и представительств), и изменениях в ранее сообщенные в налоговый орган сведения о таких обособленных подразделениях:
\par в течение одного месяца со дня создания обособленного подразделения российской организации;
\par в течение трех дней со дня изменения соответствующего сведения об обособленном подразделении российской организации;
&&& обо всех обособленных подразделениях российской организации на территории Российской Федерации, через которые прекращается деятельность этой организации (которые закрываются этой организацией):
\par в течение трех дней со дня принятия российской организацией решения о прекращении деятельности через филиал или представительство (закрытии филиала или представительства);
\par в течение трех дней со дня прекращения деятельности российской организации через иное обособленное подразделение (закрытия иного обособленного подразделения);
&& утратил силу.
& Нотариусы, занимающиеся частной практикой, и адвокаты, учредившие адвокатские кабинеты, обязаны сообщать в налоговый орган по месту своего жительства об открытии (о закрытии) счетов, предназначенных для осуществления ими профессиональной деятельности, в течение семи дней со дня открытия (закрытия) таких счетов.
& Плательщики сборов обязаны уплачивать законно установленные сборы и нести иные обязанности, установленные законодательством Российской Федерации о налогах и сборах.
& За невыполнение или ненадлежащее выполнение возложенных на него обязанностей налогоплательщик (плательщик сборов) несет ответственность в соответствии с законодательством Российской Федерации.
& Налогоплательщики, уплачивающие налоги в связи с перемещением товаров через таможенную границу Таможенного союза, также несут обязанности, предусмотренные законодательством Таможенного союза и законодательством Российской Федерации о таможенном деле.
& Сообщения, предусмотренные \ul{пунктами 2} и \ul{3} настоящей статьи, могут быть представлены в налоговый орган лично или через представителя, направлены по почте заказным письмом или переданы в электронной форме по телекоммуникационным каналам связи.
\par Если указанные сообщения переданы в электронной форме, такие сообщения должны быть заверены усиленной квалифицированной электронной подписью лица, представившего их, или усиленной квалифицированной электронной подписью его представителя.
\par Формы и форматы сообщений, представляемых на бумажном носителе или в электронной форме, а также порядок заполнения форм указанных сообщений утверждаются федеральным органом исполнительной власти, уполномоченным по контролю и надзору в области налогов и сборов.
\par Порядок представления сообщений, предусмотренных \ul{пунктами 2} и \ul{3} настоящей статьи, в электронной форме по телекоммуникационным каналам связи утверждается федеральным органом исполнительной власти, уполномоченным по контролю и надзору в области налогов и сборов.
\eEasyList
\subsubsection{{\bf Статья 24.} Налоговые агенты}
\beginEasyList
& Налоговыми агентами признаются лица, на которых в соответствии с настоящим Кодексом возложены обязанности по исчислению, удержанию у налогоплательщика и перечислению налогов в бюджетную систему Российской Федерации.
& Налоговые агенты имеют те же права, что и налогоплательщики, если иное не предусмотрено настоящим Кодексом.
\par Обеспечение и защита прав налоговых агентов осуществляются в соответствии со \ul{статьей 22} настоящего Кодекса.
& Налоговые агенты обязаны:
&& правильно и своевременно исчислять, удерживать из денежных средств, выплачиваемых налогоплательщикам, и перечислять налоги в бюджетную систему Российской Федерации на соответствующие счета Федерального казначейства;
&& письменно сообщать в налоговый орган по месту своего учета о невозможности удержать налог и о сумме задолженности налогоплательщика в течение одного месяца со дня, когда налоговому агенту стало известно о таких обстоятельствах;
&& вести учет начисленных и выплаченных налогоплательщикам доходов, исчисленных, удержанных и перечисленных в бюджетную систему Российской Федерации налогов, в том числе по каждому налогоплательщику;
&& представлять в налоговый орган по месту своего учета документы, необходимые для осуществления контроля за правильностью исчисления, удержания и перечисления налогов;
&& в течение четырех лет обеспечивать сохранность документов, необходимых для исчисления, удержания и перечисления налогов.
\par 3.1. Налоговые агенты несут также другие обязанности, предусмотренные настоящим Кодексом.
& Налоговые агенты перечисляют удержанные налоги в порядке, предусмотренном настоящим Кодексом для уплаты налога налогоплательщиком.
& За неисполнение или ненадлежащее исполнение возложенных на него обязанностей налоговый агент несет ответственность в соответствии с законодательством Российской Федерации.
\eEasyList
\subsubsection{{\bf Статья 24.1.} Участие налогоплательщика в договоре инвестиционного товарищества}
\beginEasyList
& Каждый налогоплательщик самостоятельно исполняет обязанности по уплате налога на прибыль организаций, налога на доходы физических лиц, возникающие в связи с его участием в договоре инвестиционного товарищества, с учетом особенностей, предусмотренных настоящей статьей и иными положениями настоящего Кодекса.
& Обязанность по уплате налогов и сборов, не указанных в \ul{пункте 1} настоящей статьи, но возникающих в связи с выполнением договора инвестиционного товарищества, возлагается на участника такого договора --- управляющего товарища, ответственного за ведение налогового учета (далее в настоящей статье --- управляющий товарищ, ответственный за ведение налогового учета).
& Управляющий товарищ, ответственный за ведение налогового учета, признается налоговым агентом по доходам иностранных лиц от участия в инвестиционном товариществе.
& Управляющий товарищ, ответственный за ведение налогового учета, обязан:
&& направлять в налоговый орган по месту своего учета копию договора инвестиционного товарищества (за исключением инвестиционной декларации), сообщать о его прекращении, сообщать о выполнении, прекращении выполнения функций управляющего товарища в срок не позднее пяти дней со дня заключения указанного договора, его прекращения, начала, прекращения выполнения функций управляющего товарища;
&& вести обособленный налоговый учет по операциям инвестиционного товарищества в порядке, установленном \ul{главой 25} настоящего Кодекса;
&& представлять в налоговый орган по месту своего учета расчет финансового результата инвестиционного товарищества.
\par Форма расчета финансового результата инвестиционного товарищества утверждается Министерством финансов Российской Федерации.
\par Расчет финансового результата инвестиционного товарищества представляется в налоговый орган в сроки, установленные настоящим Кодексом для представления налоговой декларации (расчета) по налогу на прибыль организаций;
&& сообщать в налоговый орган по месту своего учета об открытии или о закрытии счетов инвестиционного товарищества в течение семи дней со дня открытия или закрытия таких счетов;
&& в порядке и в сроки, установленные договором инвестиционного товарищества, но не позднее пятнадцати дней до окончания срока представления в налоговый орган налоговых деклараций (расчетов) по налогу на прибыль организаций, установленных настоящим Кодексом, предоставлять участникам договора копию расчета финансового результата инвестиционного товарищества и сведения о приходящейся на каждого из них доле прибыли (убытка) инвестиционного товарищества.
\par Управляющий товарищ предоставляет товарищам сведения о доле прибыли (убытка) инвестиционного товарищества, приходящейся на каждого из них, по каждому виду доходов, налоговая база по которым в соответствии с настоящим Кодексом определяется отдельно;
&& предоставлять участникам договора инвестиционного товарищества сведения, предусмотренные Федеральным законом «Об инвестиционных товариществах»;
&& в случае, если в расчет финансового результата инвестиционного товарищества вносятся уточнения, представлять уточненный расчет в налоговый орган по месту своего учета и предоставлять участникам договора копию уточненного расчета финансового результата инвестиционного товарищества в течение пяти дней с даты внесения уточнений.
& Управляющий товарищ, ответственный за ведение налогового учета, в отношениях, связанных с ведением дел инвестиционного товарищества, имеет те же права, что и налогоплательщики.
\eEasyList
\subsubsection{{\bf Статья 25.} Утратила силу с 1 января 2007 г.}
\subsection{{\bf Глава 3.1. Консолидированная группа налогоплательщиков}}
\subsubsection{{\bf Статья 25.1.} Общие положения о консолидированной группе налогоплательщиков}
\beginEasyList
& Консолидированной группой налогоплательщиков признается добровольное объединение налогоплательщиков налога на прибыль организаций на основе договора о создании консолидированной группы налогоплательщиков в порядке и на условиях, которые предусмотрены настоящим Кодексом, в целях исчисления и уплаты налога на прибыль организаций с учетом совокупного финансового результата хозяйственной деятельности указанных налогоплательщиков (далее --- налог на прибыль организаций по консолидированной группе налогоплательщиков).
& Участником консолидированной группы налогоплательщиков признается организация, которая является стороной действующего договора о создании консолидированной группы налогоплательщиков, соответствует критериям и условиям, предусмотренным настоящим Кодексом для участников консолидированной группы налогоплательщиков.
& Ответственным участником консолидированной группы налогоплательщиков признается участник консолидированной группы налогоплательщиков, на которого в соответствии с договором о создании консолидированной группы налогоплательщиков возложены обязанности по исчислению и уплате налога на прибыль организаций по консолидированной группе налогоплательщиков и который в правоотношениях по исчислению и уплате указанного налога осуществляет те же права и несет те же обязанности, что и налогоплательщики налога на прибыль организаций.
& Документом, подтверждающим полномочия ответственного участника консолидированной группы налогоплательщиков, является договор о создании консолидированной группы налогоплательщиков, заключенный в соответствии с настоящим Кодексом и гражданским законодательством Российской Федерации.
\eEasyList
\subsubsection{{\bf Статья 25.2.} Условия создания консолидированной группы налогоплательщиков}
\beginEasyList
& Российские организации, соответствующие условиям, предусмотренным настоящей статьей, вправе создать консолидированную группу налогоплательщиков.
\par Условия, которым должны соответствовать участники консолидированной группы налогоплательщиков, предусмотренные настоящей статьей, применяются в течение всего срока действия договора о создании указанной группы, если иное не предусмотрено настоящим Кодексом.
& Консолидированная группа налогоплательщиков может быть создана организациями при условии, что одна организация непосредственно и (или) косвенно участвует в уставном (складочном) капитале других организаций и доля такого участия в каждой такой организации составляет не менее 90 процентов. Указанное условие должно соблюдаться в течение всего срока действия договора о создании консолидированной группы налогоплательщиков.
\par Доля участия одной организации в другой организации определяется в порядке, установленном настоящим Кодексом.
& {\bf Организация} --- сторона договора о создании консолидированной группы налогоплательщиков должна соответствовать следующим условиям:
&& организация не находится в процессе реорганизации или ликвидации;
&& в отношении организации не возбуждено производство по делу о несостоятельности (банкротстве) в соответствии с законодательством Российской Федерации о несостоятельности (банкротстве);
&& размер чистых активов организации, рассчитанный на основании бухгалтерской (финансовой) отчетности на последнюю отчетную дату, предшествующую дате представления в налоговый орган документов для регистрации договора о создании (изменении) консолидированной группы налогоплательщиков, превышает размер ее уставного (складочного) капитала.
& Присоединение новой организации к существующей консолидированной группе налогоплательщиков возможно при условии, что присоединяемая организация соответствует условиям, предусмотренным \ul{пунктом 3} настоящей статьи, на дату своего присоединения.
& Все в совокупности организации, являющиеся участниками консолидированной группы налогоплательщиков, должны соответствовать следующим условиям:
&& совокупная сумма налога на добавленную стоимость, акцизов, налога на прибыль организаций и налога на добычу полезных ископаемых, уплаченная в течение календарного года, предшествующего году, в котором представляются в налоговый орган документы для регистрации договора о создании консолидированной группы налогоплательщиков, без учета сумм налогов, уплаченных в связи с перемещением товаров через таможенную границу Таможенного союза, составляет не менее 10 миллиардов рублей;
&& суммарный объем выручки от продажи товаров, продукции, выполнения работ и оказания услуг, а также от прочих доходов по данным бухгалтерской (финансовой) отчетности за календарный год, предшествующий году, в котором представляются в налоговый орган документы для регистрации договора о создании консолидированной группы налогоплательщиков, составляет не менее 100 миллиардов рублей;
&& совокупная стоимость активов по данным бухгалтерской (финансовой) отчетности на 31 декабря календарного года, предшествующего году, в котором представляются в налоговый орган документы для регистрации договора о создании консолидированной группы налогоплательщиков, составляет не менее 300 миллиардов рублей.
& Участниками консолидированной группы налогоплательщиков не могут являться следующие организации:
&& организации, являющиеся резидентами особых экономических зон;
&& организации, применяющие специальные налоговые режимы;
&& банки, за исключением случая, когда все другие организации, входящие в эту группу, являются банками;
&& страховые организации, за исключением случая, когда все другие организации, входящие в эту группу, являются страховыми организациями;
&& негосударственные пенсионные фонды, за исключением случая, когда все другие организации, входящие в эту группу, являются негосударственными пенсионными фондами;
&& профессиональные участники рынка ценных бумаг, не являющиеся банками, за исключением случая, когда все другие организации, входящие в эту группу, являются профессиональными участниками рынка ценных бумаг, не являющимися банками;
&& организации, являющиеся участниками иной консолидированной группы налогоплательщиков;
&& организации, не признаваемые налогоплательщиками налога на прибыль организаций, а также использующие право на освобождение от обязанностей налогоплательщика налога на прибыль организаций в соответствии с \ul{главой 25} настоящего Кодекса;
&& организации, осуществляющие образовательную и (или) медицинскую деятельность и применяющие налоговую ставку 0 процентов по налогу на прибыль организаций в соответствии с \ul{главой 25} настоящего Кодекса;
&& организации, являющиеся налогоплательщиками налога на игорный бизнес;
&& клиринговые организации;
&& кредитные потребительские кооперативы;
&& микрофинансовые организации.
& Консолидированная группа налогоплательщиков создается не менее чем на два налоговых периода по налогу на прибыль организаций.
\eEasyList
\subsubsection{{\bf Статья 25.3.} Договор о создании консолидированной группы налогоплательщиков}
\beginEasyList
& В соответствии с договором о создании консолидированной группы налогоплательщиков организации, соответствующие условиям, установленным \ul{статьей 25.2} настоящего Кодекса, объединяются на добровольной основе без создания юридического лица в целях исчисления и уплаты налога на прибыль организаций по консолидированной группе налогоплательщиков в порядке и на условиях, которые установлены настоящим Кодексом.
& Договор о создании консолидированной группы налогоплательщиков должен содержать следующие положения:
&& предмет договора о создании консолидированной группы налогоплательщиков;
&& перечень и реквизиты организаций --- участников консолидированной группы налогоплательщиков;
&& наименование организации --- ответственного участника консолидированной группы налогоплательщиков;
&& перечень полномочий, которые участники консолидированной группы налогоплательщиков передают ответственному участнику этой группы в соответствии с настоящей главой;
&& порядок и сроки исполнения обязанностей и осуществления прав ответственным участником и другими участниками консолидированной группы налогоплательщиков, не предусмотренных настоящим Кодексом, ответственность за невыполнение установленных обязанностей;
&& срок, исчисляемый в календарных годах, на который создается консолидированная группа налогоплательщиков, если она создается на определенный срок, либо указание на отсутствие определенного срока, на который создается эта группа;
&& показатели, необходимые для определения налоговой базы и уплаты налога на прибыль организаций по каждому участнику консолидированной группы налогоплательщиков с учетом особенностей, предусмотренных \ul{статьей 288} настоящего Кодекса.
& К правоотношениям, основанным на договоре о создании консолидированной группы налогоплательщиков, применяется законодательство о налогах и сборах, а в части, не урегулированной законодательством о налогах и сборах, --- гражданское законодательство Российской Федерации.
\par Любые положения договора о создании консолидированной группы налогоплательщиков (включая сам такой договор), если они не соответствуют законодательству Российской Федерации, могут быть признаны недействительными в судебном порядке участником этой группы или налоговым органом.
& Договор о создании консолидированной группы налогоплательщиков действует до наступления наиболее ранней из следующих дат:
&& даты прекращения действия указанного договора, предусмотренной этим договором и (или) настоящим Кодексом;
&& даты расторжения договора;
&& 1-го числа \ul{налогового периода} по налогу на прибыль организаций, следующего за датой отказа налоговым органом в регистрации указанного договора.
& Договор о создании консолидированной группы налогоплательщиков подлежит регистрации в налоговом органе по месту нахождения организации --- ответственного участника консолидированной группы налогоплательщиков.
\par В случае, если ответственный участник консолидированной группы налогоплательщиков в соответствии со \ul{статьей 83} настоящего Кодекса отнесен к категории крупнейших налогоплательщиков, договор о создании консолидированной группы налогоплательщиков подлежит регистрации в налоговом органе по месту учета указанного ответственного участника консолидированной группы в качестве крупнейшего налогоплательщика.
& Для регистрации договора о создании консолидированной группы налогоплательщиков ответственный участник этой группы представляет в налоговый орган следующие документы:
&& подписанное уполномоченными лицами всех участников создаваемой консолидированной группы заявление о регистрации договора о создании консолидированной группы налогоплательщиков;
&& два экземпляра договора о создании консолидированной группы налогоплательщиков;
&& документы, подтверждающие выполнение условий, предусмотренных \ul{пунктами 2}, \ul{3} и \ul{5 статьи 25.2} настоящего Кодекса, заверенные ответственным участником консолидированной группы налогоплательщиков, в том числе копии платежных поручений на уплату налога на добавленную стоимость, акцизов, налога на прибыль организаций и налога на добычу полезных ископаемых (копии решений налогового органа о проведении зачета по перечисленным выше налогам), бухгалтерских балансов, отчетов о финансовых результатах за предшествующий календарный год для каждого из участников группы;
&& документы, подтверждающие полномочия лиц, подписавших договор о создании консолидированной группы налогоплательщиков.
& Документы, указанные в \ul{пункте 6} настоящей статьи, представляются в налоговый орган не позднее 30 октября года, предшествующего налоговому периоду, начиная с которого исчисляется и уплачивается налог на прибыль организаций по консолидированной группе налогоплательщиков.
& Руководитель (заместитель руководителя) налогового органа в течение одного месяца со дня представления в налоговый орган документов, указанных в \ul{пункте 6} настоящей статьи, производит регистрацию договора о создании консолидированной группы налогоплательщиков либо принимает мотивированное решение об отказе в его регистрации.
\par При обнаружении нарушений, устранимых в пределах срока, установленного настоящим пунктом, налоговый орган обязан уведомить о них ответственного участника консолидированной группы налогоплательщиков.
\par До истечения срока, установленного настоящим пунктом, ответственный участник консолидированной группы налогоплательщиков вправе устранить выявленные нарушения.
& При соблюдении условий, предусмотренных \ul{статьей 25.2} настоящего Кодекса и \ul{пунктами 1-7} настоящей статьи, налоговый орган обязан зарегистрировать договор о создании консолидированной группы налогоплательщиков и в течение пяти дней с даты его регистрации выдать один экземпляр этого договора с отметкой о его регистрации ответственному участнику консолидированной группы налогоплательщиков лично под расписку или иным способом, свидетельствующим о дате получения.
\par В течение пяти дней с даты регистрации договора о создании консолидированной группы налогоплательщиков информация о регистрации договора о создании консолидированной группы налогоплательщиков направляется налоговым органом в налоговые органы по месту нахождения организаций --- участников консолидированной группы налогоплательщиков, а также по месту нахождения обособленных подразделений организаций --- участников консолидированной группы налогоплательщиков.
& Консолидированная группа налогоплательщиков признается созданной с 1-го числа \ul{налогового периода} по налогу на прибыль организаций, следующего за календарным годом, в котором налоговым органом зарегистрирован договор о создании этой группы.
& Отказ налогового органа в регистрации договора о создании консолидированной группы налогоплательщиков допускается исключительно при наличии хотя бы одного из следующих обстоятельств:
&& несоответствия условиям создания консолидированной группы налогоплательщиков, предусмотренным \ul{статьей 25.2} настоящего Кодекса;
&& несоответствия договора о создании консолидированной группы налогоплательщиков требованиям, указанным в \ul{пункте 2} настоящей статьи;
&& непредставления (представления не в полном объеме) или нарушения срока представления в уполномоченный налоговый орган документов для регистрации договора о создании консолидированной группы налогоплательщиков, предусмотренных \ul{пунктами 5--7} настоящей статьи;
&& в случае подписания документов не уполномоченными на это лицами.
& В случае отказа налогового органа в регистрации договора о создании консолидированной группы налогоплательщиков ответственный участник консолидированной группы налогоплательщиков вправе повторно представить документы о регистрации такого договора.
& Копия решения об отказе в регистрации договора о создании консолидированной группы налогоплательщиков в течение пяти дней со дня его принятия передается налоговым органом уполномоченному представителю лица, указанного в таком договоре в качестве ответственного участника консолидированной группы налогоплательщиков, лично под расписку или иным способом, свидетельствующим о дате получения.
& Отказ в регистрации договора о создании консолидированной группы налогоплательщиков может быть обжалован лицом, указанным в таком договоре в качестве ответственного участника консолидированной группы налогоплательщиков, в порядке и сроки, которые установлены настоящим Кодексом для обжалования актов, действий или бездействия налоговых органов и их должностных лиц.
\par При удовлетворении заявления (жалобы), если для регистрации договора о создании консолидированной группы налогоплательщиков не имеется иных препятствий, установленных настоящей главой, налоговый орган обязан зарегистрировать указанный договор, а указанная группа признается созданной с 1-го числа \ul{налогового периода} по налогу на прибыль организаций, следующего за календарным годом, в котором такая группа подлежала регистрации в соответствии с \ul{пунктом 8} настоящей статьи.
\eEasyList
\subsubsection{{\bf Статья 25.4.} Изменение договора о создании консолидированной группы налогоплательщиков и продление срока его действия}
\beginEasyList
& Договор о создании консолидированной группы налогоплательщиков может быть изменен в порядке и на условиях, которые предусмотрены настоящей статьей.
& Стороны договора о создании консолидированной группы налогоплательщиков обязаны внести изменения в указанный договор в случае:
&& принятия решения о ликвидации одной или нескольких организаций --- участников консолидированной группы налогоплательщиков;
&& принятия решения о реорганизации (в форме слияния, присоединения, выделения и разделения) одной или нескольких организаций --- участников консолидированной группы налогоплательщиков;
&& присоединения организации к консолидированной группе налогоплательщиков;
&& выхода организации из консолидированной группы налогоплательщиков (в том числе в случаях, когда такая организация перестает удовлетворять условиям, предусмотренным \ul{статьей 25.2} настоящего Кодекса, включая случаи ее слияния с организацией, не являющейся участником указанной группы, разделения (выделения) организации, являющейся участником этой группы);
&& принятия решения о продлении срока действия договора о создании консолидированной группы налогоплательщиков.
& Соглашение об изменении договора о создании консолидированной группы налогоплательщиков (решение о продлении срока действия указанного договора) принимается всеми участниками такой группы, включая вновь присоединяющихся участников и исключая участников, выходящих из группы.
& Соглашение об изменении договора о создании консолидированной группы налогоплательщиков (решение о продлении срока действия указанного договора) представляется для регистрации в налоговый орган в следующие сроки:
&& не позднее одного месяца до начала очередного налогового периода по налогу на прибыль организаций --- при внесении изменений, связанных с присоединением к группе новых участников (за исключением случаев реорганизации участников указанной группы);
&& не позднее одного месяца до истечения срока действия договора о создании консолидированной группы налогоплательщиков --- при принятии решения о продлении срока действия указанного договора;
&& в течение одного месяца со дня возникновения обстоятельств для изменения договора о создании консолидированной группы налогоплательщиков --- в прочих случаях.
& Для регистрации соглашения об изменении договора о создании консолидированной группы налогоплательщиков (решения о продлении срока действия указанного договора) ее ответственный участник представляет в налоговый орган следующие документы:
&& уведомление о внесении изменений в договор;
&& подписанные уполномоченными лицами участников консолидированной группы налогоплательщиков два экземпляра соглашения об изменении договора;
&& документы, подтверждающие полномочия лиц, подписавших соглашение о внесении изменений в договор;
&& документы, подтверждающие выполнение условий, предусмотренных \ul{статьей 25.2} настоящего Кодекса, с учетом внесенных изменений в договор;
&& два экземпляра решения о продлении срока действия договора.
& Налоговый орган обязан зарегистрировать изменения договора о создании консолидированной группы налогоплательщиков в течение 10 дней со дня представления документов, указанных в \ul{пункте 5} настоящей статьи, и выдать уполномоченному представителю ответственного участника указанной группы один экземпляр изменений с отметкой о его регистрации.
& Основаниями для отказа в регистрации изменений договора о создании консолидированной группы налогоплательщиков являются:
&& невыполнение условий, предусмотренных \ul{статьей 25.2} настоящего Кодекса, в отношении хотя бы одного участника консолидированной группы налогоплательщиков;
&& подписание документов не уполномоченными на это лицами;
&& нарушение срока представления документов на изменение указанного договора;
&& непредставление (представление не в полном объеме) документов, предусмотренных \ul{пунктом 5} настоящей статьи.
& Изменения договора о создании консолидированной группы налогоплательщиков вступают в силу в следующем порядке:
&& изменения договора о создании консолидированной группы налогоплательщиков, связанные с присоединением к такой группе новых организаций (за исключением случаев реорганизации участников группы), вступают в силу не ранее 1-го числа \ul{налогового периода} по налогу на прибыль организаций, следующего за календарным годом, в котором соответствующие изменения договора зарегистрированы налоговым органом;
&& изменения договора о создании консолидированной группы налогоплательщиков, связанные с выходом участников из состава такой группы, вступают в силу с 1-го числа налогового периода по налогу на прибыль организаций, в котором возникли обстоятельства для внесения соответствующих изменений в договор (если иное не предусмотрено \ul{подпунктом 3} настоящего пункта);
&& изменения договора о создании консолидированной группы налогоплательщиков, связанные с выходом участников из состава такой группы, которые на момент регистрации налоговым органом соответствующих изменений договора соответствуют условиям, предусмотренным \ul{статьей 25.2} настоящего Кодекса, вступают в силу с 1-го числа налогового периода по налогу на прибыль организаций, следующего за календарным годом, в котором соответствующие изменения договора зарегистрированы налоговым органом;
&& в прочих случаях изменения договора о создании консолидированной группы налогоплательщиков вступают в силу с даты, указанной его сторонами, но не ранее даты регистрации соответствующих изменений налоговым органом.
& Уклонение от внесения обязательных изменений в договор о создании консолидированной группы налогоплательщиков влечет прекращение действия договора с 1-го числа налогового периода по налогу на прибыль организаций, в котором соответствующие обязательные изменения договора должны были бы вступить в силу.
\eEasyList
\subsubsection{{\bf Статья 25.5.} Права и обязанности ответственного участника и других участников консолидированной группы налогоплательщиков}
\beginEasyList
& Ответственный участник консолидированной группы налогоплательщиков, если иное не предусмотрено настоящим Кодексом, осуществляет права и несет обязанности, предусмотренные настоящим Кодексом для налогоплательщиков налога на прибыль организаций, в отношениях, регулируемых законодательством о налогах и сборах, возникающих в связи с действием консолидированной группы налогоплательщиков.
& Ответственный участник консолидированной группы налогоплательщиков имеет право:
&& представлять налоговым органам и их должностным лицам любые пояснения по исчислению и уплате налога на прибыль организаций (авансовых платежей) по консолидированной группе налогоплательщиков;
&& присутствовать при проведении выездных налоговых проверок, проводимых в связи с уплатой налога на прибыль организаций по консолидированной группе налогоплательщиков, по месту нахождения любого участника такой группы и его обособленных подразделений;
&& получать копии актов налоговых проверок и решений налогового органа, вынесенных по результатам проверок, проводимых в связи с уплатой налога на прибыль организаций по консолидированной группе налогоплательщиков, а также получать требования об уплате налога на прибыль организаций (авансовых платежей) и иные документы, связанные с действием консолидированной группы налогоплательщиков;
&& участвовать при рассмотрении руководителем (заместителем руководителя) налогового органа материалов налоговых проверок и дополнительных мероприятий налогового контроля, проводимых в связи с уплатой налога на прибыль организаций по консолидированной группе налогоплательщиков, в случаях и порядке, которые предусмотрены \ul{статьей 101} настоящего Кодекса;
&& получать от налоговых органов сведения об участниках консолидированной группы налогоплательщиков, составляющие налоговую тайну;
&& обжаловать в установленном порядке акты налоговых органов, иных уполномоченных органов и действия или бездействие их должностных лиц, в том числе в интересах отдельных участников консолидированной группы налогоплательщиков в связи с исполнением ими обязанностей (осуществлением прав) при исчислении налога на прибыль организаций по консолидированной группе налогоплательщиков;
&& обращаться в налоговый орган с заявлением о зачете (возврате) излишне уплаченного налога на прибыль организаций по консолидированной группе налогоплательщиков.
& Ответственный участник консолидированной группы налогоплательщиков обязан:
&& представлять в порядке и сроки, которые предусмотрены настоящим Кодексом, в налоговый орган для регистрации договор о создании консолидированной группы налогоплательщиков, изменения договора о создании консолидированной группы налогоплательщиков, решение или уведомление о прекращении действия консолидированной группы налогоплательщиков;
&& вести налоговый учет, исчислять и уплачивать налог на прибыль организаций (авансовые платежи) по консолидированной группе налогоплательщиков в порядке, установленном \ul{главой 25} настоящего Кодекса;
&& представлять в налоговый орган \ul{налоговую декларацию} по налогу на прибыль организаций по консолидированной группе налогоплательщиков, а также документы, полученные от других участников этой группы, в порядке и сроки, которые установлены настоящим Кодексом;
&& в случаях прекращения действия консолидированной группы налогоплательщиков и (или) выхода организации из состава консолидированной группы налогоплательщиков представлять другим участникам этой группы (в том числе вышедшим из состава группы или реорганизованным) сведения, необходимые для исчисления и уплаты налога на прибыль организаций (авансовых платежей) и составления налоговых деклараций за соответствующие \ul{отчетные} и \ul{налоговый периоды}, в порядке и сроки, которые предусмотрены договором о создании консолидированной группы налогоплательщиков;
&& уплачивать недоимку, пени и штрафы, возникающие в связи с исполнением обязанностей налогоплательщика налога на прибыль организаций по консолидированной группе налогоплательщиков;
&& информировать участников консолидированной группы налогоплательщиков о получении требования об уплате налогов и сборов в течение пяти дней со дня его получения;
&& истребовать у участников консолидированной группы налогоплательщиков документы, пояснения и иную информацию, необходимую для осуществления налоговыми органами мероприятий налогового контроля и исполнения обязанностей налогоплательщика налога на прибыль организаций по консолидированной группе налогоплательщиков;
&& представлять первичные документы, регистры налогового учета и иную информацию по консолидированной группе налогоплательщиков, истребованную в рамках мероприятий налогового контроля налоговым органом, которым зарегистрирован договор о создании указанной группы.
& Ответственный участник консолидированной группы налогоплательщиков в пределах предоставленных ему полномочий имеет иные права и несет другие обязанности налогоплательщика, предусмотренные настоящим Кодексом.
& Участники консолидированной группы налогоплательщиков обязаны:
&& представлять (в том числе в электронной форме) ответственному участнику консолидированной группы налогоплательщиков расчеты налоговой базы по налогу на прибыль организаций в отношении полученных ими доходов и расходов, данные регистров налогового учета и иные документы, необходимые ответственному участнику указанной группы для исполнения им обязанностей и осуществления прав налогоплательщика налога на прибыль организаций по консолидированной группе налогоплательщиков;
&& представлять в налоговые органы в установленные настоящим Кодексом сроки и порядке истребуемые документы и иную информацию при осуществлении налоговым органом мероприятий налогового контроля в связи с действием консолидированной группы налогоплательщиков;
&& исполнять обязанность по уплате налога на прибыль организаций (авансовых платежей) по консолидированной группе налогоплательщиков, соответствующих пеней и штрафов в случае неисполнения или ненадлежащего исполнения такой обязанности ответственным участником этой группы в порядке, установленном \ul{статьями 45--47} настоящего Кодекса;
&& осуществлять все действия и предоставлять все документы, необходимые для регистрации договора о создании консолидированной группы налогоплательщиков и его изменений;
&& в случае несоблюдения условий, предусмотренных \ul{статьей 25.2} настоящего Кодекса, незамедлительно уведомить ответственного участника консолидированной группы налогоплательщиков и налоговый орган, в котором зарегистрирован договор о создании указанной группы;
&& вести налоговый учет в порядке, предусмотренном \ul{главой 25} настоящего Кодекса.
& В случае неисполнения или ненадлежащего исполнения ответственным участником консолидированной группы налогоплательщиков обязанности по уплате налога на прибыль организаций (авансовых платежей, соответствующих пеней и штрафов) участник (участники) этой группы, исполнивший (исполнившие) указанную обязанность, приобретает (приобретают) право регрессного требования в размерах и порядке, которые предусмотрены гражданским законодательством Российской Федерации и договором о создании указанной группы.
& Участники консолидированной группы налогоплательщиков вправе:
&& получать от ответственного участника указанной группы копии актов, решений, требований, актов сверки и иных документов, предоставленных ответственному участнику налоговым органом в связи с действием консолидированной группы налогоплательщиков;
&& самостоятельно обжаловать в вышестоящий налоговый орган или в суд акты налоговых органов, действия или бездействие их должностных лиц с учетом особенностей, предусмотренных настоящим Кодексом;
&& добровольно исполнять обязанность ответственного участника консолидированной группы налогоплательщиков по уплате налога на прибыль организаций по консолидированной группе налогоплательщиков;
&& присутствовать при проведении налоговых проверок, проводимых в связи с исчислением и уплатой налога на прибыль организаций по консолидированной группе налогоплательщиков у такого участника, а также участвовать при рассмотрении материалов таких налоговых проверок.
& Организация при выходе из состава консолидированной группы налогоплательщиков обязана:
&& внести изменения в налоговый учет с начала налогового периода по налогу на прибыль организаций, с 1-го числа которого указанная организация вышла из состава консолидированной группы налогоплательщиков, направленные на соблюдение требований \ul{главы 25} настоящего Кодекса по налоговому учету налогоплательщика, не являющегося участником консолидированной группы налогоплательщиков;
&& исчислить и уплатить налог на прибыль организаций (авансовые платежи) исходя из фактически полученной прибыли за соответствующие \ul{отчетные} и \ul{налоговый периоды} в сроки, установленные \ul{главой 25} настоящего Кодекса применительно к налоговому периоду, с 1-го числа которого организация вышла из состава консолидированной группы налогоплательщиков;
&& по окончании налогового периода, с 1-го числа которого указанная организация вышла из состава консолидированной группы налогоплательщиков, представить в налоговый орган по месту своего учета налоговую декларацию по налогу на прибыль организаций в сроки, предусмотренные \ul{главой 25} настоящего Кодекса.
& Ответственный участник консолидированной группы налогоплательщиков при выходе из состава указанной группы одного или нескольких участников обязан:
&& внести соответствующие изменения в налоговый учет с начала налогового периода по налогу на прибыль организаций, в котором участник (участники) вышел (вышли) из состава консолидированной группы налогоплательщиков;
&& произвести перерасчет авансовых платежей по налогу на прибыль организаций по истекшим отчетным периодам и представить в налоговый орган по месту учета уточненные налоговые декларации по налогу на прибыль организаций по консолидированной группе налогоплательщиков.
& Выход организации из состава консолидированной группы налогоплательщиков не освобождает ее от исполнения в соответствии со \ul{статьями 45--47} настоящего Кодекса обязанности по уплате налога на прибыль организаций, соответствующих пеней и штрафов, возникших в период, когда организация являлась участником такой группы.
\par Настоящее положение применяется независимо от того, было или не было известно указанной организации до ее выхода из состава консолидированной группы налогоплательщиков о неисполнении указанной обязанности или нарушении законодательства Российской Федерации о налогах и сборах либо соответствующие обстоятельства стали известны организации после ее выхода из состава консолидированной группы налогоплательщиков.
& \ul{Пункты 8--10} настоящей статьи применяются также в случае прекращения действия консолидированной группы налогоплательщиков до истечения срока, на который она была создана.
\eEasyList
\subsubsection{{\bf Статья 25.6.} Прекращение действия консолидированной группы налогоплательщиков}
\beginEasyList
& Консолидированная группа налогоплательщиков прекращает действовать при наличии хотя бы одного из следующих обстоятельств:
&& окончание срока действия договора о создании консолидированной группы налогоплательщиков;
&& расторжение договора о создании консолидированной группы налогоплательщиков по соглашению сторон;
&& вступление в законную силу решения суда о признании договора о создании консолидированной группы налогоплательщиков недействительным;
&& непредставление в налоговый орган в установленные сроки соглашения об изменении договора о создании консолидированной группы налогоплательщиков в связи с выходом из состава указанной группы организации, нарушившей условия, установленные \ul{статьей 25.2} настоящего Кодекса;
&& реорганизация (за исключением реорганизации в форме преобразования), ликвидация ответственного участника консолидированной группы налогоплательщиков;
&& возбуждение в отношении ответственного участника консолидированной группы налогоплательщиков производства по делу о несостоятельности (банкротстве) в соответствии с законодательством Российской Федерации о несостоятельности (банкротстве);
&& несоответствие ответственного участника консолидированной группы налогоплательщиков условиям, предусмотренным \ul{статьей 25.2} настоящего Кодекса;
&& уклонение от внесения обязательных изменений в договор о создании консолидированной группы налогоплательщиков.
& Приобретение (продажа) акций (долей) в уставном (складочном) капитале (фонде) организации --- участника консолидированной группы налогоплательщиков, не приводящее к нарушению условий, предусмотренных \ul{пунктом 2 статьи 25.2} настоящего Кодекса, не влечет прекращения действия консолидированной группы налогоплательщиков.
& При наличии обстоятельства, указанного в \ul{подпункте 2 пункта 1} настоящей статьи, ответственный участник консолидированной группы налогоплательщиков обязан направить в налоговый орган, зарегистрировавший договор о создании этой группы, решение о прекращении действия такой группы, подписанное уполномоченными представителями всех организаций --- участников консолидированной группы налогоплательщиков, в срок не позднее пяти дней со дня принятия соответствующего решения.
\par При наличии обстоятельств, указанных в \ul{подпунктах 1}, \ul{3--7 пункта 1} настоящей статьи, ответственный участник консолидированной группы налогоплательщиков обязан направить в налоговый орган, зарегистрировавший договор о создании этой группы, уведомление, составленное в произвольной форме, с указанием даты возникновения таких обстоятельств.
\par В течение пяти дней с даты получения документов, указанных в абзацах первом и втором настоящего пункта, информация о прекращении действия консолидированной группы налогоплательщиков направляется налоговым органом в налоговые органы по месту нахождения организаций --- участников консолидированной группы налогоплательщиков, а также по месту нахождения обособленных подразделений организаций --- участников консолидированной группы налогоплательщиков.
& Консолидированная группа налогоплательщиков прекращает действие с 1-го числа \ul{налогового периода} по налогу на прибыль организаций, следующего за налоговым периодом, в котором возникли обстоятельства, указанные в \ul{пункте 1} настоящей статьи, если иное не предусмотрено настоящим Кодексом.
& При наличии основания, предусмотренного \ul{подпунктом 3 пункта 1} настоящей статьи, консолидированная группа налогоплательщиков прекращает действие с 1-го числа \ul{отчетного периода} по налогу на прибыль организаций, в котором вступило в законную силу решение суда, указанное в подпункте 3 пункта 1 настоящей статьи.
& При наличии основания, предусмотренного \ul{подпунктом 4 пункта 1} настоящей статьи, консолидированная группа налогоплательщиков прекращает действие с 1-го числа налогового периода по налогу на прибыль организаций, в котором участник этой группы нарушил условия, установленные \ul{статьей 25.2} настоящего Кодекса.
& При наличии оснований, предусмотренных \ul{подпунктами 5--7 пункта 1} настоящей статьи, консолидированная группа налогоплательщиков прекращает действие с 1-го числа налогового периода по налогу на прибыль организаций, в котором соответственно была осуществлена реорганизация (за исключением реорганизации в форме преобразования) или ликвидация ответственного участника указанной группы, либо в отношении такого участника было возбуждено производство по делу о несостоятельности (банкротстве) в соответствии с законодательством Российской Федерации о несостоятельности (банкротстве), либо имело место несоблюдение этим ответственным участником условий, предусмотренных \ul{статьей 25.2} настоящего Кодекса.
\eEasyList
\subsection{{\bf Глава 3.2. Оператор нового морского месторождения углеводородного сырья}}
\subsubsection{{\bf Статья 25.7.} Оператор нового морского месторождения углеводородного сырья}
\beginEasyList
& В целях настоящего Кодекса организация признается оператором нового морского месторождения углеводородного сырья в случае, если такая организация одновременно удовлетворяет следующим условиям:
&& в уставном капитале организации прямо или косвенно участвует организация, владеющая лицензией на пользование участком недр, в границах которого предполагается осуществлять поиск, оценку, разведку и (или) разработку нового морского месторождения углеводородного сырья, либо организация, являющаяся взаимозависимым лицом с организацией, владеющей такой лицензией;
&& организация осуществляет хотя бы один из видов деятельности, связанной с добычей углеводородного сырья на новом морском месторождении углеводородного сырья, собственными силами и (или) с привлечением подрядных организаций;
&& организация осуществляет деятельность, связанную с добычей углеводородного сырья на новом морском месторождении углеводородного сырья, на основе договора, заключенного с владельцем лицензии в отношении нового морского месторождения углеводородного сырья и (или) участка недр, указанного в \ul{подпункте 1} настоящего пункта, и такой договор предусматривает выплату организации-оператору вознаграждения, размер которого зависит в том числе от объема добытого углеводородного сырья на соответствующем морском месторождении углеводородного сырья и (или) выручки от реализации этого сырья (далее в настоящем Кодексе --- операторский договор).
& Организация признается оператором нового морского месторождения углеводородного сырья с даты заключения операторского договора, указанного в \ul{подпункте 3 пункта 1} настоящей статьи, если налоговый орган был уведомлен о заключении договора в соответствии с пунктом 3 настоящей статьи.
& Организация --- владелец лицензии на пользование участком недр, указанная в \ul{подпункте 3 пункта 1} настоящей статьи, в течение десяти рабочих дней с даты заключения операторского договора уведомляет налоговый орган по месту своего учета о заключении операторского договора путем представления в налоговый орган следующих документов:
&& уведомления о заключении операторского договора с указанием информации о новых морских месторождениях углеводородного сырья (при наличии такой информации на дату представления уведомления);
&& заверенной копии подписанного операторского договора;
&& копии лицензии на пользование участком недр, в границах которого осуществляются поиск, оценка, разведка и (или) разработка новых морских месторождений углеводородного сырья или расположено (расположены) новое морское месторождение (новые морские месторождения) углеводородного сырья.
& В целях настоящего Кодекса не допускается, чтобы одновременно на одном и том же новом морском месторождении углеводородного сырья осуществляли деятельность, связанную с добычей углеводородного сырья на указанном новом морском месторождении углеводородного сырья, два и более оператора нового морского месторождения углеводородного сырья.
\par В случае заключения организацией --- владельцем лицензии на пользование участком недр, в границах которого осуществляются поиск, оценка, разведка и (или) разработка нового морского месторождения углеводородного сырья, нового операторского договора с иной организацией, одновременно удовлетворяющей условиям, установленным \ul{пунктом 1} настоящей статьи, указанная иная организация получает статус оператора нового морского месторождения углеводородного сырья в целях настоящего Кодекса с даты уведомления налогового органа о заключении операторского договора в соответствии с \ul{пунктом 3} настоящей статьи.
& Для целей настоящего Кодекса организация утрачивает статус оператора нового морского месторождения углеводородного сырья при наступлении наиболее ранней из следующих дат:
&& даты прекращения действия операторского договора, предусмотренной указанным договором;
&& даты истечения срока лицензии на пользование участком недр, в границах которого расположено указанное новое морское месторождение углеводородного сырья, или прекращения права пользования таким участком недр по иным основаниям, предусмотренным законом;
&& даты ликвидации организации --- владельца лицензии на пользование участком недр, в границах которого расположено указанное новое морское месторождение углеводородного сырья.
\eEasyList
\subsection{{\bf Глава 3.3. Особенности налогообложения при реализации региональных инвестиционных проектов}}
\subsubsection{{\bf Статья 25.8}. Общие положения о региональных инвестиционных проектах}
\beginEasyList
& Региональным инвестиционным проектом для целей настоящего Кодекса признается инвестиционный проект, целью которого является производство товаров и который удовлетворяет одновременно следующим требованиям:
&& производство товаров в результате реализации такого инвестиционного проекта осуществляется, если иное не предусмотрено настоящей статьей, исключительно на территории одного из следующих субъектов Российской Федерации:
\par Республика Бурятия,
\par Республика Саха (Якутия),
\par Республика Тыва,
\par Забайкальский край,
\par Камчатский край,
\par Приморский край,
\par Хабаровский край,
\par Амурская область,
\par Иркутская область,
\par Магаданская область,
\par Сахалинская область,
\par Еврейская автономная область,
\par Чукотский автономный округ;
&& региональный инвестиционный проект не может быть направлен на следующие цели:
\par добыча и (или) переработка нефти, добыча природного газа и (или) газового конденсата, оказание услуг по транспортировке нефти и (или) нефтепродуктов, газа и (или) газового конденсата;
\par производство подакцизных товаров (за исключением легковых автомобилей и мотоциклов);
\par осуществление деятельности, по которой применяется налоговая ставка по налогу на прибыль организаций в размере 0 процентов;
&& на земельных участках, на которых предполагается реализация такого инвестиционного проекта, не располагаются здания, сооружения, находящиеся в собственности физических лиц или организации, не являющейся участником такого инвестиционного проекта (за исключением подъездных путей, коммуникаций, трубопроводов, электрических кабелей, дренажа и других объектов инфраструктуры);
&& объем капитальных вложений в соответствии с инвестиционной декларацией не может быть менее:
\par 50 миллионов рублей при условии осуществления капитальных вложений в срок, не превышающий трех лет со дня включения организации в реестр участников региональных инвестиционных проектов;
\par 500 миллионов рублей при условии осуществления капитальных вложений в срок, не превышающий пяти лет со дня включения организации в реестр участников региональных инвестиционных проектов;
&& каждый региональный инвестиционный проект реализуется единственным участником.
& Требование, установленное \ul{подпунктом 1 пункта 1} настоящей статьи, также признается выполненным в случаях, если:
&& региональный инвестиционный проект предусматривает производство товаров в рамках единого технологического процесса на территориях нескольких указанных в \ul{подпункте 1 пункта 1} настоящей статьи субъектов Российской Федерации;
&& региональный инвестиционный проект направлен на добычу полезных ископаемых и соответствующий участок недр частично расположен за пределами территорий указанных в \ul{подпункте 1 пункта 1} настоящей статьи субъектов Российской Федерации.
& При определении объема капитальных вложений учитываются затраты на создание (приобретение) амортизируемого имущества, в том числе затраты на осуществление проектно-изыскательских работ, новое строительство, техническое перевооружение, модернизацию основных средств, реконструкцию зданий, приобретение машин, оборудования, инструментов, инвентаря (за исключением затрат на приобретение легковых автомобилей, мотоциклов, спортивных, туристских и прогулочных судов, а также затрат на строительство и реконструкцию жилых помещений).
\par При этом не учитываются:
\par полученные участником регионального инвестиционного проекта машины, оборудование, транспортные средства и иное амортизируемое имущество, затраты на которые ранее включались в объем капитальных вложений участниками других региональных инвестиционных проектов;
\par затраты на создание (приобретение) зданий, сооружений, расположенных на земельных участках, на которых осуществляется реализация инвестиционного проекта, на дату включения организации в реестр участников региональных инвестиционных проектов.
& Определение фактического объема капитальных вложений, осуществленных в ходе реализации регионального инвестиционного проекта, осуществляется на основании цен товаров (работ, услуг), определяемых в соответствии со \ul{статьей 105.3} настоящего Кодекса без учета налога на добавленную стоимость.
& Законом субъекта Российской Федерации может быть увеличен минимальный объем капитальных вложений, указанный в \ul{подпункте 4 пункта 1} настоящей статьи, а также установлены иные требования в дополнение к требованиям, предусмотренным настоящей статьей.
\eEasyList
\subsubsection{{\bf Статья 25.9.} Налогоплательщики --- участники региональных инвестиционных проектов}
\beginEasyList
& Налогоплательщиком --- участником регионального инвестиционного проекта признается российская организация, которая получила в порядке, установленном настоящей главой, статус участника регионального инвестиционного проекта и которая непрерывно в течение указанных в \ul{пунктах 2--5 статьи 284.3} настоящего Кодекса налоговых периодов применения налоговых ставок, установленных \ul{пунктом 1.5 статьи 284} настоящего Кодекса, отвечает одновременно следующим требованиям:
&& государственная регистрация юридического лица осуществлена на территории субъекта Российской Федерации, в котором реализуется региональный инвестиционный проект;
&& организация не имеет в своем составе обособленных подразделений, расположенных за пределами территории субъекта (территорий субъектов) Российской Федерации, в котором (которых) реализуется региональный инвестиционный проект;
&& организация не применяет специальных налоговых режимов, предусмотренных \ul{частью второй} настоящего Кодекса;
&& организация не является участником консолидированной группы налогоплательщиков;
&& организация не является некоммерческой организацией, банком, страховой организацией (страховщиком), негосударственным пенсионным фондом, профессиональным участником рынка ценных бумаг, клиринговой организацией;
&& организация ранее не была участником регионального инвестиционного проекта и не является участником (правопреемником участника) иного реализуемого регионального инвестиционного проекта;
&& организация имеет в собственности (в аренде на срок не менее чем до 1 января 2024 года) земельный участок (земельные участки), на котором (которых) планируется реализация регионального инвестиционного проекта;
&& организация имеет разрешение на строительство в случае, если наличие такого разрешения является обязательным для реализации регионального инвестиционного проекта;
&& организация не является резидентом особой экономической зоны любого типа.
& Организация получает статус участника регионального инвестиционного проекта со дня включения ее в реестр участников региональных инвестиционных проектов в порядке, установленном настоящей главой.
\eEasyList
\subsubsection{{\bf Статья 25.10.} Реестр участников региональных инвестиционных проектов}
\beginEasyList
& Реестр участников региональных инвестиционных проектов (далее в настоящей главе --- реестр) ведется федеральным органом исполнительной власти, уполномоченным по контролю и надзору в области налогов и сборов, на основании решений и сведений, направляемых в порядке, предусмотренном настоящей статьей, налоговым органом по месту нахождения налогоплательщика --- участника регионального инвестиционного проекта (по месту учета в качестве крупнейшего налогоплательщика) и уполномоченным органом государственной власти соответствующего субъекта Российской Федерации.
\par В реестре отражаются сведения об участниках региональных инвестиционных проектов, а также сведения о региональных инвестиционных проектах, содержащиеся в соответствующих инвестиционных декларациях. Порядок ведения реестра, состав сведений, содержащихся в реестре, и форма инвестиционной декларации устанавливаются федеральным органом исполнительной власти, уполномоченным по контролю и надзору в области налогов и сборов.
& Решения о включении организации в реестр, а также о внесении изменений в реестр принимаются уполномоченным органом государственной власти субъекта Российской Федерации с учетом положений \ul{статьи 25.11} и \ul{пунктов 1--3 статьи 25.12} настоящего Кодекса.
\par Решение о прекращении статуса участника регионального инвестиционного проекта принимается налоговым органом по месту нахождения налогоплательщика --- участника регионального инвестиционного проекта (по месту учета в качестве крупнейшего налогоплательщика) по основаниям, установленным \ul{пунктом 4 статьи 25.12} настоящего Кодекса.
& Указанные в \ul{пункте 2} настоящей статьи решения, а также иные необходимые сведения направляются в электронной форме в федеральный орган исполнительной власти, уполномоченный по контролю и надзору в области налогов и сборов, в течение трех рабочих дней со дня принятия соответствующего решения.
\eEasyList
\subsubsection{{\bf Статья 25.11.} Порядок включения организации в реестр}
\beginEasyList
& Для включения в реестр организация направляет в уполномоченный орган государственной власти субъекта Российской Федерации составленное в произвольной форме заявление о включении в реестр с приложением следующих документов:
&& копии учредительных документов организации, удостоверенные в установленном порядке;
&& копия документа, подтверждающего факт внесения записи о государственной регистрации организации в Единый государственный реестр юридических лиц;
&& копия свидетельства о постановке организации на учет в налоговом органе;
&& инвестиционная декларация (с приложением инвестиционного проекта);
&& иные документы, подтверждающие соответствие требованиям к региональным инвестиционным проектам и (или) их участникам, установленным настоящим Кодексом и (или) законами соответствующих субъектов Российской Федерации.
& В случае реализации регионального инвестиционного проекта на территориях нескольких субъектов Российской Федерации в соответствии с \ul{пунктом 2 статьи 25.8} настоящего Кодекса заявление о включении в реестр подается организацией в уполномоченный орган государственной власти того субъекта Российской Федерации, в котором организация стоит на учете в налоговом органе по месту своего нахождения.
& В случае, если документы, указанные в \ul{подпунктах 2} и \ul{3 пункта 1} настоящей статьи, не представлены организацией, то по межведомственному запросу уполномоченного органа государственной власти субъекта Российской Федерации федеральный орган исполнительной власти, осуществляющий государственную регистрацию юридических лиц, физических лиц в качестве индивидуальных предпринимателей и крестьянских (фермерских) хозяйств, представляет сведения, подтверждающие факт внесения записи о государственной регистрации этой организации в Единый государственный реестр юридических лиц, а федеральный орган исполнительной власти, уполномоченный по контролю и надзору в области налогов и сборов, представляет сведения, подтверждающие факт постановки такой организации на учет в налоговом органе.
& Сведения, подтверждающие соответствие организации требованиям, установленным \ul{подпунктами 1--6 пункта 1 статьи 25.9} настоящего Кодекса, представляются федеральным органом исполнительной власти, уполномоченным по контролю и надзору в области налогов и сборов, по межведомственному запросу уполномоченного органа государственной власти субъекта Российской Федерации.
& Уполномоченный орган государственной власти субъекта Российской Федерации проверяет соответствие документов, приложенных к заявлению о включении в реестр, перечню документов, указанных в \ul{пункте 1} настоящей статьи, в срок не более чем три рабочих дня со дня их представления в указанный уполномоченный орган и на основании результатов указанной проверки направляет организации одно из следующих решений:
&& о принятии указанного заявления к рассмотрению;
&& об отказе в принятии указанного заявления к рассмотрению в случае непредставления документов, указанных в \ul{подпунктах 1}, \ul{4} и \ul{5 пункта 1} настоящей статьи.
& Если иное не предусмотрено настоящим пунктом, в течение тридцати дней со дня направления решения о принятии указанного в \ul{пункте 1} настоящей статьи заявления к рассмотрению уполномоченный орган государственной власти субъекта Российской Федерации в порядке, установленном законом субъекта Российской Федерации, принимает решение о включении организации в реестр или об отказе во включении организации в реестр в случае несоблюдения требований, установленных к региональным инвестиционным проектам, и не позднее пяти дней со дня принятия соответствующего решения направляет его организации.
\par В случае реализации регионального инвестиционного проекта на территориях нескольких субъектов Российской Федерации в соответствии с \ul{пунктом 2 статьи 25.8} настоящего Кодекса уполномоченный орган государственной власти субъекта Российской Федерации, принявший заявление о включении в реестр к рассмотрению, по согласованию с уполномоченными органами государственной власти субъектов Российской Федерации, на территориях которых реализуется региональный инвестиционный проект, принимает одно из решений, указанных в абзаце первом настоящего пункта, в течение сорока дней со дня направления организации решения о принятии заявления о включении в реестр к рассмотрению.
& Включение организации в реестр производится с 1-го числа календарного месяца, следующего за месяцем, в котором было принято соответствующее решение.
\eEasyList
\subsubsection{{\bf Статья 25.12.} Внесение изменений в сведения, содержащиеся в реестре, и прекращение статуса участника регионального инвестиционного проекта}
\beginEasyList
& Решение о внесении изменений в реестр, не связанных с прекращением статуса участника регионального инвестиционного проекта, принимается в случае внесения изменений в инвестиционную декларацию в порядке и на условиях, которые устанавливаются законом субъекта Российской Федерации в соответствии с настоящей статьей, при условии соблюдения требований, предъявляемых к региональным инвестиционным проектам и (или) их участникам, установленных настоящим \ul{Кодексом} и (или) законами соответствующих субъектов Российской Федерации.
& Внесение в инвестиционную декларацию изменений, касающихся условий реализации регионального инвестиционного проекта, осуществляется уполномоченным органом государственной власти субъекта Российской Федерации на основании заявления участника регионального инвестиционного проекта, составленного в произвольной форме, содержащего обоснование необходимости внесения таких изменений, в порядке, предусмотренном \ul{статьей 25.11} настоящего Кодекса для включения организации в реестр.
& Основаниями для отказа во внесении изменений в инвестиционную декларацию являются:
&& изменение цели регионального инвестиционного проекта;
&& снижение общего объема финансирования регионального инвестиционного проекта в совокупности более чем на 10 процентов по отношению к уровню, заявленному в первоначальной инвестиционной декларации;
&& изменение графика ежегодного объема инвестиций, исключающее возможность реализации регионального инвестиционного проекта с соблюдением установленных требований;
&& в результате вносимых изменений региональный инвестиционный проект перестанет удовлетворять иным требованиям, предусмотренным настоящим Кодексом и (или) законами соответствующих субъектов Российской Федерации.
& Статус участника регионального инвестиционного проекта подлежит прекращению:
&& на основании заявления участника регионального инвестиционного проекта о прекращении статуса участника регионального инвестиционного проекта --- со дня, указанного в заявлении;
&& на основании вступившего в силу решения по результатам налоговой проверки, проведенной в порядке, установленном настоящим Кодексом, выявившей несоответствие регионального инвестиционного проекта и (или) его участника требованиям, установленным настоящим Кодексом и (или) законодательством субъекта Российской Федерации, --- со дня включения организации в реестр;
&& в случае внесения в Единый государственный реестр юридических лиц записи о том, что организация --- участник регионального инвестиционного проекта находится в процессе ликвидации, --- со дня, следующего за днем внесения соответствующей записи в Единый государственный реестр юридических лиц;
&& в случае прекращения деятельности организации --- участника регионального инвестиционного проекта в результате реорганизации в форме слияния, разделения, присоединения к другому юридическому лицу или преобразования --- со дня, следующего за днем внесения соответствующей записи в Единый государственный реестр юридических лиц;
&& на основании вступившего в законную силу решения арбитражного суда о признании должника банкротом --- со дня, следующего за днем вступления в законную силу такого решения.
\eEasyList
\subsection{{\bf Глава 4. Представительство в отношениях, регулируемых}\\{\bf законодательством о налогах и сборах}}
\subsubsection{{\bf Статья 26.} Право на представительство в отношениях, регулируемых законодательством о налогах и сборах}
\beginEasyList
& Налогоплательщик может участвовать в отношениях, регулируемых законодательством о налогах и сборах через законного или уполномоченного представителя, если иное не предусмотрено настоящим Кодексом.
& Личное участие налогоплательщика в отношениях, регулируемых законодательством о налогах и сборах, не лишает его права иметь представителя, равно как участие представителя не лишает налогоплательщика права на личное участие в указанных правоотношениях.
& Полномочия представителя должны быть документально подтверждены в соответствии с настоящим Кодексом и иными федеральными законами.
& Правила, предусмотренные настоящей главой, распространяются на плательщиков сборов и налоговых агентов.
\eEasyList
\subsubsection{{\bf Статья 27.} Законный представитель налогоплательщика}
\beginEasyList
& Законными представителями налогоплательщика --- организации признаются лица, уполномоченные представлять указанную организацию на основании закона или ее учредительных документов.
& Законными представителями налогоплательщика --- физического лица признаются лица, выступающие в качестве его представителей в соответствии с гражданским законодательством Российской Федерации.
\eEasyList
\subsubsection{{\bf Статья 28.} Действия (бездействие) законных представителей организации}
\par Действия (бездействие) законных представителей организации, совершенные в связи с участием этой организации в отношениях, регулируемых законодательством о налогах и сборах, признаются действиями (бездействием) этой организации.
\subsubsection{{\bf Статья 29.} Уполномоченный представитель налогоплательщика}
\beginEasyList
& Уполномоченным представителем налогоплательщика признается физическое или юридическое лицо, уполномоченное налогоплательщиком представлять его интересы в отношениях с налоговыми органами (таможенными органами), иными участниками отношений, регулируемых законодательством о налогах и сборах.
& Не могут быть уполномоченными представителями налогоплательщика должностные лица налоговых органов, таможенных органов, органов внутренних дел, судьи, следователи и прокуроры.
& Уполномоченный представитель налогоплательщика --- \ul{организации} осуществляет свои полномочия на основании доверенности, выдаваемой в порядке, установленном гражданским законодательством Российской Федерации, если \ul{иное} не предусмотрено настоящим Кодексом.
\par Уполномоченный представитель налогоплательщика --- \ul{физического лица} осуществляет свои полномочия на основании нотариально удостоверенной доверенности или доверенности, приравненной к нотариально удостоверенной в соответствии с гражданским законодательством Российской Федерации.
& Ответственный участник консолидированной группы налогоплательщиков является уполномоченным представителем всех участников консолидированной группы налогоплательщиков на основании закона. Независимо от положений договора о создании консолидированной группы налогоплательщиков ответственный участник этой группы вправе представлять интересы участников указанной консолидированной группы в следующих правоотношениях:
&& в правоотношениях, связанных с регистрацией в налоговых органах договора о создании консолидированной группы налогоплательщиков, а также изменений указанного договора, решения о продлении срока действия договора и его прекращения;
&& в правоотношениях, связанных с принудительным взысканием с участника консолидированной группы налогоплательщиков недоимки по налогу на прибыль организаций по консолидированной группе налогоплательщиков;
&& в правоотношениях, связанных с привлечением организации к ответственности за налоговые правонарушения, совершенные в связи с участием в консолидированной группе налогоплательщиков;
&& в других случаях, когда по характеру совершаемых налоговым органом действий (бездействия) они непосредственно затрагивают права организации, являющейся участником консолидированной группы налогоплательщиков.
& По окончании срока действия, при досрочном расторжении или прекращении договора о создании консолидированной группы налогоплательщиков лицо, являвшееся ответственным участником этой группы, сохраняет полномочия, предусмотренные \ul{пунктом 4} настоящей статьи.
& Лицо, являющееся ответственным участником консолидированной группы налогоплательщиков, вправе делегировать предоставленные ему настоящим Кодексом полномочия по представлению интересов участников этой группы третьим лицам на основании доверенности, выданной в порядке, установленном гражданским законодательством Российской Федерации.
\eEasyList
\section{{\bf Раздел III. Налоговые органы. Таможенные органы. Финансовые органы. Органы внутренних дел. Следственные органы. Ответственность налоговых органов, таможенных органов, органов внутренних дел, следственных органов, их должностных лиц}}
\subsection{{\bf Глава 5. Налоговые органы. Таможенные органы. Финансовые органы. Ответственность налоговых органов, таможенных органов, их должностных лиц}}
\subsubsection{{\bf Статья 30.} Налоговые органы в Российской Федерации}
\beginEasyList
& Налоговые органы составляют единую централизованную систему контроля за соблюдением законодательства о налогах и сборах, за правильностью исчисления, полнотой и своевременностью уплаты (перечисления) в бюджетную систему Российской Федерации налогов и сборов, а в случаях, предусмотренных законодательством Российской Федерации, за правильностью исчисления, полнотой и своевременностью уплаты (перечисления) в бюджетную систему Российской Федерации иных обязательных платежей. В указанную систему входят федеральный орган исполнительной власти, уполномоченный по контролю и надзору в области налогов и сборов, и его территориальные органы.
& Утратил силу.
& Налоговые органы действуют в пределах своей компетенции и в соответствии с законодательством Российской Федерации.
& Налоговые органы осуществляют свои функции и взаимодействуют с федеральными органами исполнительной власти, органами исполнительной власти субъектов Российской Федерации, органами местного самоуправления и государственными внебюджетными фондами посредством реализации полномочий, предусмотренных настоящим Кодексом и иными нормативными правовыми актами Российской Федерации.
\eEasyList
\subsubsection{{\bf Статья 31.} Права налоговых органов}
\beginEasyList
& Налоговые органы вправе:
&& требовать в соответствии с законодательством о налогах и сборах от налогоплательщика, плательщика сбора или налогового агента документы по формам и (или) форматам в электронной форме, установленным государственными органами и органами местного самоуправления, служащие основаниями для исчисления и уплаты (удержания и перечисления) налогов, сборов, а также документы, подтверждающие правильность исчисления и своевременность уплаты (удержания и перечисления) налогов, сборов;
&& проводить налоговые проверки в порядке, установленном настоящим \ul{Кодексом};
&& производить выемку документов у налогоплательщика, плательщика сбора или налогового агента при проведении налоговых проверок в случаях, когда есть достаточные основания полагать, что эти документы будут уничтожены, сокрыты, изменены или заменены;
&& вызывать на основании письменного уведомления в налоговые органы налогоплательщиков, плательщиков сборов или налоговых агентов для дачи пояснений в связи с уплатой (удержанием и перечислением) ими налогов и сборов либо в связи с налоговой проверкой, а также в иных случаях, связанных с исполнением ими законодательства о налогах и сборах;
&& приостанавливать операции по счетам налогоплательщика, плательщика сбора или налогового агента в банках и налагать арест на имущество налогоплательщика, плательщика сбора или налогового агента в порядке, предусмотренном настоящим Кодексом;
&& в порядке, предусмотренном \ul{статьей 92} настоящего Кодекса, осматривать любые используемые налогоплательщиком для извлечения дохода либо связанные с содержанием объектов налогообложения независимо от места их нахождения производственные, складские, торговые и иные помещения и территории, проводить инвентаризацию принадлежащего налогоплательщику имущества. Порядок проведения инвентаризации имущества налогоплательщика при налоговой проверке утверждается Министерством финансов Российской Федерации;
&& определять суммы налогов, подлежащие уплате налогоплательщиками в бюджетную систему Российской Федерации, расчетным путем на основании имеющейся у них информации о налогоплательщике, а также данных об иных аналогичных налогоплательщиках в случаях отказа налогоплательщика допустить должностных лиц налогового органа к осмотру производственных, складских, торговых и иных помещений и территорий, используемых налогоплательщиком для извлечения дохода либо связанных с содержанием объектов налогообложения, непредставления в течение более двух месяцев налоговому органу необходимых для расчета налогов документов, отсутствия учета доходов и расходов, учета объектов налогообложения или ведения учета с нарушением установленного порядка, приведшего к невозможности исчислить налоги;
&& требовать от налогоплательщиков, плательщиков сборов, налоговых агентов, их представителей устранения выявленных нарушений законодательства о налогах и сборах и контролировать выполнение указанных требований;
&& взыскивать недоимки, а также пени, проценты и штрафы в случаях и порядке, которые установлены настоящим \ul{Кодексом};
&& требовать от банков документы, подтверждающие факт списания со счетов налогоплательщика, плательщика сбора или налогового агента и с корреспондентских счетов банков сумм налогов, сборов, пеней и штрафов и перечисления этих сумм в бюджетную систему Российской Федерации;
&& привлекать для проведения налогового контроля специалистов, экспертов и переводчиков;
&& вызывать в качестве свидетелей лиц, которым могут быть известны какие-либо обстоятельства, имеющие значение для проведения налогового контроля;
&& заявлять ходатайства об аннулировании или о приостановлении действия выданных юридическим и физическим лицам лицензий на право осуществления определенных видов деятельности;
&& предъявлять в суды общей юрисдикции или арбитражные суды иски (заявления):
\par о взыскании недоимки, пеней и штрафов за налоговые правонарушения в случаях, предусмотренных настоящим Кодексом;
\par о возмещении ущерба, причиненного государству и (или) муниципальному образованию вследствие неправомерных действий банка по списанию денежных средств со счета налогоплательщика после получения решения налогового органа о приостановлении операций, в результате которых стало невозможным взыскание налоговым органом недоимки, задолженности по пеням, штрафам с налогоплательщика в порядке, предусмотренном настоящим Кодексом;
\par о \ul{досрочном расторжении договора} об инвестиционном налоговом кредите;
\par в иных случаях, предусмотренных настоящим Кодексом.
& Налоговые органы осуществляют также другие права, предусмотренные настоящим Кодексом.
& Вышестоящие налоговые органы вправе отменять и изменять решения нижестоящих налоговых органов в случае несоответствия указанных решений законодательству о налогах и сборах.
& Формы и форматы документов, предусмотренных настоящим Кодексом и используемых налоговыми органами при реализации своих полномочий в отношениях, регулируемых законодательством о налогах и сборах, документов, необходимых для обеспечения электронного документооборота в отношениях, регулируемых законодательством о налогах и сборах, а также порядок заполнения форм указанных документов и порядок направления и получения таких документов на бумажном носителе или в электронной форме по телекоммуникационным каналам связи утверждаются федеральным органом исполнительной власти, уполномоченным по контролю и надзору в области налогов и сборов, если полномочия по их утверждению не возложены настоящим Кодексом на иной федеральный орган исполнительной власти.
\par Документы, используемые налоговыми органами при реализации своих полномочий в отношениях, регулируемых законодательством о налогах и сборах, могут быть переданы налоговым органом лицу, которому они адресованы, или его представителю непосредственно под расписку, направлены по почте заказным письмом или переданы в электронной форме по телекоммуникационным каналам связи через оператора электронного документооборота, если порядок их передачи прямо не предусмотрен настоящим Кодексом. Лицам, на которых настоящим Кодексом возложена обязанность представлять налоговую декларацию (расчет) в электронной форме, указанные документы передаются налоговым органом в электронной форме по телекоммуникационным каналам связи через оператора электронного документооборота.
\par В случаях направления документа налоговым органом по почте заказным письмом датой его получения считается шестой день со дня отправки заказного письма.
& В случае направления документов, которые используются налоговыми органами при реализации своих полномочий в отношениях, регулируемых законодательством о налогах и сборах, по почте такие документы направляются налоговым органом:
\par налогоплательщику --- российской организации (ее филиалу, представительству) --- по адресу места ее нахождения (места нахождения ее филиала, представительства), содержащемуся в Едином государственном реестре юридических лиц;
\par налогоплательщику --- иностранной организации --- по адресу места осуществления им деятельности на территории Российской Федерации, содержащемуся в Едином государственном реестре налогоплательщиков;
\par налогоплательщику --- индивидуальному предпринимателю, нотариусу, занимающемуся частной практикой, адвокату, учредившему адвокатский кабинет, физическому лицу, не являющемуся индивидуальным предпринимателем, --- по адресу места его жительства (места пребывания) или по предоставленному налоговому органу адресу для направления документов, указанных в настоящем пункте, содержащемуся в Едином государственном реестре налогоплательщиков.
\par Форма заявления о предоставлении налогоплательщиком --- индивидуальным предпринимателем, нотариусом, занимающимся частной практикой, адвокатом, учредившим адвокатский кабинет, физическим лицом, не являющимся индивидуальным предпринимателем, налоговому органу адреса для направления по почте документов, которые используются налоговыми органами при реализации своих полномочий в отношениях, регулируемых законодательством о налогах и сборах, утверждается федеральным органом исполнительной власти, уполномоченным по контролю и надзору в области налогов и сборов.
\eEasyList
\subsubsection{{\bf Статья 32.} Обязанности налоговых органов}
\beginEasyList
& Налоговые органы обязаны:
&& соблюдать законодательство о налогах и сборах;
&& осуществлять контроль за соблюдением законодательства о налогах и сборах, а также принятых в соответствии с ним нормативных правовых актов;
&& вести в установленном порядке учет организаций и физических лиц;
&& бесплатно информировать (в том числе в письменной форме) налогоплательщиков, плательщиков сборов и налоговых агентов о действующих налогах и сборах, законодательстве о налогах и сборах и о принятых в соответствии с ним нормативных правовых актах, порядке исчисления и уплаты налогов и сборов, правах и обязанностях налогоплательщиков, плательщиков сборов и налоговых агентов, полномочиях налоговых органов и их должностных лиц, а также представлять формы налоговых деклараций (расчетов) и разъяснять порядок их заполнения;
&& руководствоваться письменными разъяснениями Министерства финансов Российской Федерации по вопросам применения законодательства Российской Федерации о налогах и сборах;
&& сообщать налогоплательщикам, плательщикам сборов и налоговым агентам при их постановке на учет в налоговых органах сведения о реквизитах соответствующих счетов Федерального казначейства, а также в порядке, определяемом федеральным органом исполнительной власти, уполномоченным по контролю и надзору в области налогов и сборов, доводить до налогоплательщиков, плательщиков сборов и налоговых агентов сведения об изменении реквизитов этих счетов и иные сведения, необходимые для заполнения поручений на перечисление налогов, сборов, пеней и штрафов в бюджетную систему Российской Федерации;
&& принимать решения о возврате налогоплательщику, плательщику сбора или налоговому агенту сумм излишне уплаченных или излишне взысканных налогов, сборов, пеней и штрафов, направлять оформленные на основании этих решений поручения соответствующим территориальным органам Федерального казначейства для исполнения и осуществлять зачет сумм излишне уплаченных или излишне взысканных налогов, сборов, пеней и штрафов в порядке, предусмотренном настоящим Кодексом;
&& соблюдать \ul{налоговую тайну} и обеспечивать ее сохранение;
&& направлять налогоплательщику, плательщику сбора или налоговому агенту копии акта налоговой проверки и решения налогового органа, а также в случаях, предусмотренных настоящим Кодексом, налоговое уведомление и (или) требование об уплате налога и сбора;
&& представлять налогоплательщику, плательщику сбора или налоговому агенту по его запросу справки о состоянии расчетов указанного лица по налогам, сборам, пеням, штрафам, процентам и справки об исполнении обязанности по уплате налогов, сборов, пеней, штрафов, процентов на основании данных налогового органа.
\par Справка о состоянии расчетов по налогам, сборам, пеням, штрафам, процентам передается (направляется) указанному лицу (его представителю) в течение пяти дней, справка об исполнении обязанности по уплате налогов, сборов, пеней, штрафов, процентов --- в течение десяти дней со дня поступления в налоговый орган соответствующего запроса;
&&& представлять ответственному участнику консолидированной группы налогоплательщиков по его запросу, направленному в пределах предоставленных ему полномочий, справки о состоянии расчетов консолидированной группы налогоплательщиков по налогу на прибыль организаций;
&& осуществлять по заявлению налогоплательщика, ответственного участника консолидированной группы налогоплательщиков, плательщика сбора или налогового агента совместную сверку расчетов по налогам, сборам, пеням, штрафам, процентам. Результаты совместной сверки расчетов по налогам, сборам, пеням, штрафам, процентам оформляются актом. Акт совместной сверки расчетов по налогам, сборам, пеням, штрафам, процентам вручается (направляется по почте заказным письмом) или передается налогоплательщику (ответственному участнику консолидированной группы налогоплательщиков, плательщику сбора, налоговому агенту) в электронной форме по телекоммуникационным каналам связи в течение следующего дня после дня составления такого акта.
\par Порядок проведения совместной сверки расчетов по налогам, сборам, пеням, штрафам, процентам, форма и формат акта совместной сверки расчетов по налогам, сборам, пеням, штрафам, процентам, а также порядок его передачи в электронной форме по телекоммуникационным каналам связи утверждаются федеральным органом исполнительной власти, уполномоченным по контролю и надзору в области налогов и сборов;
&& по заявлению налогоплательщика, плательщика сбора или налогового агента выдавать копии решений, принятых налоговым органом в отношении этого налогоплательщика, плательщика сбора или налогового агента;
&& по заявлению ответственного участника консолидированной группы налогоплательщиков выдавать копии решений, принятых налоговым органом в отношении консолидированной группы налогоплательщиков;
&& представлять пользователям выписки из Единого государственного реестра налогоплательщиков.
& Налоговые органы несут также другие обязанности, предусмотренные настоящим Кодексом и иными федеральными законами.
& Если в течение двух месяцев со дня истечения \ul{срока} исполнения требования об уплате налога (сбора), направленного налогоплательщику (плательщику сбора, налоговому агенту) на основании решения о привлечении к ответственности за совершение налогового правонарушения, налогоплательщик (плательщик сбора, налоговый агент) не уплатил (не перечислил) в полном объеме указанные в данном требовании суммы недоимки, размер которой позволяет предполагать факт совершения нарушения законодательства о налогах и сборах, содержащего признаки преступления, соответствующих пеней и штрафов, налоговые органы обязаны в течение 10 дней со дня выявления указанных обстоятельств направить материалы в следственные органы, уполномоченные производить предварительное следствие по уголовным делам о преступлениях, предусмотренных статьями 198-199.2 Уголовного кодекса Российской Федерации (далее --- следственные органы), для решения вопроса о возбуждении уголовного дела.
\eEasyList
\subsubsection{{\bf Статья 33.} Обязанности должностных лиц налоговых органов}
\par Должностные лица налоговых органов обязаны:
\beginEasyList
&& действовать в строгом соответствии с настоящим Кодексом и иными федеральными законами;
&& реализовывать в пределах своей компетенции права и обязанности налоговых органов;
&& корректно и внимательно относиться к налогоплательщикам, их представителям и иным участникам отношений, регулируемых законодательством о налогах и сборах, не унижать их честь и достоинство.
\eEasyList
\subsubsection{{\bf Статья 34.} Полномочия таможенных органов и обязанности их должностных лиц в области налогообложения и сборов}
\beginEasyList
& Таможенные органы пользуются правами и несут обязанности налоговых органов по взиманию налогов при перемещении товаров через таможенную границу Таможенного союза в соответствии с таможенным законодательством Таможенного союза и законодательством Российской Федерации о таможенном деле, настоящим Кодексом, иными федеральными законами о налогах, а также иными федеральными законами.
& Должностные лица таможенных органов несут обязанности, предусмотренные \ul{статьей 33} настоящего Кодекса, а также другие обязанности в соответствии с таможенным законодательством Таможенного союза и законодательством Российской Федерации о таможенном деле.
& Утратил силу с 1 января 2005 г.
\eEasyList
\subsubsection{{\bf Статья 34.1.} Утратила силу.}
\subsubsection{{\bf Статья 34.2.} Полномочия финансовых органов в области налогов и сборов}
\beginEasyList
& Министерство финансов Российской Федерации дает письменные разъяснения налоговым органам, налогоплательщикам, ответственному участнику консолидированной группы налогоплательщиков, плательщикам сборов и налоговым агентам по вопросам применения законодательства Российской Федерации о налогах и сборах.
& Финансовые органы субъектов Российской Федерации и муниципальных образований дают письменные разъяснения налогоплательщикам и налоговым агентам по вопросам применения соответственно законодательства субъектов Российской Федерации о налогах и сборах и нормативных правовых актов муниципальных образований о местных налогах и сборах.
& Министерство финансов Российской Федерации, финансовые органы субъектов Российской Федерации и муниципальных образований дают письменные разъяснения в пределах своей компетенции в течение двух месяцев со дня поступления соответствующего запроса. По решению руководителя (заместителя руководителя) соответствующего финансового органа указанный срок может быть продлен, но не более чем на один месяц.
\eEasyList
\subsubsection{{\bf Статья 35.} Ответственность налоговых органов, таможенных органов, а также их должностных лиц}
\beginEasyList
& Налоговые и таможенные органы несут ответственность за убытки, причиненные налогоплательщикам, плательщикам сборов и налоговым агентам вследствие своих неправомерных действий (решений) или бездействия, а равно неправомерных действий (решений) или бездействия должностных лиц и других работников указанных органов при исполнении ими служебных обязанностей.
\par Причиненные налогоплательщикам, плательщикам сборов и налоговым агентам убытки возмещаются за счет федерального бюджета в порядке, предусмотренном настоящим Кодексом и иными федеральными законами.
& Утратил силу.
& За неправомерные действия или бездействие должностные лица и другие работники органов, указанных в пункте 1 настоящей статьи, несут ответственность в соответствии с законодательством Российской Федерации.
\eEasyList
\subsection{{\bf Глава 6. Органы внутренних дел. Следственные органы}}
\subsubsection{{\bf Статья 36.} Полномочия органов внутренних дел, следственных органов}
\beginEasyList
& По запросу налоговых органов органы внутренних дел участвуют вместе с налоговыми органами в проводимых налоговыми органами выездных налоговых проверках.
& При выявлении обстоятельств, требующих совершения действий, отнесенных настоящим Кодексом к полномочиям налоговых органов, органы внутренних дел, следственные органы обязаны в десятидневный срок со дня выявления указанных обстоятельств направить материалы в соответствующий налоговый орган для принятия по ним решения.
\eEasyList
\subsubsection{{\bf Статья 37.} Ответственность органов внутренних дел, следственных органов и их должностных лиц}
\beginEasyList
& Органы внутренних дел, следственные органы несут ответственность за убытки, причиненные налогоплательщикам, плательщикам сборов и налоговым агентам вследствие своих неправомерных действий (решений) или бездействия, а равно неправомерных действий (решений) или бездействия должностных лиц и других работников этих органов при исполнении ими служебных обязанностей.
\par Причиненные налогоплательщикам, плательщикам сборов и налоговым агентам при проведении мероприятий, предусмотренных \ul{пунктом 1 статьи 36} настоящего Кодекса, убытки возмещаются за счет федерального бюджета в порядке, предусмотренном настоящим Кодексом и иными федеральными законами.
& За неправомерные действия или бездействие должностные лица и другие работники органов внутренних дел, следственных органов несут ответственность в соответствии с законодательством Российской Федерации.
\eEasyList
\section{{\bf Раздел IV. Общие правила исполнения обязанности}\\{\bf по уплате налогов и сборов}}
\subsection{{\bf Глава 7. Объекты налогообложения}}
\subsubsection{{\bf Статья 38.} Объект налогообложения}
\beginEasyList
& Объект налогообложения --- реализация товаров (работ, услуг), имущество, прибыль, доход, расход или иное обстоятельство, имеющее стоимостную, количественную или физическую характеристику, с наличием которого законодательство о налогах и сборах связывает возникновение у налогоплательщика обязанности по уплате налога.
\par Каждый налог имеет самостоятельный объект налогообложения, определяемый в соответствии с \ul{частью второй} настоящего Кодекса и с учетом положений настоящей статьи.
& Под имуществом в настоящем Кодексе понимаются виды объектов гражданских прав (за исключением имущественных прав), относящихся к имуществу в соответствии с Гражданским кодексом Российской Федерации.
& Товаром для целей настоящего Кодекса признается любое имущество, реализуемое либо предназначенное для реализации. В целях регулирования отношений, связанных с взиманием таможенных платежей, к товарам относится и иное имущество, определяемое в соответствии с таможенным законодательством Таможенного союза и законодательством Российской Федерации о таможенном деле.
& Работой для целей налогообложения признается деятельность, результаты которой имеют материальное выражение и могут быть реализованы для удовлетворения потребностей \ul{организации} и (или) \ul{физических лиц.}
& Услугой для целей налогообложения признается деятельность, результаты которой не имеют материального выражения, реализуются и потребляются в процессе осуществления этой деятельности.
& Идентичными товарами (работами, услугами) в целях настоящего Кодекса признаются товары (работы, услуги), имеющие одинаковые характерные для них основные признаки. При определении идентичности товаров незначительные различия во внешнем виде таких товаров могут не учитываться.
\par При определении идентичности товаров учитываются их физические характеристики, качество, функциональное назначение, страна происхождения и производитель, его деловая репутация на рынке и используемый товарный знак.
\par При определении идентичности работ (услуг) учитываются характеристики подрядчика (исполнителя), его деловая репутация на рынке и используемый товарный знак.
& Однородными товарами в целях настоящего Кодекса признаются товары, которые, не являясь идентичными, имеют сходные характеристики и состоят из схожих компонентов, что позволяет им выполнять одни и те же функции и (или) быть коммерчески взаимозаменяемыми. При определении однородности товаров учитываются их качество, репутация на рынке, товарный знак, страна происхождения.
\par Однородными работами (услугами) признаются работы (услуги), которые, не являясь идентичными, имеют сходные характеристики, что позволяет им быть коммерчески и (или) функционально взаимозаменяемыми. При определении однородности работ (услуг) учитываются их качество, товарный знак, репутация на рынке, а также вид работ (услуг), их объем, уникальность и коммерческая взаимозаменяемость.
\eEasyList
\subsubsection{{\bf Статья 39.} Реализация товаров, работ или услуг}
\beginEasyList
& Реализацией товаров, работ или услуг \ul{организацией} или \ul{индивидуальным предпринимателем} признается соответственно передача на возмездной основе (в том числе обмен товарами, работами или услугами) права собственности на товары, результатов выполненных работ одним лицом для другого лица, возмездное оказание услуг одним лицом другому лицу, а в случаях, предусмотренных настоящим Кодексом, передача права собственности на товары, результатов выполненных работ одним лицом для другого лица, оказание услуг одним лицом другому лицу --- на безвозмездной основе.
& Место и момент фактической реализации товаров, работ или услуг определяются в соответствии с \ul{частью второй} настоящего Кодекса.
& Не признается реализацией товаров, работ или услуг:
&& осуществление операций, связанных с обращением российской или иностранной валюты (за исключением целей нумизматики);
&& передача основных средств, нематериальных активов и (или) иного имущества организации ее правопреемнику (правопреемникам) при реорганизации этой организации;
&& передача основных средств, нематериальных активов и (или) иного имущества некоммерческим организациям на осуществление основной уставной деятельности, не связанной с предпринимательской деятельностью;
&& передача имущества, если такая передача носит инвестиционный характер (в частности, вклады в уставный (складочный) капитал хозяйственных обществ и товариществ, вклады по договору простого товарищества (договору о совместной деятельности), договору инвестиционного товарищества, паевые взносы в паевые фонды кооперативов);
&&& передача имущества и (или) имущественных прав по концессионному соглашению в соответствии с законодательством Российской Федерации;
&& передача имущества в пределах первоначального взноса участнику хозяйственного общества или товарищества (его правопреемнику или наследнику) при выходе (выбытии) из хозяйственного общества или товарищества, а также при распределении имущества ликвидируемого хозяйственного общества или товарищества между его участниками;
&& передача имущества в пределах первоначального взноса участнику договора простого товарищества (договора о совместной деятельности), договора инвестиционного товарищества или его правопреемнику в случае выдела его доли из имущества, находящегося в общей собственности участников договора, или раздела такого имущества;
&& передача жилых помещений \ul{физическим лицам} в домах государственного или муниципального жилищного фонда при проведении приватизации;
&& изъятие имущества путем конфискации, наследование имущества, а также обращение в собственность иных лиц бесхозяйных и брошенных вещей, бесхозяйных животных, находки, клада в соответствии с нормами Гражданского кодекса Российской Федерации;
&&& передача имущества участникам хозяйственного общества или товарищества при распределении имущества и имущественных прав ликвидируемой организации, являющейся иностранным организатором XXII Олимпийских зимних игр и XI Паралимпийских зимних игр 2014 года в городе Сочи или маркетинговым партнером Международного олимпийского комитета в соответствии со статьей 3.1 Федерального закона от 1 декабря 2007 года N 310-ФЗ «Об организации и о проведении XXII Олимпийских зимних игр и XI Паралимпийских зимних игр 2014 года в городе Сочи, развитии города Сочи как горноклиматического курорта и внесении изменений в отдельные законодательные акты Российской Федерации». Настоящее положение применяется в случае, если создание и ликвидация организации, являющейся иностранным организатором XXII Олимпийских зимних игр и XI Паралимпийских зимних игр 2014 года в городе Сочи или маркетинговым партнером Международного олимпийского комитета в соответствии со статьей 3.1 указанного Федерального закона, осуществляются в период организации XXII Олимпийских зимних игр и XI Паралимпийских зимних игр 2014 года в городе Сочи, установленный частью 1 статьи 2 указанного Федерального закона;
&& иные операции в случаях, предусмотренных настоящим Кодексом.
\eEasyList
\subsubsection{{\bf Статья 40.} Принципы определения цены товаров, работ или услуг для целей налогообложения}
\beginEasyList
& Если иное не предусмотрено настоящей статьей, для целей налогообложения принимается цена товаров, работ или услуг, указанная сторонами сделки. Пока не доказано обратное, предполагается, что эта цена соответствует уровню рыночных цен.
& Налоговые органы при осуществлении контроля за полнотой исчисления налогов вправе проверять правильность применения цен по сделкам лишь в следующих случаях:
&& между \ul{взаимозависимыми лицами};
&& по товарообменным (бартерным) операциям;
&& при совершении внешнеторговых сделок;
&& при отклонении более чем на 20 процентов в сторону повышения или в сторону понижения от уровня цен, применяемых налогоплательщиком по идентичным (однородным) товарам (работам, услугам) в пределах непродолжительного периода времени.
& В случаях, предусмотренных \ul{пунктом 2} настоящей статьи, когда цены товаров, работ или услуг, примененные сторонами сделки, отклоняются в сторону повышения или в сторону понижения более чем на 20 процентов от рыночной цены идентичных (однородных) товаров (работ или услуг), налоговый орган вправе вынести мотивированное решение о доначислении налога и пени, рассчитанных таким образом, как если бы результаты этой сделки были оценены исходя из применения рыночных цен на соответствующие товары, работы или услуги.
\par Рыночная цена определяется с учетом положений, предусмотренных \ul{пунктами 4-11} настоящей статьи. При этом учитываются обычные при заключении сделок между невзаимозависимыми лицами надбавки к цене или скидки. В частности, учитываются скидки, вызванные:
\par сезонными и иными колебаниями потребительского спроса на товары (работы, услуги);
\par потерей товарами качества или иных потребительских свойств;
\par истечением (приближением даты истечения) сроков годности или реализации товаров;
\par маркетинговой политикой, в том числе при продвижении на рынки новых товаров, не имеющих аналогов, а также при продвижении товаров (работ, услуг) на новые рынки;
\par реализацией опытных моделей и образцов товаров в целях ознакомления с ними потребителей.
& Рыночной ценой товара (работы, услуги) признается цена, сложившаяся при взаимодействии спроса и предложения на рынке идентичных (а при их отсутствии --- однородных) товаров (работ, услуг) в сопоставимых экономических (коммерческих) условиях.
& Рынком товаров (работ, услуг) признается сфера обращения этих товаров (работ, услуг), определяемая исходя из возможности покупателя (продавца) реально и без значительных дополнительных затрат приобрести (реализовать) товар (работу, услугу) на ближайшей по отношению к покупателю (продавцу) территории Российской Федерации или за пределами Российской Федерации.
& Идентичными признаются товары, имеющие одинаковые характерные для них основные признаки.
\par При определении идентичности товаров учитываются, в частности, их физические характеристики, качество и репутация на рынке, страна происхождения и производитель. При определении идентичности товаров незначительные различия в их внешнем виде могут не учитываться.
& Однородными признаются товары, которые, не являясь идентичными, имеют сходные характеристики и состоят из схожих компонентов, что позволяет им выполнять одни и те же функции и (или) быть коммерчески взаимозаменяемыми.
\par При определении однородности товаров учитываются, в частности, их качество, наличие товарного знака, репутация на рынке, страна происхождения.
\par Абзац третий исключен.
& При определении рыночных цен товаров, работ или услуг принимаются во внимание сделки между лицами, не являющимися \ul{взаимозависимыми.} Сделки между взаимозависимыми лицами могут приниматься во внимание только в тех случаях, когда взаимозависимость этих лиц не повлияла на результаты таких сделок.
& При определении рыночных цен товара, работы или услуги учитывается информация о заключенных на момент реализации этого товара, работы или услуги сделках с идентичными (однородными) товарами, работами или услугами в сопоставимых условиях. В частности, учитываются такие условия сделок, как количество (объем) поставляемых товаров (например, объем товарной партии), сроки исполнения обязательств, условия платежей, обычно применяемые в сделках данного вида, а также иные разумные условия, которые могут оказывать влияние на цены.
\par При этом условия сделок на рынке идентичных (а при их отсутствии --- однородных) товаров, работ или услуг признаются сопоставимыми, если различие между такими условиями либо существенно не влияет на цену таких товаров, работ или услуг, либо может быть учтено с помощью поправок.
& При отсутствии на соответствующем рынке товаров, работ или услуг сделок по идентичным (однородным) товарам, работам, услугам или из-за отсутствия предложения на этом рынке таких товаров, работ или услуг, а также при невозможности определения соответствующих цен ввиду отсутствия либо недоступности информационных источников для определения рыночной цены используется метод цены последующей реализации, при котором рыночная цена товаров, работ или услуг, реализуемых продавцом, определяется как разность цены, по которой такие товары, работы или услуги реализованы покупателем этих товаров, работ или услуг при последующей их реализации (перепродаже), и обычных в подобных случаях затрат, понесенных этим покупателем при перепродаже (без учета цены, по которой были приобретены указанным покупателем у продавца товары, работы или услуги) и продвижении на рынок приобретенных у покупателя товаров, работ или услуг, а также обычной для данной сферы деятельности прибыли покупателя.
\par При невозможности использования метода цены последующей реализации (в частности, при отсутствии информации о цене товаров, работ или услуг, в последующем реализованных покупателем) используется затратный метод, при котором рыночная цена товаров, работ или услуг, реализуемых продавцом, определяется как сумма произведенных затрат и обычной для данной сферы деятельности прибыли. При этом учитываются обычные в подобных случаях прямые и косвенные затраты на производство (приобретение) и (или) реализацию товаров, работ или услуг, обычные в подобных случаях затраты на транспортировку, хранение, страхование и иные подобные затраты.
& При определении и признании рыночной цены товара, работы или услуги используются официальные источники информации о рыночных ценах на товары, работы или услуги и биржевых котировках.
& При рассмотрении дела суд вправе учесть любые обстоятельства, имеющие значение для определения результатов сделки, не ограничиваясь обстоятельствами, перечисленными в \ul{пунктах 4-11} настоящей статьи.
& При реализации товаров (работ, услуг) по государственным регулируемым ценам (тарифам), установленным в соответствии с законодательством Российской Федерации, для целей налогообложения принимаются указанные цены (тарифы).
& Положения, предусмотренные \ul{пунктами 3} и \ul{10} настоящей статьи, при определении рыночных цен финансовых инструментов срочных сделок и рыночных цен ценных бумаг применяются с учетом особенностей, предусмотренных \ul{главой 23} настоящего Кодекса «Налог на доходы физических лиц» и \ul{главой 25} настоящего Кодекса «Налог на прибыль организаций».
\eEasyList
\subsubsection{{\bf Статья 41.} Принципы определения доходов}
\par В соответствии с настоящим Кодексом доходом признается экономическая выгода в денежной или натуральной форме, учитываемая в случае возможности ее оценки и в той мере, в которой такую выгоду можно оценить, и определяемая в соответствии с главами \ul{«Налог на доходы физических лиц»}, \ul{«Налог на прибыль организаций»} настоящего Кодекса.
\subsubsection{{\bf Статья 42.} Доходы от источников в Российской Федерации и от источников за пределами Российской Федерации}
\beginEasyList
& Доходы налогоплательщика могут быть отнесены к доходам от источников в Российской Федерации или к доходам от источников за пределами Российской Федерации в соответствии с главами \ul{«Налог на прибыль организаций»}, \ul{«Налог на доходы физических лиц»} настоящего Кодекса.
& Если положения настоящего Кодекса не позволяют однозначно отнести полученные налогоплательщиком доходы к доходам от источников в Российской Федерации либо к доходам от источников за пределами Российской Федерации, отнесение дохода к тому или иному источнику осуществляется федеральным органом исполнительной власти, уполномоченным по контролю и надзору в области налогов и сборов. В аналогичном порядке в указанных доходах определяется доля, которая может быть отнесена к доходам от источников в Российской Федерации, и доли, которые могут быть отнесены к доходам от источников за пределами Российской Федерации.
\eEasyList
\subsubsection{{\bf Статья 43.} Дивиденды и проценты}
\beginEasyList
& Дивидендом признается любой доход, полученный акционером (участником) от \ul{организации} при распределении прибыли, остающейся после налогообложения (в том числе в виде процентов по привилегированным акциям), по принадлежащим акционеру (участнику) акциям (долям) пропорционально долям акционеров (участников) в уставном (складочном) капитале этой организации.
\par К дивидендам также относятся любые доходы, получаемые из источников за пределами Российской Федерации, относящиеся к дивидендам в соответствии с законодательствами иностранных государств.
& Не признаются дивидендами:
&& выплаты при ликвидации организации акционеру (участнику) этой организации в денежной или натуральной форме, не превышающие взноса этого акционера (участника) в уставный (складочный) капитал организации;
&& выплаты акционерам (участникам) организации в виде передачи акций этой же организации в собственность;
&& выплаты некоммерческой организации на осуществление ее основной уставной деятельности (не связанной с предпринимательской деятельностью), произведенные хозяйственными обществами, уставный капитал которых состоит полностью из вкладов этой некоммерческой организации.
& Процентами признается любой заранее заявленный (установленный) доход, в том числе в виде дисконта, полученный по долговому обязательству любого вида (независимо от способа его оформления). При этом процентами признаются, в частности, доходы, полученные по денежным вкладам и долговым обязательствам.
\eEasyList
\subsection{{\bf Глава 8. Исполнение обязанности по уплате налогов и сборов}}
\subsubsection{{\bf Статья 44.} Возникновение, изменение и прекращение обязанности по уплате налога или сбора}
\beginEasyList
& Обязанность по уплате налога или сбора возникает, изменяется и прекращается при наличии оснований, установленных настоящим Кодексом или иным актом законодательства о налогах и сборах.
& Обязанность по уплате конкретного налога или сбора возлагается на налогоплательщика и плательщика сбора с момента возникновения установленных законодательством о налогах и сборах обстоятельств, предусматривающих уплату данного налога или сбора.
& Обязанность по уплате налога и (или) сбора прекращается:
&& с уплатой налога и (или) сбора налогоплательщиком, плательщиком сбора и (или) участником консолидированной группы налогоплательщиков в случаях, предусмотренных настоящим Кодексом;
&& утратил силу с 1 января 2007 г.;
&& со смертью физического лица --- налогоплательщика или с объявлением его умершим в порядке, установленном гражданским процессуальным законодательством Российской Федерации. Задолженность по налогам, указанным в \ul{пункте 3 статьи 14} и \ul{статье 15} настоящего Кодекса, умершего лица либо лица, объявленного умершим, погашается наследниками в пределах стоимости наследственного имущества в порядке, установленном гражданским законодательством Российской Федерации для оплаты наследниками долгов наследодателя;
&& с ликвидацией организации-налогоплательщика после проведения всех расчетов с бюджетной системой Российской Федерации в соответствии со \ul{статьей 49} настоящего Кодекса;
&& с возникновением иных обстоятельств, с которыми законодательство о налогах и сборах связывает прекращение обязанности по уплате соответствующего налога или сбора.
\eEasyList
\subsubsection{{\bf Статья 45.} Исполнение обязанности по уплате налога или сбора}
\beginEasyList
& Налогоплательщик обязан самостоятельно исполнить обязанность по уплате налога, если иное не предусмотрено законодательством о налогах и сборах. Обязанность по уплате налога на прибыль организаций по консолидированной группе налогоплательщиков исполняется ответственным участником этой группы, если иное не предусмотрено настоящим Кодексом.
\par Обязанность по уплате налога должна быть выполнена в срок, установленный законодательством о налогах и сборах. Налогоплательщик либо в случаях, установленных настоящим Кодексом, участник консолидированной группы налогоплательщиков вправе исполнить обязанность по уплате налога досрочно.
\par Неисполнение или ненадлежащее исполнение обязанности по уплате налога является основанием для направления налоговым органом или таможенным органом налогоплательщику (ответственному участнику консолидированной группы налогоплательщиков) требования об уплате налога.
& В случае неуплаты или неполной уплаты налога в установленный срок производится взыскание налога в порядке, предусмотренном настоящим Кодексом.
\par Взыскание налога с организации или индивидуального предпринимателя производится в порядке, предусмотренном \ul{статьями 46} и \ul{47} настоящего Кодекса. Взыскание налога с физического лица, не являющегося индивидуальным предпринимателем, производится в порядке, предусмотренном \ul{статьей 48} настоящего Кодекса.
\par Взыскание налога в судебном порядке производится:
&& с организации, которой открыт лицевой счет;
&& в целях взыскания недоимки, возникшей по итогам проведенной налоговой проверки, числящейся более трех месяцев:
\par за организациями, являющимися в соответствии с гражданским законодательством Российской Федерации зависимыми (дочерними) обществами (предприятиями), --- с соответствующих основных (преобладающих, участвующих) обществ (предприятий), когда на их счета в банках поступает выручка за реализуемые товары (работы, услуги) зависимых (дочерних) обществ (предприятий);
\par за организациями, являющимися в соответствии с гражданским законодательством Российской Федерации основными (преобладающими, участвующими) обществами (предприятиями), --- с зависимых (дочерних) обществ (предприятий), когда на их счета в банках поступает выручка за реализуемые товары (работы, услуги) основных (преобладающих, участвующих) обществ (предприятий);
\par за организациями, являющимися в соответствии с гражданским законодательством Российской Федерации зависимыми (дочерними) обществами (предприятиями), --- с соответствующих основных (преобладающих, участвующих) обществ (предприятий), если с момента, когда организация, за которой числится недоимка, узнала или должна была узнать о назначении выездной налоговой проверки или о начале проведения камеральной налоговой проверки, произошла передача денежных средств, иного имущества основному (преобладающему, участвующему) обществу (предприятию) и если такая передача привела к невозможности взыскания указанной недоимки;
\par за организациями, являющимися в соответствии с гражданским законодательством Российской Федерации основными (преобладающими, участвующими) обществами (предприятиями), --- с зависимых (дочерних) обществ (предприятий), если с момента, когда организация, за которой числится недоимка, узнала или должна была узнать о назначении выездной налоговой проверки или о начале проведения камеральной налоговой проверки, произошла передача денежных средств, иного имущества зависимому (дочернему) обществу (предприятию) и если такая передача привела к невозможности взыскания указанной недоимки.
\par Если налоговым органом в указанных случаях будет установлено, что выручка за реализуемые товары (работы, услуги) поступает на счета нескольких организаций или если с момента, когда организация, за которой числится недоимка, узнала или должна была узнать о назначении выездной налоговой проверки или о начале проведения камеральной налоговой проверки, произошла передача денежных средств, иного имущества нескольким основным (преобладающим, участвующим) обществам (предприятиям), зависимым (дочерним) обществам (предприятиям), взыскание недоимки производится с соответствующих организаций пропорционально доле поступившей им выручки за реализуемые товары (работы, услуги), доле переданных денежных средств, стоимости иного имущества.
\par Положения настоящего подпункта также применяются, если налоговым органом в указанных случаях будет установлено, что перечисление выручки за реализуемые товары (работы, услуги), передача денежных средств, иного имущества основным (преобладающим, участвующим) обществам (предприятиям), зависимым (дочерним) обществам (предприятиям) были произведены через совокупность взаимосвязанных операций, в том числе в случае, если участники указанных операций не являются основными (преобладающими, участвующими) обществами (предприятиями), зависимыми (дочерними) обществами (предприятиями).
\par Положения настоящего подпункта также применяются, если налоговым органом в указанных случаях будет установлено, что перечисление выручки за реализуемые товары (работы, услуги), передача денежных средств, иного имущества производятся организациям, признанным судом иным образом зависимыми с налогоплательщиком, за которым числится недоимка.
\par При применении положений настоящего подпункта взыскание может производиться в пределах поступившей основным (преобладающим, участвующим) обществам (предприятиям), зависимым (дочерним) обществам (предприятиям), организациям, признанным судом иным образом зависимыми с налогоплательщиком, за которым числится недоимка, выручки за реализуемые товары (работы, услуги), переданных денежных средств, иного имущества.
\par Стоимость имущества в указанных в настоящем подпункте случаях определяется как остаточная стоимость имущества, отраженная в бухгалтерском учете организации на момент, когда организация, за которой числится недоимка, узнала или должна была узнать о назначении выездной налоговой проверки или о начале проведения камеральной налоговой проверки;
&& с организации или индивидуального предпринимателя, если их обязанность по уплате налога основана на изменении налоговым органом юридической квалификации сделки, совершенной таким налогоплательщиком, или статуса и характера деятельности этого налогоплательщика;
&& с организации или индивидуального предпринимателя, если их обязанность по уплате налога возникла по результатам проверки федеральным органом исполнительной власти, уполномоченным по контролю и надзору в области налогов и сборов, полноты исчисления и уплаты налогов в связи с совершением сделок между взаимозависимыми лицами.
& Обязанность по уплате налога считается исполненной налогоплательщиком либо в случаях, установленных настоящим Кодексом, участником консолидированной группы налогоплательщиков, если иное не предусмотрено \ul{пунктом 4} настоящей статьи:
&& с момента предъявления в банк поручения на перечисление в бюджетную систему Российской Федерации на соответствующий счет Федерального казначейства денежных средств со счета налогоплательщика в банке при наличии на нем достаточного денежного остатка на день платежа;
&&& с момента передачи физическим лицом в банк поручения на перечисление в бюджетную систему Российской Федерации на соответствующий счет Федерального казначейства без открытия счета в банке денежных средств, предоставленных банку физическим лицом, при условии их достаточности для перечисления;
&& с момента отражения на лицевом счете организации, которой открыт лицевой счет, операции по перечислению соответствующих денежных средств в бюджетную систему Российской Федерации;
&& со дня внесения физическим лицом в банк, кассу местной администрации либо в организацию федеральной почтовой связи наличных денежных средств для их перечисления в бюджетную систему Российской Федерации на соответствующий счет Федерального казначейства;
&& со дня вынесения налоговым органом в соответствии с настоящим Кодексом решения о зачете сумм излишне уплаченных или сумм излишне взысканных налогов, пеней, штрафов в счет исполнения обязанности по уплате соответствующего налога;
&& со дня удержания сумм налога налоговым агентом, если обязанность по исчислению и удержанию налога из денежных средств налогоплательщика возложена в соответствии с настоящим Кодексом на налогового агента;
&& со дня уплаты декларационного платежа в соответствии с федеральным законом об упрощенном порядке декларирования доходов физическими лицами.
& Обязанность по уплате налога не признается исполненной в следующих случаях:
&& отзыва налогоплательщиком или возврата банком налогоплательщику неисполненного поручения на перечисление соответствующих денежных средств в бюджетную систему Российской Федерации;
&& отзыва налогоплательщиком-организацией, которой открыт лицевой счет, или возврата органом Федерального казначейства (иным уполномоченным органом, осуществляющим открытие и ведение лицевых счетов) налогоплательщику неисполненного поручения на перечисление соответствующих денежных средств в бюджетную систему Российской Федерации;
&& возврата местной администрацией либо организацией федеральной почтовой связи налогоплательщику --- физическому лицу наличных денежных средств, принятых для их перечисления в бюджетную систему Российской Федерации;
&& неправильного указания налогоплательщиком в поручении на перечисление суммы налога номера счета Федерального казначейства и наименования банка получателя, повлекшего неперечисление этой суммы в бюджетную систему Российской Федерации на соответствующий счет Федерального казначейства;
&& если на день предъявления налогоплательщиком в банк (орган Федерального казначейства, иной уполномоченный орган, осуществляющий открытие и ведение лицевых счетов) поручения на перечисление денежных средств в счет уплаты налога этот налогоплательщик имеет иные неисполненные требования, которые предъявлены к его счету (лицевому счету) и в соответствии с гражданским законодательством Российской Федерации исполняются в первоочередном порядке, и если на этом счете (лицевом счете) нет достаточного остатка для удовлетворения всех требований.
& Обязанность по уплате налога исполняется в валюте Российской Федерации, если иное не предусмотрено настоящим \ul{Кодексом}. Пересчет суммы налога, исчисленной в предусмотренных настоящим Кодексом случаях в иностранной валюте, в валюту Российской Федерации осуществляется по официальному курсу Центрального банка Российской Федерации на дату уплаты налога.
& Неисполнение обязанности по уплате налога является основанием для применения мер принудительного исполнения обязанности по уплате налога, предусмотренных настоящим \ul{Кодексом}.
& Поручение на перечисление налога в бюджетную систему Российской Федерации на соответствующий счет Федерального казначейства заполняется налогоплательщиком в соответствии с правилами заполнения поручений. Указанные правила устанавливаются Министерством финансов Российской Федерации по согласованию с Центральным банком Российской Федерации.
\par При обнаружении налогоплательщиком ошибки в оформлении поручения на перечисление налога, не повлекшей неперечисления этого налога в бюджетную систему Российской Федерации на соответствующий счет Федерального казначейства, налогоплательщик вправе подать в налоговый орган по месту своего учета заявление о допущенной ошибке с приложением документов, подтверждающих уплату им указанного налога и его перечисление в бюджетную систему Российской Федерации на соответствующий счет Федерального казначейства, с просьбой уточнить основание, тип и принадлежность платежа, налоговый период или статус плательщика.
\par По предложению налогового органа или налогоплательщика может быть проведена совместная сверка уплаченных налогоплательщиком налогов. Результаты сверки оформляются актом, который подписывается налогоплательщиком и уполномоченным должностным лицом налогового органа.
\par Налоговый орган вправе требовать от банка копию поручения налогоплательщика на перечисление налога в бюджетную систему Российской Федерации на соответствующий счет Федерального казначейства, оформленного налогоплательщиком на бумажном носителе. Банк обязан представить в налоговый орган копию указанного поручения в течение пяти дней со дня получения требования налогового органа.
\par В случае, предусмотренном настоящим пунктом, на основании заявления налогоплательщика и акта совместной сверки расчетов по налогам, сборам, пеням и штрафам, если такая совместная сверка проводилась, налоговый орган принимает решение об уточнении платежа на день фактической уплаты налогоплательщиком налога в бюджетную систему Российской Федерации на соответствующий счет Федерального казначейства. При этом налоговый орган осуществляет пересчет пеней, начисленных на сумму налога, за период со дня его фактической уплаты в бюджетную систему Российской Федерации на соответствующий счет Федерального казначейства до дня принятия налоговым органом решения об уточнении платежа.
\par О принятом решении об уточнении платежа налоговый орган уведомляет налогоплательщика в течение пяти дней после принятия данного решения.
& Правила, предусмотренные настоящей статьей, применяются также в отношении сборов, пеней, штрафов и распространяются на плательщиков сборов, налоговых агентов и ответственного участника консолидированной группы налогоплательщиков.
\eEasyList

\subsubsection{{\bf Статья 52.} Порядок исчисления налога}
\beginEasyList
& Налогоплательщик самостоятельно исчисляет сумму налога, подлежащую уплате за налоговый период, исходя из налоговой базы, налоговой ставки и налоговых льгот, если иное не предусмотрено настоящим Кодексом.
& В случаях, предусмотренных законодательством Российской Федерации о налогах и сборах, обязанность по исчислению суммы налога может быть возложена на налоговый орган или налогового агента.
\par В случае, если обязанность по исчислению суммы налога возлагается на налоговый орган, не позднее 30 дней до наступления срока платежа налоговый орган направляет налогоплательщику налоговое уведомление.
& В налоговом уведомлении должны быть указаны сумма налога, подлежащая уплате, расчет налоговой базы, а также срок уплаты налога.
\par В налоговом уведомлении могут быть указаны данные по нескольким подлежащим уплате налогам.
\par Форма налогового уведомления утверждается федеральным органом исполнительной власти, уполномоченным по контролю и надзору в области налогов и сборов.
& Налоговое уведомление может быть передано руководителю организации (ее законному или уполномоченному представителю) или физическому лицу (его законному или уполномоченному представителю) лично под расписку, направлено по почте заказным письмом или передано в электронной форме по телекоммуникационным каналам связи. В случае направления налогового уведомления по почте заказным письмом налоговое уведомление считается полученным по истечении шести дней с даты направления заказного письма.
\par Форматы и порядок направления налогоплательщику налогового уведомления в электронной форме по телекоммуникационным каналам связи устанавливаются федеральным органом исполнительной власти, уполномоченным по контролю и надзору в области налогов и сборов.
& Сумма налога на прибыль организаций, исчисляемая по консолидированной группе налогоплательщиков, исчисляется ответственным участником этой группы на основании имеющихся у него данных, включая данные, предоставленные иными участниками консолидированной группы.
& Сумма налога исчисляется в полных рублях. Сумма налога менее 50 копеек отбрасывается, а сумма налога 50 копеек и более округляется до полного рубля.
\eEasyList

\section{{\bf Раздел V. Налоговая декларация и налоговый контроль}}
\subsection{{\bf Глава 13. Налоговая декларация}}
\subsubsection{{\bf Статья 80.} Налоговая декларация}
\beginEasyList
& Налоговая декларация представляет собой письменное заявление или заявление, составленное в электронной форме и переданное по телекоммуникационным каналам связи с применением усиленной квалифицированной электронной подписи, налогоплательщика об объектах налогообложения, о полученных доходах и произведенных расходах, об источниках доходов, о налоговой базе, налоговых льготах, об исчисленной сумме налога и (или) о других данных, служащих основанием для исчисления и уплаты налога.
\par Налоговая декларация представляется каждым налогоплательщиком по каждому налогу, подлежащему уплате этим налогоплательщиком, если иное не предусмотрено законодательством о налогах и сборах.
\par Расчет авансового платежа представляет собой письменное заявление или заявление, составленное в электронной форме и переданное по телекоммуникационным каналам связи с применением усиленной квалифицированной электронной подписи, налогоплательщика о базе исчисления, об используемых льготах, исчисленной сумме авансового платежа и (или) о других данных, служащих основанием для исчисления и уплаты авансового платежа. Расчет авансового платежа представляется в случаях, предусмотренных настоящим Кодексом применительно к конкретному налогу.
\par Расчет сбора представляет собой письменное заявление или заявление, составленное в электронной форме и переданное по телекоммуникационным каналам связи с применением усиленной квалифицированной электронной подписи, плательщика сбора об объектах обложения, облагаемой базе, используемых льготах, исчисленной сумме сбора и (или) о других данных, служащих основанием для исчисления и уплаты сбора, если иное не предусмотрено настоящим Кодексом. Расчет сбора представляется в случаях, предусмотренных \ul{частью второй} настоящего Кодекса применительно к каждому сбору.
\par Налоговый агент представляет в налоговые органы расчеты, предусмотренные \ul{частью второй} настоящего Кодекса. Указанные расчеты представляются в порядке, установленном частью второй настоящего Кодекса применительно к конкретному налогу.
& Не подлежат представлению в налоговые органы налоговые декларации (расчеты) по тем налогам, по которым налогоплательщики освобождены от обязанности по их уплате в связи с применением специальных налоговых режимов, в части деятельности, осуществление которой влечет применение специальных налоговых режимов, либо имущества, используемого для осуществления такой деятельности.
\par Лицо, признаваемое налогоплательщиком по одному или нескольким налогам, не осуществляющее операций, в результате которых происходит движение денежных средств на его счетах в банках (в кассе организации), и не имеющее по этим налогам объектов налогообложения, представляет по данным налогам единую (упрощенную) налоговую декларацию.
\par Форма единой (упрощенной) налоговой декларации и порядок ее заполнения утверждаются федеральным органом исполнительной власти, уполномоченным по контролю и надзору в области налогов и сборов, по согласованию с Министерством финансов Российской Федерации.
\par Единая (упрощенная) налоговая декларация представляется в налоговый орган по месту нахождения организации или месту жительства физического лица не позднее 20-го числа месяца, следующего за истекшими кварталом, полугодием, 9 месяцами, календарным годом.
& Налоговая декларация (расчет) представляется в налоговый орган по месту учета налогоплательщика (плательщика сбора, налогового агента) по установленной форме на бумажном носителе или по установленным форматам в электронной форме вместе с документами, которые в соответствии с настоящим Кодексом должны прилагаться к налоговой декларации (расчету). Налогоплательщики вправе представить документы, которые в соответствии с настоящим Кодексом должны прилагаться к налоговой декларации (расчету), в электронной форме.
\par Налоговые декларации (расчеты) представляются в налоговый орган по месту учета налогоплательщика (плательщика сбора, налогового агента) по установленным форматам в электронной форме по телекоммуникационным каналам связи через оператора электронного документооборота, являющегося российской организацией и соответствующего требованиям, утверждаемым федеральным органом исполнительной власти, уполномоченным осуществлять функции по контролю и надзору в сфере налогов и сборов, если иной порядок представления информации, отнесенной к государственной тайне, не предусмотрен законодательством Российской Федерации, следующими категориями налогоплательщиков:
\par налогоплательщиками, среднесписочная численность работников которых за предшествующий календарный год превышает 100 человек;
\par вновь созданными (в том числе при реорганизации) организациями, численность работников которых превышает 100 человек;
\par налогоплательщиками, не указанными в абзацах третьем и четвертом настоящего пункта, для которых такая обязанность предусмотрена \ul{частью второй} настоящего Кодекса применительно к конкретному налогу.
\par Сведения о среднесписочной численности работников за предшествующий календарный год представляются организацией (индивидуальным предпринимателем, привлекавшим в указанный период наемных работников) в налоговый орган не позднее 20 января текущего года, а в случае создания (реорганизации) организации --- не позднее 20-го числа месяца, следующего за месяцем, в котором организация была создана (реорганизована). Указанные сведения представляются по форме, утвержденной федеральным органом исполнительной власти, уполномоченным по контролю и надзору в области налогов и сборов, в налоговый орган по месту нахождения организации (по месту жительства индивидуального предпринимателя).
\par Налогоплательщики, в соответствии со \ul{статьей 83} настоящего Кодекса отнесенные к категории крупнейших, представляют все налоговые декларации (расчеты), которые они обязаны представлять в соответствии с настоящим Кодексом, в налоговый орган по месту учета в качестве крупнейших налогоплательщиков по установленным форматам в электронной форме, если иной порядок представления информации, отнесенной к государственной тайне, не предусмотрен законодательством Российской Федерации.
\par Бланки налоговых деклараций (расчетов) предоставляются налоговыми органами бесплатно.
& Налоговая декларация (расчет) может быть представлена налогоплательщиком (плательщиком сбора, налоговым агентом) в налоговый орган лично или через представителя, направлена в виде почтового отправления с описью вложения или передана в электронной форме по телекоммуникационным каналам связи.
\par Налоговый орган не вправе отказать в принятии налоговой декларации (расчета), представленной налогоплательщиком (плательщиком сборов, налоговым агентом) по установленной форме (установленному формату), и обязан проставить по просьбе налогоплательщика (плательщика сбора, налогового агента) на копии налоговой декларации (копии расчета) отметку о принятии и дату ее получения при получении налоговой декларации (расчета) на бумажном носителе либо передать налогоплательщику (плательщику сбора, налоговому агенту) квитанцию о приеме в электронной форме --- при получении налоговой декларации (расчета) по телекоммуникационным каналам связи.
\par При отправке налоговой декларации (расчета) по почте днем ее представления считается дата отправки почтового отправления с описью вложения. При передаче налоговой декларации (расчета) по телекоммуникационным каналам связи днем ее представления считается дата ее отправки.
\par Абзац четвертый утратил силу.
& Налоговая декларация (расчет) представляется с указанием идентификационного номера налогоплательщика, если иное не предусмотрено настоящим Кодексом.
\par Налогоплательщик (плательщик сбора, налоговый агент) или его представитель подписывает налоговую декларацию (расчет), подтверждая достоверность и полноту сведений, указанных в налоговой декларации (расчете).
\par Если достоверность и полноту сведений, указанных в налоговой декларации (расчете), в том числе с применением усиленной квалифицированной электронной подписи при представлении налоговой декларации (расчета) в электронной форме, подтверждает уполномоченный представитель налогоплательщика (плательщика сбора, налогового агента), в налоговой декларации (расчете) указывается основание представительства (наименование документа, подтверждающего наличие полномочий на подписание налоговой декларации (расчета). При этом к налоговой декларации (расчету) прилагается копия документа, подтверждающего полномочия представителя на подписание налоговой декларации (расчета).
\par При представлении налоговой декларации (расчета) в электронной форме копия документа, подтверждающего полномочия представителя на подписание налоговой декларации (расчета), может быть представлена в электронной форме по телекоммуникационным каналам связи.
& Налоговая декларация (расчет) представляется в установленные законодательством о налогах и сборах сроки.
& Формы и порядок заполнения форм налоговых деклараций (расчетов), а также форматы и порядок представления налоговых деклараций (расчетов) и прилагаемых к ним документов в соответствии с настоящим Кодексом в электронной форме утверждаются федеральным органом исполнительной власти, уполномоченным по контролю и надзору в области налогов и сборов, по согласованию с Министерством финансов Российской Федерации.
\par Абзац второй утратил силу.
\par Федеральный орган исполнительной власти, уполномоченный по контролю и надзору в области налогов и сборов, не вправе включать в форму налоговой декларации (расчета), а налоговые органы не вправе требовать от налогоплательщиков (плательщиков сборов, налоговых агентов) включения в налоговую декларацию (расчет) сведений, не связанных с исчислением и (или) уплатой налогов и сборов, за исключением:
&& вида документа: первичный (корректирующий);
&& наименования налогового органа;
&& места нахождения организации (ее обособленного подразделения) или места жительства физического лица;
&& фамилии, имени, отчества физического лица или полного наименования организации (ее обособленного подразделения);
&& номера контактного телефона налогоплательщика;
&& сведений, подлежащих включению в налоговую декларацию в соответствии с \ul{главой 21} настоящего Кодекса.
& Утратил силу с 1 января 2011 г.
& Особенности представления налоговых деклараций при выполнении соглашений о разделе продукции определяются \ul{главой 26.4} настоящего Кодекса.
& Особенности исполнения обязанности по представлению налоговых деклараций посредством уплаты декларационного платежа определяются федеральным законом об упрощенном порядке декларирования доходов физическими лицами.
& Особенности представления в налоговый орган налоговой декларации консолидированной группы налогоплательщиков определяются \ul{главой 25} настоящего Кодекса.
& Правила, предусмотренные настоящей статьей, распространяются также на иных лиц, на которых возложена обязанность по представлению налоговой декларации (расчета) в соответствии с \ul{частью второй} настоящего Кодекса.
\eEasyList
\subsubsection{{\bf Статья 81.} Внесение изменений в налоговую декларацию}
\beginEasyList
& При обнаружении налогоплательщиком в поданной им в налоговый орган налоговой декларации факта неотражения или неполноты отражения сведений, а также ошибок, приводящих к занижению суммы налога, подлежащей уплате, налогоплательщик обязан внести необходимые изменения в налоговую декларацию и представить в налоговый орган уточненную налоговую декларацию в порядке, установленном настоящей статьей.
\par При обнаружении налогоплательщиком в поданной им в налоговый орган налоговой декларации недостоверных сведений, а также ошибок, не приводящих к занижению суммы налога, подлежащей уплате, налогоплательщик вправе внести необходимые изменения в налоговую декларацию и представить в налоговый орган уточненную налоговую декларацию в порядке, установленном настоящей статьей. При этом уточненная налоговая декларация, представленная после истечения установленного срока подачи декларации, не считается представленной с нарушением срока.
& Если уточненная налоговая декларация представляется в налоговый орган до истечения срока подачи налоговой декларации, она считается поданной в день подачи уточненной налоговой декларации.
& Если уточненная налоговая декларация представляется в налоговый орган после истечения срока подачи налоговой декларации, но до истечения срока уплаты налога, то налогоплательщик освобождается от ответственности, если уточненная налоговая декларация была представлена до момента, когда налогоплательщик узнал об обнаружении налоговым органом факта неотражения или неполноты отражения сведений в налоговой декларации, а также ошибок, приводящих к занижению подлежащей уплате суммы налога, либо о назначении выездной налоговой проверки.
& Если уточненная налоговая декларация представляется в налоговый орган после истечения срока подачи налоговой декларации и срока уплаты налога, то налогоплательщик освобождается от ответственности в случаях:
&& представления уточненной налоговой декларации до момента, когда налогоплательщик узнал об обнаружении налоговым органом неотражения или неполноты отражения сведений в налоговой декларации, а также ошибок, приводящих к занижению подлежащей уплате суммы налога, либо о назначении выездной налоговой проверки по данному налогу за данный период, при условии, что до представления уточненной налоговой декларации он уплатил недостающую сумму налога и соответствующие ей пени;
&& представления уточненной налоговой декларации после проведения выездной налоговой проверки за соответствующий налоговый период, по результатам которой не были обнаружены неотражение или неполнота отражения сведений в налоговой декларации, а также ошибки, приводящие к занижению подлежащей уплате суммы налога.
& Уточненная налоговая декларация представляется налогоплательщиком в налоговый орган по месту учета.
\par Уточненная налоговая декларация (расчет) представляется в налоговый орган по форме, действовавшей в налоговый период, за который вносятся соответствующие изменения.
& При обнаружении налоговым агентом в поданном им в налоговый орган расчете факта неотражения или неполноты отражения сведений, а также ошибок, приводящих к занижению или завышению суммы налога, подлежащей перечислению, налоговый агент обязан внести необходимые изменения и представить в налоговый орган уточненный расчет в порядке, установленном настоящей статьей.
\par Уточненный расчет, представляемый налоговым агентом в налоговый орган, должен содержать данные только в отношении тех налогоплательщиков, в отношении которых обнаружены факты неотражения или неполноты отражения сведений, а также ошибки, приводящие к занижению суммы налога.
\par Положения, предусмотренные \ul{пунктами 3} и \ul{4} настоящей статьи, касающиеся освобождения от ответственности, применяются также в отношении налоговых агентов при представлении ими уточненных расчетов.
\par 6.1. В случае, если участник договора инвестиционного товарищества --- управляющий товарищ, ответственный за ведение налогового учета (далее в настоящей статье --- управляющий товарищ, ответственный за ведение налогового учета), предоставил участникам договора инвестиционного товарищества копию уточненного расчета финансового результата инвестиционного товарищества, налогоплательщики, уплачивающие налог на прибыль организаций, налог на доходы физических лиц в связи с их участием в договоре инвестиционного товарищества, обязаны подавать уточненную налоговую декларацию (расчет).
\par Уточненная налоговая декларация (расчет) должна быть представлена в налоговый орган по месту учета участника договора инвестиционного товарищества не позднее пятнадцати дней со дня, когда ему была передана копия уточненного расчета финансового результата инвестиционного товарищества.
\par При этом, если уточненная налоговая декларация (расчет) представляется в налоговый орган в сроки, указанные в абзаце втором настоящего пункта, участник договора инвестиционного товарищества, не являющийся управляющим товарищем, ответственным за ведение налогового учета, освобождается от ответственности.
\par Если участник договора инвестиционного товарищества обжалует акты или решения налогового органа, которыми были изменены финансовые результаты инвестиционного товарищества, он обязан представить уточненную налоговую декларацию (расчет) не позднее пятнадцати дней со дня, когда вышестоящим налоговым органом было принято решение по результатам рассмотрения его жалобы.
& Правила, предусмотренные настоящей статьей, применяются также в отношении уточненных расчетов сборов и распространяются на плательщиков сборов.
\eEasyList

\end{document}
