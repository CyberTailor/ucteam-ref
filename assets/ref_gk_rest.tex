\documentclass{report}
    \usepackage[a4paper,margin=2cm]{geometry}
    \usepackage[utf8]{inputenc}
    \usepackage[T2A]{fontenc}
    \usepackage[russian]{babel}
    \usepackage{indentfirst}
    \setlength{\parindent}{0pt}
    \usepackage{textcomp}
    \usepackage[normalem]{ulem}
    \usepackage[ampersand]{easylist}
    \newcommand{\beginEasyList}{
        \begin{easylist}[enumerate]
            \ListProperties(Numbers=a,Hide2=1,Hide3=1,Style*=,Mark=.,FinalMark={)},FinalMark1=.)
    }
    \newcommand{\eEasyList}{\end{easylist}}
    \setcounter{secnumdepth}{-2}
\begin{document}

\section{{\bf Раздел I. Общие положения}}
\section{{\bf Подраздел 2. Лица}}
\subsection{{\bf Глава 3. Граждане (физические лица)}}

\subsubsection{{\bf Статья 17.} Правоспособность гражданина}
\beginEasyList
& Способность иметь гражданские права и нести обязанности (гражданская правоспособность) признается в равной мере за всеми гражданами.
& Правоспособность гражданина возникает в момент его рождения и прекращается смертью.
\eEasyList
\subsubsection{{\bf Статья 18.} Содержание правоспособности граждан}
\par Граждане могут иметь имущество на праве собственности; наследовать и завещать имущество; заниматься предпринимательской и любой иной не запрещенной законом деятельностью; создавать юридические лица самостоятельно или совместно с другими гражданами и юридическими лицами; совершать любые не противоречащие закону сделки и участвовать в обязательствах; избирать место жительства; иметь права авторов произведений науки, литературы и искусства, изобретений и иных охраняемых законом результатов интеллектуальной деятельности; иметь иные имущественные и личные неимущественные права.

\subsubsection{{\bf Статья 23.} Предпринимательская деятельность гражданина}
\beginEasyList
& Гражданин вправе заниматься предпринимательской деятельностью без образования юридического лица с момента государственной регистрации в качестве индивидуального предпринимателя.
& Утратил силу с 1 марта 2013 г.
& К предпринимательской деятельности граждан, осуществляемой без образования юридического лица, соответственно применяются правила настоящего Кодекса, которые регулируют деятельность юридических лиц, являющихся коммерческими организациями, если иное не вытекает из закона, иных правовых актов или существа правоотношения.
& Гражданин, осуществляющий предпринимательскую деятельность без образования юридического лица с нарушением требований \uline{пункта 1} настоящей статьи, не вправе ссылаться в отношении заключенных им при этом сделок на то, что он не является предпринимателем. Суд может применить к таким сделкам правила настоящего Кодекса об обязательствах, связанных с осуществлением предпринимательской деятельности.
& Граждане вправе заниматься производственной или иной хозяйственной деятельностью в области сельского хозяйства без образования юридического лица на основе соглашения о создании крестьянского (фермерского) хозяйства, заключенного в соответствии с законом о крестьянском (фермерском) хозяйстве.
\par Главой крестьянского (фермерского) хозяйства может быть гражданин, зарегистрированный в качестве индивидуального предпринимателя.
\eEasyList
\subsubsection{{\bf Статья 24.} Имущественная ответственность гражданина}
\par Гражданин отвечает по своим обязательствам всем принадлежащим ему имуществом, за исключением имущества, на которое в соответствии с законом не может быть обращено взыскание.
\par Перечень имущества граждан, на которое не может быть обращено взыскание, устанавливается гражданским процессуальным законодательством.
\subsubsection{{\bf Статья 25.} Несостоятельность (банкротство) индивидуального предпринимателя}
\beginEasyList
& Индивидуальный предприниматель, который не в состоянии удовлетворить требования кредиторов, связанные с осуществлением им предпринимательской деятельности, может быть признан несостоятельным (банкротом) по решению суда. С момента вынесения такого решения утрачивает силу его регистрация в качестве индивидуального предпринимателя.
& При осуществлении процедуры признания банкротом индивидуального предпринимателя его кредиторы по обязательствам, не связанным с осуществлением им предпринимательской деятельности, также вправе предъявить свои требования. Требования указанных кредиторов, не заявленные ими в таком порядке, сохраняют силу после завершения процедуры банкротства индивидуального предпринимателя.
& Требования кредиторов индивидуального предпринимателя в случае признания его банкротом удовлетворяются за счет принадлежащего ему имущества в порядке и в очередности, которые предусмотрены законом о несостоятельности (банкротстве).
& После завершения расчетов с кредиторами индивидуальный предприниматель, признанный банкротом, освобождается от исполнения оставшихся обязательств, связанных с его предпринимательской деятельностью, и иных требований, предъявленных к исполнению и учтенных при признании предпринимателя банкротом.
\par Сохраняют силу требования граждан, перед которыми лицо, объявленное банкротом, несет ответственность за причинение вреда жизни или здоровью, а также иные требования личного характера.
& Основания и порядок признания судом индивидуального предпринимателя банкротом либо объявления им о своем банкротстве устанавливаются законом о несостоятельности (банкротстве).
\eEasyList
\subsubsection{{\bf Статья 26.} Дееспособность несовершеннолетних в возрасте от четырнадцати до восемнадцати лет}
\beginEasyList
& Несовершеннолетние в возрасте от четырнадцати до восемнадцати лет совершают сделки, за исключением названных в \uline{пункте 2} настоящей статьи, с письменного согласия своих законных представителей --- родителей, усыновителей или попечителя.
\par Сделка, совершенная таким несовершеннолетним, действительна также при ее последующем письменном одобрении его родителями, усыновителями или попечителем.
& Несовершеннолетние в возрасте от четырнадцати до восемнадцати лет вправе самостоятельно, без согласия родителей, усыновителей и попечителя:
&& распоряжаться своими заработком, стипендией и иными доходами;
&& осуществлять права автора произведения науки, литературы или искусства, изобретения или иного охраняемого законом результата своей интеллектуальной деятельности;
&& в соответствии с \uline{законом} вносить вклады в кредитные организации и распоряжаться ими;
&& совершать мелкие бытовые сделки и иные сделки, предусмотренные \uline{пунктом 2 статьи 28} настоящего Кодекса.
\par По достижении шестнадцати лет несовершеннолетние также вправе быть членами кооперативов в соответствии с законами о кооперативах.
& Несовершеннолетние в возрасте от четырнадцати до восемнадцати лет самостоятельно несут имущественную ответственность по сделкам, совершенным ими в соответствии с \uline{пунктами 1} и \uline{2} настоящей статьи. За причиненный ими вред такие несовершеннолетние несут ответственность в соответствии с настоящим \uline{Кодексом.}
& При наличии достаточных оснований суд по ходатайству родителей, усыновителей или попечителя либо органа опеки и попечительства может ограничить или лишить несовершеннолетнего в возрасте от четырнадцати до восемнадцати лет права самостоятельно распоряжаться своими заработком, стипендией или иными доходами, за исключением случаев, когда такой несовершеннолетний приобрел дееспособность в полном объеме в соответствии с \uline{пунктом 2 статьи 21} или со \uline{статьей 27} настоящего Кодекса.
\eEasyList
\subsubsection{{\bf Статья 27.} Эмансипация}
\beginEasyList
& Несовершеннолетний, достигший шестнадцати лет, может быть объявлен полностью дееспособным, если он работает по трудовому договору, в том числе по контракту, или с согласия родителей, усыновителей или попечителя занимается предпринимательской деятельностью.
\par Объявление несовершеннолетнего полностью дееспособным (эмансипация) производится по решению органа опеки и попечительства --- с согласия обоих родителей, усыновителей или попечителя либо при отсутствии такого согласия --- по решению суда.
& Родители, усыновители и попечитель не несут ответственности по обязательствам эмансипированного несовершеннолетнего, в частности по обязательствам, возникшим вследствие причинения им вреда.
\eEasyList
\subsubsection{{\bf Статья 28.} Дееспособность малолетних}
\beginEasyList
& За несовершеннолетних, не достигших четырнадцати лет (малолетних), сделки, за исключением указанных в \uline{пункте 2} настоящей статьи, могут совершать от их имени только их родители, усыновители или опекуны.
\par К сделкам законных представителей несовершеннолетнего с его имуществом применяются правила, предусмотренные \uline{пунктами 2} и \uline{3 статьи 37} настоящего Кодекса.
& Малолетние в возрасте от шести до четырнадцати лет вправе самостоятельно совершать:
&& мелкие бытовые сделки;
&& сделки, направленные на безвозмездное получение выгоды, не требующие нотариального удостоверения либо государственной регистрации;
&& сделки по распоряжению средствами, предоставленными законным представителем или с согласия последнего третьим лицом для определенной цели или для свободного распоряжения.
& Имущественную ответственность по сделкам малолетнего, в том числе по сделкам, совершенным им самостоятельно, несут его родители, усыновители или опекуны, если не докажут, что обязательство было нарушено не по их вине. Эти лица в соответствии с \uline{законом} также отвечают за вред, причиненный малолетними.
\eEasyList

\subsection{{\bf Глава 4. Юридические лица}}
\subsection{{\bf § 1. Основные положения}}
\subsubsection{{\bf Статья 48.} Понятие юридического лица}
\beginEasyList
& Юридическим лицом признается организация, которая имеет в собственности, хозяйственном ведении или оперативном управлении обособленное имущество и отвечает по своим обязательствам этим имуществом, может от своего имени приобретать и осуществлять имущественные и личные неимущественные права, нести обязанности, быть истцом и ответчиком в суде.
\par Юридические лица должны иметь самостоятельный баланс и (или) смету.
& В связи с участием в образовании имущества юридического лица его учредители (участники) могут иметь обязательственные права в отношении этого юридического лица либо вещные права на его имущество.
\par К юридическим лицам, в отношении которых их участники имеют обязательственные права, относятся хозяйственные товарищества и общества, производственные и потребительские кооперативы.
\par К юридическим лицам, на имущество которых их учредители имеют право собственности или иное вещное право, относятся государственные и муниципальные унитарные предприятия, а также учреждения.
& К юридическим лицам, в отношении которых их учредители (участники) не имеют имущественных прав, относятся общественные и религиозные организации (объединения), благотворительные и иные фонды, объединения юридических лиц (ассоциации и союзы).
\eEasyList
\subsubsection{{\bf Статья 49.} Правоспособность юридического лица}
\beginEasyList
& Юридическое лицо может иметь гражданские права, соответствующие целям деятельности, предусмотренным в его учредительных документах, и нести связанные с этой деятельностью обязанности.
\par Коммерческие организации, за исключением унитарных предприятий и иных видов организаций, предусмотренных законом, могут иметь гражданские права и нести гражданские обязанности, необходимые для осуществления любых видов деятельности, не запрещенных законом.
\par Отдельными видами деятельности, перечень которых определяется законом, юридическое лицо может заниматься только на основании специального разрешения (лицензии).
& Юридическое лицо может быть ограничено в правах лишь в случаях и в порядке, предусмотренных законом. Решение об ограничении прав может быть оспорено юридическим лицом в суде.
& Правоспособность юридического лица возникает в момент его создания и прекращается в момент внесения записи о его исключении из единого государственного реестра юридических лиц.
\par Право юридического лица осуществлять деятельность, на занятие которой необходимо получение лицензии, возникает с момента получения такой лицензии или в указанный в ней срок и прекращается по истечении срока ее действия, если иное не установлено законом или иными правовыми актами.
\eEasyList
\subsubsection{{\bf Статья 50.} Коммерческие и некоммерческие организации}
\beginEasyList
& Юридическими лицами могут быть организации, преследующие извлечение прибыли в качестве основной цели своей деятельности (коммерческие организации) либо не имеющие извлечение прибыли в качестве такой цели и не распределяющие полученную прибыль между участниками (некоммерческие организации).
& Юридические лица, являющиеся коммерческими организациями, могут создаваться в форме хозяйственных товариществ и обществ, хозяйственных партнерств, производственных кооперативов, государственных и муниципальных унитарных предприятий.
& Юридические лица, являющиеся некоммерческими организациями, могут создаваться в форме потребительских кооперативов, общественных или религиозных организаций (объединений), учреждений, благотворительных и иных фондов, а также в других формах, предусмотренных законом.
\par Некоммерческие организации могут осуществлять предпринимательскую деятельность лишь постольку, поскольку это служит достижению целей, ради которых они созданы, и соответствующую этим целям.
& Допускается создание объединений коммерческих и (или) некоммерческих организаций в форме ассоциаций и союзов.
\eEasyList
\subsubsection{{\bf Статья 51.} Государственная регистрация юридических лиц}
\beginEasyList
& Юридическое лицо подлежит государственной регистрации в уполномоченном государственном органе в порядке, предусмотренном законом о государственной регистрации юридических лиц.
& Данные государственной регистрации включаются в единый государственный реестр юридических лиц, открытый для всеобщего ознакомления.
\par Лицо, добросовестно полагающееся на данные единого государственного реестра юридических лиц, вправе исходить из того, что они соответствуют действительным обстоятельствам. Юридическое лицо не вправе в отношениях с лицом, полагавшимся на данные единого государственного реестра юридических лиц, ссылаться на данные, не включенные в указанный реестр, а также на недостоверные данные, содержащиеся в нем, за исключением случаев, если соответствующие данные включены в указанный реестр в результате неправомерных действий третьих лиц или иным путем помимо воли юридического лица.
\par Юридическое лицо обязано возместить убытки, причиненные другим участникам гражданского оборота вследствие непредставления, несвоевременного представления или представления недостоверных данных о нем в единый государственный реестр юридических лиц.
& До государственной регистрации юридического лица, изменений его устава или до включения иных данных, не связанных с изменениями устава, в единый государственный реестр юридических лиц уполномоченный государственный орган обязан провести в порядке и в срок, которые предусмотрены законом, проверку достоверности данных, включаемых в указанный реестр.
& В случаях и в порядке, которые предусмотрены законом о государственной регистрации юридических лиц, уполномоченный государственный орган обязан заблаговременно сообщить заинтересованным лицам о предстоящей государственной регистрации изменений устава юридического лица и о предстоящем включении данных в единый государственный реестр юридических лиц.
\par Заинтересованные лица вправе направить в уполномоченный государственный орган возражения относительно предстоящей государственной регистрации изменений устава юридического лица или предстоящего включения данных в единый государственный реестр юридических лиц в порядке, предусмотренном законом о государственной регистрации юридических лиц. Уполномоченный государственный орган обязан рассмотреть эти возражения и принять соответствующее решение в порядке и в срок, которые предусмотрены законом о государственной регистрации юридических лиц.
& Отказ в государственной регистрации юридического лица, а также во включении данных о нем в единый государственный реестр юридических лиц допускается только в случаях, предусмотренных законом о государственной регистрации юридических лиц.
\par Отказ в государственной регистрации юридического лица и уклонение от такой регистрации могут быть оспорены в суде.
& Государственная регистрация юридического лица может быть признана судом недействительной в связи с допущенными при его создании грубыми нарушениями закона, если эти нарушения носят неустранимый характер.
\par Включение в единый государственный реестр юридических лиц данных о юридическом лице может быть оспорено в суде, если такие данные недостоверны или включены в указанный реестр с нарушением закона.
& Убытки, причиненные незаконным отказом в государственной регистрации юридического лица, уклонением от государственной регистрации, включением в единый государственный реестр юридических лиц недостоверных данных о юридическом лице либо нарушением порядка государственной регистрации, предусмотренного законом о государственной регистрации юридических лиц, по вине уполномоченного государственного органа, подлежат возмещению за счет казны Российской Федерации.
& Юридическое лицо считается созданным, а данные о юридическом лице считаются включенными в единый государственный реестр юридических лиц со дня внесения соответствующей записи в этот реестр.
\eEasyList
\subsubsection{{\bf Статья 52.} Учредительные документы юридического лица}
\beginEasyList
& Юридическое лицо действует на основании устава, либо учредительного договора и устава, либо только учредительного договора. В случаях, предусмотренных законом, юридическое лицо, не являющееся коммерческой организацией, может действовать на основании общего положения об организациях данного вида.
\par Учредительный договор юридического лица заключается, а устав утверждается его учредителями (участниками).
\par Юридическое лицо, созданное в соответствии с настоящим Кодексом одним учредителем, действует на основании устава, утвержденного этим учредителем.
& В учредительных документах юридического лица должны определяться наименование юридического лица, место его нахождения, порядок управления деятельностью юридического лица, а также содержаться другие сведения, предусмотренные законом для юридических лиц соответствующего вида. В учредительных документах некоммерческих организаций и унитарных предприятий, а в предусмотренных законом случаях и других коммерческих организаций должны быть определены предмет и цели деятельности юридического лица. Предмет и определенные цели деятельности коммерческой организации могут быть предусмотрены учредительными документами и в случаях, когда по закону это не является обязательным.
\par В учредительном договоре учредители обязуются создать юридическое лицо, определяют порядок совместной деятельности по его созданию, условия передачи ему своего имущества и участия в его деятельности. Договором определяются также условия и порядок распределения между участниками прибыли и убытков, управления деятельностью юридического лица, выхода учредителей (участников) из его состава.
& Изменения учредительных документов приобретают силу для третьих лиц с момента их государственной регистрации, а в случаях, установленных законом, --- с момента уведомления органа, осуществляющего государственную регистрацию, о таких изменениях. Однако юридические лица и их учредители (участники) не вправе ссылаться на отсутствие регистрации таких изменений в отношениях с третьими лицами, действовавшими с учетом этих изменений.
\eEasyList
\subsubsection{{\bf Статья 53.} Органы юридического лица}
\beginEasyList
& Юридическое лицо приобретает гражданские права и принимает на себя гражданские обязанности через свои органы, действующие в соответствии с законом, иными правовыми актами и учредительными документами.
\par Порядок назначения или избрания органов юридического лица определяется законом и учредительными документами.
& В предусмотренных законом случаях юридическое лицо может приобретать гражданские права и принимать на себя гражданские обязанности через своих участников.
& Лицо, которое в силу закона или учредительных документов юридического лица выступает от его имени, должно действовать в интересах представляемого им юридического лица добросовестно и разумно. Оно обязано по требованию учредителей (участников) юридического лица, если иное не предусмотрено законом или договором, возместить убытки, причиненные им юридическому лицу.
\eEasyList
\subsubsection{{\bf Статья 54.} Наименование и место нахождения юридического лица}
\beginEasyList
& Юридическое лицо имеет свое наименование, содержащее указание на его организационно-правовую форму. Наименования некоммерческих организаций, а в предусмотренных законом случаях наименования коммерческих организаций должны содержать указание на характер деятельности юридического лица.
& Место нахождения юридического лица определяется местом его государственной регистрации. Государственная регистрация юридического лица осуществляется по месту нахождения его постоянно действующего исполнительного органа, а в случае отсутствия постоянно действующего исполнительного органа --- иного органа или лица, имеющих право действовать от имени юридического лица без доверенности.
& Наименование и место нахождения юридического лица указываются в его учредительных документах.
& Юридическое лицо, являющееся коммерческой организацией, должно иметь фирменное наименование.
\par Требования к фирменному наименованию устанавливаются настоящим Кодексом и другими законами. Права на фирменное наименование определяются в соответствии с правилами \uline{раздела VII} настоящего Кодекса.
\par Абзац третий утратил силу с 1 января 2008 г.
\par Абзац четвертый утратил силу с 1 января 2008 г.
\eEasyList
\subsubsection{{\bf Статья 55.} Представительства и филиалы}
\beginEasyList
& Представительством является обособленное подразделение юридического лица, расположенное вне места его нахождения, которое представляет интересы юридического лица и осуществляет их защиту.
& Филиалом является обособленное подразделение юридического лица, расположенное вне места его нахождения и осуществляющее все его функции или их часть, в том числе функции представительства.
& Представительства и филиалы не являются юридическими лицами. Они наделяются имуществом создавшим их юридическим лицом и действуют на основании утвержденных им положений.
\par Руководители представительств и филиалов назначаются юридическим лицом и действуют на основании его доверенности.
\par Представительства и филиалы должны быть указаны в учредительных документах создавшего их юридического лица.
\eEasyList
\subsubsection{{\bf Статья 56.} Ответственность юридического лица}
\beginEasyList
& Юридические лица, кроме учреждений, отвечают по своим обязательствам всем принадлежащим им имуществом.
& Казенное предприятие и учреждение отвечают по своим обязательствам в порядке и на условиях, предусмотренных \uline{пунктом 5 статьи 113}, \uline{статьями 115} и \uline{120} настоящего Кодекса.
& Учредитель (участник) юридического лица или собственник его имущества не отвечают по обязательствам юридического лица, а юридическое лицо не отвечает по обязательствам учредителя (участника) или собственника, за исключением случаев, предусмотренных настоящим Кодексом либо учредительными документами юридического лица.
\par Если несостоятельность (банкротство) юридического лица вызвана учредителями (участниками), собственником имущества юридического лица или другими лицами, которые имеют право давать обязательные для этого юридического лица указания либо иным образом имеют возможность определять его действия, на таких лиц в случае недостаточности имущества юридического лица может быть возложена субсидиарная ответственность по его обязательствам.
\eEasyList
\subsubsection{{\bf Статья 57.} Реорганизация юридического лица}
\beginEasyList
& Реорганизация юридического лица (слияние, присоединение, разделение, выделение, преобразование) может быть осуществлена по решению его учредителей (участников) либо органа юридического лица, уполномоченного на то учредительными документами.
& В случаях, установленных законом, реорганизация юридического лица в форме его разделения или выделения из его состава одного или нескольких юридических лиц осуществляется по решению уполномоченных государственных органов или по решению суда.
\par Если учредители (участники) юридического лица, уполномоченный ими орган или орган юридического лица, уполномоченный на реорганизацию его учредительными документами, не осуществят реорганизацию юридического лица в срок, определенный в решении уполномоченного государственного органа, суд по иску указанного государственного органа назначает внешнего управляющего юридическим лицом и поручает ему осуществить реорганизацию этого юридического лица. С момента назначения внешнего управляющего к нему переходят полномочия по управлению делами юридического лица. Внешний управляющий выступает от имени юридического лица в суде, составляет разделительный баланс и передает его на рассмотрение суда вместе с учредительными документами возникающих в результате реорганизации юридических лиц. Утверждение судом указанных документов является основанием для государственной регистрации вновь возникающих юридических лиц.
& В случаях, установленных законом, реорганизация юридических лиц в форме слияния, присоединения или преобразования может быть осуществлена лишь с согласия уполномоченных государственных органов.
& Юридическое лицо считается реорганизованным, за исключением случаев реорганизации в форме присоединения, с момента государственной регистрации вновь возникших юридических лиц.
\par При реорганизации юридического лица в форме присоединения к нему другого юридического лица первое из них считается реорганизованным с момента внесения в единый государственный реестр юридических лиц записи о прекращении деятельности присоединенного юридического лица.
\eEasyList
\subsubsection{{\bf Статья 58.} Правопреемство при реорганизации юридических лиц}
\beginEasyList
& При слиянии юридических лиц права и обязанности каждого из них переходят к вновь возникшему юридическому лицу в соответствии с передаточным актом.
& При присоединении юридического лица к другому юридическому лицу к последнему переходят права и обязанности присоединенного юридического лица в соответствии с передаточным актом.
& При разделении юридического лица его права и обязанности переходят к вновь возникшим юридическим лицам в соответствии с разделительным балансом.
& При выделении из состава юридического лица одного или нескольких юридических лиц к каждому из них переходят права и обязанности реорганизованного юридического лица в соответствии с разделительным балансом.
& При преобразовании юридического лица одного вида в юридическое лицо другого вида (изменении организационно-правовой формы) к вновь возникшему юридическому лицу переходят права и обязанности реорганизованного юридического лица в соответствии с передаточным актом.
\eEasyList
\subsubsection{{\bf Статья 59.} Передаточный акт и разделительный баланс}
\beginEasyList
& Передаточный акт и разделительный баланс должны содержать положения о правопреемстве по всем обязательствам реорганизованного юридического лица в отношении всех его кредиторов и должников, включая и обязательства, оспариваемые сторонами.
& Передаточный акт и разделительный баланс утверждаются учредителями (участниками) юридического лица или органом, принявшим решение о реорганизации юридических лиц, и представляются вместе с учредительными документами для государственной регистрации вновь возникших юридических лиц или внесения изменений в учредительные документы существующих юридических лиц.
\par Непредставление вместе с учредительными документами соответственно передаточного акта или разделительного баланса, а также отсутствие в них положений о правопреемстве по обязательствам реорганизованного юридического лица влекут отказ в государственной регистрации вновь возникших юридических лиц.
\eEasyList
\subsubsection{{\bf Статья 60.} Гарантии прав кредиторов реорганизуемого юридического лица}
\beginEasyList
& Юридическое лицо в течение трех рабочих дней после даты принятия решения о его реорганизации обязано в письменной форме сообщить в орган, осуществляющий государственную регистрацию юридических лиц, о начале процедуры реорганизации с указанием формы реорганизации. В случае участия в реорганизации двух и более юридических лиц такое уведомление направляется юридическим лицом, последним принявшим решение о реорганизации либо определенным решением о реорганизации. На основании данного уведомления орган, осуществляющий государственную регистрацию юридических лиц, вносит в единый государственный реестр юридических лиц запись о том, что юридическое лицо (юридические лица) находится (находятся) в процессе реорганизации.
\par Реорганизуемое юридическое лицо после внесения в единый государственный реестр юридических лиц записи о начале процедуры реорганизации дважды с периодичностью один раз в месяц помещает в средствах массовой информации, в которых опубликовываются данные о государственной регистрации юридических лиц, уведомление о своей реорганизации. В случае участия в реорганизации двух и более юридических лиц уведомление о реорганизации опубликовывается от имени всех участвующих в реорганизации юридических лиц юридическим лицом, последним принявшим решение о реорганизации либо определенным решением о реорганизации. В уведомлении о реорганизации указываются сведения о каждом участвующем в реорганизации, создаваемом (продолжающем деятельность) в результате реорганизации юридическом лице, форма реорганизации, описание порядка и условий заявления кредиторами своих требований, иные сведения, предусмотренные законом.
& Кредитор юридического лица, если его права требования возникли до опубликования уведомления о реорганизации юридического лица, вправе потребовать досрочного исполнения соответствующего обязательства должником, а при невозможности досрочного исполнения --- прекращения обязательства и возмещения связанных с этим убытков, за исключением случаев, установленных законом.
& Кредитор юридического лица --- открытого акционерного общества, реорганизуемого в форме слияния, присоединения или преобразования, если его права требования возникли до опубликования сообщения о реорганизации юридического лица, вправе в судебном порядке потребовать досрочного исполнения обязательства, должником по которому является это юридическое лицо, или прекращения обязательства и возмещения убытков в случае, если реорганизуемым юридическим лицом, его участниками или третьими лицами не предоставлено достаточное обеспечение исполнения соответствующих обязательств.
\par Указанные в настоящем пункте требования могут быть предъявлены кредиторами не позднее 30 дней с даты последнего опубликования уведомления о реорганизации юридического лица.
\par Требования, заявляемые кредиторами, не влекут приостановления действий, связанных с реорганизацией.
& В случае, если требования о досрочном исполнении или прекращении обязательств и возмещении убытков удовлетворены после завершения реорганизации, вновь созданные в результате реорганизации (продолжающие деятельность) юридические лица несут солидарную ответственность по обязательствам реорганизованного юридического лица.
& Исполнение реорганизуемым юридическим лицом обязательств перед кредиторами обеспечивается в порядке, установленном настоящим Кодексом.
\par В случае, если обязательства перед кредиторами реорганизуемого юридического лица --- должника обеспечены залогом, такие кредиторы не вправе требовать предоставления дополнительного обеспечения.
& Особенности реорганизации кредитных организаций, включая порядок уведомления органа, осуществляющего государственную регистрацию, о начале процедуры реорганизации кредитной организации, порядок уведомления кредиторов реорганизуемых кредитных организаций, порядок предъявления кредиторами требований о досрочном исполнении или прекращении соответствующих обязательств и возмещении убытков, а также порядок раскрытия информации, затрагивающей финансово-хозяйственную деятельность реорганизуемой кредитной организации, определяются законами, регулирующими деятельность кредитных организаций. При этом положения \uline{пунктов 1--5} настоящей статьи к кредитным организациям не применяются.
\eEasyList
\subsubsection{{\bf Статья 61.} Ликвидация юридического лица}
\beginEasyList
& Ликвидация юридического лица влечет его прекращение без перехода прав и обязанностей в порядке правопреемства к другим лицам, за исключением случаев, предусмотренных федеральным законом.
& Юридическое лицо может быть ликвидировано:
\par по решению его учредителей (участников) либо органа юридического лица, уполномоченного на то учредительными документами, в том числе в связи с истечением срока, на который создано юридическое лицо, с достижением цели, ради которой оно создано;
\par по решению суда в случае допущенных при его создании грубых нарушений закона, если эти нарушения носят неустранимый характер, либо осуществления деятельности без надлежащего разрешения (лицензии), либо запрещенной законом, либо с нарушением Конституции Российской Федерации, либо с иными неоднократными или грубыми нарушениями закона или иных правовых актов, либо при систематическом осуществлении некоммерческой организацией, в том числе общественной или религиозной организацией (объединением), благотворительным или иным фондом, деятельности, противоречащей ее уставным целям, а также в иных случаях, предусмотренных настоящим Кодексом.
& Требование о ликвидации юридического лица по основаниям, указанным в \uline{пункте 2} настоящей статьи, может быть предъявлено в суд государственным органом или органом местного самоуправления, которому право на предъявление такого требования предоставлено законом.
\par Решением суда о ликвидации юридического лица на его учредителей (участников) либо орган, уполномоченный на ликвидацию юридического лица его учредительными документами, могут быть возложены обязанности по осуществлению ликвидации юридического лица.
& Юридическое лицо, за исключением учреждения, казенного предприятия, политической партии и религиозной организации, ликвидируется также в соответствии со \uline{статьей 65} настоящего Кодекса вследствие признания его несостоятельным (банкротом). Государственная корпорация или государственная компания может быть ликвидирована вследствие признания ее несостоятельной (банкротом), если это допускается федеральным законом, предусматривающим ее создание. Фонд не может быть признан несостоятельным (банкротом), если это установлено законом, предусматривающим создание и деятельность такого фонда.
\par Если стоимость имущества такого юридического лица недостаточна для удовлетворения требований кредиторов, оно может быть ликвидировано только в порядке, предусмотренном \uline{статьей 65} настоящего Кодекса.
\par Абзац третий утратил силу.
\eEasyList
\subsubsection{{\bf Статья 62.} Обязанности лица, принявшего решение о ликвидации юридического лица}
\beginEasyList
& Учредители (участники) юридического лица или орган, принявшие решение о ликвидации юридического лица, обязаны незамедлительно письменно сообщить об этом в уполномоченный государственный орган для внесения в единый государственный реестр юридических лиц сведения о том, что юридическое лицо находится в процессе ликвидации.
& Учредители (участники) юридического лица или орган, принявшие решение о ликвидации юридического лица, назначают ликвидационную комиссию (ликвидатора) и устанавливают порядок и сроки ликвидации в соответствии с настоящим Кодексом, другими законами.
& С момента назначения ликвидационной комиссии к ней переходят полномочия по управлению делами юридического лица. Ликвидационная комиссия от имени ликвидируемого юридического лица выступает в суде.
\eEasyList
\subsubsection{{\bf Статья 63.} Порядок ликвидации юридического лица}
\beginEasyList
& Ликвидационная комиссия помещает в органах печати, в которых публикуются данные о государственной регистрации юридического лица, публикацию о его ликвидации и о порядке и сроке заявления требований его кредиторами. Этот срок не может быть менее двух месяцев с момента публикации о ликвидации.
\par Ликвидационная комиссия принимает меры к выявлению кредиторов и получению дебиторской задолженности, а также письменно уведомляет кредиторов о ликвидации юридического лица.
& После окончания срока для предъявления требований кредиторами ликвидационная комиссия составляет промежуточный ликвидационный баланс, который содержит сведения о составе имущества ликвидируемого юридического лица, перечне предъявленных кредиторами требований, а также о результатах их рассмотрения.
\par Промежуточный ликвидационный баланс утверждается учредителями (участниками) юридического лица или органом, принявшими решение о ликвидации юридического лица. В случаях, установленных законом, промежуточный ликвидационный баланс утверждается по согласованию с уполномоченным государственным органом.
& Если имеющиеся у ликвидируемого юридического лица (кроме учреждений) денежные средства недостаточны для удовлетворения требований кредиторов, ликвидационная комиссия осуществляет продажу имущества юридического лица с публичных торгов в порядке, установленном для исполнения судебных решений.
\par В случае недостаточности имущества ликвидируемого юридического лица для удовлетворения требований кредиторов либо при наличии признаков банкротства юридического лица ликвидационная комиссия обязана обратиться в арбитражный суд с заявлением о банкротстве юридического лица.
& Выплата денежных сумм кредиторам ликвидируемого юридического лица производится ликвидационной комиссией в порядке очередности, установленной \uline{статьей 64} настоящего Кодекса, в соответствии с промежуточным ликвидационным балансом, начиная со дня его утверждения, за исключением кредиторов третьей и четвертой очереди, выплаты которым производятся по истечении месяца со дня утверждения промежуточного ликвидационного баланса.
& После завершения расчетов с кредиторами ликвидационная комиссия составляет ликвидационный баланс, который утверждается учредителями (участниками) юридического лица или органом, принявшими решение о ликвидации юридического лица. В случаях, установленных законом, ликвидационный баланс утверждается по согласованию с уполномоченным государственным органом.
& При недостаточности у ликвидируемого казенного предприятия имущества, а у ликвидируемого учреждения --- денежных средств для удовлетворения требований кредиторов последние вправе обратиться в суд с иском об удовлетворении оставшейся части требований за счет собственника имущества этого предприятия или учреждения.
& Оставшееся после удовлетворения требований кредиторов имущество юридического лица передается его учредителям (участникам), имеющим вещные права на это имущество или обязательственные права в отношении этого юридического лица, если иное не предусмотрено законом, иными правовыми актами или учредительными документами юридического лица.
& Ликвидация юридического лица считается завершенной, а юридическое лицо --- прекратившим существование после внесения об этом записи в единый государственный реестр юридических лиц.
\eEasyList
\subsubsection{{\bf Статья 64.} Удовлетворение требований кредиторов}
\beginEasyList
& При ликвидации юридического лица требования его кредиторов удовлетворяются в следующей очередности:
\par в первую очередь удовлетворяются требования граждан, перед которыми ликвидируемое юридическое лицо несет ответственность за причинение вреда жизни или здоровью, путем капитализации соответствующих повременных платежей, а также по требованиям о компенсации морального вреда, о компенсации сверх возмещения вреда, причиненного вследствие разрушения, повреждения объекта капитального строительства, нарушения требований безопасности при строительстве объекта капитального строительства, требований к обеспечению безопасной эксплуатации здания, сооружения;
\par во вторую очередь производятся расчеты по выплате выходных пособий и оплате труда лиц, работающих или работавших по трудовому договору, и по выплате вознаграждений авторам результатов интеллектуальной деятельности;
\par в третью очередь производятся расчеты по обязательным платежам в бюджет и во внебюджетные фонды;
\par в четвертую очередь производятся расчеты с другими кредиторами;
\par абзац шестой утратил силу.
\par При ликвидации банков, привлекающих средства физических лиц, в первую очередь удовлетворяются также требования физических лиц, являющихся кредиторами банков по заключенным с ними договорам банковского вклада и (или) договорам банковского счета (за исключением требований физических лиц по возмещению убытков в форме упущенной выгоды и по уплате сумм финансовых санкций и требований физических лиц, занимающихся предпринимательской деятельностью без образования юридического лица, или требований адвокатов, нотариусов, если такие счета открыты для осуществления предусмотренной законом предпринимательской или профессиональной деятельности указанных лиц), требования организации, осуществляющей функции по обязательному страхованию вкладов, в связи с выплатой возмещения по вкладам в соответствии с законом о страховании вкладов физических лиц в банках и Банка России в связи с осуществлением выплат по вкладам физических лиц в банках в соответствии с законом.
& Требования кредиторов каждой очереди удовлетворяются после полного удовлетворения требований кредиторов предыдущей очереди, за исключением требований кредиторов по обязательствам, обеспеченным залогом имущества ликвидируемого юридического лица.
\par Требования кредиторов по обязательствам, обеспеченным залогом имущества ликвидируемого юридического лица, удовлетворяются за счет средств, полученных от продажи предмета залога, преимущественно перед иными кредиторами, за исключением обязательств перед кредиторами первой и второй очереди, права требования по которым возникли до заключения соответствующего договора залога.
\par Не удовлетворенные за счет средств, полученных от продажи предмета залога, требования кредиторов по обязательствам, обеспеченным залогом имущества ликвидируемого юридического лица, удовлетворяются в составе требований кредиторов четвертой очереди.
& При недостаточности имущества ликвидируемого юридического лица оно распределяется между кредиторами соответствующей очереди пропорционально суммам требований, подлежащих удовлетворению, если иное не установлено законом.
& В случае отказа ликвидационной комиссии в удовлетворении требований кредитора либо уклонения от их рассмотрения кредитор вправе до утверждения ликвидационного баланса юридического лица обратиться в суд с иском к ликвидационной комиссии. По решению суда требования кредитора могут быть удовлетворены за счет оставшегося имущества ликвидируемого юридического лица.
& Требования кредитора, заявленные после истечения срока, установленного ликвидационной комиссией для их предъявления, удовлетворяются из имущества ликвидируемого юридического лица, оставшегося после удовлетворения требований кредиторов, заявленных в срок.
& Требования кредиторов, не удовлетворенные из-за недостаточности имущества ликвидируемого юридического лица, считаются погашенными. Погашенными считаются также требования кредиторов, не признанные ликвидационной комиссией, если кредитор не обращался с иском в суд, а также требования, в удовлетворении которых решением суда кредитору отказано.
\eEasyList
\subsubsection{{\bf Статья 65.} Несостоятельность (банкротство) юридического лица}
\beginEasyList
& Юридическое лицо, за исключением казенного предприятия, учреждения, политической партии и религиозной организации, по решению суда может быть признано несостоятельным (банкротом). Государственная корпорация или государственная компания может быть признана несостоятельной (банкротом), если это допускается федеральным законом, предусматривающим ее создание. Фонд не может быть признан несостоятельным (банкротом), если это установлено законом, предусматривающим создание и деятельность такого фонда.
\par Признание юридического лица банкротом судом влечет его ликвидацию.
& Утратил силу.
& Основания признания судом юридического лица несостоятельным (банкротом), порядок ликвидации такого юридического лица, а также очередность удовлетворения требований кредиторов устанавливается законом о несостоятельности (банкротстве).
\eEasyList
\subsection{{\bf § 2. Хозяйственные товарищества и общества}}
\subsection{{\bf 1. Общие положения}}
\subsubsection{{\bf Статья 66.} Основные положения о хозяйственных товариществах и обществах}
\beginEasyList
& Хозяйственными товариществами и обществами признаются коммерческие организации с разделенным на доли (вклады) учредителей (участников) уставным (складочным) капиталом. Имущество, созданное за счет вкладов учредителей (участников), а также произведенное и приобретенное хозяйственным товариществом или обществом в процессе его деятельности, принадлежит ему на праве собственности.
\par В случаях, предусмотренных настоящим Кодексом, хозяйственное общество может быть создано одним лицом, которое становится его единственным участником.
& Хозяйственные товарищества могут создаваться в форме полного товарищества и товарищества на вере (коммандитного товарищества).
& Хозяйственные общества могут создаваться в форме акционерного общества, общества с ограниченной или с дополнительной ответственностью.
& Участниками полных товариществ и полными товарищами в товариществах на вере могут быть индивидуальные предприниматели и (или) коммерческие организации.
\par Участниками хозяйственных обществ и вкладчиками в товариществах на вере могут быть граждане и юридические лица.
\par Государственные органы и органы местного самоуправления не вправе выступать участниками хозяйственных обществ и вкладчиками в товариществах на вере, если иное не установлено законом.
\par Учреждения могут быть участниками хозяйственных обществ и вкладчиками в товариществах с разрешения собственника, если иное не установлено законом.
\par Законом может быть запрещено или ограничено участие отдельных категорий граждан в хозяйственных товариществах и обществах, за исключением открытых акционерных обществ.
& Хозяйственные товарищества и общества могут быть учредителями (участниками) других хозяйственных товариществ и обществ, за исключением случаев, предусмотренных настоящим Кодексом и другими законами.
& Вкладом в имущество хозяйственного товарищества или общества могут быть деньги, ценные бумаги, другие вещи или имущественные права либо иные права, имеющие денежную оценку.
\par Денежная оценка вклада участника хозяйственного общества производится по соглашению между учредителями (участниками) общества и в случаях, предусмотренных законом, подлежит независимой экспертной проверке.
& Хозяйственные товарищества, а также общества с ограниченной и дополнительной ответственностью не вправе выпускать акции.
\eEasyList
\subsubsection{{\bf Статья 67.} Права и обязанности участников хозяйственного товарищества или общества}
\beginEasyList
& Участники хозяйственного товарищества или общества вправе:
\par участвовать в управлении делами товарищества или общества, за исключением случаев, предусмотренных \uline{пунктом 2 статьи 84} настоящего Кодекса и законом об акционерных обществах;
\par получать информацию о деятельности товарищества или общества и знакомиться с его бухгалтерскими книгами и иной документацией в установленном учредительными документами порядке;
\par принимать участие в распределении прибыли;
\par получать в случае ликвидации товарищества или общества часть имущества, оставшегося после расчетов с кредиторами, или его стоимость.
\par Участники хозяйственного товарищества или общества могут иметь и другие права, предусмотренные настоящим Кодексом, законами о хозяйственных обществах, учредительными документами товарищества или общества.
& Участники хозяйственного товарищества или общества обязаны:
\par вносить вклады в порядке, размерах, способами и в сроки, которые предусмотрены учредительными документами;
\par не разглашать конфиденциальную информацию о деятельности товарищества или общества.
\par Участники хозяйственного товарищества или общества могут нести и другие обязанности, предусмотренные его учредительными документами.
\eEasyList
\subsubsection{{\bf Статья 68.} Преобразование хозяйственных товариществ и обществ}
\beginEasyList
& Хозяйственные товарищества и общества одного вида могут преобразовываться в хозяйственные товарищества и общества другого вида или в производственные кооперативы по решению общего собрания участников в порядке, установленном настоящим Кодексом.
& При преобразовании товарищества в общество каждый полный товарищ, ставший участником (акционером) общества, в течение двух лет несет субсидиарную ответственность всем своим имуществом по обязательствам, перешедшим к обществу от товарищества. Отчуждение бывшим товарищем принадлежавших ему долей (акций) не освобождает его от такой ответственности. Правила, изложенные в настоящем пункте, соответственно применяются при преобразовании товарищества в производственный кооператив.
\eEasyList
\subsection{{\bf 2. Полное товарищество}}
\subsubsection{{\bf Статья 69.} Основные положения о полном товариществе}
\beginEasyList
& Полным признается товарищество, участники которого (полные товарищи) в соответствии с заключенным между ними договором занимаются предпринимательской деятельностью от имени товарищества и несут ответственность по его обязательствам принадлежащим им имуществом.
& Лицо может быть участником только одного полного товарищества.
& Фирменное наименование полного товарищества должно содержать либо имена (наименования) всех его участников и слова \symbol{34}полное товарищество\symbol{34}, либо имя (наименование) одного или нескольких участников с добавлением слов \symbol{34}и компания\symbol{34} и слова \symbol{34}полное товарищество\symbol{34}.
\eEasyList
\subsubsection{{\bf Статья 70.} Учредительный договор полного товарищества}
\beginEasyList
& Полное товарищество создается и действует на основании учредительного договора. Учредительный договор подписывается всеми его участниками.
& Учредительный договор полного товарищества должен содержать помимо сведений, указанных в \uline{пункте 2 статьи 52} настоящего Кодекса, условия о размере и составе складочного капитала товарищества; о размере и порядке изменения долей каждого из участников в складочном капитале; о размере, составе, сроках и порядке внесения ими вкладов; об ответственности участников за нарушение обязанностей по внесению вкладов.
\eEasyList
\subsubsection{{\bf Статья 71.} Управление в полном товариществе}
\beginEasyList
& Управление деятельностью полного товарищества осуществляется по общему согласию всех участников. Учредительным договором товарищества могут быть предусмотрены случаи, когда решение принимается большинством голосов участников.
& Каждый участник полного товарищества имеет один голос, если учредительным договором не предусмотрен иной порядок определения количества голосов его участников.
& Каждый участник товарищества независимо от того, уполномочен ли он вести дела товарищества, вправе знакомиться со всей документацией по ведению дел. Отказ от этого права или его ограничение, в том числе по соглашению участников товарищества, ничтожны.
\eEasyList
\subsubsection{{\bf Статья 72.} Ведение дел полного товарищества}
\beginEasyList
& Каждый участник полного товарищества вправе действовать от имени товарищества, если учредительным договором не установлено, что все его участники ведут дела совместно, либо ведение дел поручено отдельным участникам.
\par При совместном ведении дел товарищества его участниками для совершения каждой сделки требуется согласие всех участников товарищества.
\par Если ведение дел товарищества поручается его участниками одному или некоторым из них, остальные участники для совершения сделок от имени товарищества должны иметь доверенность от участника (участников), на которого возложено ведение дел товарищества.
\par В отношениях с третьими лицами товарищество не вправе ссылаться на положения учредительного договора, ограничивающие полномочия участников товарищества, за исключением случаев, когда товарищество докажет, что третье лицо в момент совершения сделки знало или заведомо должно было знать об отсутствии у участника товарищества права действовать от имени товарищества.
& Полномочия на ведение дел товарищества, предоставленные одному или нескольким участникам, могут быть прекращены судом по требованию одного или нескольких других участников товарищества при наличии к тому серьезных оснований, в частности вследствие грубого нарушения уполномоченным лицом (лицами) своих обязанностей или обнаружившейся неспособности его к разумному ведению дел. На основании судебного решения в учредительный договор товарищества вносятся необходимые изменения.
\eEasyList
\subsubsection{{\bf Статья 73.} Обязанности участника полного товарищества}
\beginEasyList
& Участник полного товарищества обязан участвовать в его деятельности в соответствии с условиями учредительного договора.
& Участник полного товарищества обязан внести не менее половины своего вклада в складочный капитал товарищества к моменту его регистрации. Остальная часть должна быть внесена участником в сроки, установленные учредительным договором. При невыполнении указанной обязанности участник обязан уплатить товариществу десять процентов годовых с невнесенной части вклада и возместить причиненные убытки, если иные последствия не установлены учредительным договором.
& Участник полного товарищества не вправе без согласия остальных участников совершать от своего имени в своих интересах или в интересах третьих лиц сделки, однородные с теми, которые составляют предмет деятельности товарищества.
\par При нарушении этого правила товарищество вправе по своему выбору потребовать от такого участника возмещения причиненных товариществу убытков либо передачи товариществу всей приобретенной по таким сделкам выгоды.
\eEasyList
\subsubsection{{\bf Статья 74.} Распределение прибыли и убытков полного товарищества}
\beginEasyList
& Прибыль и убытки полного товарищества распределяются между его участниками пропорционально их долям в складочном капитале, если иное не предусмотрено учредительным договором или иным соглашением участников. Не допускается соглашение об устранении кого-либо из участников товарищества от участия в прибыли или в убытках.
& Если вследствие понесенных товариществом убытков стоимость его чистых активов станет меньше размера его складочного капитала, полученная товариществом прибыль не распределяется между участниками до тех пор, пока стоимость чистых активов не превысит размер складочного капитала.
\eEasyList
\subsubsection{{\bf Статья 75.} Ответственность участников полного товарищества по его обязательствам}
\beginEasyList
& Участники полного товарищества солидарно несут субсидиарную ответственность своим имуществом по обязательствам товарищества.
& Участник полного товарищества, не являющийся его учредителем, отвечает наравне с другими участниками по обязательствам, возникшим до его вступления в товарищество.
\par Участник, выбывший из товарищества, отвечает по обязательствам товарищества, возникшим до момента его выбытия, наравне с оставшимися участниками в течение двух лет со дня утверждения отчета о деятельности товарищества за год, в котором он выбыл из товарищества.
& Соглашение участников товарищества об ограничении или устранении ответственности, предусмотренной в настоящей статье, ничтожно.
\eEasyList
\subsubsection{{\bf Статья 76.} Изменение состава участников полного товарищества}
\beginEasyList
& В случаях выхода или смерти кого-либо из участников полного товарищества, признания одного из них безвестно отсутствующим, недееспособным, или ограниченно дееспособным, либо несостоятельным (банкротом), открытия в отношении одного из участников реорганизационных процедур по решению суда, ликвидации участвующего в товариществе юридического лица либо обращения кредитором одного из участников взыскания на часть имущества, соответствующую его доле в складочном капитале, товарищество может продолжить свою деятельность, если это предусмотрено учредительным договором товарищества или соглашением остающихся участников.
& Участники полного товарищества вправе требовать в судебном порядке исключения кого-либо из участников из товарищества по единогласному решению остающихся участников и при наличии к тому серьезных оснований, в частности вследствие грубого нарушения этим участником своих обязанностей или обнаружившейся неспособности его к разумному ведению дел.
\eEasyList
\subsubsection{{\bf Статья 77.} Выход участника из полного товарищества}
\beginEasyList
& Участник полного товарищества вправе выйти из него, заявив об отказе от участия в товариществе.
\par Отказ от участия в полном товариществе, учрежденном без указания срока, должен быть заявлен участником не менее чем за шесть месяцев до фактического выхода из товарищества. Досрочный отказ от участия в полном товариществе, учрежденном на определенный срок, допускается лишь по уважительной причине.
& Соглашение между участниками товарищества об отказе от права выйти из товарищества ничтожно.
\eEasyList
\subsubsection{{\bf Статья 78.} Последствия выбытия участника из полного товарищества}
\beginEasyList
& Участнику, выбывшему из полного товарищества, выплачивается стоимость части имущества товарищества, соответствующей доле этого участника в складочном капитале, если иное не предусмотрено учредительным договором. По соглашению выбывающего участника с остающимися участниками выплата стоимости имущества может быть заменена выдачей имущества в натуре.
\par Причитающаяся выбывающему участнику часть имущества товарищества или ее стоимость определяется по балансу, составляемому, за исключением случая, предусмотренного в \uline{статье 80} настоящего Кодекса, на момент его выбытия.
& В случае смерти участника полного товарищества его наследник может вступить в полное товарищество лишь с согласия других участников.
\par Юридическое лицо, являющееся правопреемником участвовавшего в полном товариществе реорганизованного юридического лица, вправе вступить в товарищество с согласия других его участников, если иное не предусмотрено учредительным договором товарищества.
\par Расчеты с наследником (правопреемником), не вступившим в товарищество, производятся в соответствии с \uline{пунктом 1} настоящей статьи. Наследник (правопреемник) участника полного товарищества несет ответственность по обязательствам товарищества перед третьими лицами, по которым в соответствии с \uline{пунктом 2 статьи 75} настоящего Кодекса отвечал бы выбывший участник, в пределах перешедшего к нему имущества выбывшего участника товарищества.
& Если один из участников выбыл из товарищества, доли оставшихся участников в складочном капитале товарищества соответственно увеличиваются, если иное не предусмотрено учредительным договором или иным соглашением участников.
\eEasyList
\subsubsection{{\bf Статья 79.} Передача доли участника в складочном капитале полного товарищества}
\par Участник полного товарищества вправе с согласия остальных его участников передать свою долю в складочном капитале или ее часть другому участнику товарищества либо третьему лицу.
\par При передаче доли (части доли) иному лицу к нему переходят полностью или в соответствующей части права, принадлежавшие участнику, передавшему долю (часть доли). Лицо, которому передана доля (часть доли), несет ответственность по обязательствам товарищества в порядке, установленном абзацем первым \uline{пункта 2 статьи 75} настоящего Кодекса.
\par Передача всей доли иному лицу участником товарищества прекращает его участие в товариществе и влечет последствия, предусмотренные \uline{пунктом 2 статьи 75} настоящего Кодекса.
\subsubsection{{\bf Статья 80.} Обращение взыскания на долю участника в складочном капитале полного товарищества}
\par Обращение взыскания на долю участника в складочном капитале полного товарищества по собственным долгам участника допускается лишь при недостатке иного его имущества для покрытия долгов. Кредиторы такого участника вправе потребовать от полного товарищества выдела части имущества товарищества, соответствующей доле должника в складочном капитале, с целью обращения взыскания на это имущество. Подлежащая выделу часть имущества товарищества или его стоимость определяется по балансу, составленному на момент предъявления кредиторами требования о выделе.
\par Обращение взыскания на имущество, соответствующее доле участника в складочном капитале полного товарищества, прекращает его участие в товариществе и влечет последствия, предусмотренные абзацем вторым \uline{пункта 2 статьи 75} настоящего Кодекса.
\subsubsection{{\bf Статья 81.} Ликвидация полного товарищества}
\par Полное товарищество ликвидируется по основаниям, указанным в \uline{статье 61} настоящего Кодекса, а также в случае, когда в товариществе остается единственный участник. Такой участник вправе в течение шести месяцев с момента, когда он стал единственным участником товарищества, преобразовать такое товарищество в хозяйственное общество в порядке, установленном настоящим Кодексом.
\par Полное товарищество ликвидируется также в случаях, указанных в \uline{пункте 1 статьи 76} настоящего Кодекса, если учредительным договором товарищества или соглашением остающихся участников не предусмотрено, что товарищество продолжит свою деятельность.
\subsection{{\bf 3. Товарищество на вере}}
\subsubsection{{\bf Статья 82.} Основные положения о товариществе на вере}
\beginEasyList
& Товариществом на вере (коммандитным товариществом) признается товарищество, в котором наряду с участниками, осуществляющими от имени товарищества предпринимательскую деятельность и отвечающими по обязательствам товарищества своим имуществом (полными товарищами), имеется один или несколько участников --- вкладчиков (коммандитистов), которые несут риск убытков, связанных с деятельностью товарищества, в пределах сумм внесенных ими вкладов и не принимают участия в осуществлении товариществом предпринимательской деятельности.
& Положение полных товарищей, участвующих в товариществе на вере, и их ответственность по обязательствам товарищества определяются правилами настоящего \uline{Кодекса} об участниках полного товарищества.
& Лицо может быть полным товарищем только в одном товариществе на вере.
\par Участник полного товарищества не может быть полным товарищем в товариществе на вере.
\par Полный товарищ в товариществе на вере не может быть участником полного товарищества.
& Фирменное наименование товарищества на вере должно содержать либо имена (наименования) всех полных товарищей и слова \symbol{34}товарищество на вере\symbol{34} или \symbol{34}коммандитное товарищество\symbol{34}, либо имя (наименование) не менее чем одного полного товарища с добавлением слов \symbol{34}и компания\symbol{34} и слова \symbol{34}товарищество на вере\symbol{34} или \symbol{34}коммандитное товарищество\symbol{34}.
\par Если в фирменное наименование товарищества на вере включено имя вкладчика, такой вкладчик становится полным товарищем.
& К товариществу на вере применяются правила настоящего \uline{Кодекса} о полном товариществе постольку, поскольку это не противоречит правилам настоящего Кодекса о товариществе на вере.
\eEasyList
\subsubsection{{\bf Статья 83.} Учредительный договор товарищества на вере}
\beginEasyList
& Товарищество на вере создается и действует на основании учредительного договора. Учредительный договор подписывается всеми полными товарищами.
& Учредительный договор товарищества на вере должен содержать помимо сведений, указанных в \uline{пункте 2 статьи 52} настоящего Кодекса, условия о размере и составе складочного капитала товарищества; о размере и порядке изменения долей каждого из полных товарищей в складочном капитале; о размере, составе, сроках и порядке внесения ими вкладов, их ответственности за нарушение обязанностей по внесению вкладов; о совокупном размере вкладов, вносимых вкладчиками.
\eEasyList
\subsubsection{{\bf Статья 84.} Управление в товариществе на вере и ведение его дел}
\beginEasyList
& Управление деятельностью товарищества на вере осуществляется полными товарищами. Порядок управления и ведения дел такого товарищества его полными товарищами устанавливается ими по правилам настоящего \uline{Кодекса} о полном товариществе.
& Вкладчики не вправе участвовать в управлении и ведении дел товарищества на вере, выступать от его имени иначе, как по доверенности. Они не вправе оспаривать действия полных товарищей по управлению и ведению дел товарищества.
\eEasyList
\subsubsection{{\bf Статья 85.} Права и обязанности вкладчика товарищества на вере}
\beginEasyList
& Вкладчик товарищества на вере обязан внести вклад в складочный капитал. Внесение вклада удостоверяется свидетельством об участии, выдаваемым вкладчику товариществом.
& Вкладчик товарищества на вере имеет право:
&& получать часть прибыли товарищества, причитающуюся на его долю в складочном капитале, в порядке, предусмотренном учредительным договором;
&& знакомиться с годовыми отчетами и балансами товарищества;
&& по окончании финансового года выйти из товарищества и получить свой вклад в порядке, предусмотренном учредительным договором;
&& передать свою долю в складочном капитале или ее часть другому вкладчику или третьему лицу. Вкладчики пользуются преимущественным перед третьими лицами правом покупки доли (ее части) применительно к условиям и порядку, предусмотренным \uline{пунктом 2 статьи 93} настоящего Кодекса. Передача всей доли иному лицу вкладчиком прекращает его участие в товариществе.
\par Учредительным договором товарищества на вере могут предусматриваться и иные права вкладчика.
\eEasyList
\subsubsection{{\bf Статья 86.} Ликвидация товарищества на вере}
\beginEasyList
& Товарищество на вере ликвидируется при выбытии всех участвовавших в нем вкладчиков. Однако полные товарищи вправе вместо ликвидации преобразовать товарищество на вере в полное товарищество.
\par Товарищество на вере ликвидируется также по основаниям ликвидации полного товарищества (\uline{статья 81}). Однако товарищество на вере сохраняется, если в нем остаются по крайней мере один полный товарищ и один вкладчик.
& При ликвидации товарищества на вере, в том числе в случае банкротства, вкладчики имеют преимущественное перед полными товарищами право на получение вкладов из имущества товарищества, оставшегося после удовлетворения требований его кредиторов.
\par Оставшееся после этого имущество товарищества распределяется между полными товарищами и вкладчиками пропорционально их долям в складочном капитале товарищества, если иной порядок не установлен учредительным договором или соглашением полных товарищей и вкладчиков.
\eEasyList
\subsection{{\bf 3.1. Крестьянское (фермерское) хозяйство}}
\subsubsection{{\bf Статья 86.1.} Крестьянское (фермерское) хозяйство}
\beginEasyList
& Граждане, ведущие совместную деятельность в области сельского хозяйства без образования юридического лица на основе соглашения о создании крестьянского (фермерского) хозяйства (\uline{статья 23}), вправе создать юридическое лицо --- крестьянское (фермерское) хозяйство.
\par Крестьянским (фермерским) хозяйством, создаваемым в соответствии с настоящей статьей в качестве юридического лица, признается добровольное объединение граждан на основе членства для совместной производственной или иной хозяйственной деятельности в области сельского хозяйства, основанной на их личном участии и объединении членами крестьянского (фермерского) хозяйства имущественных вкладов.
& Имущество крестьянского (фермерского) хозяйства принадлежит ему на праве собственности.
& Гражданин может быть членом только одного крестьянского (фермерского) хозяйства, созданного в качестве юридического лица.
& При обращении взыскания кредиторов крестьянского (фермерского) хозяйства на земельный участок, находящийся в собственности хозяйства, земельный участок подлежит продаже с публичных торгов в пользу лица, которое в соответствии с законом вправе продолжать использование земельного участка по целевому назначению.
\par Члены крестьянского (фермерского) хозяйства, созданного в качестве юридического лица, несут по обязательствам крестьянского (фермерского) хозяйства субсидиарную ответственность.
& Особенности правового положения крестьянского (фермерского) хозяйства, созданного в качестве юридического лица, определяются законом.
\eEasyList
\subsection{{\bf 4. Общество с ограниченной ответственностью}}
\subsubsection{{\bf Статья 87.} Основные положения об обществе с ограниченной ответственностью}
\beginEasyList
& Обществом с ограниченной ответственностью признается общество, уставный капитал которого разделен на доли; участники общества с ограниченной ответственностью не отвечают по его обязательствам и несут риск убытков, связанных с деятельностью общества, в пределах стоимости принадлежащих им долей.
\par Участники общества, не полностью оплатившие доли, несут солидарную ответственность по обязательствам общества в пределах стоимости неоплаченной части доли каждого из участников.
& Фирменное наименование общества с ограниченной ответственностью должно содержать наименование общества и слова \symbol{34}с ограниченной ответственностью\symbol{34}.
& Правовое положение общества с ограниченной ответственностью и права и обязанности его участников определяются настоящим Кодексом и законом об обществах с ограниченной ответственностью.
\par Особенности правового положения кредитных организаций, созданных в форме обществ с ограниченной ответственностью, права и обязанности их участников определяются также законами, регулирующими деятельность кредитных организаций.
\eEasyList
\subsubsection{{\bf Статья 88.} Участники общества с ограниченной ответственностью}
\beginEasyList
& Число участников общества с ограниченной ответственностью не должно превышать предела, установленного законом об обществах с ограниченной ответственностью. В противном случае оно подлежит преобразованию в акционерное общество в течение года, а по истечении этого срока --- ликвидации в судебном порядке, если число его участников не уменьшится до установленного законом предела.
& Общество с ограниченной ответственностью может быть учреждено одним лицом или может состоять из одного лица, в том числе при создании в результате реорганизации.
\par Общество с ограниченной ответственностью не может иметь в качестве единственного участника другое хозяйственное общество, состоящее из одного лица.
\eEasyList
\subsubsection{{\bf Статья 89.} Учреждение общества с ограниченной ответственностью и его учредительный документ}
\beginEasyList
& Учредители общества с ограниченной ответственностью заключают между собой договор об учреждении общества с ограниченной ответственностью, определяющий порядок осуществления ими совместной деятельности по учреждению общества, размер уставного капитала общества, размер их долей в уставном капитале общества и иные установленные законом об обществах с ограниченной ответственностью условия.
\par Договор об учреждении общества с ограниченной ответственностью заключается в письменной форме.
& Учредители общества с ограниченной ответственностью несут солидарную ответственность по обязательствам, связанным с его учреждением и возникшим до его государственной регистрации.
\par Общество с ограниченной ответственностью несет ответственность по обязательствам учредителей общества, связанным с его учреждением, только в случае последующего одобрения действий учредителей общества общим собранием участников общества. Размер ответственности общества по этим обязательствам учредителей общества может быть ограничен законом об обществах с ограниченной ответственностью.
& Учредительным документом общества с ограниченной ответственностью является его устав.
\par Устав общества с ограниченной ответственностью наряду со сведениями, указанными в \uline{пункте 2 статьи 52} настоящего Кодекса, должен содержать сведения о размере уставного капитала общества, составе и компетенции его органов управления, порядке принятия ими решений (в том числе решений по вопросам, принимаемым единогласно или квалифицированным большинством голосов) и иные предусмотренные законом об обществах с ограниченной ответственностью сведения.
& Порядок совершения иных действий по учреждению общества с ограниченной ответственностью определяется законом об обществах с ограниченной ответственностью.
\eEasyList
\subsubsection{{\bf Статья 90.} Уставный капитал общества с ограниченной ответственностью}
\beginEasyList
& Уставный капитал общества с ограниченной ответственностью составляется из стоимости долей, приобретенных его участниками.
\par Уставный капитал определяет минимальный размер имущества общества, гарантирующего интересы его кредиторов. Размер уставного капитала общества не может быть менее суммы, определенной законом об обществах с ограниченной ответственностью.
& Не допускается освобождение участника общества с ограниченной ответственностью от обязанности оплаты доли в уставном капитале общества.
\par Оплата уставного капитала общества с ограниченной ответственностью при увеличении уставного капитала путем зачета требований к обществу допускается в случаях, предусмотренных законом об обществах с ограниченной ответственностью.
& Уставный капитал общества с ограниченной ответственностью должен быть на момент регистрации общества оплачен его участниками не менее чем наполовину. Оставшаяся неоплаченной часть уставного капитала общества подлежит оплате его участниками в течение первого года деятельности общества. Последствия нарушения этой обязанности определяются законом об обществах с ограниченной ответственностью.
& Если по окончании второго или каждого последующего финансового года стоимость чистых активов общества с ограниченной ответственностью окажется меньше уставного капитала, общество обязано объявить об уменьшении своего уставного капитала и зарегистрировать его уменьшение в установленном порядке. Если стоимость указанных активов общества становится меньше определенного законом минимального размера уставного капитала, общество подлежит ликвидации.
& Уменьшение уставного капитала общества с ограниченной ответственностью допускается после уведомления всех его кредиторов. Последние вправе в этом случае потребовать досрочного прекращения или исполнения соответствующих обязательств общества и возмещения им убытков.
\par Права и обязанности кредиторов кредитных организаций, созданных в форме обществ с ограниченной ответственностью, определяются также законами, регулирующими деятельность кредитных организаций.
& Увеличение уставного капитала общества допускается после полной оплаты всех его долей.
\eEasyList
\subsubsection{{\bf Статья 91.} Управление в обществе с ограниченной ответственностью}
\beginEasyList
& Высшим органом общества с ограниченной ответственностью является общее собрание его участников.
\par В обществе с ограниченной ответственностью создается исполнительный орган (коллегиальный и (или) единоличный), осуществляющий текущее руководство его деятельностью и подотчетный общему собранию его участников. Единоличный орган управления обществом может быть избран также и не из числа его участников.
& Компетенция органов управления обществом, а также порядок принятия ими решений и выступления от имени общества определяются в соответствии с настоящим Кодексом законом об обществах с ограниченной ответственностью и уставом общества.
& К компетенции общего собрания участников общества с ограниченной ответственностью относятся:
&& изменение устава общества, изменение размера его уставного капитала;
&& образование исполнительных органов общества и досрочное прекращение их полномочий, а также принятие решения о передаче полномочий единоличного исполнительного органа общества управляющему, утверждение такого управляющего и условий договора с ним, если уставом общества решение указанных вопросов не отнесено к компетенции совета директоров (наблюдательного совета) общества;
&& утверждение годовых отчетов и бухгалтерских балансов общества и распределение его прибылей и убытков;
&& решение о реорганизации или ликвидации общества;
&& избрание ревизионной комиссии (ревизора) общества.
\par Законом об обществах с ограниченной ответственностью к компетенции общего собрания может быть также отнесено решение иных вопросов.
\par Вопросы, отнесенные к компетенции общего собрания участников общества, не могут быть переданы им на решение исполнительного органа общества.
& Для проверки и подтверждения правильности годовой финансовой отчетности общества с ограниченной ответственностью оно вправе ежегодно привлекать профессионального аудитора, не связанного имущественными интересами с обществом или его участниками (внешний аудит). Аудиторская проверка годовой финансовой отчетности общества может быть также проведена по требованию любого из его участников.
\par Порядок проведения аудиторских проверок деятельности общества определяется законом и уставом общества.
& Опубликование обществом сведений о результатах ведения его дел (публичная отчетность) не требуется, за исключением случаев, предусмотренных законом об обществах с ограниченной ответственностью.
\eEasyList
\subsubsection{{\bf Статья 92.} Реорганизация и ликвидация общества с ограниченной ответственностью}
\beginEasyList
& Общество с ограниченной ответственностью может быть реорганизовано или ликвидировано добровольно по единогласному решению его участников.
\par Иные основания реорганизации и ликвидации общества, а также порядок его реорганизации и ликвидации определяются настоящим Кодексом и другими законами.
& Общество с ограниченной ответственностью вправе преобразоваться в хозяйственное общество другого вида, хозяйственное товарищество или производственный кооператив.
\eEasyList
\subsubsection{{\bf Статья 93.} Переход доли в уставном капитале общества с ограниченной ответственностью к другому лицу}
\beginEasyList
& Переход доли или части доли участника общества в уставном капитале общества с ограниченной ответственностью к другому лицу допускается на основании сделки или в порядке правопреемства либо на ином законном основании с учетом особенностей, предусмотренных настоящим Кодексом и законом об обществах с ограниченной ответственностью.
& Продажа либо отчуждение иным образом доли или части доли в уставном капитале общества с ограниченной ответственностью третьим лицам допускается с соблюдением требований, предусмотренных законом об обществах с ограниченной ответственностью, если это не запрещено уставом общества.
\par Участники общества пользуются преимущественным правом покупки доли или части доли участника общества. Порядок осуществления преимущественного права и срок, в течение которого участники общества могут воспользоваться указанным правом, определяются законом об обществах с ограниченной ответственностью и уставом общества. Уставом общества также может быть предусмотрено преимущественное право покупки обществом доли или части доли участника общества, если другие участники общества не использовали свое преимущественное право покупки доли или части доли в уставном капитале общества.
& В случае, если уставом общества отчуждение доли или части доли, принадлежащих участнику общества, третьим лицам запрещено и другие участники общества отказались от их приобретения либо не получено согласие на отчуждение доли или части доли участнику общества или третьему лицу при условии, что необходимость получить такое согласие предусмотрена уставом общества, общество обязано выплатить участнику действительную стоимость доли или части доли либо выдать ему в натуре имущество, соответствующее такой стоимости.
& Доля участника общества с ограниченной ответственностью может быть отчуждена до полной ее оплаты только в части, в которой она уже оплачена.
& В случае приобретения доли или части доли участника самим обществом с ограниченной ответственностью оно обязано реализовать их другим участникам или третьим лицам в сроки и в порядке, которые предусмотрены законом об обществах с ограниченной ответственностью и уставом, либо уменьшить свой уставный капитал в соответствии с \uline{пунктами 4} и \uline{5 статьи 90} настоящего Кодекса.
& Доли в уставном капитале общества переходят к наследникам граждан и к правопреемникам юридических лиц, являвшихся участниками общества, если иное не предусмотрено уставом общества с ограниченной ответственностью. Уставом общества может быть предусмотрено, что переход доли в уставном капитале общества к наследникам граждан и правопреемникам юридических лиц, являвшихся участниками общества, передача доли, принадлежавшей ликвидированному юридическому лицу, его учредителям (участникам), имеющим вещные права на его имущество или обязательственные права в отношении этого юридического лица, допускаются только с согласия остальных участников общества. Отказ в согласии на переход доли влечет за собой обязанность общества выплатить указанным лицам ее действительную стоимость или выдать им в натуре имущество, соответствующее такой стоимости, в порядке и на условиях, которые предусмотрены законом об обществах с ограниченной ответственностью и уставом общества.
& Переход доли участника общества с ограниченной ответственностью к другому лицу влечет за собой прекращение его участия в обществе.
\eEasyList
\subsubsection{{\bf Статья 94.} Выход участника общества с ограниченной ответственностью из общества}
\beginEasyList
& Участник общества с ограниченной ответственностью вправе выйти из общества путем отчуждения обществу своей доли в его уставном капитале независимо от согласия других его участников или общества, если это предусмотрено уставом общества.
& При выходе участника общества с ограниченной ответственностью из общества ему должна быть выплачена действительная стоимость его доли в уставном капитале общества или выдано в натуре имущество, соответствующее такой стоимости, в порядке, способом и в сроки, которые предусмотрены законом об обществах с ограниченной ответственностью и уставом общества.
\eEasyList
\subsection{{\bf 5. Общество с дополнительной ответственностью}}
\subsubsection{{\bf Статья 95.} Основные положения об обществах с дополнительной ответственностью}
\beginEasyList
& Обществом с дополнительной ответственностью признается общество, уставный капитал которого разделен на доли; участники такого общества солидарно несут субсидиарную ответственность по его обязательствам своим имуществом в одинаковом для всех кратном размере к стоимости их долей, определенном уставом общества. При банкротстве одного из участников его ответственность по обязательствам общества распределяется между остальными участниками пропорционально их вкладам, если иной порядок распределения ответственности не предусмотрен учредительными документами общества.
& Фирменное наименование общества с дополнительной ответственностью должно содержать наименование общества и слова \symbol{34}с дополнительной ответственностью\symbol{34}.
& К обществу с дополнительной ответственностью применяются правила настоящего Кодекса об обществе с ограниченной ответственностью и закона об обществах с ограниченной ответственностью постольку, поскольку иное не предусмотрено настоящей статьей.
\eEasyList
\subsection{{\bf 6. Акционерное общество}}
\subsubsection{{\bf Статья 96.} Основные положения об акционерном обществе}
\beginEasyList
& Акционерным обществом признается общество, уставный капитал которого разделен на определенное число акций; участники акционерного общества (акционеры) не отвечают по его обязательствам и несут риск убытков, связанных с деятельностью общества, в пределах стоимости принадлежащих им акций.
\par Акционеры, не полностью оплатившие акции, несут солидарную ответственность по обязательствам акционерного общества в пределах неоплаченной части стоимости принадлежащих им акций.
& Фирменное наименование акционерного общества должно содержать его наименование и указание на то, что общество является акционерным.
& Правовое положение акционерного общества и права и обязанности акционеров определяются в соответствии с настоящим Кодексом и законом об акционерных обществах.
\par Особенности правового положения акционерных обществ, созданных путем приватизации государственных и муниципальных предприятий, определяются также законами и иными правовыми актами о приватизации этих предприятий.
\par Особенности правового положения кредитных организаций, созданных в форме акционерных обществ, права и обязанности их акционеров определяются также законами, регулирующими деятельность кредитных организаций.
\eEasyList
\subsubsection{{\bf Статья 97.} Открытые и закрытые акционерные общества}
\beginEasyList
& Акционерное общество, участники которого могут отчуждать принадлежащие им акции без согласия других акционеров, признается открытым акционерным обществом. Такое акционерное общество вправе проводить открытую подписку на выпускаемые им акции и их свободную продажу на условиях, устанавливаемых законом и иными правовыми актами.
\par Открытое акционерное общество обязано ежегодно публиковать для всеобщего сведения годовой отчет, бухгалтерский баланс, счет прибылей и убытков.
& Акционерное общество, акции которого распределяются только среди его учредителей или иного заранее определенного круга лиц, признается закрытым акционерным обществом. Такое общество не вправе проводить открытую подписку на выпускаемые им акции либо иным образом предлагать их для приобретения неограниченному кругу лиц.
\par Акционеры закрытого акционерного общества имеют преимущественное право приобретения акций, продаваемых другими акционерами этого общества.
\par Число участников закрытого акционерного общества не должно превышать числа, установленного законом об акционерных обществах, в противном случае оно подлежит преобразованию в открытое акционерное общество в течение года, а по истечении этого срока --- ликвидации в судебном порядке, если их число не уменьшится до установленного законом предела.
\par В случаях, предусмотренных законом об акционерных обществах, закрытое акционерное общество может быть обязано публиковать для всеобщего сведения документы, указанные в \uline{пункте 1} настоящей статьи.
\eEasyList
\subsubsection{{\bf Статья 98.} Образование акционерного общества}
\beginEasyList
& Учредители акционерного общества заключают между собой договор, определяющий порядок осуществления ими совместной деятельности по созданию общества, размер уставного капитала общества, категории выпускаемых акций и порядок их размещения, а также иные условия, предусмотренные законом об акционерных обществах.
\par Договор о создании акционерного общества заключается в письменной форме.
& Учредители акционерного общества несут солидарную ответственность по обязательствам, возникшим до регистрации общества.
\par Общество несет ответственность по обязательствам учредителей, связанным с его созданием, только в случае последующего одобрения их действий общим собранием акционеров.
& Учредительным документом акционерного общества является его устав, утвержденный учредителями.
\par Устав акционерного общества помимо сведений, указанных в \uline{пункте 2 статьи 52} настоящего Кодекса, должен содержать условия о категориях выпускаемых обществом акций, их номинальной стоимости и количестве; о размере уставного капитала общества; о правах акционеров; о составе и компетенции органов управления обществом и порядке принятия ими решений, в том числе о вопросах, решения по которым принимаются единогласно или квалифицированным большинством голосов. В уставе акционерного общества должны также содержаться иные сведения, предусмотренные законом об акционерных обществах.
& Порядок совершения иных действий по созданию акционерного общества, в том числе компетенция учредительного собрания, определяется законом об акционерных обществах.
& Особенности создания акционерных обществ при приватизации государственных и муниципальных предприятий определяются законами и иными правовыми актами о приватизации этих предприятий.
& Акционерное общество может быть создано одним лицом или состоять из одного лица в случае приобретения одним акционером всех акций общества. Сведения об этом должны содержаться в уставе общества, быть зарегистрированы и опубликованы для всеобщего сведения.
\par Акционерное общество не может иметь в качестве единственного участника другое хозяйственное общество, состоящее из одного лица, если иное не установлено законом.
\eEasyList
\subsubsection{{\bf Статья 99.} Уставный капитал акционерного общества}
\beginEasyList
& Уставный капитал акционерного общества составляется из номинальной стоимости акций общества, приобретенных акционерами.
\par Уставный капитал общества определяет минимальный размер имущества общества, гарантирующего интересы его кредиторов. Он не может быть менее размера, предусмотренного законом об акционерных обществах.
& Не допускается освобождение акционера от обязанности оплаты акций общества.
\par Оплата размещаемых обществом дополнительных акций путем зачета требований к обществу допускается в случаях, предусмотренных законом об акционерных обществах.
& Открытая подписка на акции акционерного общества не допускается до полной оплаты уставного капитала. При учреждении акционерного общества все его акции должны быть распределены среди учредителей.
& Если по окончании второго или каждого последующего финансового года стоимость чистых активов общества окажется меньше его уставного капитала, общество обязано принять меры, предусмотренные законом об акционерных обществах.
& Законом или уставом общества могут быть установлены ограничения числа, суммарной номинальной стоимости акций или максимального числа голосов, принадлежащих одному акционеру.
\eEasyList
\subsubsection{{\bf Статья 100.} Увеличение уставного капитала акционерного общества}
\beginEasyList
& Акционерное общество вправе по решению общего собрания акционеров увеличить уставный капитал путем увеличения номинальной стоимости акций или выпуска дополнительных акций.
& Увеличение уставного капитала акционерного общества допускается после его полной оплаты.
& В случаях, предусмотренных законом об акционерных обществах, уставом общества может быть установлено преимущественное право акционеров, владеющих простыми (обыкновенными) или иными голосующими акциями, на покупку дополнительно выпускаемых обществом акций.
\eEasyList
\subsubsection{{\bf Статья 101.} Уменьшение уставного капитала акционерного общества}
\beginEasyList
& Акционерное общество вправе по решению общего собрания акционеров уменьшить уставный капитал путем уменьшения номинальной стоимости акций либо путем покупки части акций в целях сокращения их общего количества.
\par Уменьшение уставного капитала общества допускается после уведомления всех его кредиторов в порядке, определяемом законом об акционерных обществах. Права кредиторов в случае уменьшения уставного капитала общества или снижения стоимости его чистых активов определяются законом об акционерных обществах.
\par Права и обязанности кредиторов кредитных организаций, созданных в форме акционерных обществ, определяются также законами, регулирующими деятельность кредитных организаций.
& Уменьшение уставного капитала акционерного общества путем покупки и погашения части акций допускается, если такая возможность предусмотрена в уставе общества.
\eEasyList
\subsubsection{{\bf Статья 102.} Ограничения на выпуск ценных бумаг и выплату дивидендов акционерного общества}
\beginEasyList
& Доля привилегированных акций в общем объеме уставного капитала акционерного общества не должна превышать двадцати пяти процентов.
& Акционерное общество вправе выпускать облигации только после полной оплаты уставного капитала.
\par Абзац второй утратил силу со 2 января 2013 г.
& Акционерное общество не вправе объявлять и выплачивать дивиденды:
\par до полной оплаты всего уставного капитала;
\par если стоимость чистых активов акционерного общества меньше его уставного капитала и резервного фонда либо станет меньше их размера в результате выплаты дивидендов.
\eEasyList
\subsubsection{{\bf Статья 103.} Управление в акционерном обществе}
\beginEasyList
& Высшим органом управления акционерным обществом является общее собрание его акционеров.
\par К исключительной компетенции общего собрания акционеров относятся:
&& изменение устава общества, в том числе изменение размера его уставного капитала;
&& избрание членов совета директоров (наблюдательного совета) и ревизионной комиссии (ревизора) общества и досрочное прекращение их полномочий;
&& образование исполнительных органов общества и досрочное прекращение их полномочий, если уставом общества решение этих вопросов не отнесено к компетенции совета директоров (наблюдательного совета);
&& утверждение годовых отчетов, бухгалтерских балансов, счетов прибылей и убытков общества и распределение его прибылей и убытков;
&& решение о реорганизации или ликвидации общества.
\par Законом об акционерных обществах к исключительной компетенции общего собрания акционеров может быть также отнесено решение иных вопросов.
\par Вопросы, отнесенные законом к исключительной компетенции общего собрания акционеров, не могут быть переданы им на решение исполнительных органов общества.
& В обществе с числом акционеров более пятидесяти создается совет директоров (наблюдательный совет).
\par В случае создания совета директоров (наблюдательного совета) уставом общества в соответствии с законом об акционерных обществах должна быть определена его исключительная компетенция. Вопросы, отнесенные уставом к исключительной компетенции совета директоров (наблюдательного совета), не могут быть переданы им на решение исполнительных органов общества.
& Исполнительный орган общества может быть коллегиальным (правление, дирекция) и (или) единоличным (директор, генеральный директор). Он осуществляет текущее руководство деятельностью общества и подотчетен совету директоров (наблюдательному совету) и общему собранию акционеров.
\par К компетенции исполнительного органа общества относится решение всех вопросов, не составляющих исключительную компетенцию других органов управления обществом, определенную законом или уставом общества.
\par По решению общего собрания акционеров полномочия исполнительного органа общества могут быть переданы по договору другой коммерческой организации или индивидуальному предпринимателю (управляющему).
& Компетенция органов управления акционерным обществом, а также порядок принятия ими решений и выступления от имени общества определяются в соответствии с настоящим Кодексом законом об акционерных обществах и уставом общества.
& Акционерное общество, обязанное в соответствии с настоящим Кодексом или законом об акционерных обществах публиковать для всеобщего сведения документы, указанные в \uline{пункте 1 статьи 97} настоящего Кодекса, должно для проверки и подтверждения правильности годовой финансовой отчетности ежегодно привлекать профессионального аудитора, не связанного имущественными интересами с обществом или его участниками.
\par Аудиторская проверка деятельности акционерного общества, в том числе и не обязанного публиковать для всеобщего сведения указанные документы, должна быть проведена во всякое время по требованию акционеров, совокупная доля которых в уставном капитале составляет десять или более процентов.
\par Порядок проведения аудиторских проверок деятельности акционерного общества определяется законом и уставом общества.
\eEasyList
\subsubsection{{\bf Статья 104.} Реорганизация и ликвидация акционерного общества}
\beginEasyList
& Акционерное общество может быть реорганизовано или ликвидировано добровольно по решению общего собрания акционеров.
\par Иные основания и порядок реорганизации и ликвидации акционерного общества определяются настоящим Кодексом и другими законами.
& Акционерное общество вправе преобразоваться в общество с ограниченной ответственностью или в производственный кооператив, а также в некоммерческую организацию в соответствии с законом.
\eEasyList
\subsection{{\bf 7. Дочерние и зависимые общества}}
\subsubsection{{\bf Статья 105.} Дочернее хозяйственное общество}
\beginEasyList
& Хозяйственное общество признается дочерним, если другое (основное) хозяйственное общество или товарищество в силу преобладающего участия в его уставном капитале, либо в соответствии с заключенным между ними договором, либо иным образом имеет возможность определять решения, принимаемые таким обществом.
& Дочернее общество не отвечает по долгам основного общества (товарищества).
\par Основное общество (товарищество), которое имеет право давать дочернему обществу, в том числе по договору с ним, обязательные для него указания, отвечает солидарно с дочерним обществом по сделкам, заключенным последним во исполнение таких указаний.
\par В случае несостоятельности (банкротства) дочернего общества по вине основного общества (товарищества) последнее несет субсидиарную ответственность по его долгам.
& Участники (акционеры) дочернего общества вправе требовать возмещения основным обществом (товариществом) убытков, причиненных по его вине дочернему обществу, если иное не установлено законами о хозяйственных обществах.
\eEasyList
\subsubsection{{\bf Статья 106.} Зависимое хозяйственное общество}
\beginEasyList
& Хозяйственное общество признается зависимым, если другое (преобладающее, участвующее) общество имеет более двадцати процентов голосующих акций акционерного общества или двадцати процентов уставного капитала общества с ограниченной ответственностью.
& Хозяйственное общество, которое приобрело более двадцати процентов голосующих акций акционерного общества или двадцати процентов уставного капитала общества с ограниченной ответственностью, обязано незамедлительно публиковать сведения об этом в порядке, предусмотренном законами о хозяйственных обществах.
& Пределы взаимного участия хозяйственных обществ в уставных капиталах друг друга и число голосов, которыми одно из таких обществ может пользоваться на общем собрании участников или акционеров другого общества, определяются законом.
\eEasyList
\subsection{{\bf § 3. Производственные кооперативы}}
\subsubsection{{\bf Статья 107.} Понятие производственного кооператива}
\beginEasyList
& Производственным кооперативом (артелью) признается добровольное объединение граждан на основе членства для совместной производственной или иной хозяйственной деятельности (производство, переработка, сбыт промышленной, сельскохозяйственной и иной продукции, выполнение работ, торговля, бытовое обслуживание, оказание других услуг), основанной на их личном трудовом и ином участии и объединении его членами (участниками) имущественных паевых взносов. Законом и учредительными документами производственного кооператива может быть предусмотрено участие в его деятельности юридических лиц. Производственный кооператив является коммерческой организацией.
& Члены производственного кооператива несут по обязательствам кооператива субсидиарную ответственность в размерах и в порядке, предусмотренных законом о производственных кооперативах и уставом кооператива.
& Фирменное наименование кооператива должно содержать его наименование и слова \symbol{34}производственный кооператив\symbol{34} или \symbol{34}артель\symbol{34}.
& Правовое положение производственных кооперативов и права и обязанности их членов определяются в соответствии с настоящим Кодексом законами о производственных кооперативах.
\eEasyList
\subsubsection{{\bf Статья 108.} Образование производственных кооперативов}
\beginEasyList
& Учредительным документом производственного кооператива является его устав, утверждаемый общим собранием его членов.
& Устав кооператива должен содержать помимо сведений, указанных в \uline{пункте 2 статьи 52} настоящего Кодекса, условия о размере паевых взносов членов кооператива; о составе и порядке внесения паевых взносов членами кооператива и их ответственности за нарушение обязательства по внесению паевых взносов; о характере и порядке трудового участия его членов в деятельности кооператива и их ответственности за нарушение обязательства по личному трудовому участию; о порядке распределения прибыли и убытков кооператива; о размере и условиях субсидиарной ответственности его членов по долгам кооператива; о составе и компетенции органов управления кооперативом и порядке принятия ими решений, в том числе о вопросах, решения по которым принимаются единогласно или квалифицированным большинством голосов.
& Число членов кооператива не должно быть менее пяти.
\eEasyList
\subsubsection{{\bf Статья 109.} Имущество производственного кооператива}
\beginEasyList
& Имущество, находящееся в собственности производственного кооператива, делится на паи его членов в соответствии с уставом кооператива.
\par Уставом кооператива может быть установлено, что определенная часть принадлежащего кооперативу имущества составляет неделимые фонды, используемые на цели, определяемые уставом.
\par Решение об образовании неделимых фондов принимается членами кооператива единогласно, если иное не предусмотрено уставом кооператива.
& Член кооператива обязан внести к моменту регистрации кооператива не менее десяти процентов паевого взноса, а остальную часть --- в течение года с момента регистрации.
& Кооператив не вправе выпускать акции.
& Прибыль кооператива распределяется между его членами в соответствии с их трудовым участием, если иной порядок не предусмотрен законом и уставом кооператива.
\par В таком же порядке распределяется имущество, оставшееся после ликвидации кооператива и удовлетворения требований его кредиторов.
\eEasyList
\subsubsection{{\bf Статья 110.} Управление в производственном кооперативе}
\beginEasyList
& Высшим органом управления кооперативом является общее собрание его членов.
\par В кооперативе с числом членов более пятидесяти может быть создан наблюдательный совет, который осуществляет контроль за деятельностью исполнительных органов кооператива.
\par Исполнительными органами кооператива являются правление и (или) его председатель. Они осуществляют текущее руководство деятельностью кооператива и подотчетны наблюдательному совету и общему собранию членов кооператива.
\par Членами наблюдательного совета и правления кооператива, а также председателем кооператива могут быть только члены кооператива. Член кооператива не может одновременно быть членом наблюдательного совета и членом правления либо председателем кооператива.
& Компетенция органов управления кооперативом и порядок принятия ими решений определяются законом и уставом кооператива.
& К исключительной компетенции общего собрания членов кооператива относятся:
&& изменение устава кооператива;
&& образование наблюдательного совета и прекращение полномочий его членов, а также образование и прекращение полномочий исполнительных органов кооператива, если это право по уставу кооператива не передано его наблюдательному совету;
&& прием и исключение членов кооператива;
&& утверждение годовых отчетов и бухгалтерских балансов кооператива и распределение его прибыли и убытков;
&& решение о реорганизации и ликвидации кооператива.
\par Законом о производственных кооперативах и уставом кооператива к исключительной компетенции общего собрания может быть также отнесено решение иных вопросов.
\par Вопросы, отнесенные к исключительной компетенции общего собрания или наблюдательного совета кооператива, не могут быть переданы ими на решение исполнительных органов кооператива.
& Член кооператива имеет один голос при принятии решений общим собранием.
\eEasyList
\subsubsection{{\bf Статья 111.} Прекращение членства в производственном кооперативе и переход пая}
\beginEasyList
& Член кооператива вправе по своему усмотрению выйти из кооператива. В этом случае ему должна быть выплачена стоимость пая или выдано имущество, соответствующее его паю, а также осуществлены другие выплаты, предусмотренные уставом кооператива.
\par Выплата стоимости пая или выдача другого имущества выходящему члену кооператива производится по окончании финансового года и утверждении бухгалтерского баланса кооператива, если иное не предусмотрено уставом кооператива.
& Член кооператива может быть исключен из кооператива по решению общего собрания в случае неисполнения или ненадлежащего исполнения обязанностей, возложенных на него уставом кооператива, а также в других случаях, предусмотренных законом и уставом кооператива.
\par Член наблюдательного совета или исполнительного органа может быть исключен из кооператива по решению общего собрания в связи с его членством в аналогичном кооперативе.
\par Член кооператива, исключенный из него, имеет право на получение пая и других выплат, предусмотренных уставом кооператива, в соответствии с \uline{пунктом 1} настоящей статьи.
& Член кооператива вправе передать свой пай или его часть другому члену кооператива, если иное не предусмотрено законом и уставом кооператива.
\par Передача пая (его части) гражданину, не являющемуся членом кооператива, допускается лишь с согласия кооператива. В этом случае другие члены кооператива пользуются преимущественным правом покупки такого пая (его части).
& В случае смерти члена производственного кооператива его наследники могут быть приняты в члены кооператива, если иное не предусмотрено уставом кооператива. В противном случае кооператив выплачивает наследникам стоимость пая умершего члена кооператива.
& Обращение взыскания на пай члена производственного кооператива по собственным долгам члена кооператива допускается лишь при недостатке иного его имущества для покрытия таких долгов в порядке, предусмотренном законом и уставом кооператива. Взыскание по долгам члена кооператива не может быть обращено на неделимые фонды кооператива.
\eEasyList
\subsubsection{{\bf Статья 112.} Реорганизация и ликвидация производственных кооперативов}
\beginEasyList
& Производственный кооператив может быть добровольно реорганизован или ликвидирован по решению общего собрания его членов.
\par Иные основания и порядок реорганизации и ликвидации кооператива определяются настоящим Кодексом и другими законами.
& Производственный кооператив по единогласному решению его членов может преобразоваться в хозяйственное товарищество или общество.
\eEasyList
\subsection{{\bf § 4. Государственные и муниципальные унитарные предприятия}}
\subsubsection{{\bf Статья 113.} Унитарное предприятие}
\beginEasyList
& Унитарным предприятием признается коммерческая организация, не наделенная правом собственности на закрепленное за ней собственником имущество. Имущество унитарного предприятия является неделимым и не может быть распределено по вкладам (долям, паям), в том числе между работниками предприятия.
\par Устав унитарного предприятия должен содержать помимо сведений, указанных в \uline{пункте 2 статьи 52} настоящего Кодекса, сведения о предмете и целях деятельности предприятия, а также о размере уставного фонда предприятия, порядке и источниках его формирования, за исключением казенных предприятий.
\par В форме унитарных предприятий могут быть созданы только государственные и муниципальные предприятия.
& Имущество государственного или муниципального унитарного предприятия находится соответственно в государственной или муниципальной собственности и принадлежит такому предприятию на праве хозяйственного ведения или оперативного управления.
& Фирменное наименование унитарного предприятия должно содержать указание на собственника его имущества.
& Органом унитарного предприятия является руководитель, который назначается собственником либо уполномоченным собственником органом и им подотчетен.
& Унитарное предприятие отвечает по своим обязательствам всем принадлежащим ему имуществом.
\par Унитарное предприятие не несет ответственности по обязательствам собственника его имущества.
& Правовое положение государственных и муниципальных унитарных предприятий определяется настоящим Кодексом и законом о государственных и муниципальных унитарных предприятиях.
\eEasyList
\subsubsection{{\bf Статья 114.} Унитарное предприятие, основанное на праве хозяйственного ведения}
\beginEasyList
& Унитарное предприятие, основанное на праве хозяйственного ведения, создается по решению уполномоченного на то государственного органа или органа местного самоуправления, если иное не предусмотрено законом.
& Учредительным документом предприятия, основанного на праве хозяйственного ведения, является его устав, утверждаемый уполномоченным на то государственным органом или органом местного самоуправления, если иное не предусмотрено законом.
& Размер уставного фонда предприятия, основанного на праве хозяйственного ведения, не может быть менее суммы, определенной законом о государственных и муниципальных унитарных предприятиях.
& Порядок формирования уставного фонда предприятия, основанного на праве хозяйственного ведения, определяется законом о государственных и муниципальных унитарных предприятиях.
& Если по окончании финансового года стоимость чистых активов предприятия, основанного на праве хозяйственного ведения, окажется меньше размера уставного фонда, орган, уполномоченный создавать такие предприятия, обязан произвести в установленном порядке уменьшение уставного фонда. Если стоимость чистых активов становится меньше размера, определенного законом, предприятие может быть ликвидировано по решению суда.
& В случае принятия решения об уменьшении уставного фонда предприятие обязано письменно уведомить об этом своих кредиторов.
\par Кредитор предприятия вправе потребовать прекращения или досрочного исполнения обязательства, должником по которому является это предприятие, и возмещения убытков.
& Собственник имущества предприятия, основанного на праве хозяйственного ведения, не отвечает по обязательствам предприятия, за исключением случаев, предусмотренных \uline{пунктом 3 статьи 56} настоящего Кодекса. Это правило также применяется к ответственности предприятия, учредившего дочернее предприятие, по обязательствам последнего.
\eEasyList
\subsubsection{{\bf Статья 115.} Унитарное предприятие, основанное на праве оперативного управления}
\beginEasyList
& В случаях и в порядке, которые предусмотрены законом о государственных и муниципальных унитарных предприятиях, на базе государственного или муниципального имущества может быть создано унитарное предприятие на праве оперативного управления (казенное предприятие).
& Учредительным документом казенного предприятия является его устав, утверждаемый уполномоченным на то государственным органом или органом местного самоуправления.
& Фирменное наименование унитарного предприятия, основанного на праве оперативного управления, должно содержать указание на то, что такое предприятие является казенным.
& Права казенного предприятия на закрепленное за ним имущество определяются в соответствии со \uline{статьями 296} и \uline{297} настоящего Кодекса и законом о государственных и муниципальных унитарных предприятиях.
& Собственник имущества казенного предприятия несет субсидиарную ответственность по обязательствам такого предприятия при недостаточности его имущества.
& Казенное предприятие может быть реорганизовано или ликвидировано в соответствии с законом о государственных и муниципальных унитарных предприятиях.
\eEasyList
\subsection{{\bf § 5. Некоммерческие организации}}
\subsubsection{{\bf Статья 116.} Потребительский кооператив}
\beginEasyList
& Потребительским кооперативом признается добровольное объединение граждан и юридических лиц на основе членства с целью удовлетворения материальных и иных потребностей участников, осуществляемое путем объединения его членами имущественных паевых взносов.
& Устав потребительского кооператива должен содержать помимо сведений, указанных в \uline{пункте 2 статьи 52} настоящего Кодекса, условия о размере паевых взносов членов кооператива; о составе и порядке внесения паевых взносов членами кооператива и об их ответственности за нарушение обязательства по внесению паевых взносов; о составе и компетенции органов управления кооперативом и порядке принятия ими решений, в том числе о вопросах, решения по которым принимаются единогласно или квалифицированным большинством голосов; о порядке покрытия членами кооператива понесенных им убытков.
& Наименование потребительского кооператива должно содержать указание на основную цель его деятельности, а также или слово \symbol{34}кооператив\symbol{34}, или слова \symbol{34}потребительский союз\symbol{34} либо \symbol{34}потребительское общество\symbol{34}.
& Члены потребительского кооператива обязаны в течение трех месяцев после утверждения ежегодного баланса покрыть образовавшиеся убытки путем дополнительных взносов. В случае невыполнения этой обязанности кооператив может быть ликвидирован в судебном порядке по требованию кредиторов.
\par Члены потребительского кооператива солидарно несут субсидиарную ответственность по его обязательствам в пределах невнесенной части дополнительного взноса каждого из членов кооператива.
& Доходы, полученные потребительским кооперативом от предпринимательской деятельности, осуществляемой кооперативом в соответствии с законом и уставом, распределяются между его членами.
& Правовое положение потребительских кооперативов, а также права и обязанности их членов определяются в соответствии с настоящим Кодексом законами о потребительских кооперативах.
\eEasyList
\subsubsection{{\bf Статья 117.} Общественные и религиозные организации (объединения)}
\beginEasyList
& Общественными и религиозными организациями (объединениями) признаются добровольные объединения граждан, в установленном законом порядке объединившихся на основе общности их интересов для удовлетворения духовных или иных нематериальных потребностей.
\par Общественные и религиозные организации являются некоммерческими организациями. Они вправе осуществлять предпринимательскую деятельность лишь для достижения целей, ради которых они созданы, и соответствующую этим целям.
& Участники (члены) общественных и религиозных организаций не сохраняют прав на переданное ими этим организациям в собственность имущество, в том числе на членские взносы. Они не отвечают по обязательствам общественных и религиозных организаций, в которых участвуют в качестве их членов, а указанные организации не отвечают по обязательствам своих членов.
& Особенности правового положения общественных и религиозных организаций как участников отношений, регулируемых настоящим Кодексом, определяются законом.
\eEasyList
\subsubsection{{\bf Статья 118.} Фонды}
\beginEasyList
& Фондом для целей настоящего Кодекса признается не имеющая членства некоммерческая организация, учрежденная гражданами и (или) юридическими лицами на основе добровольных имущественных взносов, преследующая социальные, благотворительные, культурные, образовательные или иные общественно полезные цели.
\par Имущество, переданное фонду его учредителями (учредителем), является собственностью фонда. Учредители не отвечают по обязательствам созданного ими фонда, а фонд не отвечает по обязательствам своих учредителей.
& Фонд использует имущество для целей, определенных в его уставе. Фонд вправе заниматься предпринимательской деятельностью, необходимой для достижения общественно полезных целей, ради которых создан фонд, и соответствующей этим целям. Для осуществления предпринимательской деятельности фонды вправе создавать хозяйственные общества или участвовать в них.
\par Фонд обязан ежегодно публиковать отчеты об использовании своего имущества.
& Порядок управления фондом и порядок формирования его органов определяются его уставом, утверждаемым учредителями.
& Устав фонда помимо сведений, указанных в \uline{пункте 2 статьи 52} настоящего Кодекса, должен содержать: наименование фонда, включающее слово \symbol{34}фонд\symbol{34}, сведения о цели фонда; указания об органах фонда, в том числе о попечительском совете, осуществляющем надзор за деятельностью фонда, о порядке назначения должностных лиц фонда и их освобождения, о месте нахождения фонда, о судьбе имущества фонда в случае его ликвидации.
\eEasyList
\subsubsection{{\bf Статья 119.} Изменение устава и ликвидация фонда}
\beginEasyList
& Устав фонда может быть изменен органами фонда, если уставом предусмотрена возможность его изменения в таком порядке.
\par Если сохранение устава в неизменном виде влечет последствия, которые было невозможно предвидеть при учреждении фонда, а возможность изменения устава в нем не предусмотрена либо устав не изменяется уполномоченными лицами, право внесения изменений принадлежит суду по заявлению органов фонда или органа, уполномоченного осуществлять надзор за его деятельностью.
& Решение о ликвидации фонда может принять только суд по заявлению заинтересованных лиц.
\par Фонд может быть ликвидирован:
&& если имущества фонда недостаточно для осуществления его целей и вероятность получения необходимого имущества нереальна;
&& если цели фонда не могут быть достигнуты, а необходимые изменения целей фонда не могут быть произведены;
&& в случае уклонения фонда в его деятельности от целей, предусмотренных уставом;
&& в других случаях, предусмотренных законом.
& В случае ликвидации фонда его имущество, оставшееся после удовлетворения требований кредиторов, направляется на цели, указанные в уставе фонда.
\eEasyList
\subsubsection{{\bf Статья 120.} Учреждения}
\beginEasyList
& Учреждением признается некоммерческая организация, созданная собственником для осуществления управленческих, социально-культурных или иных функций некоммерческого характера.
\par Права учреждения на имущество, закрепленное за ним собственником, а также на имущество, приобретенное учреждением, определяются в соответствии со \uline{статьей 296} настоящего Кодекса.
& Учреждение может быть создано гражданином или юридическим лицом (частное учреждение) либо соответственно Российской Федерацией, субъектом Российской Федерации, муниципальным образованием (государственное или муниципальное учреждение).
\par Государственное или муниципальное учреждение может быть автономным, бюджетным или казенным учреждением.
\par Частное учреждение полностью или частично финансируется собственником его имущества. Порядок финансового обеспечения деятельности государственных и муниципальных учреждений определяется законом.
\par Частное или казенное учреждение отвечает по своим обязательствам находящимися в его распоряжении денежными средствами. При недостаточности указанных денежных средств субсидиарную ответственность по обязательствам такого учреждения несет собственник его имущества.
\par Автономное учреждение отвечает по своим обязательствам всем находящимся у него на праве оперативного управления имуществом, за исключением недвижимого имущества и особо ценного движимого имущества, закрепленных за автономным учреждением собственником этого имущества или приобретенных автономным учреждением за счет выделенных таким собственником средств. Собственник имущества автономного учреждения не несет ответственность по обязательствам автономного учреждения.
\par Бюджетное учреждение отвечает по своим обязательствам всем находящимся у него на праве оперативного управления имуществом, как закрепленным за бюджетным учреждением собственником имущества, так и приобретенным за счет доходов, полученных от приносящей доход деятельности, за исключением особо ценного движимого имущества, закрепленного за бюджетным учреждением собственником этого имущества или приобретенного бюджетным учреждением за счет выделенных собственником имущества бюджетного учреждения средств, а также недвижимого имущества. Собственник имущества бюджетного учреждения не несет ответственности по обязательствам бюджетного учреждения.
& Особенности правового положения отдельных видов государственных и иных учреждений определяются законом и иными правовыми актами.
\eEasyList
\subsubsection{{\bf Статья 121.} Основные положения об ассоциациях (союзах)}
\beginEasyList
& Ассоциацией (союзом) признается объединение юридических лиц и (или) граждан, основанное на добровольном или в установленных законом случаях на обязательном членстве и созданное для представления и защиты общих, в том числе профессиональных, интересов, для достижения общественно полезных, а также иных не противоречащих закону и имеющих некоммерческий характер целей.
\par Ассоциация (союз) является некоммерческой организацией.
\par В организационно-правовой форме ассоциаций (союзов) создаются, в частности, объединения юридических лиц и (или) граждан, имеющие целями координацию их предпринимательской деятельности, представление и защиту общих имущественных интересов, профессиональные объединения граждан, не имеющие целью защиту трудовых прав и интересов своих членов, профессиональные объединения граждан независимо от наличия или отсутствия у них трудовых отношений с работодателями (объединения адвокатов, нотариусов, оценщиков, лиц творческих профессий и другие), объединения саморегулируемых организаций.
& Ассоциации (союзы) могут иметь гражданские права и нести гражданские обязанности, соответствующие целям создания и деятельности, предусмотренным уставами таких ассоциаций (союзов).
& Ассоциация (союз) является собственником своего имущества. Ассоциация (союз) отвечает по своим обязательствам всем своим имуществом, если иное не предусмотрено законом в отношении отдельных видов ассоциаций (союзов).
& Ассоциация (союз) не отвечает по обязательствам своих членов, если иное не предусмотрено законом.
\par Члены ассоциации (союза) не отвечают по ее обязательствам, за исключением случаев, если законом и (или) уставом ассоциации (союза) предусмотрена субсидиарная ответственность ее членов.
& Ассоциация (союз) по решению своих членов может быть преобразована в общественную организацию, автономную некоммерческую организацию, некоммерческое партнерство или фонд.
& Особенности правового положения ассоциаций (союзов) отдельных видов могут быть установлены законами и иными правовыми актами.
\eEasyList
\subsubsection{{\bf Статья 121.1.} Учредители ассоциации (союза) и устав ассоциации (союза)}
\beginEasyList
& Количество учредителей ассоциации (союза) не может быть менее пяти. Законами, устанавливающими особенности правового положения ассоциаций (союзов) отдельных видов, могут быть установлены иные требования к минимальному количеству учредителей таких ассоциаций (союзов).
& Устав ассоциации (союза) должен определять ее наименование и место нахождения, предмет и цели ее деятельности, содержать условия о порядке вступления (принятия) членов в ассоциацию (союз) и выхода из нее, о составе и компетенции органов ассоциации (союза) и порядке принятия ими решений, в том числе по вопросам, решения по которым принимаются единогласно или квалифицированным большинством голосов, об имущественных правах и обязанностях членов ассоциации (союза), о порядке распределения имущества, оставшегося после ликвидации ассоциации (союза).
\eEasyList
\subsubsection{{\bf Статья 121.2.} Особенности управления ассоциацией (союзом)}
\beginEasyList
& К исключительной компетенции высшего органа ассоциации (союза) относятся:
&& определение приоритетных направлений деятельности ассоциации (союза), принципов образования и использования ее имущества;
&& изменение устава ассоциации (союза);
&& образование других органов ассоциации (союза) и досрочное прекращение их полномочий, если уставом ассоциации (союза) в соответствии с законом это правомочие не отнесено к компетенции иных коллегиальных органов ассоциации (союза);
&& утверждение годовых отчетов и годовых бухгалтерских балансов ассоциации (союза), если уставом ассоциации (союза) в соответствии с законом это правомочие не отнесено к компетенции иных коллегиальных органов ассоциации (союза);
&& принятие решений о создании ассоциацией (союзом) других юридических лиц;
&& принятие решений об участии ассоциации (союза) в других юридических лицах, о создании филиалов и об открытии представительств ассоциации (союза);
&& принятие решений о реорганизации или ликвидации ассоциации (союза), о назначении ликвидационной комиссии (ликвидатора) и об утверждении ликвидационного баланса;
&& избрание ревизионной комиссии (ревизора) и назначение аудиторской организации или индивидуального аудитора (профессионального аудитора) ассоциации (союза);
&& принятие решения о порядке определения размера и способа уплаты членских взносов;
&& принятие решений о дополнительных имущественных взносах членов ассоциации (союза).
& В ассоциации (союзе) образуется единоличный исполнительный орган (председатель, президент или другие), а также могут образовываться постоянно действующие коллегиальные исполнительные органы (совет, правление и тому подобное).
& По решению высшего органа ассоциации (союза) полномочия органов ассоциации (союза) могут быть досрочно прекращены в случаях грубого нарушения ими своих обязанностей, обнаружившейся неспособности к надлежащему ведению дел или при наличии иных серьезных оснований.
\eEasyList
\subsubsection{{\bf Статья 122.} Утратила силу.}
\subsubsection{{\bf Статья 123.} Права и обязанности члена ассоциации (союза)}
\beginEasyList
& Член ассоциации (союза) вправе:
&& в порядке, установленном законом или уставом ассоциации (союза), участвовать в управлении делами ассоциации (союза);
&& в случаях и в порядке, которые предусмотрены законом и уставом ассоциации (союза), получать информацию о деятельности ассоциации (союза), знакомиться с ее бухгалтерской и иной документацией;
&& в порядке, установленном законом, обжаловать решения органов ассоциации (союза), влекущие за собой гражданско-правовые последствия;
&& в случаях, предусмотренных законом, оспаривать совершенные ассоциацией (союзом) сделки и требовать возмещения причиненных ассоциации (союзу) убытков;
&& безвозмездно, если иное не предусмотрено законом, пользоваться оказываемыми ассоциацией (союзом) услугами на равных началах с другими ее членами;
&& по своему усмотрению выйти из ассоциации (союза) по окончании финансового года. В этом случае член ассоциации (союза) несет субсидиарную ответственность по ее обязательствам пропорционально своему взносу в течение двух лет с момента выхода;
&& осуществлять иные права, предусмотренные законом или уставом ассоциации (союза), в порядке, установленном уставом ассоциации (союза).
& Член ассоциации (союза) обязан:
&& участвовать в образовании имущества ассоциации (союза) в порядке, в размере, способом и в сроки, которые предусмотрены уставом ассоциации (союза) в соответствии с настоящим Кодексом или иным законом;
&& не разглашать конфиденциальную информацию о деятельности ассоциации (союза);
&& участвовать в принятии решений, если его участие в соответствии с законом и (или) уставом ассоциации (союза) необходимо для принятия таких решений;
&& не совершать действия, заведомо направленные на причинение вреда ассоциации (союзу), участником которой он является;
&& уплачивать предусмотренные уставом ассоциации (союза) членские взносы;
&& по решению высшего органа ассоциации (союза) вносить дополнительные имущественные взносы.
& Член ассоциации (союза) может быть исключен из нее по решению остающихся членов в случаях и в порядке, которые установлены уставом ассоциации (союза). В отношении ответственности исключенного члена ассоциации (союза) применяются правила, относящиеся к выходу из ассоциации (союза).
& С согласия членов ассоциации (союза) в нее может войти новый член. Вступление в ассоциацию (союз) нового члена может быть обусловлено его субсидиарной ответственностью по обязательствам ассоциации (союза), возникшим до его вступления.
\eEasyList

\end{document}

